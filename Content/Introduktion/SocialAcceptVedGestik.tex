\section{Er semaforiske gestikker social acceptable?}
\label{Socialaccept}
%            
Der fokuseres ikke på manipulerende gestikker i forhold til social accept, hvilket skyldes, at interaktionen og resultatet deraf er synlig. Det antages, at manipulerende gestikker er så veletableret, at både tilskuere og brugeren selv ikke nødvendigvis tænker over, at det er en gestik. \blankline
%
Der inddrages fire undersøgelser, der enten undersøger eller kommenterer på social accept i forhold til brugen af semaforiske gestikker, \parencite{PDF:AChairAsUbiquitousInputDevice, PDF:WouldYouDoThat, PDF:AreYouComfortableDoingThat, PDF:AnExploratoryStudy}. Af de fire undersøgelser fokuserer \textcite{PDF:AreYouComfortableDoingThat} og \textcite{PDF:WouldYouDoThat} på, hvordan semaforiske gestikker opfattes af andre samt i hvilke situationer, brugeren er villig til og komfortable ved at udføre bestemte former for semaforiske gestikker og hvornår de ikke er. I undersøgelsen foretaget af \textcite{PDF:AnExploratoryStudy} fokuseres der på, hvordan sociale elektroniske apparater kan være med til at fremme kontakten mellem mennesker. Hvor der i undersøgelsen, foretaget af \textcite{PDF:AChairAsUbiquitousInputDevice}, fokuseres på en interaktiv stol. Interaktionen undersøges i to scenarier; i den centrale opmærksomhed, hvor testpersonerne interagerer med en computer og i den perifere opmærksomhed, hvor testpersonerne interagerer med en musikafspiller, \parencite{PDF:AChairAsUbiquitousInputDevice}. I begge scenarier giver testpersonerne udtryk for, at de er bekymrede for, hvorvidt deres interaktion er social acceptabel, \parencite[s. 8]{PDF:AChairAsUbiquitousInputDevice}. Bekymringerne vedrører, hvorvidt andre anser gestikkerne, som værende akavede, \parencite[s. 4]{PDF:AChairAsUbiquitousInputDevice}. Dog pointerer \textcite[s. 9]{PDF:AChairAsUbiquitousInputDevice}, at det formentligt skyldes, at gestikken er synlig, men resultatet deraf er usynligt, hvorfor det kun er testpersonen, der kan høre musikken ændrer sig. \textcite[s. 9]{PDF:AChairAsUbiquitousInputDevice} argumenterer for, at hvis den interaktive stol implementeres i et, i forvejen, interaktivt miljø, kan det potentielt reducere den akavede fornemmelse.\blankline
%
Der er to overordnede måder hvorpå social accept defineres. Det defineres enten ud fra brugerens synspunkt eller tilskuerens synspunkt. Fra brugerens synspunkt vedrører social accept, hvilket indtryk brugeren får ved at udføre interaktionen i forhold til om oplevelsen var positiv eller negativ, \parencite[s. 276]{PDF:WouldYouDoThat}. Social accept fra tilskuerens synspunkt afhænger af, hvordan tilskueren vurderer brugerens adfærd som værende enten positiv eller negativ, \parencite[s. 276]{PDF:WouldYouDoThat}. Ifølge \textcite[s. 276]{PDF:WouldYouDoThat} påvirker følgende faktorer social accept: Brugertype, tid og sted samt manipulation kontra effekt. Brugertype henviser til, om brugeren hurtigt accepterer gestik, som interaktionsform eller om de modsætter sig. Tid og sted henviser til kultur og hvor længe den pågældende teknologi har været på markedet, hvilket kan være med til at øge den sociale accept. Manipulation kontra effekt henviser til de specifikke gestikker og hvad de resulterer i.

Ifølge \textcite[s. 276]{PDF:WouldYouDoThat} kan gestikker inddeles i følgende fire kategorier: Udtryksfyldte, spændingsfyldte, hemmelighedsfyldte og magiske. Hvor de udtryksfyldte gestikker både har synlig manipulation og synlig effekt, modsat de hemmelighedsfyldte gestikker, som både har usynlig manipulation og usynlig effekt. De magiske gestikker defineres som gestikker, hvor manipulationen er usynlig, men effekten er synlig, modsat de spændingsfyldte, som har synlig manipulation og usynlig effekt. Baseret på resultaterne fremsat af \textcite[s. 277]{PDF:WouldYouDoThat} fremgår det, at de spændingsfyldte gestikker ikke anses for, at være socialt acceptable, hvilket gør sig gældende både i offentlige og private sammenhænge. De hemmelighedsfyldte gestikker rangeres højt i forhold til social accept, både i det offentlig og i det private. Ydermere fremgår det af resultaterne, at både de udtryksfyldte og magiske gestikke generelt opleves som værende socialt acceptable, \parencite[s. 277]{PDF:WouldYouDoThat}. 

Gestikker hvor manipulationen er synlig må nødvendigvis have en synlig effekt, for at opnå social accept. Alternativt kan manipulationen være usynlig, hvor effekten enten kan være synlig eller usynlig for stadig at opnå social accept, \parencite[s. 278]{PDF:WouldYouDoThat}. Disse kategorier inden for social accept er uafhængige af hvilken type gestik, der udføres, i den forstand, at de ikke kobles til én specifik form for gestik men nærmere henvender sig til hvordan gestikken udføres.\blankline
%
I undersøgelsen foretaget af \textcite[s. 193]{PDF:AreYouComfortableDoingThat}, fokuseres der dels på den sociale accept og dels på, om brugeren føler sig komfortabel ved at udføre gestikker i området omkring et elektronisk apparat. Det fremgår, at der er en stærk forbindelse mellem gestikkens nødvendige bevægelsesmængde, varigheden hvorved gestikken udføres samt positionen, relativt til det elektroniske apparat, \parencite[s. 193]{PDF:AreYouComfortableDoingThat}. Derudover fremgår det ligeledes, at brugere er både selektive og bevidste om, hvor interaktionen foregår samt hvem der overværer interaktionen, \parencite[s. 193]{PDF:AreYouComfortableDoingThat}. Denne problemstilling undersøges i tre forskellige henseende; 1) området og afstanden til det elektroniske apparat, 2) bevægelsesmængden og varigheden af gestikken og 3) tilskuere, \parencite[ss. 195-200]{PDF:AreYouComfortableDoingThat}. 

Ud fra den første undersøgelse fremgår det, at testpersonerne føler sig mest komfortable ved at udføre gestikkerne i området tæt på, lige over og til højre for det elektroniske apparat, \parencite[s. 197]{PDF:AreYouComfortableDoingThat}. Endvidere vurderer \textcite[s. 201]{PDF:AreYouComfortableDoingThat}, at grænsen for, hvornår en gestik er i risiko for, at blive anset som værende social uacceptabel, er når afstanden til det elektroniske apparat når 30 cm. I forhold til hvem testpersonerne føler sig komfortable ved at udføre gestikker overfor, fremgår det, at det primært er personer, som de har et tæt forhold til, såsom venner, familie og deres partnere, \parencite[s. 196]{PDF:AreYouComfortableDoingThat}. Dertil føler testpersonerne sig mest ukomfortable overfor fremmede og kollegaer. Testpersonerne vurderer ydermere, at de to mest ukomfortable lokationer, at udføre gestikkerne på er i offentligtransport og på et museum. 

Ved den næste undersøgelse, som vedrører bevægelsesmængde og varighed, defineres en bevægelsesmængde inden for et område på 15x15 cm, som værende lille, hvorimod et område på 30x30 cm anses for at være stor, \parencite[s. 198]{PDF:AreYouComfortableDoingThat}. Gestikkerne udføres ved tre forskellige varigheder; tre sekunder, seks sekunder og ni sekunder. På baggrund af resultaterne fra undersøgelsen fremgår det, at gestikker med en varighed på mindre end seks sekunder klart er at fortrække, hvis testpersonerne skal føle sig komfortable, \parencite[s. 199]{PDF:AreYouComfortableDoingThat}. Ydermere foretrækker testpersonerne en lille bevægelsesmængde, dog pointerer \textcite[s. 199]{PDF:AreYouComfortableDoingThat}, at såfremt en gestik med en stor bevægelsesmængde udføres i en favorabel lokation, tæt på det elektroniske apparat og har en kort varighed, kan der kompenseres for den ellers ukomfortable oplevelse. 

Ved den tredje undersøgelse, som vedrører tilskuerne, fremgår det, at tilskuerne ikke fandt gestikkerne påtrængende og en del af tilskuerne tænkte ikke yderliger over de gestikker, som de lige havde overværet, \parencite[s. 200]{PDF:AreYouComfortableDoingThat}. Ifølge \textcite[s. 200]{PDF:AreYouComfortableDoingThat} har tilskuerne en tendens til at vurdere den sociale accept højere end hvad tilfældet er ved de to foregående undersøgelser. \blankline
%
\textcite{PDF:AnExploratoryStudy} undersøger, hvordan semaforiske gestikker kan understøtte interaktionen med elektroniske apparater, ved at fremme den sociale kontakt mellem mennesker. Baseret på disse resultater tyder det på, at gestikker med en stor bevægelsesmængde kan være socialt acceptable og komfortable, hvis de udføres hjemme hos ens venner, \parencite[s. 4]{PDF:AnExploratoryStudy}. Endvidere finder testpersonerne det mindre pinligt, at udføre gestikker fremfor at anvende stemmestyring eller lydkontrol, \parencite[s. 4]{PDF:AnExploratoryStudy}. Det fremgår ydermere, at testpersonerne foretrækker, at gestikkerne benyttes i en positiv sammenhæng fremfor en negativ sammenhæng, hvor de foretrækker at gestikkerne er mindre synlige, \parencite[s. 4]{PDF:AnExploratoryStudy}. \blankline
%
For at afgøre om semaforiske gestikker er socialt acceptable, afhænger det af flere aspekter; dels hvilket miljø gestikken optræder i, forholdet mellem manipulation og effekt i forhold til hvad der er synligt kontra usynligt, gestikkens bevægelsesmængde og varighed, området og afstanden til det elektroniske apparat og hvilken type tilskuere, der overvære gestikken. For at opsummere nogle af de anbefalinger, der forefindes i de fire undersøgelser, tyder det på, at brugerne i forhold til miljø er mere tilbøjelige til at føle sig komfortable i private omgivelser og såfremt gestikken udføres i det offentlige, skal den være forholdvist diskret. En af årsageren til, at testpersonerne i undersøgelsen, foretaget af \textcite[s. 4]{PDF:AChairAsUbiquitousInputDevice}, udtrykker bekymringer i forbindelse med den interaktive stol, skyldes formentlig, at det er en spændingsfyldt gestik.

I henhold til manipulation og effekt anbefaler \textcite[s. 278]{PDF:WouldYouDoThat}, at der netop ikke anvendes spændingsfyldte gestikker, hvilket efterlader de udtryksfyldte, hemmelighedsfyldte og magiske gestikker til fri afbenyttelse for at opnå social accept. I relation til bevægelsesmængden anbefaler \textcite[s. 201]{PDF:AreYouComfortableDoingThat}, at såfremt gestikken udføres i det offentlige, skal der være en forholdvist lille bevægelsesmængde i et område svarende til 15x15 cm. Ifølge \textcite[s. 278]{PDF:WouldYouDoThat}, er både små og store bevægelsesmængder socialt acceptable, så længe effekten er synlig.

Ydermere skal varigheden være kortere end seks sekunder, da risikoen for at brugeren føler sig ukomfortabel stiger. I forbindelse med området og afstanden vurderer \textcite[s. 201]{PDF:AreYouComfortableDoingThat}, at området til højre og foran det elektroniske apparat er at foretrække, særligt for højrehåndede brugere, ellers skal der kompenseres for afstanden. Ifølge \textcite[s. 201]{PDF:AreYouComfortableDoingThat} er grænsen for, hvornår det er komfortabelt samt social acceptabelt, omkring 30 cm mellem det elektroniske apparat og gestikken.\blankline
%
På baggrund af de foregående to kapitler er brugen af perifer interaktion samt hvilke forbehold, der skal tages i forbindelse med brugen af semaforiske gestikker, blevet specificeret, hvorfor projektet nu kan afgrænses.


