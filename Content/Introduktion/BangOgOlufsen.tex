\chapter{Perifer interaktion til musikkontrol}
\label{PeriferInteraktionTilMusikKontrol}
%
Da det tyder på, at interaktionen med vores elektroniske apparater i dag, primært finder sted i vores centrale opmærksomhed, vil det fremadrettet være fordelagtigt, at virksomheder overvejer andre interaktionsformer, heriblandt perifer interaktion. Da dette projekt foregår i samarbejde med en af verdens førende virksomheder inden for \textit{high-end} luksus lydudstyr; Bang $\&$ Olufsen, så vil perifer interaktion selvsagt være i forbindelse med musikkontrol. Igennem det følgende kapitel vil først og fremmest blive specificeret, hvilken sammenhæng denne perifer interaktion skal foregå i. Efterfølgende fokuseres der på det teoretiske bag perifer interaktion, samt hvor det udspringer fra. Der fokuseres ydermere på relateret undersøgelser, som netop anvender perifer interaktion til musikkontrol. Derefter bliver der gået nærmere i detaljer om, hvordan perifer interaktion rent faktisk kan foregå; ved hjælp af gestik. Afslutningsvist vil der i kapitlet ledes over i en problemformulering, som danner grundlag for det fremadrettede projektarbejde.    
%
\section{Sammenspil med Bang $\&$ Olufsen}
\label{SammenspilMedBO}
%
I vores højteknologiske verden er det særligt vigtigt for virksomheder i elektronik industrien, at være i konstant udvikling for dels at kunne følge med teknologien og dels for at skabe revolutionerende og eftertragtede produkter, der indeholder den nyeste teknologi. Det gælder, for disse virksomheder, både om at forudse, hvilke nye produkter brugeren ønsker, men også hvilken form for revolutionerende interaktion, der skal være med produktet. Specielt interaktionsdelen er interessant i et samarbejde med Bang $\&$ Olufsen, hvor implementering af nye interaktionsformer, herunder perifer interaktion, kan være med til at forbedre brugeroplevelsen.

Perifer interaktion er nemlig en interaktionsform, som i større grad vil finde indpas i fremtiden, for på den måde at give et mere gnidningsløst sammenspil mellem mennesker og produkter, \parencite[s. 1]{PDF:PIIntroduction}. I samarbejde med Bang $\&$ Olufsen er det interessant at undersøge, hvordan perifer interaktion samt tilhørende gestikker kan bruges til at styre et musikanlæg, ophængt på væggen, så det ikke er nødvendigt at rejse sig fra sofaen, finde en fjernbetjening eller afbryde en samtale, når eksempelvis lydstyrken skal justeres. Selvom perifer interaktion med elektroniske apparater er uundgåeligt, jævnfør \textcite[s. 1]{PDF:PIIntroduction}, er det ikke nødvendigvis alle funktioner, der kan og bør styres perifert. Ved et interview med Lyle Clarke hos Bang $\&$ Olufsen udspecificeres både de primære og sekundære funktioner, som Bang $\&$ Olufsens produkter skal indeholde. De primære funktioner, samt formålet med disse, er opstillet i \autoref{tab:BogOsPrimaereFunktioner}, hvor de sekundære funktioner, samt formålet med disse, er opstillet i \autoref{tab:BogOsSekundaereFunktioner}.
%
\begin{table}[H]
	\centering
	\begin{tabular}{ | l | p{8cm} |}
		\hline
		\multicolumn{1}{|l|}{\textbf{Primære funktioner}} & \multicolumn{1}{l|}{\textbf{Formål}} \\ \hline
		Start & Start musikken \\ \hline
		Pause & Paus Musikken \\ \hline
		Standby & Musik sat på pause \\ \hline
		Forward/backward in time & Skift sang frem eller tilbage \\ \hline
		Intensity up/down & Justering af intensiteten op og ned (lyd, lys osv.) \\ \hline
		Like & Sæt en sang til farvoritter \\ \hline
		Dismiss & Fjern denne og lignende sange fra playlister \\ \hline
		Shift source & Skift musikkilde (radio, CD, AUX osv.) \\ \hline
		Shift experiences & Skift mellem hvilken højtaler der skal spilles sammen med (skift mellem zoner) \\ \hline
		Initiate interaction & Start interaktion med produktet \\ \hline
		Confirm & Bekræft interaktion med produktet \\ \hline
	\end{tabular}
	\caption{Primære funktioner i Bang $\&$ Olufsens produkter.}
	\label{tab:BogOsPrimaereFunktioner}
\end{table}
\noindent
%
%
\begin{table}[H]
	\centering
	\begin{tabular}{ | l | p{8cm} |}
		\hline
		\multicolumn{1}{|l|}{\textbf{Sekundære funktioner}} & \multicolumn{1}{l|}{\textbf{Formål}} \\ \hline
		Fastforward/fastbackward & Spol frem eller tilbage i musikken \\ \hline
		Stay in mood & Bliv i pågældende genre/stemning \\ \hline
		Send/expand & Flyt den afspillede musik til en anden højtaler \\ \hline
		Store & Gem sang/kunstner/playliste \\ \hline
		Add to.. & Tilføj til.. (eksempelvis playliste) \\ \hline
		Play next & Sæt i kø som det næste nummer \\ \hline
		Play later & Sæt i kø til sidst \\ \hline
	\end{tabular}
	\caption{Sekundære funktioner i Bang $\&$ Olufsens produkter.}
	\label{tab:BogOsSekundaereFunktioner}
\end{table}
\noindent
%
Af disse funktioner er det ikke alle, der giver mening at tilknytte en specifik gestik, da interaktionen ikke nødvendigvis behøver at være perifer for samtlige funktioner. 

Eksempelvis giver det god mening, at bruge den centrale opmærksomhed, når der skal spoles i et musiknummer eller når et bestemt nummer skal tilføjes til en bestemt playliste. Det er hovedsaligt de sekundære funktioner, men også flere af de primære funktioner, der med fordel bør kunne styres på selve produktet eller i en tilhørende applikation. På den måde er det kun de vigtigste funktioner, som et musikanlæg består af, der kan styres perifert med gestikker. Ud fra interviewet med Lyle Clarke besluttes det, at disse, primære, funktioner, hvortil der ønskes en tilknyttet gestikker, er start, pause, skru op og ned for musikken samt skift nummer frem og tilbage. De resterende funktioner skal naturligvis være tilgængelig, men egner sig højst sandsynligt bedst til direkte fysisk interaktion med musikanlægget eller ved brug af en tilhørende applikation.\blankline
%
For at få en idé om hvordan perifer interaktion kan bruges i elektroniske apparater, vil der i det følgende afsnit blive beskrevet teorien bag perifer interaktion og hvordan gestikker kan bidrage til perifer interaktion. 

