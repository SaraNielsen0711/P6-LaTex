\section{Delkonklusion}
\label{ValgAfGestikkerDelkonklusion}
%
På baggrund af foregående analyse er semaforiske gestikker til de seks mest gængse funktioner på et musikanlæg valgt. Jævnfør \fullref{TestresultaterPauseStart} er GP5 valgt til at pause og starte musikken, hvor hånden lukker som et krokodillenæb for at pause musikken og åbner op igen for at starte musikken, \autoref{fig:GestikPar5Pause}. For at skifte musiknummer er GP5 ligeledes valgt, jævnfør \fullref{TestresultaterSkiftMusiknummer}, hvor der skiftes til det næste musiknummer med en swipe-bevægelse fra højre mod venstre samtidig med at tommel-, pege- og langefinger holdes strakt, som illustreret på \autoref{fig:GestikPar5Skift}. GP2 er valgt til at justere lydstyrken, jævnfør \fullref{TestresultaterVolumen}, hvor der drejes på en fiktiv drejeknap, hvilket illustreres på \autoref{fig:GestikPar2Volumen}. Størstedelen af de testpersoner, der har deltaget i undersøgelsen har en positiv indstilling til at interaktionen med et musikanlæg kan foregå via semaforiske gestikker. Det på trods af, at ingen af testpersonerne betragter sig selv som værende de første til at anskaffe sig den nyeste teknologi, jævnfør \fullref{TestresultaterInteraktioner}. Ydermere har størstedelen af testpersonerne ikke et problem med, at blive registreret af deres musikanlæg, dog med de forebehold at det skal være muligt, at deaktivere registreringen og at naturlige samtale-gestikker ikke fejlagtigt registreres af musikanlægget. For at imødekomme testpersonernes bekymringer, kan det være en fordel at de semaforiske gestikker rettes direkte mod musikanlægget samt at denne interaktionsform kun fungerer, såfremt musikanlægget allerede er tændt. \blankline
%
Da de semaforiske gestikker til de seks mest gængse funktioner i et musikanlæg er udvalgt, er det relevant at teste disse i en social kontekst. I den sociale kontekst kan både den sociale accept af gestikkerne samt muligheden for, at interagere med musikanlægget sideløbende med en primær opgave undersøges. 