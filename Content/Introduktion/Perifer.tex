\chapter{Perifer interaktion}
\label{PeriferInteratkion}
%
\begin{itemize}
  \item Perifer interaktion - de to yderpunkter
  \item Opmærksomhed (perifer opmærksomhed)
  \item Inddrag figur fra projektforslag (side 118 i bogen eller side 6, der er den røde cirkel der ikke)
  \item Hvorfor er det interesant at designe produkter, som kan reagere på perifer interaktion? (fordele og ulemper) 
  \item Inddrag at det er relativt nyt og ukendt at have produkter, som man kan interagere med perifert 
  \item Microgestures 
  \item Socialt acceptabelt 
  \item Hvordan har perifer interaktion ændret sig? (nu er det blevet skærmbaseret)
  \item Adfærdsændringer
  \item Relateret produkter  
  \item Baggrund og teori
  \item Calm og Casual
\end{itemize}
%

%Flere og flere elektroniske apparater kommer ud på markedet og kræver vores opmærksomhed. Når vi skal interagere med flere af disse apparater på en gang bliver de kognitive ressourcer belastet, i nogle tilfælde overbelastet, hvilket kan ende ud i en dårlig oplevelse af et eller flere produkter. (mangler kilde!). For bedre at udnytte de kognitive ressourcer, kan det derfor være favorabelt at interaktion med elektroniske produkter i højere grad foregår i den perifere opmærksomhed. På den måde kan interaktionen med produkterne foregå uden at den fulde(centrale?) opmærksomhed behøver at blive rettet mod dem. Udover de kognitive ressourcer, som de elektroniske apparater belaster, så kan det også være socialt hæmmende hele tiden at skulle rette blikket og opmærksomheden mod en smartphone, der vibrerer, eller et musikanlæg, der spiller for højt. I vores nymoderne verden er alt smart. Interaktionen med disse smart-produkter befinder sig i den centrale del af vores opmærksomhed - og det er næsten udelukkende det visuelle system produkterne retter sig imod \parencite[s. 249]{PDF:PeripheralInteraction}. Det kan derfor være interessant ikke blot at flytte interaktionen med produkter fra den centrale til den perifere opmærksomhed, men også flytte interaktionen med produkterne til den perifere interaktion, hvor det visuelle system ikke nødvendigvis skal bruges. Perifer interaktion er ikke noget nyt. Selvom der skal bruges fuld opmærksomhed de første gange et objekt skal interageres med, kan interaktionen gennem gentagelser blive overført til det perifere. Når vi nu alle sammen bruger den perifere interaktion i løbet af dagen til eksempelvis at flytte en kaffekop op til munden eller tage tøj på, hvorfor så ikke også interagere med vores elektronik perifert?\\


Perifer interaktion er ikke nyt. Vi gør det alle sammen, når vi i løbet af vores dagligdag foretager adskillige aktiviteter i vores perifere opmærksomhed, \parencite[s. 1]{PDF:PeripheralInteraction}. Det forudsætter dog at disse perifere interaktioner primært retter sig mod ikke-computer-relateret aktiviteter, såsom at føre en samtale imedens der laves mad. Rettes fokus derimod til \textit{Human-Computer-Interaction}, HCI, så forlanger diverse elektroniske apparater vores centrale opmærksomhed ved at blinke, ringe og vibrere, \parencite[s. 1]{PDF:PeripheralInteraction}. Ifølge \textcite[s. 3]{PDF:PeripheralInteraction} så bevæger interaktionen med elektroniske apparater sig uforudsigeligt mellem den centrale og perifere opmærksomhed, i modsætning til ikke-computer-relateret aktiviteter, som i større grad kan kontrolleres. Grunden til at disse apparater frit kan bevæge sig fra den perifere til den centrale opmærksomhed skyldes, at vi sjælendt har kontrol over hvornår et apparat kræver den centrale opmærksomhed. Så snart et elektronisk apparat befinder sig i den centrale opmærksomhed, så vil ens opmærksomhed midlertidigt fjernes fra den primære opgave, indtil interaktionen med apparatet er afsluttet. Konsekvenserne af at blive forstyrret og skulle rette opmærksomheden væk fra ens primære opgave over til apparatet, er en øget kognitiv belastning samt en større risiko for at begå fejl i den primære opgave, \parencite[ss. 188-189][s. 162]{PDF:PeripheralInteraction, PDF:ComparingInputModalities}. 

Istedet for at forstætte med at designe elektroniske apparater, som kræver den centrale opmærksomhed samt kræver at opmærksomheden flyttes fra den primære opgave over til apparatet, så bør apparaterne derimod designes så de i højere grad bliver en integreret del af vores daglige rutiner, \parencite[s. 239]{PDF:PICharacteristicsAndConsiderations}. Ved at designe apparater ud fra den tilgang så vil det tillade perifer interaktion fordi det bygger på teori omkring delt-opmærksomhed. Ifølge disse teorier råder mennesket over en bestemt mængde af mentale resourcer, som afhængigt af opgavernes sværhedsgrad, kan distribueres frit mellem opgaverne, \parencite[s. 240]{PDF:PICharacteristicsAndConsiderations}.  



 Så istedet for at elektroniske apparater kræver den fulde opmærksomhed, så er det ved perifer interaktion muligt at dele sin opmærksomhed for på den måde at foretage flere aktivitet samtidig. Derudover vil       

  



%
\begin{figure}[H]
	\centering
	\includegraphics[resolution=300,width=\textwidth]{LevelsOfInteraction}
	\caption{ny \textcite[s. 118]{PDF:PeripheralInteraction}.}
	\label{fig:LevelsOfInteraction}
\end{figure}
\noindent
%

