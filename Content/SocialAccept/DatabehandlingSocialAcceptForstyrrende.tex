\section{Forstyrrende elementer ved interaktionsformen}
\label{TestresultaterSocialAcceptForstyrrende}
%
Følgende analyse og diskusion foretages på baggrund af værternes respons til spørgsmålene: \textit{Oplevede du, at det var forstyrrende at lave gestikkerne imens I planlagde ferie?} og \textit{Følte du, at du var i stand til både at planlægge ferien og styre musikken?} samt gæsternes respons til spørgsmålene: \textit{Oplevede du at det var forstyrrende at din ven(inde)/kæreste lavede gestikker?} og \textit{Følte du, at din ven(inde)/kæreste var i stand til både at planlægge ferien sammen med dig og styre musikken?}. \blankline
%
Ud af de 12 testpersoner syntes ingen af dem, at gestikkerne var forstyrrende. Derimod syntes seks testpersoner, herunder tre værter og tre gæster, at de forskellige hints forstyrrede, fordi de afbrød flowet i samtalen. Det er primært opringningen, som er forstyrrende for samtalen og derudover påpeger gæst 3, at det var forstyrrende at musikken nogen gange blev meget høj. Tre af værterne giver udtryk for, at de koncentrerede sig lidt om at lytte efter de forskellige hints, mens de planlagde ferien. Heriblandt giver vært 4 udtryk for, at være opmærksom på om musikken begyndte at skratte, men nævner samtidig, at det i en virkelig situation ikke vil være noget problem, da der ikke kommer hints på samme måde. Ifølge vært 6, så sad værten nogle gange og lyttede til musikken for at finde ud af, om det var kvaliteten af musiknummeret, der var dårlig, eller om der var en decideret forvrængning af musikken, som værten skulle reagere på. To af testpersonerne, vært 3 og gæst 3, giver udtryk for, at det efter kort tid blev mere naturligt, at interaktionen foregik med gestikker. Gæst 3 giver udtryk for at være meget fascineret af interaktionsformen i starten, hvorefter det blev mere naturligt. Vært 3 nævner derimod, at værten i første omgang havde i baghovedet, at skulle reagere på hints, men at det med tiden blev naturligt at håndtere de forskellige hints, når de blev præsenteret. 

Når der spørges ind til om gestikkerne var forstyrrende at udføre nævner vært 5, at det ikke forstyrrer mere at interaktionen foregår via semaforiske gestikker, sammenlignet med at interaktionen foregår på sin egen mobiltelefon. Gæst 5 og vært 6 giver i denne sammenhæng udtryk for, at det er mindre forstyrrende at interaktionen foregår via semaforiske gestikker sammenliget med at benytte sig af sin mobiltelefon eller en fjernbetjening. Dette er favorabelt hvis formålet, på sigt, er at opnå en perifer interaktion ved gentagne brug af de udvalgte semaforiske gestikker, samtidig med at de, med alt forventning, ikke har en negativ effekt på det sociale samværd. \blankline
%
Resultaterne gengiver ydermere, at samtlig testpersoner, uanset om de er værter og gæster, følte at værten var i stand til både at deltage i ferieplanlægning og styre musikanlægget. Selvom gæst 1 giver udtryk for at vært 1 var i stand til at deltage i begge opgaver, gav gæsten udtryk for at værten nogle gange mistede fokus. Dog nævner gæst 1, at der selvfølgelig skulle interageres mere med musikanlægget i en testsituation end hvad tilfældet vil være i virkeligheden, hvilket kan være årsagen til værtens, til tider, manglende fokus på ferieplanlægningen. Gæst 3 nævner, at begge parter blev lidt distraheret, når musikken eksempelvis blev skruet op, hvilket resulterede i at de kortvarigt fjernede fokus fra ferieplanlægningen for at skrue ned. Dog giver gæst 3 udtryk for at vært 3 derefter bare skruede ned, hvorefter samtalen kunne genoptages efter den kortvarige afbrydelse. \blankline
%
Selvom det ud fra resultaterne ikke er muligt, entydigt, at påvise, at interaktionen kan foregå i den perifere opmærksomhed forefindes der på baggrund af resultaterne evidens for, at interaktionen kan foregå helt eller delvist sideløbende med en primæropgave, hvilket indikerer at de udvalgte semaforiske gestikker ved gentagende brug kan danne grundlag for en perifer interaktion.

