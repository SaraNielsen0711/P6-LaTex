\section{Holdning til gestik genkendelse}
\label{TestresultaterOvervaagning}
%
I exit-interviewet blev de 18 testpersoner spurgt, hvordan de vil have det med at systemet altid kan opfange deres gestikker. Hertil er det kun testperson 6, testperson 7 og testperson 18, som ikke direkte udtrykker, at det er i orden, at produktet registrerer deres bevægelser. Ifølge testperson 6 afhænger det af udbyderen; gemmes data i et datacenter og om de videregiver data, hvorimod testperson 7 foretrækker, at det er en form for sensor, som registrerer bevægelserne og ikke et kamera, da det ved et kamera potentielt kan være mulig for fremmede at tvinge sig adgang, for at se om testpersonen er hjemme eller ej. Dog kommenterer testpersonen at det er en meget lille ting. Ifølge testperson 18, så vil det være værre hvis produktet har internetforbindelse, da det så er muligt at sende data. Dog anerkender testpersonen, at det er en nødvendighed for at produktet både kan blive bedre og lære undervejs. Testperson 18 fremsætter en lister af krav som producenten, i det her tilfælde Bang $\&$ Olufsen, skal kunne forsvare, før testpersonen vil have det fint med registreringen. På testpersonens liste over ting producenten skal forsvare indgår, blandt andet; dokumentation for hvad det bruges til, hvor sendes data hen, hvem har adgang til data, hvilken kryptering anvendes der, optager den eller filmer den bare, hvem og hvordan lagres data. Såfremt testperson 18 får information om dette, så vil testpersonen være villig til at gå på kompromis. 

Fælles for testperson 6 og testperson 18 er, at de begge giver udtryk for, at der skal være mulighed for, at slukke produktet så det ikke længere registrerer gestikker. Det at antages at dette er en selvfølge og at produktet kun registrerer brugerens gestikker når de aktivt har tændt musikanlægget og at så snart det slukkes, så stopper registreringen. Selvom testperson 10 giver udtryk for at det er i orden, at produktet registrerer gestikkerne, så foreslår testpersonen ligeledes at det skal være muligt at tænde og slukke musikanlægget uden semaforiske gestikker. Lignende fremgår fra testperson 1, som ligeledes foretrækker at det kan slukkes. Testperson 16 giver udtryk for, at medmindre data kan sendes op i en sky og benyttes af andre og at testpersonen ikke bliver filmet på den måde, så er det fint at produktet registrerer gestikkerne. Lignende går igen ved testperson 11, som heller ikke vil have data sendes nogen steder hen og ved testperson 3, som ikke vil have at data gemmes. I tillæg foretrækker testperson 3, at det kun er bevægelserne, som registreres og ikke at der ikke foretages en decideret videooptagelse, som der eksempelvist blev gjort under testen.

Testperson 14 kommenterer, at det vil være ubehageligt hvis registreringen foregik via CPR-nummer og at produktet på den måde kunne registrerer testpersonens færden, men hvis det bare er testpersonens radio eller musikanlæg, så vil det være i orden. Udover testperson 14, så kommenterer tre andre testpersoner, at det er essentielt at produktet ikke fejlagtigt registrerer en tilfældig bevægelse som en semaforisk gestik, der er knyttet til en af funktionerne. Hvor testperson 8 foreslår at gestikkerne skal være retningsbestemt, så de kun registreres når de laves i en bestemt retning. I tillæg kommenterer testperson 5, at gestikkerne kun kan registreres i bestemte zoner, kommentaren kommer i forbindelse med at testpersonen kun vil have at registreringen foregår i bestemte rum. Derudover påpeger testperson 3 og testperson 12, at det kun er bevægelserne, der skal registreres. 

Det eneste forbehold testperson 4 giver udtryk for er, at testpersonen ikke vil have hele huset fyldt med kameraer i frygt for, at det bliver grimt. Ifølge testperson 13, så anses registreringen ikke for et problem for det er den vej teknologien går. Lignende holdning kommer til udtryk fra testperson 5 og testperson 6, som påpeger at alle andre produkter jo også overvåger en. I tillæg pointerer testperson 17, at hvis testpersonen betragtede registreringen som et problem, så kunne testpersonen jo bare lade være med at købe det. Endvidere kommenterer testperson 2 og testperson 15, at de ikke vil have et problem med at produktet registrerer dem.\blankline
%
I og med at størstedelen direkte giver udtryk for, at de har det fint med at produktet registrerer dem, så understøtter det kun forventningen om at denne type interaktion kan finde indpas i dagligdagen, hvert fald så længe at produktet fungerer optimalt. Et optimalt fungerende produkt vil i denne sammenhæng være et produkt, der ikke fejlagtigt reagerer på gestikker, som er en naturlig del af ens kropssprog, som værende gestikker henvendt til musikanlægget, hvilket bekymre både testperson 8, testperson 9, testperson 12 og særligt testperson 14, som giver udtryk for generelt at være meget gestikulerende. For at undgå dette kan foreslag fra både testperson 5 og testperson 8 tages til eftertragtning.  

Ud fra testpersonernes holdninger tyder det ikke på at det vil være et problem at have en form sensor eller kamera indbygget i et musikanlæg, som derfra kan reagere på brugerens gestikker. Ydermere kan det, afhængigt af de teknologiske begrænsninger, være nødvendigt gestikkerne rettes direkte mod anlægget, hvilket formentlig vil reducere risikoen for at produktet fejlfortolker gestikkerne. For at imødekomme ønsket fra de testpersoner, som efterspørger at produktet kun kan registrere gestikker, såfremt de har tændt musikanlægget på selve anlægget, så bør dette implementeres af Bang $\&$ Olufsen i deres fremtidige musikanlæg. Når brugeren ikke længere vil høre musik og derfor slukker musikanlægget, så skal registreringssystemet ligeledes deaktiveres, så brugeren dermed ikke skal bekymre sig om, hvorvidt de bliver registreret eller ej.  

