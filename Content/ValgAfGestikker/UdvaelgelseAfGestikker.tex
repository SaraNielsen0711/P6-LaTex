\section{Udvælgelse af gestikker}
\label{UdvaelgelseAfGestikker}
%
Hat - handler om at finde de gestikker fra litteraturen og dem vi selv vælger at lave test med + formål og problemformulering fra testprotokol
%

\begin{itemize}
  \item opsummering af hvilke gestikker vi ikke vil og gerne vil kigge på.
  \item Forklaring af de forskellige gestikker der kan bruges
  \begin{itemize}
  	\item Gestikker fra litteraturen
  	\item Gestikker overført fra Bang $\&$ Olufsens eksisterende produkter
  	\item Vores egne idéer til gestikker
  \end{itemize}
  \item Fordele og ulemper ved gestikkerne
\end{itemize}

For at vælge hvilke gestikker der skal bruges til at styre et musikanlæg undersøges det først og fremmest, hvilke gestikker der er blevet brugt i andre studier - enten til at styre et musikanlæg eller eksempelvis et TV med lignende funktioner. Derudover blev det undersøgt hvilke gestikker der i forvejen bruges i Bang $\&$ Olufsens produkter og hvorvidt disse kan overføres fra en 2D flade til 3D. Både gestikker fra andre studier og Bang $\&$ Olufsens egne gestikker blev ydermere brugt som inspiration til forslag på andre gestikker. Studerende i de omkringliggende grupperum blev spurgt ind til deres input og der blev designet nogle gestikker af projektgruppen selv. Alle de indsamlede forslag fremgår af \autoref{tab:IndsamledeGestikkerPause}, \autoref{tab:IndsamledeGestikkerSkift} og \autoref{tab:IndsamledeGestikkerVolumen}. Er forslaget kommet fra tidligere studier eller Bang $\&$ Olufsens produkter er der blevet henvist. Er forslaget derimod kommet fra andre studerende eller projektgruppen selv er der ikke skrevet en henvisning. 




\begin{table}[H]
	\centering
	\begin{tabular}{| p{4cm} | p{4cm} | p{4cm} |}
		\hline
		\textbf{Pause} & \textbf{Start} & \textbf{Kilde} \\ \hline
		Stationært stop tegn & Stationært stop tegn & \parencite[s. 166]{PDF:ComparingInputModalities} \\ \hline
		Dynamisk stop tegn, der starter ved brystet og ender i udstrakt arm  & Dynamisk stop tegn, der starter ved brystet og ender i udstrakt arm &  \\ \hline
		Kryds i luften & Flueben i luften & \parencite[s. 48]{PDF:UserDefinedGesturesTV} \\ \hline
		Hånden lukker sammen hen mod anlægget & Hånden åbner op hen mod anlægget & \\ \hline
		Hånden lukker sammen vertikalt nedad & Hånden åbner op vertikalt opad &  \\ \hline
		Peg og tryk i luften hen mod anlægget & Peg og tryk i luften hen mod anlægget & \parencite[s. 48]{PDF:UserDefinedGesturesTV} \\ \hline
		Krokodillenæb med hånden, der lukker sammen & Krokodillenæb med hånden, der åbner op & \parencite[s. 48]{PDF:UserDefinedGesturesTV} \\ \hline
	\end{tabular}
	\caption{Oversigt over forslag til pause- og start-gestikker. Hvis forslaget stammer fra tidligere studier eller Bang $\&$ Olufsens produkter er der lavet en henvisning, ellers stammer forslagene fra andre studerende og projektgruppen selv.}
	\label{tab:IndsamledeGestikkerPause}
\end{table}
\noindent

\begin{table}[H]
	\centering
	\begin{tabular}{| p{4cm} | p{4cm} | p{4cm} |}
		\hline
		\textbf{Skift musiknummer frem} & \textbf{Skift musiknummer tilbage} & \textbf{Kilde} \\ \hline
		Stationært stop tegn & Stationært stop tegn & \parencite[s. 166]{PDF:ComparingInputModalities} \\ \hline
		Dynamisk stop tegn, der starter ved brystet og ender i udstrakt arm  & Dynamisk stop tegn, der starter ved brystet og ender i udstrakt arm &  \\ \hline
		Kryds i luften & Flueben i luften & \parencite[s. 48]{PDF:UserDefinedGesturesTV} \\ \hline
		Hånden lukker sammen hen mod anlægget & Hånden åbner op hen mod anlægget & \\ \hline
		Hånden lukker sammen vertikalt nedad & Hånden åbner op vertikalt opad &  \\ \hline
		Peg og tryk i luften hen mod anlægget & Peg og tryk i luften hen mod anlægget & \parencite[s. 48]{PDF:UserDefinedGesturesTV} \\ \hline
		Krokodillenæb med hånden, der lukker sammen & Krokodillenæb med hånden, der åbner op & \parencite[s. 48]{PDF:UserDefinedGesturesTV} \\ \hline
	\end{tabular}
%	\caption{Oversigt over forslag til gestikker, der skifter musiknummer. Hvis forslaget stammer fra tidligere studier eller Bang $/&$ Olufsens produkter er der lavet en henvisning, ellers stammer forslagene fra andre studerende og projektgruppen selv.}
	\label{tab:IndsamledeGestikkerSkift}
\end{table}
\noindent
