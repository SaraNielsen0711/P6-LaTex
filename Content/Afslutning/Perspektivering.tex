\chapter{Perspektivering}
\label{Perspektivering}
%
I følgende afsnit vil der fokuseret på perspektiverende overvejelser omkring projektet, herunder hvordan videre arbejde kan forløbe.

Da Bang $\&$ Olufsen på nuværende tidspunkt ikke har en færdig prototype, har det ikke været muligt at lave brugertests af prototypen, hvilket vil være at foretrække, hvis en perifer interaktion skal påvises. Til videre undersøgelse af perifer interaktion vil det være fordelagtigt at undersøge musikanlægget og interaktionsformen ved en feltundersøgelse, hvor testpersonerne kan lære og vænne sig til at styre musikanlægget med semaforiske gestikker over en længere periode. Ved gentagende interaktion med musikanlægget kan det, ud fra resultaterne fundet i dette projekt, højst sandsynligt påvises, at interaktionen kan foregå i den perifere del af opmærksomheden. Udover det fordelagtige ved at kunne teste brugerne over en længere periode, så er det også fordelagtigt at kunne teste en virkende prototype, frem for at udføre et \textit{Wizard of Oz}-eksperiment. Her kan netop aspekter som musikanlæggets reaktionstid og genkendelsessytem testes, for at undersøge, hvorvidt brugerne har brug for aumenteret feedback eller hvorvidt brugerne udføre de valgte semaforiske gestikker i anden kontekst end ved interaktion med musikanlægget.\blankline
%
På baggrund af de to tests udført i dette projekt kan det udledes, at flere af testpersonerne har problemer med at forstå, hvor meget den valgte semforiske gestik til at justere lydstyrken skruer op for lydstyrken, når de dreje hånden. Der kan derfor vælges at lave flere brugertest med denne gestik i fokus, så den bedste måde at justere lydstyrken på sit anlæg findes. Da der høsjt sandsynligt ikke er nogen, der skruer fra 0 til 100 på en gang, så kan der være en idé at lade en rotation med hånden, svarende til det maksimale hånden kan rotere, svare til eksempelvis 15$\%$'s lydjustering på et anlæg. På den måde skal grebet tages igen, hvis der skal skrues højere op, uden at der ved et enkelt tag kan skrues alt for højt op. Et andet forslag kan være at tage udgangspunkt i en drejeknap, der kan skrue fra 0 til 100 på en omgang. Her kan det at dreje eksempelvis 45$^{\circ}$ på drejeknappen svare til at dreje hånden 90$^{\circ}$, hvorfor man på samme måde vil have fat i musikken. At få musikken til at følge bevægelserne synkront på denne måde, kræver højst sandsynligt at anlægget 1) er klar over at det er nu, det skal reagere og 2) at sensorene er gode nok til at opfange små præcise bevægelser.\blankline
%
I de foregående test er der hovedsaligt brugt unge studerende, der alle sammen virker til at have samme præferencer 



og muligvis også, hvordan et musikanlæg, styret ved brug af semaforiske gestikker, vil accepteres i sociale sammenhænge.  

Hvilken cool gesture kan vi lave, hvis man vil shuffle musikken?

ældre mennesker, måske laver de/forstår de gestikker anderledes?

In-situ test, når prototypen er færdig

Volumeknappen, hvordan man kan gøre med den? Den skal man nok også lave flere test af. 