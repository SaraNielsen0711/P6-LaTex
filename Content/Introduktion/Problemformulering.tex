\section{Problemformulering}
\label{Problemformulering}
%
I projektet er der fokus på, at undersøge, hvilke semaforiske gestikker, der skal afspejle de seks mest gængse funktioner et musikanlæg indeholder; start, pause, justering af lydstyrken og skift musiknummer frem og tilbage. Derudover vil det blive undersøgt, hvorvidt de valgte semaforiske gestikker kan benyttes til at interagere med Bang $\&$ Olufsens fremtidige musikanlæg sideløbende med en primær opgave, for på den måde at undersøge, om interaktionen kan blive perifer. Ydermere vil det undersøges, hvorvidt de valgte semaforiske gestikker forårsager social accept eller ej. Dette leder op til følgende problemstillinger:\blankline
%
\begin{quotation}
	\noindent
	\textit{Hvilke specifikke semaforiske gestikker skal knyttes til hver af de seks mest gængse funktioner i Bang $\&$ Olufsens fremtidige musikanlæg, for at interaktionen kan foregå i den perifere opmærksomhed?\blankline
		%
		Hvordan vil disse semaforiske gestikker påvirke den sociale accept?}\blankline
\end{quotation}
%
For at besvare den første problemstilling er det nødvendigt først at foretage en undersøgelse af, hvilke semaforiske gestikker eventuelle brugere tilknytter hver af de seks førnævnte funktioner, for derefter at undersøge, hvorvidt de valgte gestikker kan bruges til interaktion sideløbende med en primær opgave. Når den perifere interaktion undersøges, vil feedback stadig blive taget in mente og såfremt det tyder på, at feedback vil være nødvendig for at fremme den perifere interaktion, så vil dette blive undersøgt og diskuteret nærmere. Som nævnt tidligere bliver en interaktion først perifer, når den er blevet en rutine og derfor kun kræver få mentale ressourcer, jævnfør \fullref{PeriferInteratkion}. Der skal derfor tages højde for at de testpersoner, som deltager i de nødvendige undersøgelser, ikke nødvendigvis oplever interaktionen med musikanlægget som værende i den perifere opmærksomhed. Det forventes derfor ikke at, at en perifer interaktion kan påvises, men derimod undersøges det, hvorvidt interaktionen har potentiale for at blive perifer ved gentagende brug. 

Da det tyder på, at brugere generelt bliver påvirket af den sociale kontekst både i forhold til lokation og tilskuere, jævnfør \fullref{Socialaccept}, så vil det være fordelagtigt at undersøge interaktionen i to forskellige scenarier; når brugeren er alene og når brugeren er sammen med andre. På baggrund af den undersøgelse bør det være muligt, at besvare den sidst nævnte problemstilling og tilmed vurdere om de semaforiske gestikker opfattes, som værende socialt acceptable eller ej. 




