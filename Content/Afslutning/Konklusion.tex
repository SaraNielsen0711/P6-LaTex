\chapter{Konklusion}
\label{Konklusion}
%
Der vil i konklusionen tilstræbes, at besvare den fremsatte problemformulering: 
%
\begin{quotation}
	\noindent
	\textit{Hvilke semaforiske gestikker skal knyttes til hver af de seks mest gængse funktioner i Bang $\&$ Olufsens fremtidige musikanlæg, for at interaktionen kan foregå i den perifere opmærksomhed?\blankline
		%
		Hvordan vil de udvalgte semaforiske gestikker påvirke den sociale accept?}\blankline
\end{quotation}
%
Da det er valgt, at definere de seks funktioner i par af tre, som udgør henholdvis; pause og start, skift musiknummer frem og tilbage, samt justering af lydstyrken, udvælges der et tilhørende gestik-par. For at pause musikken gengives et krokodillenæb, der lukker sammen og for at starte musikken gengives et krokodillenæb, der åbner op, jævnfør \autoref{fig:GestikPar5PauseApp}. For at skifte musiknummer knyttes en swipe-bevægelse fra højre mod venstre med pege- og langefinger strakt, for at skifte til det næste musiknummer og fra venstre mod højre for at skifte til det forrige musiknummer, jævnfør \autoref{fig:GestikPar5SkiftApp}. Til at justere lydstyrken stod valget mellem tre gestik-par; GP2, GP3 og GP4, hvor det i denne undersøgelse vælges at fokusere på GP2, som gengives ved en cirkulær bevægelse med halvbøjede fingre, som hvis det var en drejeknap lydstyrken justeres på, med uret for at skrue op og mod uret for at skrue ned, jævnfør \autoref{fig:GestikPar2VolumenApp}. Det konkluderes at de tre valgte gestik-par egner sig til hver deres funktion, da de forbindes med henholdvis en \enquote{ti stille}-bevægelse, en swipe-bevægelse ligesom hvis interaktionen foregik på en smartphone eller tablet og hvordan der normalvist, ved brug af en drejeknap, justeres på lydstyrken på et musikanlæg. Derudover konkluderes det, at de tre gestik-par ikke forekommer i naturligt i kropssproget, hvorfor risikoen for et \textit{Midas touch problem} minimeres. I tillæg konkluderes det, at fordi testpersonerne direkte forbinder de udvalgte gestikker med noget velkendt, skal de ikke først lære gestikken, hvorfor det ydermere konkluderes, at gestikkerne i forvejen ikke kræver en stor mængde kognitive ressourcer.\blankline 
%
På baggrund af undersøgelsen af interaktionsformen i en social kontekst konkluderes det, at de udvalgte semaforiske gestikker ikke påvirker den social accept negativt, jævnfør \fullref{SocialAcceptDelkonklusion}. Derimod konkluderes det, at de udvalgte semaforiske gestikker både socialt accepteres i en social kontekst, hvor der er flere personer tilstede og alene. Derudover forefindes der ingen indikationer af, at de udvalgte semaforiske gestikker er forstyrrende for samtalen mellem vært og gæst, da begge parter vurderer, at værten er i stand til at interagere med musikanlægget samtidig med at føre samtalen med gæsten, jævnfør \fullref{SocialAcceptDelkonklusion}. 

I \fullref{DiskussionFunktionerCasualInteraction} diskuteres hvilke forudsætninger, der skal være mødt før en interaktion kan foregå i den perifere opmærksomhed og selvom det ikke direkte kan påvises, hvorvidt interaktionen med musikanlægget foregår i den perifere opmærksomhed, konkluderes det at forudsætningerne tilnærmelsesvist er mødt, hvorfor det ligeledes konkluderes, at denne interaktion kan foregå i den perifere opmærksomhed. Den eneste forudsætning, der ikke imødekommes relaterer sig til, at testpersonerne ikke nødvendigvis har repeteret interaktionen gentagende gange, så det bliver en indøvet rutine.\blankline
%  


















Det kan på baggrund af foregående analyse og diskussion konkluderes, at semaforiske gestikker fungerer til at styre de mest gængse funktioner på et musikanlæg. Herunder egner et krokodillenæb der lukker sammen og åbner op sig godt til henholdsvis at pause og starte musikken. Ydermere konkluderes det, at en swipe bevægelse med pege-langefinger egner sig godt til at skifte musiknummer, mens det, at tage fat i en fiktiv drejeknap og dreje enten med eller mod uret egner sig godt til at justere lydstyrken. Ud fra foregående undersøgelser konkluderes det, at interaktionen med musikanlægget kan foregå sideløbende med en primær opgave, hvorfor disse semaforiske gestikker ved gentagende brug danner grundlag for en potentiel perifer interaktion. 

Ved undersøgelse af interaktionsformen en social kontekst påvises det, at brugen af semaforiske gestikker ikke har negativ effekt på en samtale mellem to personer. Ydermere kan det ud fra testpersonernes respons konkluderes, at de valgte gestikker er socialt acceptable. 

Positiv indstilling til semaforiske gestikker

Det skal være muligt at deaktivere 

gestik-funktionen (grund overvågnings-problemer, mange mennesker i huset osv) 

Såfremt at det fungerer optimalt og er nemmere end at trykke på en knap

De er nemme at lære – egner sig godt (måske er det sagt)

Problemer ved lydjustering med GP2 – kan man lave noget aktivering her 

I stand til at udføre begge opgaver


