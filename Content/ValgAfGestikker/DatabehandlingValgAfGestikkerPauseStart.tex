\section{Udvælgelse af gestik-par til pause og start}
\label{TestresultaterPauseStart}
%
Udvælgelsen af hvilket gestik-par, der skal knyttes til pause og start foretages på baggrund af testpersonernes udsagn. I \fullref{app:TestresultaterPauseDaarlig} analyseres testpersonernes respons i forhold til hvilke gestik-par de mindst kan lide og på baggrund af den analyse ekskluderes gestik-par 2, gestik-par 3 og gestik-par 4 fra yderligere undersøgelser. Dette medfører at udvælgelsen af hvilket gestik-par, der skal knyttes til pause og start kun foretages på gestik-par 1, gestik-par 5, gestik-par 6 og gestik-par 7. Analysen bygger på testpersonernes respons til de tre spørgsmål: \textit{Hvordan vil du rangere de tre bedste gestikker?}, \textit{Hvorfor har du valgt netop de tre?} og \textit{Hvorfor har du rangeret dem, som du har?}, hvor testpersonernes samlede data er vedlagt i HENVISNING TIL BILAG.  \blankline
%  
I nedenstående \autoref{tab:GestikParITopTrePause} fremgår samtlige testpersoners top tre rangering, hvor der ikke er taget forbehold for hvorvidt testpersonerne har inkluderet et gestik-par, som der på baggrund af \fullref{app:TestresultaterPauseDaarlig} er blevet ekskluderet.
%
\begin{table}[H]
	\centering
	\begin{tabular}{ | p{3cm} | p{3cm} | p{3cm} | p{3cm} |}
	\hline
		 & 1. Plads & 2. Plads & 3. Plads \\ \hline
		Testperson 1 & Gestik-par 1 & Gestik-par 7 & Gestik-par 5 \\ \hline
		Testperson 2 & Gestik-par 7 & Gestik-par 6 & Gestik-par 1 \\ \hline
		Testperson 3 & Gestik-par 1 & Gestik-par 2 & Gestik-par 4 \\ \hline
		Testperson 4 & Gestik-par 1 & Gestik-par 5 & Gestik-par 7 \\ \hline
		Testperson 5 & Gestik-par 5 & Gestik-par 1 & Gestik-par 6 \\ \hline
		Testperson 6 & Gestik-par 1 & Gestik-par 5 & Gestik-par 7 \\ \hline 
		Testperson 7 & Gestik-par 5 & Gestik-par 2 & Gestik-par 6 \\ \hline
		Testperson 8 & Gestik-par 7 & Gestik-par 3 & Gestik-par 6 \\ \hline
		Testperson 9 & Gestik-par 7 & Gestik-par 1 & Gestik-par 6 \\ \hline
		Testperson 10 & Gestik-par 1 & Gestik-par 2 & Gestik-par 4 \\ \hline
		Testperson 11 & Gestik-par 5 & Gestik-par 7 & Gestik-par 6 \\ \hline
		Testperson 12 & Gestik-par 2 & Gestik-par 1 & Gestik-par 5 \\ \hline
		Testperson 13 & Gestik-par 6 & Gestik-par 3 & Gestik-par 5 \\ \hline
		Testperson 14 & Gestik-par 6 & Gestik-par 5 & Gestik-par 7 \\ \hline
		Testperson 15 & Gestik-par 1 & Gestik-par 4 & Gestik-par 5 \\ \hline
		Testperson 16 & Gestik-par 1 & Gestik-par 5 & Gestik-par 7 \\ \hline
		Testperson 17 & Gestik-par 1 & Gestik-par 5 & Gestik-par 7 \\ \hline
		Testperson 18 & Gestik-par 5 & Gestik-par 7 & Gestik-par 1 \\ \hline
	\end{tabular}
	\caption{Oversigt over samtlige testpersoners top tre i forbindelse med pause og start.}
	\label{tab:GestikParITopTrePause}
\end{table}
\noindent
%
Da der er tre gestik-par, som er blevet ekskluderet, kan ovenstående  \autoref{tab:GestikParITopTrePause} med fordel opsummeres både i forhold til at fjerne de tre gestik-par men også i forhold til at opsummere hvor mange gange de fire tilbageværende gestik-par indgår i top tre rangeringen. 
%
\begin{table}[H]
	\centering
	\begin{tabular}{ | p{2.4cm} | p{2.4cm} | p{2.4cm} | p{2.4cm} |p{2.4cm}|}
	\hline
		 & 1. Plads & 2. Plads & 3. Plads & I alt \\ \hline
		Gestik-par 1 & 8 & 3 & 2 & 13\\ \hline
		Gestik-par 5 & 4 & 5 & 4 & 13\\ \hline
		Gestik-par 6 & 2 & 1 & 5 & 8\\ \hline 
		Gestik-par 7 & 3 & 3 & 5 & 11\\ \hline
	\end{tabular}
	\caption{Oversigt over dels hvor mange gange hvert gestik-par indgår i samtlige testpersoners top tre i forbindelse med pause og start og dels over hvor mange gange et gestik-par sammenlagt indgår i en top tre.}
	\label{tab:GestikParITopTrePauseOversigt}
\end{table}
\noindent
%
Det tyder på at de otte testpersoner, som alle rangerer gestik-par 1 på en første plads, gør det fordi de vurderer den til at være en kombination af at være; simpel, enkel, logisk, naturlig, oplagt, nem at huske, nem at udføre, hurtig at udføre og fordi den giver mening. Ud fra \autoref{tab:GestikParITopTrePause} fremgår det at testperson 3 og testperson 10 har fuldstændig den samme top tre rangering og sammenholdes det med testpersonernes respons og videooptagelser, så er det også de eneste to testpersoner, som har gestik-par 1 på en første plads, som ikke er i tvivl om at gestik-par 1 skal være på en første plads. Ydermere giver testperson 10 udtryk for, at der ved de gestik-par testpersonen har valgt ikke er så stor risiko for at testpersonen kommer til at lave dem anderledes og derudover er det ikke en bevægelse som testpersonen hyppigt laver. Ifølge testperson 5, som har rangeret gestik-par 1 på en anden plads, så er bevægelsen; et statisik stop-tegn, normal i ens kropssproget.   

En af årsageren til at testperson 3 ikke har inkluderet hverken gestik-par 5, gestik-par 6 eller gestik-par 7 på sin top tre, skyldes at testpersonerne forbinder bevægelserne med mute. At forbinde gestikkerne med mute giver ikke mening i denne sammenhæng fordi det er ret usandsynligt at mute sin musik, fremfor at pause den. Selvom testperson 1 i højere grad forbinder gestik-par 5 med at skrue op og ned for musikken, så har testpersonen alligevel inkluderet parret i sin top tre. 	 

De seks andre testpersoner, som har rangeret gestik-par 1 på en første plads, har alle inkluderet gestik-par 5 enten på en anden plads eller en tredje plads. Testperson 1 svinger mellem at have gestik-par 1 og gestik-par 5 på en første plads, hvor testpersonen til at starte med har top tre rangeringen, der fremgår af \autoref{tab:GestikParITopTrePause}, hvorefter den ændres til; gestik-par 5, gestik-par 1 og gestik-par 7 på henholdvis en første, anden og tredje plads. Når testpersonen afslutningsvist skal gengive gestikkerne foretrækker testpersonen gestik-par 1 igen. På baggrund af det er det ikke muligt at afgøre hvorvidt testperson 1's top tre rangeringen bør være gestik-par 1, gestik-par 5 og gestik-par 7 på henholdvis en første, anden og tredje plads, dog tyder det på at det er tilfældet. Ifølge testperson 16 er de tre gestik-par som testpersonen har rangeret i sin top-tre næsten lige gode, da de alle tre er simple, det tyder derfor på at der ikke er stor forskel mellem en første, anden og tredje plads. Dog vurderer testpersonen at gestik-par 1 er oplagt og den er ikke er til at glemme. Selvom testperson 4 ikke giver udtryk for at være i tvivl om sin top tre rangering, så tyder det på at testpersonen har rangeret gestik-par 1 højere end gestik-par 5 fordi parret er mere simpel. Dog giver testperson 4 udtryk for at gestik-par 5 er meget intuitiv. Ligende tendens forefindes ved testperson 17, som argumentere for hvorfor testpersonen har rangeret gestik-par 1 som værende den bedste; det er logisk. Når testperson 17 afslutningsvist bliver bedt om at gengive sine fortrukne gestikker, så er det gestik-par 5, som testpersonen knytter til pause og start. I og med at testpersonen tidligere har givet udtryk for at gestikkerne skal falde testpersonen naturligt ind og at testpersonen ikke skal tænke over hvilken bevægelse der laves, så tyder det på at gestik-par 5 måske bør være rangeret over gestik-par 1.   

Selvom testperson 6 og testperson 15 har rangeret gestik-par 1 på en første plads, så tyder det på at de foretrækker at der er bevægelse i gestikken. Testperson 6 gengiver, som en forbedring til gestik-par 1, en bevægelse, som minder om en kombination af gestik-par 1 og gestik-par 7, hvor håndens position fra gestik-par 1 bibeholdes mens fingrenes bevægelse i gestik-par 7 bibeholdes. Derudover giver testperson 6 også udtryk for at mekanikken i gestik-par 5, hvor fingrene lukkes sammen for at pause og åbnes igen for at starte musikken, er god. Selvom det ikke nøjagtigt gengiver bevægelsen i gestik-par 5, så tyder det på at det testperson 6 efterlyser formentlig kan opfyldes ved gestik-par 5. Testperson 15 giver udtryk for godt at kunne lide, at der er forskel på pause og start i gestik-par 5, hvilket ikke er tilfældet for de to gestik-par testpersonen har rangeret over gestik-par 5. 

Af de testpersoner som har valgt gestik-par 1 er der, foruden testperson 6, kun to testpersoner, som har forbedringsforsalg. Testperson 4 foreslår at hånden ikke behøver at være i 90$^{\circ}$'s vinkel men istedet noget der minder om 45$^{\circ}$ og derudover foreslår testpersonen at det skal være ligegyldigt hvor gestikken udføres; det må gerne være i hoftehøjde. Testperson 16 foreslår at håndens position fastholdes i tre sekunder både for at pause og for at starte. Når testpersonen gengiver sin forbedring så starter testpersonen med en knyttet næve, hvorefter fingrene foldes ud og danner et stop-tegn. Selvom testpersonen ikke eksplicit giver udtryk for det, så antages det, at dette ligeledes er en forbedring. \blankline
%
Der er flere årsager til at fire testpersoner har rangeret gestik-par på en første plads, hvor den primære årsag er at testpersonerne forbinder bevægelsen med en ti stille bevægelse, som de forbinder med at lukke munden på anlægget eller at de lukker musikken. Testperson 18 pointerer at det er en familiær bevægelse og at alle ved hvad det betyder når en anden person, laver luk-delen i et krokodillenæb til en; ti stille. Derudover er det ord som logisk, intuitiv og naturlig testpersonerne beskriver gestik-par 5 med. I relation til bevægelsen i gestik-par 5 så tyder det på testperson 5 favoriserer parret dels fordi bevægelsen kan udføres tæt på kroppen, der skal ikke huskes et bestemt mønster og bevægelsen foregår ikke ud fra kroppen. Derudover kommenterer testperson 11 at det er en lille bevægelse, faktisk den mindste bevægelse der kan laves, hvor det stadig giver mening. Testperson 11 giver ydermere udtryk for at gestik-par 5 er det eneste gestik-par testpersonen egentligt bryder sig om, fordi det er en lille bevægelse i forhold til de andre forslag. Testperson 7 giver udtryk for en anderledes tilgang til hvorfor gestik-par 5 rangeres på en første plads, testpersonen forestiller sig nemlig at hvis det vises frem for andre, så ville de syntes det var sjovt og de vil efterfølgende kunne huske det. Derudover giver testpersonen også selv udtryk for at gestik-par 5 er sjov. Foruden at gestik-par 5 er en familiær bevægelse, så giver testperson 18 ydermere udtryk for at det er en gestik, som ikke laves ved en fejl. Derudover giver testpersonen udtryk for godt at kunne lide, at der er forskel på hvordan musikken pauses og startes igen, med henholdvis en luk- og åben bevægelse. 

Af de fire testpersoner, som har valgt gestik-par 5 er der to testpersoner, som har forbedringsforslag. Testperson 5 foreslår at når musikken pauses så foregår det uden ændringer men når musikken startes igen foreslår testpersonen at hånden udadroterer når krokodillenæbet åbnes, hvilket resultere i at slutpositionen er med tommelfingeren øverst. Forbedringsforslaget fra testperson 18 kan nærmere betragtes som en selvfølge end en decideret forbedring. Testpersonen foreslår nemlig at gestikken ikke behøver at blive lavet i hovedhøjde, men bør kunne laves over alt. Grunden til at forbedringsforslaget anses, som værende en selvfølge er fordi det bør gøre sig gældende for samtlige semaforiske gestikker, at de ikke defineres ud fra en bestemt højde men frit kan udføres.\blankline
% 
Ud af 18 testpersoner er der kun to testpersoner; testperson 13 og testperson 14, som har gestik-par 6 rangeret på en første plads. Baseret på testperson 13's udsagn kan det ikke udledes hvorfor testpersonen foretrækker gestik-par 6, hvilket skyldes at indtil testpersonen opfordres til at komme med et forbedringsforslag, så er det gestik-par 3, som testpersonen foretrækker. Dog kommenterer testperson 13, at det ikke er muligt at afgøre hvilket gestik-par, der er bedst af gestik-par 6 og gestik-par 7 fordi den eneste forskel er hvilken retning bevægelsen foregår i. Ifølge testperson 14 vælges gestik-par 6 fordi det er tilpas akavet, samtidig med at det er en naturlig bevægelse i forhold til at skulle åbne og lukke lyd. Derudover pointere testpersonen at det giver god meningen at bevægelsen rettes mod musikanlægget. Da der kun er to testpersoner, som har rangeret gestik-par 6 på en første plads og parret sammenlagt kun indgår otte gange i testpersonernes samlede top tre rangering, samt at flere testpersoner giver udtryk for, at de åbne omkring deres anden og tredje plads, så vurderes det at der er belæg for at ekskludere gestik-par 6. \blankline
%   
Baseret på de tre testpersoner, som har tildelt gestik-par 7 en første plads, tyder det på at testperson 2 har valgt gestik-par 7 fordi det er en simpel bevægelse, som giver meget mening og fordi der er forskel på pause og start. Det er begrænset hvad der kan udledes af testperson 8, da testpersonen til at starte med rangeret gestik-par 3 på en første plads. Det der dog kan udledes er at gestik-par 7 giver mening. Det er svært at udlede noget konkret ud fra testperson 9's udsagn, dog virkede det, ifølge testperson 9, logisk at skulle gribe ud efter musikken for at pause den og slippe musikken fri for at starte musikken igen. I og med at det er begrænset hvad der kan udledes fra de tre testpersoner, som har gestik-7 på en første plads og at gestik-par 7 trods alt sammenlagt indgår 11 gange i den samlede top tre, så vurderes det at det er nødvendigt at inddrage de otte testpersoner, som har inkluderet gestik-par 7 i deres top tre. 

De tre testpersoner, som har rangeret gestik-par 7 på en anden plads, er testperson 1, testperson 11 og testperson 18. Fælles for de tre testpersoner er, at de giver udtryk for at bevægelsen giver mening i forhold til at når musikken pauses så skal der trækkes ned og når musikken skal startes igen så er det modsat. Derudover pointerer testperson 1 at det er forholdvist enkelt og det er nemt at huske. I tillæg kommenterer testperson 11 at det er intuitivt hvordan musikken lukkes ned og testperson 18 vurdere at det ikke er en bevægelse der laves ofte. Det skal understreges at kommentarerne fra testperson 11 primært bygger på hvad testpersonen giver udtryk for i forhold til gestik-par 5. At de indgår her skyldes at testpersonen giver udtryk for at både gestik-par 6 og gestik-par 7 minder meget om gestik-par 5, hvorfor det vurderes at disse kommenterer ligeledes relaterer sig til gestik-par 7. Baseret på begrundelserne fra de fem testpersoner, som har tildelt gestik-par 7 en tredje plads, så tyder det på at det primært er fordi bevægelsen følge samme bevægelsesmønster, som enten den ene eller begge de par, som testpersonerne har rangeret højere. Fra \autoref{tab:GestikParITopTrePause} fremgår det at hver gang en testperson tildeler gestik-par 7 en tredje plads, så tildeles anden pladsen altid gestik-par 5. Ligende tendens går igen når gestik-par 7 tildeles en anden plads, så har to ud af tre testpersoner tildelt gestik-par 5 en første plads. Hvis gestik-par 7 derimod indgår på en første plads, så er der ingen af de tre testpersoner, som inkluderer gestik-par 5, til gengæld inkluderer de gestik-par 1, gestik-par 3 og gestik-par 6. 

Da det tyder på at testpersonerne i højere grad foretrækker enten gestik-par 1 eller gestik-par 5 over gestik-par 7, samtidig med at gestik-par 5 indeholder to bevægelser til henholdvis pause og start, som er noget testpersonerne, der har inkluderet gestik-par 7 efterspørger, så vurderes det at der er belæg for at ekskludere gestik-par 7. \blankline
%
Baseret på foregående analyse samt \fullref{app:TestresultaterPauseDaarlig}, hvor i alt fem gestik-par er ekskluderet, så står valget mellem gestik-par 1 og gestik-par 5. Fælles for de to gestik-par er, at de indgår lige mange gange i testpersonernes samlede top tre; 13 gange i alt, selvom gestik-par 1 er tildelt en første plads dobbelt så mange gange som gestik-par 5. Det tilstræbes at imødekomme så mange testpersoners ønsker, som muligt og flere testpersoner giver udtryk for dels at de foretrækker at der er bevægelse og dels at de foretrækker at der er forskel på hvordan musikken sættes på pause og hvordan den startes igen, hvilket opnåes i gestik-par 5. Derudover er risikoen for ved en fejl at lave gestik-par 5 mindre end for gestik-par 1. At gestik-par 1 af flere testpersoner beskrives, som værende en naturlig bevægelse er ikke nødvendigvis et godt argument for at vælge den gestik, da stop-tegnet kan opstå i urelaterede situationer. Endvidere har seks ud af de otte testpersoner, som har tildelt gestik-par 1 en første plads, også inkluderet gestik-par 5 i deres top tre og fire ud af de seks har tildelt parret en anden plads. Tages alt dette i betrækning vurderes det at der forefindes tilstrækkeligt belæg for at knytte gestik-par 5 til pause og start.            





