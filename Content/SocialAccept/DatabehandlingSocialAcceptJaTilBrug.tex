\section{Interaktion med semaforiske gestikker}
\label{TestresultaterSocialAcceptBrug}
%
Følgende analyse og diskusion foretages på baggrund af samtlig testpersoners respons til spørgsmålet: \textit{Kunne du se dig selv bruge gestikker til at styre din musik?}. \blankline
%
Ud af de 12 testpersoner er det kun gæst 2, som giver udtryk for ikke at vil bruge gestikker i forbindelse med at styre et musikanlæg, hvilket skyldes at gæsten ikke selv benytter sig af et musikanlæg. Dog påpeger gæst 2, at for folk, som normalvis benytter sig af et musikanlæg, vil det formentlig være nemmere at interagere via semaforiske gestikker, fremfor at have fysisk kontakt med eksempelvis en fjernbetjening. Selvom gæst 4 og gæst 5 begge giver udtryk for, at vil bruge gestikker i forbindelse med at styre et musikanlæg, pointerer de begge, at det afhænger af prisen på produktet. I tillæg kommenterer vært 1, at det skal være det værd, at anvende gestikker, i forhold til at anvende en mobiltelefon eller en fjernbetjening. På nuværende tidspunkt giver gæst 6 udtryk for ikke at vil erhverve sig sådan et produkt, da det ifølge gæsten kræver et stort rum, hvilket gæsten ikke har. Dog giver gæst 6 udtryk for, at såfremt at rummet er stort nok, vil det være lækkert, blandt andet fordi der så ikke er behov for en fjernbetjening. Gæst 3, vært 4 og vært 6 kommenterer, i forhold til interaktion med semaforiske gestikker, henholdvis at det er fedt, sjovt og at det er fremtiden. Ifølge vært 3 skal produktet virke fra start og ikke først igennem flere opdateringer, før produktet er i stand til at registrere gestikkerne. Gæst 1 giver udtryk for, at det er træls hele tiden at være underlagt en fysisk kontakt for at styre sin musik. Ydermere kommenterer gæst 1, at det er forstyrrende at sidde med mobiltelefonen fremme i forhold til at interagere via semaforiske gestikker, som ikke forstyrrer samtalen.

Vært 5 giver udtryk for godt at vil bruge gestikker, men er bekymret for, om værtens forlovede vil være skeptisk i forhold til brugen af gestikker. Baseret på gæst 5's udsagn er dette ikke tilfældet, derimod giver gæsten udtryk for gerne at vil bruge gestikker, men at det afhænger af prisen på produktet. Dette kan indikere, at hvis den ene part i et forhold godt kan forestille sig at benytte semaforiske gestikker til at interagere, eksempelvis, med et musikanlæg, men antager at partneren ikke kan, er der en sandsynlighed for, at førstnævnte vil være tilbageholdende og dermed ikke vil overveje at anskaffe sig sådan et produkt.    

