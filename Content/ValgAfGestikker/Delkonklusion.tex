\section{Delkonklusion}
\label{ValgAfGestikkerDelkonklusion}
%
På baggrund af foregående analyse er semaforiske gestikker til de mest gængse funktioner på et musikanlæg valgt. Jævnfør \fullref{TestresultaterPauseStart} er gestik-par 5 valgt til at pause og starte musikken, hvor hånden lukker som et krokodillenæb for at pause musikken og åbner op igen for at starte musikken. For at skifte musiknummer frem og tilbage er gestik-par 5 valgt, jævnfør \fullref{TestresultaterSkiftMusiknummer}, hvor der her skiftes frem ved en swipe bevægelse fra højre mod venstre med samtidig med at tommel-, pege- og langefinger holdes strakt. Til sidst er gestik-par 2 valgt til at skrue op og ned for musikken, jævnfør \fullref{TestresultaterVolumen}, hvor hånden her gengiver at have fat i en drejeknap. Størstedelen af de testpersoner, der har deltaget i undersøgelsen har en positiv indstilling til interaktion med et musikanlæg ved brug af semaforiske gestikker, på trods af at ingen af dem betragter sig selv som værende de første til at anskaffe sig ny teknologi, hvilket er beskrevet i \fullref{TestresultaterInteraktioner}. Ydermere har størstedelen af testpersonerne ikke et problem med at blive opfanget af deres musikanlæg, dog med forebehold som at det skal være muligt at slukke for registreringen og at naturlige samtale-gestikker ikke skal opfanges som kommandoer til musikanlægget. For at komme testpersonernes bekymringerne til livs kan det være en fordel at skulle rette de semaforiske gestikker mod musikanlægget samt at denne interaktionsform kun fungerer, når musikanlægget allerede er tændt. \blankline
%
Når de semaforiske gestikker til interaktionen med de mest gængse funktioner på et musikanlæg er udvalgt, er det relevant at teste disse i en social sammenhæng. I den sociale sammenhæng kan både den sociale accept af gestikkerne og muligheden for at interagere med musikanlægget sideløbende med en primær opgave kan undersøges. 