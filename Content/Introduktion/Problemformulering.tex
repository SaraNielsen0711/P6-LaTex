\section{Problemformulering}
\label{Problemformulering}
%
Selvom de tre føromtalte undersøgelser, primært, bygger på menneskets perception af rene toner ved bestemte frekvenser, så udelukker det ikke, at det samme gør sig gældende for musik. Hvis teorier om og undersøgelser af \textit{Loudness Level} kan overføres til musik, så vil det betyde at lytteren kan afspille musiknummeret ved et lavere lydtryksniveau end tiltænkt, såfremt de lavere frekvenser forstærkes. En forstærkning af de lave frekvenser vil resultere i at nuancerne i musikken bliver tydligere. Det er derfor interessant at undersøge om det er muligt, ved hjælp af en teknologisk løsning, at reducere lydtryksniveauet for hvor musik afspilles ved og samtidig sørge for, at musikken bliver perciperet på samme måde, som ved det produceret lydtryksniveau. I det henseende kan udviklingen af sådan en løsning, muligvis, være med til at få, særligt teenagere og unge voksne, til at skrue ned for deres personlige auditive enheder og derved reducere risikoen for at udvikle høreskader.
\blankline
Baseret på de foregående afsnit, kan følgende problemstilling formuleres:
%
%Kan genbruges overalt i dokumentet med kommandoen \problemformulering
\newcommand{\problemformulering}{
\begin{quoteemph}
Hvordan kan der kompenseres for menneskets frekvensafhængige perception af lydtryksniveau, i relation til Normal Equal-Loudness-Level Contours fremsat i ISO226, med en elektronisk løsning som bevarer lydsignalet analogt? 
\end{quoteemph}
}
\problemformulering %Skriver problemformuleringen som defineret ovenfor
%
For at besvare ovenstående problemstilling er det nødvendigt at foretage en grundigere analyse af data fremsat i \textcite{STD:ISO226}. Analysen vil fokusere på sammenhængen mellem phon-kurverne afbilledet på \autoref{fig:EqualLoudnessContoursGraph}, og en valgt reference. Baseret på analysen vil det være muligt at udarbejde en elektronisk løsning, som dels forstærker eller dæmper et lydsignal så perceptionen af signalet er ens ved forskellige lydtryksniveauer, bare x-antal dB lavere, og dels sikre at lydsignalet bevares analogt. For at bevare lydsignalet analogt vil løsningen udarbejdes ved brug af analogteknik.