\section{Forstyrrende}
\label{TestresultaterSocialAcceptForstyrrende}
%
Følgende analyse og diskusion foretages på baggrund af samtlig testpersoners respons til spørgsmålene: \textit{Oplevede du, at det var forstyrrende at lave gestikkerne imens I planlagde ferie?}, \textit{Følte du, at du var i stand til både at planlægge ferien og styre musikken?}, \textit{Oplevede du at det var forstyrrende at din ven(inde)/kæreste lavede gestikker?} og \textit{Følte du, at din ven(inde)/kæreste var i stand til både at planlægge ferien sammen med dig og styre musikken?} der er stillet til henholdsvis værten og gæsten. \blankline
%
Ud af de 12 testpersoner synes ingen af dem, at gestikkerne var forstyrrende. Derimod synes seks testpersoner, herunder tre værter og tre gæster, at de forskellige hints forstyrrede, fordi de afbrød flowet i samtalen ved værten enten blev ringet op, musikken skrattede eller musikken blev skruet op. Tre af værterne gav udtryk for, at de koncentrerede sig lidt om at lytte efter de forskellige hints. Heriblandt giver vært 4 udtryk for at vedkommende var opmærksom på om musikken begyndte at skratte, men nævner samtidig at det i en virkelig situation ikke ville være noget problem, da der ikke ville komme hints på samme måde. Vært 6 sad nogle gange og lyttede til musikken for at finde ud af om det var kvaliteten af sangen, der var dårlig, eller om det var direkte forvrængning vedkommende skulle reagere på. To af testpersonerne, vært 3 og gæst 3, giver udtryk for at det med tiden blev mere naturligt at interaktionen foregik med gestikker. Gæst 3 giver udtryk for at vedkommende var meget fascineret af interaktionsformen i starten, hvorefter det blev mere naturligt. Vært 3 nævner derimod at vedkommende i første omgang havde i baghovedet at skulle reagere på hints, men at det med tiden blev naturligt at håndtere de forskellige hints, når de blev præsenteret. 


Forstyrrende og kunne lave begge dele (vært) og (gæst)