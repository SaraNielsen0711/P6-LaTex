\section{Komplikationer forårsaget af gestikker}
\label{KomplikationerGestikker}
%
Selvom gestik egner sig til perifer interaktion, er det stadig en forholdvis ny måde at interagere med elektroniske apparater på, \parencite[s. 163]{PDF:ComparingInputModalities}, hvorfor der naturligvis forekommer komplikationer. I følgende afsnit vil nogle af disse komplikationer belyses, først i forhold til teknologiske komplikationer og derefter i forhold til komplikationer vedrørende det menneskeligeaspekt. Fokus vil primært være på sidstnævnte.
%
\subsection{Teknologiske komplikationer}
\label{TeknologiskeKomplikationer}
%
Nogle af de mest åbenlyse teknologiske komplikationer, der kan opstå relaterer sig til problemer med at genkende specifikke gestikker, \parencite[s. 27]{PDF:ATaxonomyOfGestures}. Det gør sig både gældende for mikrogestikker, som er svære at registrere på lang afstand og for makrogestikker, hvor der er risiko for at en anden træder ind i det interaktiveområde og skygger for eller ubevidst udfører gestikker, som genkendes af systemet, \parencite[s. 9]{PDF:UsabilityofMicroVsMacroGestures}. \textit{LeapMotion} udbedrer den teknologiske begrænsning, der forekommer ved mikrogestikker, så snart gestikken udføres tæt på det elektroniske apparat. \textit{LeapMotion} er en lille enhed, som egner sig til at registrere og genkende semaforiske gestikker ved en kort afstand, \parencite[s. 7]{PDF:UsabilityofMicroVsMacroGestures}. Til forskel fra \textit{LeapMotion} anvendes \textit{Microsoft Kinect} til makrogestikker, da dette system er i stand til at genkende helkropsbevægelser, \parencite[s. 4]{PDF:UsabilityofMicroVsMacroGestures}, hvor \textit{LeapMotion} primært relaterer sig til mindre bevægelser, eksempelvis med fingrene. Dog er \textit{Microsoft Kinect} ikke lige så nøjagtigt som \textit{LeapMotion}, hvilket potentielt kan forårsage fejlgenkendelser, hvis to semaforiske gestikker minder om hinanden, \parencite[s. 3]{PDF:UsabilityofMicroVsMacroGestures}. \blankline
%
Særligt ved brug af semaforiske gestikker kan der opstå komplikationer, hvis en naturlig bevægelse fejlagtigt bliver genkendt af et system. Dette velkendte, problem defineres som \textit{Midas touch problem}, der for brugeren kan skabe forvirring, frustration og mistillid til det elektroniske apparat, \parencite[s. 109]{PDF:PIMicrogesturesKap5}. Ydermere kan systemet have problemer med at registrere og genkende semaforiske gestikker, hvis de udføres samtidig med andre bevægelser, \parencite[s. 27]{PDF:ATaxonomyOfGestures}. 

Problemet med semaforiske gestikker opstår, fordi systemet kan have svært ved at registrere både hvornår interaktionen starter og slutter og hvornår det er en sekvens af gestikker fremfor en enkelt gestik, \parencite[s. 27]{PDF:ATaxonomyOfGestures}. For at udbrede dette problem bør brugerne, ifølge \textcite[s. 1]{PDF:DoThatThere}, først og fremmest henvende sig direkte til det pågældende apparat. Det kan potentielt reducere sandsynligheden for at et \textit{Midas touch problem} opstår. I tillæg til at systemet ikke er i stand til at registrere og genkende bestemte gestikker er der, ifølge \textcite[s. 37]{PDF:ATaxonomyOfGestures}, meget få undersøgelser, som fokuserer på brugerens tolerance for at et system begår fejl. Det kan være i forhold til at systemet ikke reagerer på brugerens input, men det kan også være i forhold til at systemet fejlfortolker et input eller registrerer et input, fordi brugeren ubevist udfører en bestemt gestik. \blankline
%
I takt med at teknologi er i konstant udvikling og at virksomheder i elektronikindustrien konstant skal fornye sig og finde nye revolutionerende interaktionformer, antages det, at der er andre virksomheder, som ligeledes undersøger interaktion via semaforiske gestikker. Det forventes derfor at der udvikles flere produkter, hvis interaktion foregår via semaforiske gestikker. Når det er aktuelt, vil der sandsynligvis opstå et problem, som tilnærmelsesvis minder om \textit{Midas touch problem}, hvor det ikke nødvendigvis er brugerens bevægelse, der fejlfortolkes af systemet, men systemerne der fejlfortolker, hvorvidt gestikken er rettet mod dem. For at forbygge og helt undgå et potentielt stort problem foreslår \textcite[s. 2]{PDF:DoThatThere}, at et hvert system først skal addresseres med hver deres unikke gestik, hvorefter brugeren kan interagere med det specifikke system. \blankline
\newpage
%        
En anden teknologisk komplikation kan opstå i designfasen, hvor systemet testes via en \textit{Lo-Fi}-prototype, da der skal tages højde for, at resultaterne fra en \textit{Lo-Fi}-prototype formentligt afviger fra resultaterne fundet i en feltundersøgelse. \textcite[s. 176]{PDF:ComparingInputModalities} oplevede denne afvigelse, hvor testpersonerne fortrak de semaforiske gestikker fremfor de to typer af manipulerende gestikker i testen med \textit{Lo-Fi}-prototypen, hvor det i feltundersøgelsen var de to typer af manipulerende gestikker, som testpersonerne fortrak. Årsagen til det skyldes, ifølge \textcite[s. 176]{PDF:ComparingInputModalities} tre ting; 1) tekniske problemer i forbindelse med semaforiske gestikker i feltundersøgelsen, 2) haptiske problemer og 3) manglende interaktivitet ved \textit{Lo-Fi}-prototypen. 

Derudover er det sjældent at en \textit{Lo-Fi} prototype indeholder den endelige og fuldt funktionelle teknologi, hvorfor testpersonerne, i nogle situationer, nærmere evaluerer konceptet fremfor det endelige produkt. Tages der ikke højde for det i udviklingsfasen, er der risiko for at endelige produkt enten ikke lever op til brugerens ønske eller er for kompliceret at interagere med. Ydermere er det ikke sikkert, at testpersonerne oplever de samme problemer ved en \textit{Lo-Fi}-prototype, som de vil gøre ved det færdige produkt.   
%
\subsection{Komplikationer vedrørende det menneskelige aspekt}
\label{KomplikationerVedroerendeDetMenneskelige}
%
Komplikationer vedrørende det menneskelige aspekt kan opstå ved noget så åbenlyst, som manglende evne til at udføre den specifikke gestik. I den forbindelse kan der opstå problemer ved udmattelse, hvis det er fysisk udmattende at udføre gestikken eller hvis gestikken udføres gentagende gange, \parencite[s. 28]{PDF:ATaxonomyOfGestures}. Derudover indeholder gestikker ikke visuelle hints, hvilket, ifølge \textcite[s. 6]{PDF:NaturalUserInterfaces}, kan være problematisk, hvis systemet ikke reagerer på brugerens bevægelser, eller hvis systemet reagerer forkert. Det skyldes, at brugeren ikke har mulighed for at evaluere den information systemet har registreret og efterfølgende finde ud af, hvad fejlen er. De visuelle hints referer til selve udførelsen af gestikken, hvilket medfører, at når brugeren har udført en bestemt gestik, er det ikke muligt at gense bevægelsen. En måde at undgå denne komplikation på er ved at sørge for, at brugeren modtager en form for feedback, så de dels kan lære af deres fejl og dels ved, hvordan de skal gebærde sig, \parencite[s. 10]{PDF:NaturalUserInterfaces}. I forhold til hvilken type feedback og om der overhovedet skal være feedback vil blive belyst i \fullref{Feedbackformer}. \blankline
%
En anden komplikation, der skal tages højde for, særligt når gestikker anvendes som interaktionsform til perifer interaktion, er, at gestikkerne ikke er perifere før de er lært, inkorporeret og husket, \parencite[s. 16]{PDF:PIEmbeddingHCIOnTheRelevance}. Derudover kræver det, ifølge \textcite[s. 16]{PDF:PIEmbeddingHCIOnTheRelevance}, at opmærksomheden ikke fjernes fra den primære opgave, når der skal interageres i den perifere opmærksomhed, hvilket kun kan opnås, hvis gestikkerne er naturlige. I det henseende pointerer både \textcite[s. 8]{PDF:NaturalUserInterfaces}, \textcite[s. 26]{PDF:ATaxonomyOfGestures} og \textcite[s. 19]{PDF:PIEmbeddingHCIOnTheRelevance}, at der på nuværende tidspunkt både mangler simple og intuitive sets af gestikker og en form for standardisering af gestikker. Såfremt dette efterkommes vil det resultere i, at de samme gestikker repræsenterer og aktiverer de samme funktioner på tværs af de elektroniske apparater. Hvis standardiseringen realiseres, bliver der i den grad brug for, at systemerne kan differentiere mellem, hvornår brugeren henvender sig specifikt til dem og hvornår det ikke er tilfældet, eksempelvis ved at tildele hvert system sin egen unikke aktiveringsgestik, jævnfør \fullref{TeknologiskeKomplikationer}. En af årsagerne til, at der på nuværende tidspunkt ikke forefindes en form for standard og fælles forståelse for hvilke gestikker, der tilhører hvad, skyldes formentlig både, at det er relativt nyt at anvende gestik til interaktion med allestedsværende elektroniske apparater og at det er nyt at anvende gestik i forbindelse med perifer interaktion. Ifølge \textcite[s. 28]{PDF:ATaxonomyOfGestures}, er det usikkert hvilke gestikker, særligt semaforiske gestikker, der passer til bestemte scenarier. Selvom markedet har ændret sig i takt med, at nye teknologier finder indpas og selvom vores måde at interagere med disse teknologier ligeledes har ændret sig i forhold til, hvordan det var i 2005, før smartphones florerede på markedet, er det yderst relevant at undersøge hvilke semaforiske gestikker, der egner sig bedst til specifikke scenarier, hvor interaktionen foregår i den perifere opmærksomhed.

En gennemgående diskussion vedrørende semaforiske gestikker er at de betragtes som værende unaturlige, \parencite[s. 1961]{PDF:AStudyOnTheUseOfSemaphoricGestures}, men samtidig er det den form for gestik, der er mest udbredt indenfor interaktion med allestedsværende elektroniske apparater, \parencite[s. 28]{PDF:ATaxonomyOfGestures}. Ydermere understøtter semaforiske gestikker interaktion, der ikke afhænger af den visuelle opmærksomhed, \parencite[s. 1964]{PDF:AStudyOnTheUseOfSemaphoricGestures}, hvilket er favorabelt i forbindelse med perifer interaktion. Der kan være en fordel i at semaforiske gestikke ikke nødvendigvis er fuldstændig naturlige, da helt naturlige gestikker kan skabe problemer, \parencite[s. 9]{PDF:NaturalUserInterfaces}. \textcite[s. 9]{PDF:NaturalUserInterfaces} belyser problemet med naturlige gestikker i forbindelse med bowling spillet til Nintendo Wii, som forårsagede ødelagte fjernsyn. Spillet skal gengive et almindeligt spil bowling, hvor kontrollen svinges ligesom bowlingkuglen. Når spilleren normalvis slipper bowlingkuglen, kom spilleren naturligt til at slippe Nintendo Wii kontrollen, som derefter røg direkte ind i fjernsynet og ødelagde skærmen. Det bør derfor genovervejes hvor naturlige gestikkerne ønskes, da det kan være en fordel at gøre dem mindre naturlige, som tilfældet ved semaforiske gestikker.
%
