\chapter{Udledning af overføringsfunktioner}
\label{app:Overfoeringsfunktion}
%
Med udgangspunkt i to RC-led udledes følgende overføringsfunktion:
%
\begin{equation}
        \frac{V_{out}}{V_{in}} = -\frac{R_F'||\left(R_2+\frac{1}{s*C_2}\right)}{R_i}
\end{equation}
%
\begin{equation}
        = - \frac{R_F'+\left(R_2+\frac{1}{s*C_2}\right)}{R_i*\left(s*C_2*R_F'*\left(R_2+\frac{1}{s*C_2}\right)\right)}
\end{equation}
%
Udtrykket reduceres ved at multiplicere $s*C_2$ i tæller og nævner:
%
\begin{equation}
        = - \frac{R_F'+\left(s*C_2*R_2+1\right)}{R_i*\left(s*C_2*R_F'+\left(s*C_2*R_2+1\right)\right)}
\end{equation}
%
Så kan $-R_F'$ og $R_i$ trækkes ud:
%
\begin{equation}
        = -\frac{R_F'}{R_i}*\frac{s*C_2*R_2+1}{s*C_2*\left(R_F'+R_2\right)+1}
\end{equation}
% 
Hvis $C_2$ $\longrightarrow$ $\infty$ vil det resultere i en DC-forstærkning på:
%
\begin{equation}
        \frac{V_{out}}{V_{in}} = -\frac{R_F'}{R_i}*1
\end{equation}
%
Hvis det ikke er tilfældet, og $C_2$ ikke går mod $\infty$ så vil der være et pol/nulpunkt par, hvor nulpunktet udledes ved:
%
\begin{equation}
        \omega_z = \frac{1}{R*C} 
\end{equation}
%
\begin{equation}
         s*C_2*R_2 = \frac{s}{\omega_z}
\end{equation}
%
Og polen ved:
%
\begin{equation}
        \omega_p = \frac{1}{C_2\left(R_F'+R_2\right)}
\end{equation}
%
\begin{equation}
        s*C_2\left(R_F'+R_2\right) = \frac{s}{\omega_p}
\end{equation}
%
Derfra udledes den endelige overføringsfunktion:
%
\begin{equation}
        -\frac{R_F'}{R_i}*\frac{\left(\frac{s}{\omega_z}+1\right)}{\left(\frac{s}{\omega_p}+1\right)}
\end{equation}
%
Hvis $f = \infty$ så bliver impedansen af tilbagekoblingen, parallelforbindelsen mellem $R_2||R_F'$, hvorfra det fåes:
%
\begin{equation}
        \frac{R_2||R_F'}{R_i}
\end{equation}
% 
Med udgangspunkt i tre RC-led udledes følgende overføringsfunktion:
%
\begin{equation}
        \frac{V_{out}}{V_{in}} = -\frac{R_F''||\left(R_3+\frac{1}{s*C_3}\right)}{R_i}
\end{equation}
%
\begin{equation}
        = - \frac{R_F''+\left(R_3+\frac{1}{s*C_3}\right)}{R_i*\left(s*C_3*R_F''*\left(R_3+\frac{1}{s*C_3}\right)\right)}
\end{equation}
%
Udtrykket reduceres ved at multiplicere $s*C_3$ i tæller og nævner:
%
\begin{equation}
        = - \frac{R_F''+\left(s*C_3*R_3+1\right)}{R_i*\left(s*C_3*R_F''+\left(s*C_3*R_3+1\right)\right)}
\end{equation}
%
Så kan $-R_F''$ og $R_i$ trækkes ud:
%
\begin{equation}
        = -\frac{R_F''}{R_i}*\frac{s*C_3*R_3+1}{s*C_3*\left(R_F''+R_3\right)+1}
\end{equation}
% 
Hvis $C_3$ $\longrightarrow$ $\infty$ vil det resultere i en DC-forstærkning på:
%
\begin{equation}
        \frac{V_{out}}{V_{in}} = -\frac{R_F''}{R_i}*1
\end{equation}
%
Hvis det ikke er tilfældet, og $C_3$ ikke går mod $\infty$ så vil der være et pol/nulpunkt par, hvor nulpunktet udledes ved:
%
\begin{equation}
        \omega_z = \frac{1}{R*C} 
\end{equation}
%
\begin{equation}
         s*C_3*R_3 = \frac{s}{\omega_z}
\end{equation}
%
Og polen ved:
%
\begin{equation}
        \omega_p = \frac{1}{C_3\left(R_F''+R_3\right)}
\end{equation}
%
\begin{equation}
        s*C_3\left(R_F''+R_3\right) = \frac{s}{\omega_p}
\end{equation}
%
Derfra udledes den endelige overføringsfunktion:
%
\begin{equation}
        -\frac{R_F''}{R_i}*\frac{\left(\frac{s}{\omega_z}+1\right)}{\left(\frac{s}{\omega_p}+1\right)}
\end{equation}
%
Hvis $f = \infty$ så bliver impedansen af tilbagekoblingen, parallelforbindelsen mellem $R_3||R_F''$, hvorfra det fåes:
%
\begin{equation}
        \frac{R_3||R_F''}{R_i}
\end{equation}
% 

