\section{Social accept}
\label{TestresultaterSocialAcceptSocial}
%
Følgende analyse og diskusion foretages på baggrund af værternes respons til spørgsmålene: \textit{Hvordan tror du interaktion med gestikker vil fungere, hvis man er alene og i sociale sammenhænge?}, \textit{Hvordan vil du have det med at bruge gestikker, både når du er alene og i sociale sammenhænge?} og \textit{Vil det gøre en forskel, hvor du er henne?} samt glsternes respons til spørgsmålene: \textit{Hvordan vil du føle, hvis du så din ven(inde)/kæreste lave gestikkerne i et virkeligt scenarie?}, \textit{Vil det gøre en forskel, hvem I er sammen med?} og \textit{Hvordan tror du interaktion med gestikker vil fungere i sociale sammenhænge?}. \blankline
%
Ud af de 12 testpersoner har alle et overordnet positivt indtryk af interaktion med gestikker, både alene og i sociale sammenhænge. Vært 1 udtrykker, at det vil fungere godt at interagere med gestikker, men at vedkommende måske vil have nogle problemer med overvågningen, der gør, at det ikke vil være relevant at benytte sig af interaktionsformen. Alle seks værter giver udtryk for at interaktionsformen vil fungere godt alene og at de vil have det fint med at lave gestikkerne, når de er alene. Når testpersonerne bliver spurgt ind til den sociale sammenhæng, giver ni testpersoner udtryk for, at det ikke vil være et problem at gestikulere i en social sammenhæng. Fire af værterne vil ikke have et problem med at lave gestikker hjemme hos en ven, såfremt de har fået lov til at styre musikanlægget. Fem af gæsterne, med undtagelse af gæst 4, giver udtryk for, at det ikke vil gøre en forskel, hvem de er sammen med. Gæst 4 nævner i den forbindelse at vedkommendes venner nok ville lege med systemet hele aftenen. 

Der er altså ikke nogen gæster, der giver udtryk for at et bestemt selskab vil gøre det mindre passende at gestikulere til sit anlæg, men at interaktionen måske vil ske hyppigere med andre mennesker end alene. Det kommer også til udtryk, når testpersonerne bliver spurgt ind til interaktionsformen i en social sammenhæng generelt. Der er her syv testpersoner, der giver udtryk for, at der kan være problemer, hvis man holder fest eller har børn på besøg. Gæst 2 nævner, at det kunne blive en form for konkurrence om hvem, der først kunne skrue op og ned, når interaktionen foregår ved brug af gestikker i en større forsamling, men at en lille forsamling højst sandsynligt ikke vil skabe et problem. Vært 5 nævner, at interaktionsformen netop kan være et problem til fester, hvor folk render frem og tilbage med store armbevægelser, eller med børn, der hurtigt lærer gestikkerne og laver bevægelserne hele tiden. For at løse denne problematik giver gæst 2, gæst 4, vært 5 og vært 6 udtryk for nogenlunde det samme: At det ville være fint, hvis gestikkerne kunne slås fra efter behov eller at der kun var bestemte personer, der kunne styre anlægget med gestikker. 

Selvom der kan opstå nogle problematikker, når der er flere mennesker samlet, så virker det ikke til at testpersonerne har et problem med den sociale accept af disse gestikker, hvilket kun er positivt. To testpersoner giver ydermere udtryk for, at det er en god interaktionsform, fordi musikanlægget kan styres uden at ødelægge den pågældende samtale. Derudover nævner fem testpersoner, at det er fedt ikke at behøve at rejse sig for at interagere sig eller lede efter fjernbetjeningen og én enkelt vært nævner, at det er dejligt at have hænderne frie. Gæst 3 er, som nævnt, meget fascineret og giver udtryk for, at vedkommende har lyst til at vise det frem til sine venner, men at der nok er behov for tilvænning, så folk vil forstå hvad der sker, når en vært lige pludselig gestikulerer til sit anlæg. \blankline
%
Ud fra resultaterne virker det altså til at både interaktionsformen og gestikkerne er socialt acceptable, når man er på tomandshånd med en man kender på forhånd. Derudover indikerer testpersonernes svar på de forskellige spørgsmål, at gestikkerne også vil være socialt acceptable i større forsamlinger, selvom der kan være nogle problemer med interaktionsformen, hvis flere vil prøve at interagere med musikanlægget. Når den sociale accept er fastslået, er der interessant at undersøge, hvorvidt testpersonerne overhoved har lyst til at interagere med musikanlægget på den pågældende måde. 



