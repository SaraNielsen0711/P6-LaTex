%\fxwrite{Synopsis}
Projektet har til formål at designe semaforiske gestikker til de mest gængse funktioner i et musikanlæg. Det er ønsket, at disse semaforiske gestikker kan benyttes, når interaktionen skal foregå i den perifere opmærksomhed samtidig med, at interaktionsformen er social acceptabel. 

Først udføres en undersøgelse på 18 testpersoner, hvor det besluttes, hvilke semaforiske gestikker, der skal arbejdes videre med; et krokodillenæb, der lukker og åbner for henholdsvis at pause og starte musikken, en swipe-bevægelse med to fingre, for at skifte musiknummer og et greb om en fiktiv drejeknap, for at justere lydstyrken. Efterfølgende undersøges det, hvorvidt gestikkerne egner sig til musikkontrol i en social sammenhæng. Her deltager seks par, hvor der i hvert par er en vært og en gæst. Af resultaterne fra testen i den sociale kontekst, kan det konkluderes, at de valgte gestikker egner sig til interaktion med et musikanlæg sideløbende med en primær opgave; samtale med en gæst, hvilket indikerer, at interaktion med gestikkerne ved gentaget brug har potentiale for at blive perifer. Ydermere konkluderes det, at gestikkerne er sociale acceptable, da disse ikke påvirker samtale mellem to personer negativt. 