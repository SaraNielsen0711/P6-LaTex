\chapter{Perifer interaktion til musikkontrol}
\label{PeriferInteraktionTilMusikKontrol}
%
Da interaktionen med vores elektroniske apparater i dag primært finder sted i vores centrale opmærksomhed, vil det fremadrettet være fordelagtigt, at virksomheder overvejer andre interaktionsformer, heriblandt perifer interaktion. Da dette projekt foregår i samarbejde med en af verdens førende virksomheder inden for \textit{high-end} luksus lydudstyr; Bang $\&$ Olufsen, vil perifer interaktion selvsagt være i forbindelse med musikkontrol. Igennem det følgende kapitel vil det først og fremmest blive specificeret, hvilken sammenhæng denne perifer interaktion skal foregå i. Efterfølgende fokuseres der på teorien bag perifer interaktion, samt hvor det udspringer fra. Der fokuseres ydermere på relaterede undersøgelser, som netop anvender perifer interaktion til musikkontrol. Derefter bliver der gået nærmere i detaljer om, hvordan perifer interaktion rent faktisk kan foregå; ved hjælp af gestik. Afslutningsvist vil der i kapitlet ledes over i en problemformulering, som danner grundlag for det fremadrettede projektarbejde.    
%
\section{Samspil med Bang $\&$ Olufsen}
\label{SamspilMedBO}
%
I vores højteknologiske verden er det særligt vigtigt for virksomheder i elektronikindustrien, at være i konstant udvikling for dels at følge med teknologien og dels for at skabe revolutionerende og eftertragtede produkter, som indeholder den nyeste teknologi. Det gælder, for disse virksomheder, både om at forudse, hvilke nye produkter brugeren ønsker, men også hvilken form for revolutionerende interaktion, der skal være med produktet. Specielt interaktionsdelen er interessant i et samarbejde med Bang $\&$ Olufsen, hvor implementering af nye interaktionsformer, herunder perifer interaktion, kan være med til at forbedre brugeroplevelsen.

Perifer interaktion er en interaktionsform, som i større grad vil finde indpas i den nærmeste fremtid, for på den måde at fordre et gnidningsløst sammenspil mellem mennesker og produkter, \parencite[s. 1]{PDF:PIIntroduction}. I samarbejde med Bang $\&$ Olufsen er det interessant at undersøge, hvordan perifer interaktion samt tilhørende gestikker kan designes til at styre et musikanlæg, ophængt på væggen, så det ikke er nødvendigt at rejse sig fra sofaen, finde en fjernbetjening eller afbryde en samtale. Selvom perifer interaktion med elektroniske apparater er uundgåeligt, jævnfør \textcite[s. 1]{PDF:PIIntroduction}, er det ikke nødvendigvis alle funktioner, der kan og bør styres i den perifere opmærksomhed. Ved et interview med Lyle Clarke og Kashmiri Stec hos Bang $\&$ Olufsen udspecificeres både de primære og sekundære funktioner, som Bang $\&$ Olufsens produkter skal indeholde, jævnfør \autoref{app:InterviewLyleClarke}. De primære funktioner, samt formålet med disse, er opstillet i \autoref{tab:BogOsPrimaereFunktioner}, hvor de sekundære funktioner, samt formålet med disse, er opstillet i \autoref{tab:BogOsSekundaereFunktioner}.
%
\begin{table}[H]
	\centering
	\begin{tabular}{ | l | p{8cm} |}
		\hline
		\multicolumn{1}{|l|}{\textbf{Primære funktioner}} & \multicolumn{1}{l|}{\textbf{Formål}} \\ \hline
		Start & Musikken startes \\ \hline
		Pause & Musikken pauses \\ \hline
		Standby & Hviletilstand \\ \hline
		Forward/backward in time & Skift musiknummer frem eller tilbage \\ \hline
		Intensity up/down & Justering af intensiteten op og ned (lyd, lys osv.) \\ \hline
		Like & Tilføj til favoritter \\ \hline
		Dismiss & Fjern denne og lignende musiknumre fra playlister \\ \hline
		Shift source & Skift musikkilde (radio, CD, AUX osv.) \\ \hline
		Shift experiences & Skift mellem hvilken højtaler der spilles sammen med (skift mellem zoner) \\ \hline
		Initiate interaction & Start interaktion med produktet \\ \hline
		Confirm & Bekræft interaktion med produktet \\ \hline
	\end{tabular}
	\caption{Primære funktioner i Bang $\&$ Olufsens produkter.}
	\label{tab:BogOsPrimaereFunktioner}
\end{table}
\noindent
%
%
\begin{table}[H]
	\centering
	\begin{tabular}{ | l | p{8cm} |}
		\hline
		\multicolumn{1}{|l|}{\textbf{Sekundære funktioner}} & \multicolumn{1}{l|}{\textbf{Formål}} \\ \hline
		Fastforward/fastbackward & Spol frem eller tilbage i musiknummeret \\ \hline
		Stay in mood & Bliv i pågældende genre/stemning \\ \hline
		Send/expand & Flyt den afspillede musik til en anden højtaler \\ \hline
		Store & Gem musiknummer/kunstner/playliste \\ \hline
		Add to.. & Tilføj til.. (eksempelvis playliste) \\ \hline
		Play next & Sæt i kø som det næste musiknummer \\ \hline
		Play later & Sæt sidst i køen \\ \hline
	\end{tabular}
	\caption{Sekundære funktioner i Bang $\&$ Olufsens produkter.}
	\label{tab:BogOsSekundaereFunktioner}
\end{table}
\noindent
%
Hvis alt interaktion skal foregå i den perifere opmærksomhed, skal samtlige 18 funktioner allokeres fra den centrale til den perifere opmærksomhed, hvilket formentlig vil forringe både brugeroplevelsen og brugervenligheden. Det er derfor uhensigtsmæssigt, at samtlige funktioner i Bang $\&$ Olufsens fremtidige produkter skal understøtte perifer interaktion. Derudover er der funktioner, primært de sekundære funktioner, som ikke nødvendigvis anvendes lige så hyppigt som andre, hvorfor de ikke egner sig til perifer interaktion. Perifer interaktion er kun mulig såfremt en interaktion er blevet rutine, hvilket uddybes i \fullref{PeriferInteratkion}. Ydermere er nogle af funktionerne relativt komplekse, særligt hvis interaktionen ikke længere skal foregå i den centrale opmærksomhed. Det gør sig gældende for funktioner såsom; \textit{shift source}, \textit{shift experiences} og \textit{add to..}, som fordelagtigt kan være til rådighed i en tilhørende applikation eller ved nærbetjening.  

Ved at begrænse sig til kun at arbejde med de mest gængse og hyppigst anvendte funktioner i et musikanlæg, så bør det på længere sigt muliggøre perifer interaktion. Ud fra interviewet med Lyle Clarke og Kashmiri Stec fastlægges det hvilke funktioner, der skal fokuseres på i forhold til perifer interaktion. De valgte funktioner dækker over; start, pause, skift frem og tilbage samt justering af lydstyrken, jævnfør \autoref{app:InterviewLyleClarke}.\blankline
%
For at få en idé om hvordan perifer interaktion kan bruges i elektroniske apparater, vil teorien bag perifer interaktion samt hvordan gestikker kan bidrage til perifer interaktion beskrives i følgende afsnit. 

