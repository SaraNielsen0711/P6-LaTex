\section{Testpersoner}
\label{TestpersonerSocialAccept}
%
Foruden de krav, der er fremsat i \fullref{TestpersonerValgAfGestikker} i forhold til hvilke testpersoner, der må deltage i testen, så tilstræbes det kun at teste på personer, som i forvejen kender hinanden og som ikke har deltaget i udvælgelsen af semaforiske gestikker. At testpersonerne skal kende hinanden på forhånd skyldes, at testen udføres på to testpersoner ad gangen, hvor de skal agere henholdvist vært og gæst. Under testen er det fordel, at de to testpersoner kender hinanden, da det vil gøre situationen mere virkelighedsnær i forhold til hvis de ikke kendte hinanden. Ydermere antages det, at når testpersonerne kender hinanden, så er risikoen for, at der opstår pinlig tavshed eller andre akavede situationer mindre end hvis de ikke kendte hinanden. Det antages derudover, at fordi de to testpersoner kender hinanden, så bør det være nemmere for værten at styre musikken samtidig med at deltage i ferieplanlægningen med gæsten, da værten ikke først skal føle sig tryg i en fremmeds selskab. 

Det tilstræbes, at de testpersoner, der skal agere vært, som udgangspunkt skal være højrehåndet. Dette tilstræbes dels for at undgå forvirring relateret til, hvordan gestikkerne skal udføres, særligt i forhold til swipe-bevægelsen, når der skal skiftes musiknummer og dels for at undgå, at gestikkerne maskeres af en eller begge testpersoner. For at undgå denne maskering positioneres testpersonerne i henhold til \autoref{fig:Testopstilling2}. Endvidere tilstræbes det at teste på mellem 6 og 10 par, svarenden til henholdvis 12 og 20 testpersoner. 