\section{Familiarisering af værten}
\label{FamiliariseringSocialAccept}
% 
Familiariseringen af værten er bygget op, så værten gradvist introduceres til både gestikker og hints. Inden familiariseringen påbegyndes, introducerer testleder 1 værten til, hvad der kommer til at foregå. Derefter starter testleder 1 videoen til familiariseringen, som uddybes i \autoref{app:VideooptagelseSocialAccept}. I videooptagelsen præsenteres værten ad to omgange for de udvalgte gestikker, hvor der forinden angives, hvilken funktion gestikken tilhører. Efterfølgende præsenteres værten igen for gestikkerne, men uden informationen om, hvilken funktion gestikken tilhører. Til slut i videooptagelsen præsenteres gestikkerne endnu en gang, uden information om den tilhørende funktion, hvorefter testleder 1 spørger værten om, hvilken funktion den specifikke gestik tilhører. Når værten har gennemgået videooptagelsen, så forklarer testleder 1 den næste del af familiariseringen; værten får lov til at gengive gestikkerne til musik, som startes af testleder 2 inde fra kontrolrummet. Værten får i den forbindelse mulighed for at gengive de udvalgte semaforiske gestikker samt stille opklarende spørgsmål, hvis de har behov for det. Når værten er familiariseret med gestikkerne, forklarer testleder 1 hvilke hints, der præsenteres samt hvordan værten skal reagerer. Testleder 1 signalerer til testleder 2, at afspilningslisten for familiariseringen skal startes, således at værten lærer de forskellige hints og i den forbindelse, hvordan værten skal reagere på dem. Når værten giver udtryk for at have forstået koblingen mellem hints og reaktion, forlader testleder 1 lokalet, hvorefter afspilningslisten startes forfra. I løbet af de syv musiknumre, som udgør afspilningslisten, præsenteres værten for 18 hints, hvis præsentationsrækkefølge fremgår af \autoref{tab:PraesentationsraekkefoelgeHints}.

For at sikre at værten efterfølgende er i stand til både at interagere med musikanlægget og samtidig føre samtalen med gæsten, vil familiariseringen forlænges i tilfælde af, at værten begår gentagende fejl i gestikuleringen. Udover en fejl i gestikuleringen, betragtes det ligeledes som en fejl, hvis værten ikke formår at reagere på hintet inden 30 sekunder. Da hintet for at pause og starte musikken er en telefonopringning, er der mulighed for, at testleder 2 kan instruerer værten yderligere, hvis der er behov for det, og for at værten kan stille opklarende spørgsmål. Når værten har gennemgået de 18 hints, er familiariseringen færdig.\blankline
%
På baggrund af \fullref{TestresultaterVolumen} fremgår det, at bevægelsen i gestik-par 2, til justering af lydstyrken, af nogle testpersoner, opleves som ubehagelig og ukomfortabel, hvorfor det vælges, at foretage nye videooptagelser. I den forbindelse vælges det at optage de tre gestik-par fra ny, for at de har det samme udtryk både i forhold til lokation, vinkel på videokameraet, demonstratorens beklædning og andre karakteristikas. Ydermere vurderes det, at demonstratoren denne gang skal gengive gestikkerne siddende for, at værten præsenteres for gestikkerne i noget der minder om en dagligstue. I \autoref{app:VideooptagelseSocialAccept} vil videoptagelsen til familiariseringen uddybes og den færdigredigerede videooptagelse fremgår af \autoref{app:VideooptagelseFam}.

