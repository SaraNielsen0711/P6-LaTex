\section{Udførelsen af gestik}
\label{UdfoerelseAfGestik}
%
I dette afsnit fokuseres der på to overordnet former for input-gestikker, som et elektronisk apparat kan reagere på. Dernæst fokuseres der på fordelene og ulemperne ved bevægelsesmængden i gestikker. \textcite[s. 9]{PDF:ATaxonomyOfGestures} definerer fem kategorier af gestikker; deiktiske, manipulerende, semaforsike, gestikulerende samt tegnsprog, som uddybes nærmere i \autoref{app:KategoriseringerAfGestikker}. \blankline
%
\textcite[s. 9]{PDF:ATaxonomyOfGestures} definerer to former for inputs et elektronisk apparat kan modtage. Det ene input forekommer enten ved fysisk kontakt med et elektronisk apparat eller et andet objekt, \parencite[s. 10]{PDF:ATaxonomyOfGestures}. Denne form for input henvender sig til at interaktionen foregår via manipulerende gestikker, som defineres i \autoref{app:KategoriseringerAfGestikker}. Ifølge \textcite[s. 12]{PDF:ATaxonomyOfGestures}, er den anden form for input, som et elektronisk apparat kan modtage, baseret på gestikker, som apparatet kan genkende og reagere på ved hjælp af lys-, lyd- eller bevægelsessensorer. Ved denne type input er det derfor muligt at undgå interaktion med eller lokalisering af endnu et elektronisk apparat eller iklæde sig en form for elektronik, som eksempelvis en elektronisk handske, \parencite[s. 12]{PDF:ATaxonomyOfGestures}. Sidst nævnte inputform henvender sig til at interaktionen kan foregå via semaforiske gestikker, gestikulerende gestikker og/eller deiktiske gestikker, som defineres i \autoref{app:KategoriseringerAfGestikker}. \blankline
%
Bevægelsesmængde indenfor gestik defineres ud fra mikro- og makrogestikker, \parencite[s. 6]{PDF:UsabilityofMicroVsMacroGestures}. Mikrogestikker er små bevægelser, der ikke nødvendigvis kræver, at hele hånden bevæger sig og da de kan udføres samtidig med andre manuelle opgaver, er de specielt lovende indenfor perifer interaktion, \parencite[s. 95]{PDF:PIMicrogesturesKap5}. Derudover er mikrogestikker hurtige at udføre, hvilket, ifølge \textcite[s. 96]{PDF:PIMicrogesturesKap5}, egner sig til perifer interaktion. Eksempelvis tillader mikrogestikker kommunikation, som en sekundær opgave, samtidig med at fokus holdes på en samtale, som er den primære opgave, uden at samtalepartneren finder én uhøflig, \parencite[s. 97]{PDF:PIMicrogesturesKap5}.

Makrogestikker er derimod store bevægelser, der kræver signifikant bevægelse, som eksempelvis at løfte en arm eller rejse sig fra en siddende position, \parencite[s. 6]{PDF:UsabilityofMicroVsMacroGestures}. Ifølge \textcite[s. 9]{PDF:UsabilityofMicroVsMacroGestures} kan makrogestikker udføres meget mindre præcist end mikrogestikker, da en løftet arm vil tolkes som en løftet arm, uanset om den er løftet 60$^{\circ}$ eller 90$^{\circ}$. En af ulemperne ved makrogestikker kan dog være, at der skal bruges et stort areal til at udføre de pågældende gestikker og at en sensor, som opfanger makrogestikker ofte vil dække et helt lokale, \parencite[s. 9]{PDF:UsabilityofMicroVsMacroGestures}. Der kan derfor opstå et problem, hvis en anden træder ind i interaktionsområdet og ved et uheld laver en bevægelse, som genkendes af systemet. Da det kan være svært at træde ud af interaktionsområdet, kræver det at brugeren rent fysisk flytter sig langt nok væk, så en ubevidst interaktion undgåes. Til gengæld er en af fordelene ved makrogestikker, at de specifikke gestikker er så tilpas forskellige, at sandsynligheden for at systemet forveksler dem er lille, \parencite[s. 9]{PDF:UsabilityofMicroVsMacroGestures}.  

Mikrogestikker behøver, modsat makrogestikker, ikke et stort interaktionsområde, men skal til gengæld udføres meget mere præcist, \parencite[s. 10]{PDF:UsabilityofMicroVsMacroGestures}. En af fordelene ved mikrogestikker er, at de tillader en større diversitet end makrogestikker, hvilket gør det muligt at designe flere mikrogestikker end makrogestikker, \parencite[s. 10]{PDF:UsabilityofMicroVsMacroGestures}. En anden fordel ved mikrogestikker er, at interaktionen med systemet sjældent sker ved en fejl, da det er mindre sandsynligt at specifikke gestikker udføres ubevidst, \parencite[s. 10]{PDF:UsabilityofMicroVsMacroGestures}. Ulemperne ved mikrogestikker er dels, at desto større afstanden til systemet er, desto større er usikkerheden for, hvorvidt systemet kan registrerer gestikkerne, \parencite[s. 10]{PDF:UsabilityofMicroVsMacroGestures}. Derudover skal der ydermere kompenseres for situationer, hvor brugere kommer til at skygge for hele eller dele af gestikken, så systemet ikke længere er i stand til at registre og genkende den specifikke gestik. En anden ulempe ved mikrogestikker er brugerens evne til at gengive dem korrekt og uden at forveksle dem. Ifølge \textcite[s. 10]{PDF:UsabilityofMicroVsMacroGestures}, så egner mikrogestikker sig bedst til situationer, hvor der foretrækkes et mere professionelt udtryk, eksempelvis på arbejdspladsen. Derudover egner mikrogestikker sig også til underholdnings apparater.\blankline
%
I undersøgelsen foretaget af \textcite[s. 823]{PDF:UnderstandingNaturalness} fremgår det, at det ikke er hensigtmæssigt at benytte sine egne kropsdele til at simulere værktøjer i gestikker, da det vil virke unaturligt. Med værktøjer refereres der til det værktøj, som ellers vil blive anvendt såfremt det havde været en virkelig situation. Baseret på resultaterne, fremsat i \textcite[s. 823]{PDF:UnderstandingNaturalness}, fremgår det at 77.5\% af gangene opstår der fejl, når testpersonerne skal anvende en kropsdel som værktøj.

Derimod anbefaler \textcite[s. 823]{PDF:UnderstandingNaturalness} at gestikkerne udføres som pantomime, hvor brugere udfører den tiltænkte aktivitet og forestiller sig, at have værktøjet i hånden. Derudover anbefaler \textcite[s. 824]{PDF:UnderstandingNaturalness}, at hvis gestikkerne skal udføres i rummet, altså væk fra det elektroniske apparat, så bør gestikkerne udføres med begge hænder. Hvor den ikke-dominante hånd skal danne en form for ramme eller reference for den dominante hånd, som udfører gestikken.      
%
