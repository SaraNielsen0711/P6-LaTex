\section{Testforberedelse}
\label{Testforberedelse}
%
Hat
%
\begin{itemize}
  \item Materialer (iTunes 12.6)
  \item Formål
  \item Instruktioner
  \item Hvordan videoer er filmet
  \item Præsentationsrækkefølge 
  \item Deres opgaver
  \item Fremgangsmåde for testen 
  \item Pilottest (playliste, nyt spørgsmål, observatørens position)
\end{itemize}
%
\subsection{Præsentationsrækkefølge}
\label{PraesentationsraekkefoelgeValgAfGestikker}
%


%
\begin{table}[H]
	\centering
	\begin{tabular}{ | l | l | l | p{8cm} |}
		\hline
		Pause og start & Skift musiknummer & Skru op og ned \\ \hline
		Skru op og ned & Pause og start & Skift musiknummer\\ \hline
		Skift musiknummer & Skru op og ned & Pause og start \\ \hline
		Pause og start & Skru op og ned & Skift musiknummer\\ \hline
		Skru op og ned & Skift musiknummer & Pause og start\\ \hline
		Skift musiknummer & Pause og start & Skru op og ned \\ \hline
	\end{tabular}
	\caption{Oversigt over præsentationsrækkefølgen for de tre videoer; start og pause, skru op og ned og skift musiknummer.}
	\label{tab:PraesentationsraekkefoelgeValgAfGestikker}
\end{table}
\noindent
%

\subsection{Testpersoner}
\label{TestpersonerValgAfGestikker}
%
Testen udføres på studerende fra Aalborg Universitet. Det tilstræbes ikke at teste på studerende med forhåndsviden om design, brugertest og lignende eller studerende, hvis hovedfokus er elektronik og software. Da det antages at denne målgruppe primært vil fokusere på teknologien bag produktet eller andre specifikke designaspekter. Derudover bliver det af \textcite[s. 110]{Book:OUE} blandt andet anbefalet ikke at teste på personer med en baggrund inden for brugervenlighed og design af brugergrænseflader, da denne målgruppe har for stor viden inden for det område og de formentlig ikke vil være i stand til at give et upåvirket svar.

Det vil være favorabelt at teste på personer, som allerede har erhvervet sig et eller flere Bang $\&$ Olufsen produkter eller andet \textit{high-end} musikudstyr. Udfordringen ved at sætte det som et kriterie er, at det er begrænset hvor mange studerende, der på nuværende tidspunkt har råd til Bang $\&$ Olufsen's produkter, ikke sagt at de efter endt uddannelse ikke vil får råd. Da produktet stadig befinder sig på konceptbasis må det antages, at der kan gå år før produktet overhovedet bliver lanceret og til den tid kan nuværende universitetsstuderende potentielt nå at færdiggøre deres uddannelser og få en høj nok indkomst så de har råd til erhverve sig produkter fra Bang $\&$ Olufsen. \blankline
%        
Da præsentationsrækkefølgende af de tre videoer foregår efter et \textit{counterbalanced design} tilstræbes det at antallet af testpersoner kan deles med seks. At antallet af testpersoner skal kunne deles med seks skyldes at de tre videoer kan præsenteres i seks forskellige kombinationer, hvorfor der sigtes efter at hver kombination minimum repræsenteres to gange, svarende til 12 testpersoner. Det tilstræbes dog at udføre testen på 18 testpersoner, hvilket forudsætter at hver kombination repræsenteres tre gange. 

Ydermere tilstræbes det at begge køn er repræsenteret, da produkter fra Bang $\&$ Olufsen ikke er kønsafhængige. 
%

\subsection{Testopstilling}
\label{TestopstillingValgAfGestikker}
%

\subsubsection{Rollefordeling}
\label{RollerfordelingValgAfGestikker}
%
For at udføre testen skal der både udpeges to roller; en testleder og en observatør. Testlederens opgave er at instruere testpersonerne om hvordan testen vil forløbe, opfordre testpersonerne til at læse og underskrive samtykkeerklæringen, starte optagelsen efter testpersonerne har underskrevet samtykkeerklæringen, sørger for at testpersonerne udfylder spørgeskemaet, formidle testpersonernes opgaver samt at afvikle exit-interviewet. Testlederens instruktioner fremgår af BILAG og samtykkeerklæringen fremgår af BILAG. 

Observatørens opgave er at tage noter til testpersonernes respons og sørger for at pause og starte musikken, skrue op og ned for musikken samt skift musiknummer når testpersonerne opfordres til at forbedre deres fortrukne gestik til den specifikke video og når de afslutningsvist opfordres til at gengive gestikkerne igen. 

\subsubsection{Materialer og testlokation}
\label{MaterialeOgTestlokationValgAfGestikker}
%
Til testen indgår følgende materialer:
%
\begin{itemize}
  \item Videokamera + stativ
  \item Tre videoer med gestikker
  \item Farveskærm 
  \item To højtalere af typen Genelec 1031A
  \item iTunes 12.6  
  \item Computer med internetforbindelse\blankline
\end{itemize}
% 
Videokamera og stativ indgår da det ønskes at optage testpersonernes forbedringerne til at pause og starte musikken, skrue op og ned for musikken og til at skift musiknummer. Ydermere benyttes videokameraet til at optage testpersonernes respons undervejs og afslutningsvist i exit-interviewet. Det er derfor essentielt at videokameraet kan optage lyd og at videokameraet placeres dels så det kan optage testpersonernes bevægelser og dels optage deres orale respons. De tre videoer med gestikker er optaget på forhånd og gengiver, parvist, de seks mest gængse funktioner på et musikanlæg. Farveskærmen indgår for at præsenterer de tre videoer på en skærm, der er større end skærmen på en bærbarcomputer. De to højtalere skal tilkobles computeren hvorfra skærmen, hvor videoerne afspilles på, ligeledes er koblet til. iTunes 12.6 indgår i forbindelse med at testpersonerne opfordres til at forbedre deres fortrukne gestik og afslutningsvist hvor de opfordres til at gengive gestikkerne igen. Computer med internetforbindelse er nødvendig da det er der testpersonerne skal udfylde spørgeskemaet, i denne opstilling er det også den computer hvor farveskærmen og de to højtalere er tilkoblet.\blankline
% 
Testen afvikles i akustikafdelingen på Aalborg Universitet, Fredrik Bajers Vej 7B, i lokale: Lytterummet B4-107. 



