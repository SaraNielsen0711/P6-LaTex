\section{Udvælgelse af gestik-par til at skifte musiknummer}
\label{TestresultaterSkiftMusiknummer}
%
Udvælgelsen af hvilket gestik-par, der skal knyttes til at skifte musiknummer frem og tilbage foretages på baggrund af testpersonernes udsagn. I \fullref{app:TestresultaterSkiftDaarlig} analyseres testpersonernes respons i forhold til hvilke gestik-par de mindst kan lide og på baggrund af den analyse ekskluderes gestik-par 2, gestik-par 4 og gestik-par 7 fra yderligere undersøgelser. Dette medfører at udvælgelsen af hvilket gestik-par, der skal knyttes til at skifte musiknummer kun foretages på gestik-par 1, gestik-par 3, gestik-par 5 og gestik-par 6.\blankline
%  
I nedenstående \autoref{tab:GestikParITopTreSkift} fremgår samtlige testpersoners top tre rangering, hvor der ikke er taget forbehold for hvorvidt testpersonerne har inkluderet et gestik-par, som der på baggrund af \fullref{app:TestresultaterSkiftDaarlig} er blevet ekskluderet.
%
\begin{table}[H]
	\centering
	\begin{tabular}{ | p{3cm} | p{3cm} | p{3cm} | p{3cm} |}
		\hline
		& 1. Plads & 2. Plads & 3. Plads \\ \hline
		Testperson 1 & Gestik-par 1 & Gestik-par 5 & Gestik-par 6 \\ \hline
		Testperson 2 & Gestik-par 1 & Gestik-par 5 & Gestik-par 3 \\ \hline
		Testperson 3 & Gestik-par 1 & Gestik-par 5 & Gestik-par 4 \\ \hline
		Testperson 4 & Gestik-par 2 & Gestik-par 5 & Gestik-par 3 \\ \hline
		Testperson 5 & Gestik-par 1 & Gestik-par 5 & Gestik-par 3 \\ \hline
		Testperson 6 & Gestik-par 1 & Gestik-par 3 & Gestik-par 6 \\ \hline 
		Testperson 7 & Gestik-par 1 & Gestik-par 7 & Gestik-par 3 \\ \hline
		Testperson 8 & Gestik-par 3 & Gestik-par 2 & Gestik-par 7 \\ \hline
		Testperson 9 & Gestik-par 5 & Gestik-par 1 & Gestik-par 3 \\ \hline
		Testperson 10 & Gestik-par 1 & Gestik-par 5 & Gestik-par 3 \\ \hline
		Testperson 11 & Gestik-par 5 & Gestik-par 6 & Gestik-par 4 \\ \hline
		Testperson 12 & Gestik-par 1 & Gestik-par 2 & Gestik-par 5 \\ \hline
		Testperson 13 & Gestik-par 6 & Gestik-par 5 & Gestik-par 4 \\ \hline
		Testperson 14 & Gestik-par 5 & Gestik-par 6 & Gestik-par 4 \\ \hline
		Testperson 15 & Gestik-par 1 & Gestik-par 2 & Gestik-par 5 \\ \hline
		Testperson 16 & Gestik-par 1 & Gestik-par 3 & Gestik-par 7 \\ \hline
		Testperson 17 & Gestik-par 2 & Gestik-par 3 & Gestik-par 4 \\ \hline
		Testperson 18 & Gestik-par 2 & Gestik-par 1 & Gestik-par 3 \\ \hline
	\end{tabular}
	\caption{Oversigt over samtlige testpersoners top tre i forbindelse med at skifte musiknummer.}
	\label{tab:GestikParITopTreSkift}
\end{table}
\noindent
%
Da der er tre gestik-par, som er blevet ekskluderet, kan ovenstående  \autoref{tab:GestikParITopTreSkift} med fordel opsummeres både i forhold til at fjerne de tre gestik-par men også i forhold til at opsummere hvor mange gange de fire tilbageværende gestik-par indgår i top tre rangeringen. 
%
\begin{table}[H]
	\centering
	\begin{tabular}{ | p{2.4cm} | p{2.4cm} | p{2.4cm} | p{2.4cm} |p{2.4cm}|}
		\hline
		& 1. Plads & 2. Plads & 3. Plads & I alt \\ \hline
		Gestik-par 1 & 10 & 2 & 0 & 12\\ \hline
		Gestik-par 3 & 1 & 3 & 7 & 11\\ \hline
		Gestik-par 5 & 3 & 7 & 2 & 12\\ \hline 
		Gestik-par 6 & 1 & 2 & 2 & 5\\ \hline
	\end{tabular}
	\caption{Oversigt over dels hvor mange gange hvert gestik-par indgår i samtlige testpersoners top tre i forbindelse med at skifte musiknummer og dels over hvor mange gange et gestik-par sammenlagt indgår i en top tre.}
	\label{tab:GestikParITopTreSkiftOversigt}
\end{table}
\noindent
%
Der er forskellige årsager til hvorfor de 10 testpersoner har rangeret gestik-par 1 på en første plads. Fire ud af de 10 testpersoner, forbinder gestik-par 1 med hvordan de normalvist interagerer med en tablet eller en smartphone, hvor de i forvejen er bekendt med swipe-bevægelsen. Blandt de fire er testperson 1, som ydermere kommenterer at i gestik-par 1 så er der intet besværligt håndtegn, hvilket går igen ved testperson 2 og testperson 10. Derudover foretrækker testperson 1 at der er en bevægelse for at skifte musiknummer, hvilket ligeledes bliver fortrukket af testperson 3 og testperson 5. Endvidere giver testperson 1 udtryk for at gestik-par 1 er naturlig, hvilket testperson 5 og testperson 6 ligeledes giver udtryk for. En anden gennemgående tendens er at fire testpersoner giver udtryk for at gestik-par 1 giver meningen. De fire testpersoner, som giver udtryk for at gestik-par 1 giver meningen er ikke de samme fire testpersoner, som forbinder gestik-par 1 med at swipe på en tablet eller en smartphone. Derudover beskriver de 10 testpersoner gestik-par 1 som værende; simpel, intuitiv, logisk, enkel, nem at udføre og nem at forstå. Testperson 6 påpeger et potentielt problem når der skal skiftes tilbage til det forrige musiknummer og der ikke skiftes tilbage for at genspille musiknummeret. Det er selvfølgelig noget, der skal tages højde for, men som det er kendt fra blandt andet iTunes; så hvis der skiftes tilbage inden for relativt kort tid, så er det det forrige musiknummer, som afspilles og ikke det pågældende musiknummer som genstartes. Det kan være en løsning at gøre implementere samme princip i Bang $\&$ Olufsen's fremtidige musikanlæg. 

Sammenholdes udtalelserne fra de 10 testpersoner med hvad de rent faktisk gør så er der igen fire, som ikke formår at udføre bevægelsen i gestik-par 1 konsekvent. Testperson 1 skifter mellem at lave gestik-par 1 og gestik-par 5, når testpersonen referer til en swipe-bevægelse, men når testpersonen afslutningsvist bliver bedt om at gengive de fortrukne gestikker igen, så er er det gestik-par 1 testpersonen laver. Testperson 7 lægger ud med bedst at kunne lide gestik-par 7, hvorefter testpersonen ræsonnerer sig frem til at det istedet skal være gestik-par 1 med forbehold for at testpersonen foretrækker en mindre bevægelse. Når testpersonen afslutningsvist skal gengive sine fortrukne gestikker, så er det gestik-par 5 testpersonen laver, dog i den modsatte retning. Til at begynde med laver testperson 15 gestik-par 2 og når testpersonen taler om gestik-par 2, så er det egentlig gestik-par 1 testpersonen laver. Når testpersonen opfordres til at prøve gestikkerne til musik så bruger testpersonen både højre og venstre hånd, hvor begge bevægelser er modsat af hvad der foregår i gestik-par 1. Da testperson 15 afslutningsvist skal gengive sine fortrukne gestikker, så er det gestik-par 1, som testpersonen laver. Samme tendens forefindes ved testperson 16, som til at starte med laver gestik-par 1, men afslutningsvist så forklares og udføres bevægelsen, som i gestik-par 2. Testlederen gengiver gestik-par 1 for at være sikker på hvad testpersonen foretrækker og testpersonen giver udtryk for at det er sådan det skal være. Det tyder på at både testperson 15 og testperson 16 er i tvivl om hvilken retning swipe-bevægelsen skal være for enten at skifte til det næste eller forrige musiknummer. 

Ud af de 10 testpersoner har fire testpersoner forbedringer til gestik-par 1. Ifølge testperson 2 så skal det være ligegyldigt hvor mange fingre der bruges, da det er swipe-bevægelsen, der er essentielt. Det tyder derfor på at testpersonen ikke vil have noget imod at bruge gestik-par 5 til at skifte musiknummer, testpersonen har rangeret gestik-par 5 på en anden plads. Ifølge testperson 5 så er forbedringsforslaget at gøre bevægelsen til et ninja-hug, hvor fingrene samles, håndfladen peger opad og bevæges skrot ned fra højre mod venstre for at skifte til det næste musiknummer. For at skifte til det forrige musiknummer, foreslår testperson 5, igen at fingrene samles men denne gang skal håndfladen pege nedad og bevæges skrot ned fra venstre mod højre. Testperson 6, som påpegede et potentielt problem ved at skulle skifte til det forrige musiknummer og ikke bare genspille det nuværende musiknummer, foreslår at gestik-par 3 bruges til dette formål. Testperson 10's forbedringsforslag relaterer sig til at for at skifte til det næste musiknummer så skal højre hånd føres ind for foran kroppen og for at skifte til det forrige musiknummer skal venstre hånd føres ind foran kroppen. \blankline
%
Ud fra \autoref{tab:GestikParITopTreSkiftOversigt} fremgår det, at der kun er én testperson, som har tildelt gestik-par 3 en første plads. Årsagen til at testperson 8 har valgt gestik-par 3 er fordi det er klart og tydeligt hvad der skal gøres. Foruden gestik-par 3 så har testperson 8 rangeret gestik-par 7 på en tredje plads, jævnfør \autoref{tab:GestikParITopTreSkift}, hvilket indikerer at testpersonen foretrækker en lille bevægelsesmængde. Når gestik-par 3 indgår på en anden plads, så er det i to ud af tre tilfælde efter gestik-par 1. Testperson 16 har rangeret gestik-par 3 over gestik-par 7 fordi den er mere diskret. Derudover pointerer testpersonen at de er nemme at udføre samt at det er de eneste to par, hvor der ikke skal en bevægelse til, hvilket ifølge testpersonen er ret smart. At det er nemt at udføre pointerer testperson 6 også. Gestik-par 3 rangeres på en tredje plads syv gange, hvor testperson 2 og testperson 5 har inkluderet parret i top tre enten fordi der ikke var andre, som kunne accepteres eller fordi det var det par, der kom tættest på en tredje plads. Det tyder ligeledes på at testperson 9 har brugt udelukkelses metoden og i forhold til gestik-par 3 kommenterer testpersonen at det er en meget lille bevægelse som er nem at overse. Testperson 10 giver udtryk for at årsagen til at gestik-par 3 inkluderes er fordi den er enorm simpel. Testperson 18 foretrækker både sin første og anden plads over gestik-par 3 da de er mere naturlige, men derudover kommenterer testpersonen at gestik-par 3 kan laves forholdvist passivt. Hverken testperson 4 eller testperson 7 forklarer hvorfor de har inkluderet gestik-par 3. 

I og med at der kun er en enkelt testperson, som har rangeret gestik-par 3 på en første plads og at det tyder på at største delen af de andre testpersoner, som har inkluderet gestik-par 3 har gjort det ud fra udelukkelsesmetoden og sammenholdt med \fullref{app:TestresultaterSkiftDaarlig}, hvor tre testpersoner har fravalgt gestik-par 3, så vurderes det at der belæg for at ekskludere gestik-par 3. \blankline
%
Selvom gestik-par 5 kun er tildelt en første plads 3 gange, så indgår gestik-par 5 i testpersonernes samlede top tre ligeså ofte, som gestik par 1, jævnfør \autoref{tab:GestikParITopTreSkiftOversigt}. Testperson 9 giver udtryk for at de gestik-par, som indgår i top tre'en er valgt ud fra hvad testpersonen kunne huske og hvad der så mest elegant ud. I relation til gestik-par 5 kommenterer testperson 9 at det virkede mest logisk, men at det samtidig virkede pistolagtigt og at der var noget pege over det, uden at det blev et underviser-peg, hvor underviser-peget vurderes til at være negativt. Testperson 11 giver udtryk for at de gestik-par, som indgår i top tre'en er valgt ud fra hvad der er mindst naturligt i en samtale, hvorfor testpersonen har rangeret gestik-par 5 på en første plads. I forhold til gestik-par 5 vurderer testperson 11 at det giver intuitivt god mening at lave den bevægelse. Lignende synspunkter kommer til udtryk hos testperson 14, som igen vælger ud fra hvad der ikke føles naturligt og som testpersonen ikke vil komme til at lave ved en fejl, hvorfor et bestemt håndtegn er at foretrække. Desuden forbinder testperson 14 gestik-par 5 med en generel swipe-bevægelse, hvilket testpersonen finder naturligt. 

Tre ud af de syv testpersoner, testperson 1, testperson 2 og testperson 10, som har tildelt gestik-par 5 en anden plads gør det fordi det minder mest om gestik-par 1, som de alle tre har på en første plads. Årsagen til at testperson 10 har rangeret gestik-par 5 på en anden plads er udover, at det minder om gestik-par 1, så vurderer testpersonen at det er nødvendigt at tænke mere over gestik-par 5 i forhold til gestik-par 1. Det vurderes at testperson 4 ligeledes vælger gestik-par 5 fordi det minder mest om testpersonens første plads; gestik-par 2, så i det her tilfælde vælges gestik-par 5 kun hvis det fungerer efter samme princip, som i gestik-par 2. Hos testperson 3 går lignende tendenser igen, hvor testpersonen har tildelt gestik-par 5 en anden plads, fordi parret er næstmest naturligt, hvor gestik-par 1 er mest naturligt. Dog kommenterer testpersonen at det giver mere mening at bruge to fingre til at flytte på noget. Udover at testperson 5 giver udtryk for at gestik-par 5 er mere feminin sammenlignet med gestik-par 1, hvilket er årsagen til at gestik-par 5 tildeles en anden plads, så pointerer testpersonen, at gestik-par 5 tillader kontrol over hvad der skubbes til. Derudover tilføjer testperson 5 er gestik-par 5 er mindre voldsom og nemmere at udføre siddende end gestik-par 1. Testperson 13 giver udtryk for at fortrække gestikker hvor der både er bevægelse og hvor hånden skal være i et bestemt tegn, hvilket formentligt er årsagen til at gestik-par 1 ikke fremgår på top tre'en, jævnfør \autoref{tab:GestikParITopTreSkift}. Testperson 15, som har tildelt gestik-par 5 en tredje plads, giver udtryk for at gestik-par 1 er logisk og havde et mere simpel tegn en nogen af de andre foreslag. 

Af de 12 gange gestik-par 5 indgår på en top tre er der fire testpersoner, som ikke også har inkluderet gestik-par 1. Fokuseres der på hvad de fire testpersoner ellers har inkluderet i deres top tre, så er det kun testperson 4, som har inkluderet en statisk bevægelse; gestik-par 3, de tre andre har alle inkluderet gestik-par med en eller anden form for swipe-bevægelse, jævnfør \autoref{tab:GestikParITopTreSkift}. Endvidere er der kun fem testperson ud af de 12, som har inkluderet gestik-par 5 i deres top tre, som også har inkluderet en statisk gestik; gestik-par 3 og når det er tilfældet rangeres gestik-par 5 altid højere. Med udgangspunkt i det, samt testpersonernes udsagn så tyder det i høj grad på, at de fortrækker bevægelse, særligt en swipe-bevægelse som forbindes med interaktionen på en tablet eller en smartphone, såfremt der skal skiftes musiknummer. Derudover tyder det på at en af de største årsager til at testpersonerne hyppigst tildeler gestik-par 5 en anden plads er fordi det minder mest om gestik-par 1.\blankline
%                
Gestik-par 6 er kun tildelt en første plads en gang, af testperson 13 




 
TP13: Jeg kan bedste lide dem hvor man sådan bevæger hånden på en eller anden måde (laver bevægelser) istedet for at man bare skulle, men det er nok er sådan er vant til en tablet eller sådan noget hvor man ligesom skal køre frem og tilbage (swipe bevægelser). Og så kan jeg nok bedst lide dem hvor man skal lave et specielt håndtegn (laver gestik 4) der var den der (laver swipe med to fingre) og hvad de var. Hvad var forskellen på 5’eren og 6’eren (demonstrer) så vil jeg nok sige at 6’eren er bedst så kommer 5’eren og så 4’eren.
TP13: Fordi at det er et specielt håndtegn man skal lave for at det virker, så det ikke bare man lige vifter med hånden (vifter med hånden) og så sker der noget. Og så fordi at jeg godt kan lide at man skal bevæge hånden (laver en bevægelse) og ikke bare (peger i en retningen) sætte hånden op og så skifter den. 
TP13: For det første kan jeg bedst lide den der hvor man har to fingre (gestikker med to fingre) istedet for den der (gestik 4) fordi den var sådan lidt en sjov hånd og så den der (gestik med to fingre i en bue) det ved jeg ikke jeg syntes bare hyggeligt istedet det var sådan under (gestik med to fingre der swiper)





TP1: Jeg kan bedst lide 1’eren. Så 5’eren som nummer 2. Og 6’eren som nummer 3. 
TP1: Jeg tror det er fordi de har meget af de bevægelser som man er vant til fra tablets og sådan noget. Når man skal have den næste, så swiper du til højre, så det virker meget naturligt at gøre det. Og så synes jeg også det virker bedre med noget bevægelse frem for noget stilstand.
TP1: 1’eren fordi den er mest enkel. Der skal du ikke noget med med håndtegn og en speciel bevægelse. Det ligger meget naturligt, når man swiper. Selvom det gør 5’eren egentlig også. Det er derfor jeg har sat den nummer 2. 6’eren er den der er længst fra, derfor er den nummer 3. 


TP2: 1, 5, 3, tror jeg. 
TP2: Retningen blandt andet, jeg synes det gav mest mening at scrolle den vej i musikken. Og jeg synes ikke at arto-pistolen og telefonen - det var jeg ikke lige så fan af, så.
TP2: Jeg synes 1 den var mest simpel. Det var bare en hånd. 5’eren var ikke så meget anderledes. Og 3’eren var sådan den eneste der var tilbage, som jeg kunne acceptere. 

TP3: 1’eren er nok den der giver mest mening. Og så 5 og så 4 vil jeg sige.
TP3: Det der med at man skal holde hånden stille, det synes jeg er underligt. Det giver mening mening, hvor man bare swiper til en af siderne, men det der med hele håndfladen det synes jeg giver mere mening end med to fingre. (Sagt under første spørgsmål)
TP3: Gestik 4 synes jeg er underlig. Det giver mest mening for mig, at det bare er hele hånden. 5’eren er så den næstmest naturligt vil jeg sige. Det giver mere mening at bruge to fingre, når du skal flytte et eller andet. 

TP5: Jeg synes den her den virkede godt (gestik 1), hvor jeg synes det her skal være frem, fordi jeg er højrehåndet, så det giver god mening. Den giver god mening. Den her (gestik 5) er sådan lidt mere feminin, den er lidt mærkelig synes jeg. Og så var der den her (gestik 4), den skal man passe på med, for det betyder noget på tegnsprog. Så har ville jeg vælge gestik nummer 1, den ville jeg vælge som nummer 1. Og så nummer 2 rangeringen det bliver nummer 5. Og så nummer 3 det - der er jeg blank. Ellers så er det gestik 3 der kommer tættest på som 3. plads synes jeg.
TP5: Grunden til jeg tager gestik 1 først, det er fordi det ligger meget naturligt for mig at gøre sådan her, hvis jeg skal videre eller fremad eller hvis jeg skynder på nogen sådan “hey, kom nu, det skal gå lidt tjept”. Det er sådan standard. Det bruger man også ved kropssprog meget. Jeg kan godt lide, at når jeg skifter sang, så er der en bevægelse på, så det er noget hvor jeg bevæger mig i stedet for at jeg står stille og peger. Det føles meget tamt. Det var derfor at nummer 5 kom på andenpladsen. Grunden til jeg synes den er lidt mere - jeg synes bare det var lidt mere feminint at pege med to fingre. 5’eren er også god, fordi du har kontrol over hvad du skubber til, når du gør sådan her. Den er mindre voldsom og nemmere at sidde med i stuen. Dem hvor man bare står stille, dem synes jeg er dårlige, men jeg har 3’eren med, fordi jeg kan ikke lide at gøre det den anden vej. Jeg kan ikke lide at gøre det til højre, selvom at det er det på skærme. Ved 6’eren der skal man også bevæge sig meget i rummet, og det ved jeg bare føler at det er meget akavet. Jeg tror hvis jeg sad derhjemme, så ville jeg ikke orke at gøre sådan her (gestik 6) for at skifte sang, så skal det bare være sådan her (gestik 1).
 

TP6: Muligvis gestik 1 som nok værende den bedste. Den virker nemmest at forstå hvad der sker, nemmest at lave. Det eneste problem der bare kan komme, det er at du nogengange skal gå to gange tilbage, som der bliver vist hvis du bare går tilbage, så spiller den bare den samme sang, men det plejer det at være at lige så snart du trykker tilbage, så starter den forfra med sangen, men hvis du faktisk gerne vil tilbage til den tidligere sang, så kan det være at det bliver lidt for mange gange man skal svinge i luften, så muligvis mikse det op med at man laver frem og tilbage med hånden (gestik 1) og så muligvis skifte ud med den her (gestik 3) med skift tilbage til sidste sang. Gestik 3 vil så nok være nummer 2, og så tror jeg 6’eren er god. 
TP6: Jeg synes 1’eren det virker mest naturligt. Og så 3’eren fordi jeg synes også den har et fint element, hvis det kan registrere det på en måde, så fint, dejlig nem at gøre. De andre der er sådan lidt sjove, men ja, det virker stadig medgørligt. 


TP7: Jeg tror den bedste det var 7’eren, den anden det var nok 3’eren og så 1’eren. (Ved forbedringsforslag) Altså hvis man kunne gøre 1’eren bedre, så ville det være den bedste. Den bliver bedre ved at gøre armbevægelserne lidt mindre og gøre det muligt at lave bevægelserne lidt hurtigere. (ny rangering: 1, 7, 3)
TP7: Det bygger lidt på at hvis nu jeg ville høre musik, så kunne jeg godt finde på at skulle scrolle igennem en masse musik, så jeg tænkte det skulle være nogle små hurtige bevægelser. Og det virkede som om det var dem der var mindst ved. Jeg kunne godt forestille mig, hvis du skulle til at lave store armbevægelser, og du skulle 10 sange frem, så kunne det godt ligne sådan en vejrmølle. 
TP7: Jamen det er sådan set bare det der lige virkede nemmest til den funktion. 


TP10: Jeg synes 1’eren den var bedst. Og så synes jeg nok at 5’eren den var næstbedst. Og så 2’eren på en 3. plads. Nej vent, det vil jeg egentlig gerne lave om. 3’eren er egentlig god nok, den er meget nem. Så 1, 5, 3.
TP10: Igen 1’eren, jeg synes det var meget sådan intuitivt at man ligesom bladrer i sangene bare med en håndbevægelse sådan der. 5’eren fordi den minder mest om 1’eren.
TP10: Det var fordi jeg kunne bedst lide 1’eren, fordi jeg synes den var mest intuitiv for mig at skulle gøre det. Og 5’eren af samme grund, som bare kommer efter 1’eren af den ene grund at man nok lige skulle tænke en lille smule mere over det at hånden den skulle, ja, det var 3 fingre i stedet for alle 5 ud. Og så 3’eren næstefter fordi at den virkede også som en enorm simpel måde at gøre det på. 

TP12: Jeg synes 1’eren umiddelbart var den bedste og så 2’eren, netop fordi jeg synes det var lidt svært at kende forskel. Og så synes jeg også 5’eren var okay med den der slide bevægelse. 
TP12: Jamen det er på grund af jeg synes det der slide, der giver mig følelsen af at jeg skifter, frem for en retning. 
TP12: 1’eren synes jeg gjorde det mest lige, og det er nok fordi jeg tænker sådan lidt på en smartphone, hvor man også gør det sådan lige. Og så var det meget bevægelsen i det med de to andre. Og 2’eren synes jeg mindede utrolig meget om 1’eren.

TP15: Jeg tror 1 og 2 og 5
TP15: Fordi at jeg syntes det var dem der var mest simple, 
TP15: Altså de to første syntes jeg var sådan meget logiske at man skal videre så var det den vej (laver bevægelsen), nummer 2 var så den modsatte vej af hvad jeg sådan lige ville tænke var logisk men igen var der sådan en strygende bevægelse og så syntes jeg at det tegn der var ved nummer 5 var lidt mere simpelt end nogen af de andre



TP16: Arh det må nok være 1, 3, 7 - jeg kunne ikke lige gennemskue forskellen på 1’eren og 2’eren (det er den samme bevægelse bare modsat - demonstrere) nåår (.. se under mindst kan lide). 
TP16: Igen fordi de alle tre giver meget god mening i forhold til hvordan det ligesom ser ud oppe i mit hovede, og de er alle tre nemme at udføre
TP16: Hvordan de sådan indbyrdes rangeres. Jeg tror 7’eren er lavest af de tre fordi den er alligevel, ej men den er selvfølgelig også lidt smart, jeg ved det ikke de er lige smarte vil jeg sige, måske er 1’eren nummer 1 fordi den giver aller mest mening for mig. 3 og 7 er jo egentlig også ret ens, lidt i hvertfald, de er også nemme at udføre, det er også de eneste hvor der ikke ligesom skal en bevægelse til som sådan, altså hvor den er lidt mere stationær, det syntes jeg ellers os er ret smart også er 3’eren lidt mere diskret end 7’eren












TP4: Hvad er forskellen på 1 og 2? (får vist bevægelserne) Så kan jeg bedst lide 2’eren. Jeg kan godt lide deres bevægelser (1, 4, 5), de kører bare den forkerte vej. 1’eren og 2’eren - den bevægelse kan jeg bedst lide, åben hånd. Og så den med pegefingeren (gestik 5), den er nummer 2, men det skal være den samme retning (som nummer 2). Så er det 3’eren. 


TP17: Jeg tror det var, jeg syntes bare 1’eren, nå okay 1’eren og 2’eren er dæleme ens (1’eren og 2’eren er fuldstændig de samme bare modsat hinanden - testleder illustrerer). Så er 2’eren den mest logiske og så ved jeg faktisk ikke helt hvad jeg vil, så er det nok 3, ja så ved jeg faktisk ikke helt hvad jeg vil have som den sidste, 4’eren måske.  
TP17: Jeg syntes 2’eren er meget logisk at man sådan lige næste i rækken (laver bevægelser) og 3’eren det er nok den jeg ville tage, jeg ved ikke de andre, der kunne jeg måske lige så godt have taget nogen af de andre, det var hvertfald lige i mens jeg så det så var det den jeg tænkte det er da åbenlyst, (med 2’eren?) - Ja. Jeg havde så ikke lige set at der var forskel på 1’eren og 2’eren men jeg syntes at det giver mening at man skifter sang sådan der (laver swipe bevægelser) 


TP18: 2, 1, 3
TP18: Fordi de ikke kræver - 1 og 2 har begge to samme princip med at det er en flad hånd, der er ikke en speciel bevægelse, du skal ikke have hånden på bestemt måde eller andet. Det virker logisk at du swiper til siden ligesom du gør på en touchskærm, så vil du trække på til siden, så det virker meget, som den samme gestus. Og den der (referer til gestik 3) fordi du kan lave den forholdvist passivt, altså du kan sidde i sofaen og du behøver ikke at lave store armbevægelser du kan bare bevæge hånden og dreje den i den retning du vil have sangen i, derfor giver 3’eren også mening. Men det er mere naturligt for mig ville være at swipe til siden eller den anden vej, alt efter hvad der er relevant.   






















