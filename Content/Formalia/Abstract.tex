\begin{center}
	\textbf{{\huge Abstract}}
\end{center}
%
As most of our electronic devices demands our central attention and a focused interaction, the need for designing devices which can be operated in our peripheral attention increases. Peripheral interaction relates to the possibility of operating a device with minimum attention and cognitive resources, which allow the user to solve a secondary task while focusing on a primary task. The primary goals for this project are: 1) to investigate how an interaction with a Bang $\&$ Olufsen soundsystem can take place in the peripheral attention, 2) to select which semaphoric gestures that should be connected to the following chosen functions; pause, play, previous, next and volume adjustment, 3) to investigate the chosen semaphoric gestures in a social context, where the interaction with a soundsystem is defined as a sidetask to a primary task: a conversation with a guest, and 4) to determine if and how the chosen semaphoric gestures affects social acceptance according to the host and the guest.            

The first goal is achieved by a theoretical approch, where it is determined that semaphoric gestures supports peripheral interaction. The second goal is achieved by an elicitation technique performed on 18 test subjects, preferably students unfamiliar with user tests. According to the elicitation a gesture similar to a crocodile beak defines how to pause and play by closing and opening the beak respectively. The semaphoric gesture chosen to previous and next is a swipe motion from right to left for next and left to right for previous. The swipe motion is defined by a streched index and middle finger. For volume adjustment the semaphoric gesture is similar to how one adjusts the volume on a physical volume-knob. The third goal is achieved by an elicitation technique performed on 12 test subjects divided into pairs of two, preferably students who know each other and are unfamilar with user tests. The elicitation is sought to determine the hosts' ability to perform the interaction with a soundsystem while conducting a conversation with a guest. Based on the elicitation it is concluded that the hosts' manage to solve both tasks simultaneous. The fourth goal is achieved by conducting seperate exit-interviews, where it is concluded that semaphoric gestures is social acceptable as an interaction form with a soundsystem.