\section{Relateret produkter og studier}
\label{RelateretProdukterOgStudier}

I det følgende afsnit er der kigget på hvilke relaterede produkter der findes, samt hvordan perifer interaktion tidligere er blevet studeret og brugt til at styre musik eller lignende.

Tages der udgangspunkt i konceptet omkring et interaktivt kunstværk til væggen findes der flere elektroniske billedrammer på markedet. Nogle af disse billedrammer kan endda styres med gestures, herilandt Merual \parencite{WEB:Meural} og Framed 2.0 \parencite{WEB:Framed2.0}. Begge disse er billedrammer er en platform, hvor brugeren selv kan vælge hvad der skal ske på displayet. Den viste kunst kan findes i en database, hvor kunstnere kan uploade og sælge kunst. Brugeren kan på den måde selv vælge hvilket maleri eller andet kunstværk skal vises på billedrammen. Interaktionen med billedrammerne kan ske ved hjælp af en tilknyttet applikation, men begge billedrammer har også bevægelsesensorer knyttet til, så brugeren på den måde kan skifte billedet på skærmen til det næste eller forrige ved hjælp af gestures. Interaktion ved hjælp af gestures forudsætter dog, at brugeren er tæt på billedrammen for at sensorene kan opfange bevægelsen. Vendes fokus mod B$\&$O's ønske om at kunne styre et interaktivt kunstværk fra en større afstand, kan der ikke nødendigvis drages mange paralleller mellem det, Merual og Framed 2.0. Dog kan det ud fra billedrammerne forstås, at brugerne er i stand til at interagere med billedrammerne kun ved brug af \textit{free hand} gestures, hvilket kan medtages til interaktionen med B$\&$O's interaktive kunstværk.

 Udover produkter, der minder om B$\&$O's interaktive kunstværk findes der studier, der har kigget på perifer interaktion til styring af musik og lys. 
 
 


%
\begin{itemize}
  \item Meural \parencite{WEB:Meural}
  \item Framed 2.0 
  \item Musikkontrol (kilde med 3 forskellige måder at styre det perifer)
  \item kilde om perifer styring af lysintentises/lysstyrke
\end{itemize}
