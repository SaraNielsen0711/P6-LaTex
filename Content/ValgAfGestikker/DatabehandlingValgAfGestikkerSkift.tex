\section{Udvælgelse af gestik-par til at skifte musiknummer}
\label{TestresultaterSkiftMusiknummer}
%
Udvælgelsen af hvilket gestik-par, der skal knyttes til at skifte musiknummer frem og tilbage foretages på baggrund af testpersonernes udsagn. I \fullref{app:TestresultaterSkiftDaarlig} analyseres testpersonernes respons i forhold til hvilke gestik-par de mindst kan lide og på baggrund af den analyse ekskluderes gestik-par 2, gestik-par 4 og gestik-par 7 fra yderligere undersøgelser. Dette medfører at udvælgelsen af hvilket gestik-par, der skal knyttes til at skifte musiknummer kun foretages på gestik-par 1, gestik-par 3, gestik-par 5 og gestik-par 6.\blankline
%  
I nedenstående \autoref{tab:GestikParITopTreSkift} fremgår samtlige testpersoners top tre rangering, hvor der ikke er taget forbehold for hvorvidt testpersonerne har inkluderet et gestik-par, som der på baggrund af \fullref{app:TestresultaterSkiftDaarlig} er blevet ekskluderet.
%
\begin{table}[H]
	\centering
	\begin{tabular}{ | p{3cm} | p{3cm} | p{3cm} | p{3cm} |}
		\hline
		& 1. Plads & 2. Plads & 3. Plads \\ \hline
		Testperson 1 & Gestik-par 1 & Gestik-par 5 & Gestik-par 6 \\ \hline
		Testperson 2 & Gestik-par 1 & Gestik-par 5 & Gestik-par 3 \\ \hline
		Testperson 3 & Gestik-par 1 & Gestik-par 5 & Gestik-par 4 \\ \hline
		Testperson 4 & Gestik-par 2 & Gestik-par 5 & Gestik-par 3 \\ \hline
		Testperson 5 & Gestik-par 1 & Gestik-par 5 & Gestik-par 3 \\ \hline
		Testperson 6 & Gestik-par 1 & Gestik-par 3 & Gestik-par 6 \\ \hline 
		Testperson 7 & Gestik-par 1 & Gestik-par 7 & Gestik-par 3 \\ \hline
		Testperson 8 & Gestik-par 3 & Gestik-par 2 & Gestik-par 7 \\ \hline
		Testperson 9 & Gestik-par 5 & Gestik-par 1 & Gestik-par 3 \\ \hline
		Testperson 10 & Gestik-par 1 & Gestik-par 5 & Gestik-par 3 \\ \hline
		Testperson 11 & Gestik-par 5 & Gestik-par 6 & Gestik-par 4 \\ \hline
		Testperson 12 & Gestik-par 1 & Gestik-par 2 & Gestik-par 5 \\ \hline
		Testperson 13 & Gestik-par 6 & Gestik-par 5 & Gestik-par 4 \\ \hline
		Testperson 14 & Gestik-par 5 & Gestik-par 6 & Gestik-par 4 \\ \hline
		Testperson 15 & Gestik-par 1 & Gestik-par 2 & Gestik-par 5 \\ \hline
		Testperson 16 & Gestik-par 1 & Gestik-par 3 & Gestik-par 7 \\ \hline
		Testperson 17 & Gestik-par 2 & Gestik-par 3 & Gestik-par 4 \\ \hline
		Testperson 18 & Gestik-par 2 & Gestik-par 1 & Gestik-par 3 \\ \hline
	\end{tabular}
	\caption{Oversigt over samtlige testpersoners top tre i forbindelse med at skifte musiknummer.}
	\label{tab:GestikParITopTreSkift}
\end{table}
\noindent
%
Da der er tre gestik-par, som er blevet ekskluderet, kan ovenstående  \autoref{tab:GestikParITopTreSkift} med fordel opsummeres både i forhold til at fjerne de tre gestik-par men også i forhold til at opsummere hvor mange gange de fire tilbageværende gestik-par indgår i top tre rangeringen. 
%
\begin{table}[H]
	\centering
	\begin{tabular}{ | p{2.4cm} | p{2.4cm} | p{2.4cm} | p{2.4cm} |p{2.4cm}|}
		\hline
		& 1. Plads & 2. Plads & 3. Plads & I alt \\ \hline
		Gestik-par 1 & 10 & 2 & 0 & 12\\ \hline
		Gestik-par 3 & 1 & 3 & 7 & 11\\ \hline
		Gestik-par 5 & 3 & 7 & 2 & 12\\ \hline 
		Gestik-par 6 & 1 & 2 & 2 & 5\\ \hline
	\end{tabular}
	\caption{Oversigt over dels hvor mange gange hvert gestik-par indgår i samtlige testpersoners top tre i forbindelse med at skifte musiknummer og dels over hvor mange gange et gestik-par sammenlagt indgår i en top tre.}
	\label{tab:GestikParITopTreSkiftOversigt}
\end{table}
\noindent
%