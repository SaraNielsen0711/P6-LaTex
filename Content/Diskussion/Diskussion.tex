\chapter{Diskussion}
\label{Samlet Diskussion}
%
\begin{itemize}
  \item Har i et bud på hvorfor der er et gab, mellem fokuseret og automatisk interaktion (perifer) Lars Bos kommentar!
  \item Gamle telefoner med knapper var perifert
\end{itemize}
%
%Taget fra afgrænsningen men passer bedre i diskussionen 
Selvom der i \fullref{RelateretUndersoegelser} blev præsenteret nogle forskellige undersøgelser, som alle vedrører perifer interaktion med en musikafspiller, kan disse kun bruges som inspiration. Det skyldes blandt andet at interaktionen var med en musikafspiller, som eksempelvist iTunes, og ikke et decideret musikanlæg. Derudover var brugeren positioneret ved et skrivebord med en computer, hvor musikafspilleren befinder sig, og hvor interaktionen foregik i området tæt omkring computeren. Grunden til at undersøgelserne kun kan anvendes som inspiration er, at interaktionen, som undersøges i dette projekt, er med et musikanlæg og at selve brugssituationen kan være anderledes i og med, at brugeren ikke nødvendigvis behøver, at sidde ved sit skrivebord foran sin computer, men frit kan bevæge sig rundt i rummet. Der kan drages inspiration fra undersøgelserne i forhold til valg af semaforiske gestikker og hvordan disse kan testes.\blankline
%
Retning på swipe\blankline
%
\textbf{Ting der kan diskuteres:}
\begin{itemize}
	\item Noget om B$\&$O's ønske til projektet/perifer interaktion
	\item Der er kun nogle funktioner, der egner sig til perifer interaktion/interaktion med semaforiske gestikker, som TP1 i test 1 også siger. 
	\item Calm techlogy kan måske være interessant at sige noget om efter test 2 er udført.
	\item Casual interaction - det virker til at testpersonerne, når de ikke får andre muligheder, sagtens kan opgive den fulde kontrol over musikanlæget. Bliver de spurgt, er der der dog flere der kommer ind på (manglende?) kontrol ved justering af lydstyrke.
	\item Delt opmærksomhed - man kan holde fokus på samtalen og interagere med musikanlægget
	\item AChairAsUbiquitours... understreger vigtigtheden af et godt genkendelsessystem, da testpersonerne ikke får andet end funktional feedback fra musikken - manglen på feedback under tests - hvd kan man foreslå af feeback - generelt teori vi har om feedback
	\item Noget om at skygge for sine egne gestikker
	\item Midas touch problem. Der er allerede under komplikationer ved gestik åbnet op for at gestikkerne skal rettes mod anlægget. Så skygger man heller ikke for gestikken.
	\item Det her med aktiveringsgestikker/blik hen imod anlægget osv. 
	\item Standardisering af gestikker
	\item prototype-test (passer det bedre ind i perspektivering?)
	\item Social accept, naturligvis. Social accept fra begge synsvinkler, det har vi også skrevet noget om i teorien. 
	\item Er der et socialt problem, hvis de skal rette interaktionen mod venstre? Teorien siger de har det bedst med at rette den mod højre
	\item Synlige, usylinge; Hemmelighedsfyldte, udtryksfyldte, 
	magiske og spændingsfyldte gestikker
	\item Det med om gestikkerne er magiske, hvis interaktionen er perifer og diskret. 
	\item Social accept i forhold til tid, lokation og hvad vi fandt ud af ud fra test 2. 
	\item Diskuter metoder for begge forsøg. Test 1: Videoer og præsentation af gestikker (hvis det overhoved er noget ved). Test 2: Kigger testpersonerne mere imod anlægget, fordi hintet kommer herfra? Kan holdes op med noget teori omkring opmærksomhed, måske?
	\item 
\end{itemize}


