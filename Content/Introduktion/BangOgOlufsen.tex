\section{Sammenspil mellem perifer interaktion og B$\&$O}
\label{Sammenspil mellem perifer interaktion og BO}

Med viden om perifer interaktion er det muligt at forestille sig situationer, hvor ens musikanlæg kan styres perifert og uden at bryde for meget med de opgaver, der ellers udføres. B$\&$O har i den sammenhæng et ønske om at udvikle et interaktivt kunstværk, der kan styre musikken. Med dette interaktive kunstværk er ønsket ikke, at det skal fange opmærksomheden, hver gang brugeren går forbi. Derimod er ønsket, at det kan bruges til at starte musikken, hvorefter det igen opfører sig som et passivt stykke kunst på væggen. Når der lægges vægt på, at dette interaktive kunstværk ikke skal forstyrre brugeren i tide og utide er det også at ønske at kunne styre musikken ved hjælp af perifer interaktion, så illusionen om et passivt stykke kunst opretholdes. 

Produktet henvender sig til mennesker, der ønsker at have deres musik synligt, uden nødvendigvis at have flere reoler fyldt op med gamle CD'er. Derudover vil produktet henvende sig til mennesker, der ikke har et problem med at opgive noget af kontrollen ved at styre et musikanlæg. Netop ved brug af perifer interaktion fralægges den præcise kontrol, der normalt haves ved at bruge en fjernbetjening eller trykke direkte på interfacet og derfor henvender produktet sig ikke så meget til brugere, der kan lide altid at være i kontrol. Da B$\&$O generelt bestræber sig efter at lave produkter, der virker magiske, er målgruppen ved dette produkt også herefter. Et magisk produkt lægger op til, at de gestures der bruges til den perifere interaktion også fremstår magiske, så helhedsbilledet af magien bevares. 

Der er forskellige brugtssituationer med sådan et interaktivt kunstværk, da det både kan benyttes, når brugeren ønsker at høre musik i eget selskab og i selskab med andre. Når brugeren sidder alene i stuen er det højst sandsynligt nemmere at lave de rigtige gestures for at skrue op eller skifte sang, men hvordan ændrer det sig, når det er gæster? Ifølge \textcite[s. 276]{PDF:WouldYouDoThat} findes der flere måder at udføre gestures på, hvoraf de mest socialt acceptable er dem 

Ved interview med Lyle Clarke hos B$\&$O findes der frem til at de primære funktioner er ... og de sekundære funktioner er..


\begin{itemize}
  \item Brugssituationer 
  \item Kobling
  \item Hvem henvender produktet sig til?
  \item Magi
  \item Hvad er deres mål med projektet
\end{itemize}

