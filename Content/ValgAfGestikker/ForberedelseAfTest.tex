\section{Testforberedelse}
\label{Testforberedelse}
%
I de følgende afsnit vil forskellige aspekter vedrørende den test, som skal klarlægge hvilke gestikker, der skal knyttes til de seks mest gængse funktioner på et musikanlæg, blive belyst. De forskellige aspekter vil afspejle indholdet af en testprotokol, såsom; formål, problemformulering, testpersoner, metode, opgaver til testpersonerne, testopstilling, fremgangsmåde og evaluering. Dog med forbehold for andre tilføjelser, anderledes struktur og navngivning. 
%

\subsection{Testens omfang}
\label{TestensOmfangValgAfGestikker}
%
Med udgangspunkt i og for at besvare problemstillingen: \textit{Hvilke specifikke semaforiske gestikker skal knyttes til hver af de seks mest gængse funktioner i Bang $\&$ Olufsens fremtidige musikanlæg, for at interaktionen kan foregå i den perifere opmærksomhed?} vil omfanget af denne test primært indebære en undersøgelse af hvilke specifikke gestikker, der skal knyttes til hver af de seks mest gængse funktioner, samt hvordan testpersonerne forholder sig til denne form for interaktion. Formålet med testen er derfor hverken at undersøge om interaktionen med et musikanlæg kan foregå i den perifere opmærksomhed eller at undersøge hvordan gestikkerne påvirker social accept. 

Der blev i \fullref{UdvaelgelseAfGestikker} udvalgt forskellige semaforiske gestikker til henholdvis pause og start, \autoref{tab:IndsamledeGestikkerPause}, skift musiknummer frem og tilbage, \autoref{tab:IndsamledeGestikkerSkift}, og til at skrue op og ned for musikken, \autoref{tab:IndsamledeGestikkerVolumen}. Formålet med at udvælge flere forskellige semaforiske gestikker til hver af de seks funktioner er dels for at illustrere hvordan interaktionen potentielt kan foregå, dels for at undersøge hvilke egenskaber testpersonerne foretrækker og dels for at inspirere testpersonerne til at komme med et forbedringsforslag. 
%

\subsection{Fremgangsmåde}
\label{FremgangsmaadeValgAfGestikker}
%
Testpersonerne får først en kort introduktion til hvad formålet med testen er; at undersøge hvilke semaforiske gestikke, der bedst egner sig til de mest gængse funktioner på et musikanlæg og da det ønskes at optage testpersonerne undervejs får de udleveret en samtykkeerklæring. Derefter bliver testpersonerne bedt om at udfylde et spørgeskema, hvori de, foruden at angive køn og alder, blandt andet skal besvare spørgsmål omkring deres erfaringer med gestikker. Når testpersonerne har udfyldt spørgeskemaet bliver de introduceret yderligere til hvad der kommer til at foregå og hvad deres opgaver er; efter videoen er slut skal de rangere de tre bedste gestikker og forklare hvorfor de har valgt netop dem og hvorfor de er rangeret som de er. Efterfølgende skal de udpege den gestik som de syntes mindst om og forklare årsagen til hvorfor de har valgt den. Inden de præsenteres for den næste af de tre videoer, opfordres de til at komme med et forbedringsforslag til deres favorit gestik og i den forbindelse får de mulighed for at prøve med musik i baggrunden, så hvis de eksempelvis har en forbedring til hvordan musikken pausen bliver musikken tændt, hvorefter de pauser den med deres forbedring. Årsagen til at der sættes musik på når testpersonerne skal forbedre deres favorit er for at lade dem prøve det i kontekst og så de kan vurdere hvordan det føles og foretage justeringer, hvis de har lyst.

Når testpersonerne er blevet præsenteret for alle tre videoer afvikles der et exit-interview, hvor testlederen stiller følgende spørgsmål; \blankline
%
\begin{itemize}
  \item Hvordan synes du det gik? 
  \item Hvor nemt syntes du det var at udpege hvilke gestikker du bedst og mindst kunne lide? 
  \item Hvordan syntes du filmene illustrerede de forskellige gesikker? 
  \item Hvad synes du om, at skulle interagere med et interface ved brug af gestikker? 
  \item Hvis dette system skal reagere på dine gestikker hver gang du laver en, så skal systemet altid kunne opfange dine gestikker. Hvordan vil du have det med det? 
  \item Har du andre kommentarer?\blankline
\end{itemize}
\noindent
%
Exit-interviewet afvikles som et semi-struktureret interview hvilket tillader en flydende dialog mellem testperson og testleder og giver mulighed for testlederen at stille opfølgende og uddybende spørgsmål, hvis nødvendigt.  



\begin{itemize}
  \item Exit-interview
  \item Spørgeskema
  \item Hvilke metoder vi anvender 
\end{itemize}

\subsubsection{Præsentationsrækkefølge}
\label{PraesentationsraekkefoelgeValgAfGestikker}
%

%
\begin{table}[H]
	\centering
	\begin{tabular}{ | l | l | l | p{8cm} |}
		\hline
		Pause og start & Skift musiknummer & Skru op og ned \\ \hline
		Skru op og ned & Pause og start & Skift musiknummer\\ \hline
		Skift musiknummer & Skru op og ned & Pause og start \\ \hline
		Pause og start & Skru op og ned & Skift musiknummer\\ \hline
		Skru op og ned & Skift musiknummer & Pause og start\\ \hline
		Skift musiknummer & Pause og start & Skru op og ned \\ \hline
	\end{tabular}
	\caption{Oversigt over præsentationsrækkefølgen for de tre videoer; start og pause, skru op og ned og skift musiknummer.}
	\label{tab:PraesentationsraekkefoelgeValgAfGestikker}
\end{table}
\noindent
%




\subsection{Test session}
\label{TestSessionValgAfGestikker}
%
Testens varighed afhænger af testpersonernes respons, men anslåes til at vare mellem et kvarter og en halvtime, inklusiv testlederens instruktioner og det afsluttende exit-interview.

Test sessionen begynder med at testlederen byder testpersonen velkommen og introducerer testpersonen til hvad der kommer til at foregå undervejs i testen og at det ønskes at optage testpersonen undervejs, hvorefter testpersonen får udleveret en samtykkeerklæring, som de opfordres til at læse og underskrive. Så snart samtykkeerklæringen er underskrevet starter testlederen videokameraet. Dernæst bliver testpersonerne bedt om at udfylde spørgeskemaet, hvor de opfordres til at stille spørgsmål hvis der opstår tvivl. Inden testen påbegyndes bliver testpersonerne yderligere instrueret i hvad der kommer til at foregå samt hvilke opgaver de skal løse. 

Når testpersonerne har gennemgået de tre videoer og løst de pågældende opgaver, afvikles exit-interviewet. 
%

\subsection{Testpersoner}
\label{TestpersonerValgAfGestikker}
%
Testen udføres på studerende fra Aalborg Universitet. Det tilstræbes ikke at teste på studerende med forhåndsviden om design, brugertest og lignende eller studerende, hvis hovedfokus er elektronik og software. Da det antages at denne målgruppe primært vil fokusere på teknologien bag produktet eller andre specifikke designaspekter. Derudover bliver det af \textcite[s. 110]{Book:OUE} blandt andet anbefalet ikke at teste på personer med en baggrund inden for brugervenlighed og design af brugergrænseflader, da denne målgruppe har for stor viden inden for det område og de formentlig ikke vil være i stand til at give et upåvirket svar.

Det vil være favorabelt at teste på personer, som allerede har erhvervet sig et eller flere Bang $\&$ Olufsen produkter eller andet \textit{high-end} musikudstyr. Udfordringen ved at sætte det som et kriterie er, at det er begrænset hvor mange studerende, der på nuværende tidspunkt har råd til Bang $\&$ Olufsen's produkter, ikke sagt at de efter endt uddannelse ikke vil får råd. Da produktet stadig befinder sig på konceptbasis må det antages, at der kan gå år før produktet overhovedet bliver lanceret og til den tid kan nuværende universitetsstuderende potentielt nå at færdiggøre deres uddannelser og få en høj nok indkomst så de har råd til erhverve sig produkter fra Bang $\&$ Olufsen. \blankline
%        
Da præsentationsrækkefølgende af de tre videoer foregår efter et \textit{counterbalanced design} tilstræbes det at antallet af testpersoner kan deles med seks. At antallet af testpersoner skal kunne deles med seks skyldes at de tre videoer kan præsenteres i seks forskellige kombinationer, hvorfor der sigtes efter at hver kombination minimum repræsenteres to gange, svarende til 12 testpersoner. Det tilstræbes dog at udføre testen på 18 testpersoner, hvilket forudsætter at hver kombination repræsenteres tre gange. 

Ydermere tilstræbes det at begge køn er repræsenteret, da produkter fra Bang $\&$ Olufsen ikke er kønsafhængige. 
%

\subsection{Rollefordeling}
\label{RollerfordelingValgAfGestikker}
%
For at udføre testen skal der både udpeges to roller; en testleder og en observatør. Testlederens opgave er at instruere testpersonerne om hvordan testen vil forløbe, opfordre testpersonerne til at læse og underskrive samtykkeerklæringen, starte optagelsen efter testpersonerne har underskrevet samtykkeerklæringen, sørger for at testpersonerne udfylder spørgeskemaet, formidle testpersonernes opgaver samt at afvikle exit-interviewet. Testlederens instruktioner fremgår af BILAG og samtykkeerklæringen fremgår af BILAG. 

Observatørens opgave er at tage noter til testpersonernes respons og sørger for at pause og starte musikken, skrue op og ned for musikken samt skift musiknummer når testpersonerne opfordres til at forbedre deres fortrukne gestik til den specifikke video og når de afslutningsvist opfordres til at gengive gestikkerne igen. 

\subsection{Materialer og testlokation}
\label{MaterialeOgTestlokationValgAfGestikker}
%
Til testen indgår følgende materialer:
%
\begin{itemize}
  \item Videokamera + stativ
  \item Tre videoer med gestikker
  \item Farveskærm 
  \item To højtalere af typen Genelec 1031A
  \item iTunes 12.6  
  \item Computer med internetforbindelse\blankline
\end{itemize}
% 
Videokamera og stativ indgår da det ønskes at optage testpersonernes forbedringerne til at pause og starte musikken, skrue op og ned for musikken og til at skift musiknummer. Ydermere benyttes videokameraet til at optage testpersonernes respons undervejs og afslutningsvist i exit-interviewet. Det er derfor essentielt at videokameraet kan optage lyd og at videokameraet placeres dels så det kan optage testpersonernes bevægelser og dels optage deres orale respons. De tre videoer med gestikker er optaget på forhånd og gengiver, parvist, de seks mest gængse funktioner på et musikanlæg. Farveskærmen indgår for at præsenterer de tre videoer på en skærm, der er større end skærmen på en bærbarcomputer. De to højtalere skal tilkobles computeren hvorfra skærmen, hvor videoerne afspilles på, ligeledes er koblet til. iTunes 12.6 indgår i forbindelse med at testpersonerne opfordres til at forbedre deres fortrukne gestik og afslutningsvist hvor de opfordres til at gengive gestikkerne igen. Computer med internetforbindelse er nødvendig da det er der testpersonerne skal udfylde spørgeskemaet, i denne opstilling er det også den computer hvor farveskærmen og de to højtalere er tilkoblet.\blankline
% 
Testen afvikles i akustikafdelingen på Aalborg Universitet, Fredrik Bajers Vej 7B, i lokale: Lytterummet B4-107. 
%
\subsection{Evalueringsmetode}
\label{EvalueringsmetodeValgAfGestikker}
