\section{Metodevalg}
\label{DiskussionMetodevalg}
%
I den første undersøgelse, hvor de semaforiske gestikker udvælges, præsenteres testpersonerne for de forskellige forslag til semaforiske gestikker, hvilket foregik igennem tre videoer, jævnfør \autoref{app:VideooptagelserValgAfGestikker}. Herefter blev de udspurgt om, hvor nemt de syntes det var at udpege, hvilke gestikker de bedst og mindst kunne lide samt hvordan de syntes videoerne illustrerede de forskellige gestikker.

Den gennemgående tendens er, at det er nemt for testpersonerne at udpege, hvilke gestikker de bedst og mindst kan lide, men at det i flere tilfælde er svært at tildele et gestik-par tredjepladsen. Årsagen til at det har været svært, at tildele tredjepladsen, skyldes blandt andet, at testpersonerne gjorde brug af udelukkelsesmetoden, gestikken mindede mest om første- og andenpladsen eller så valgte testpersonerne det gestik-par, der egnede sig bedst til en tredjeplads. 

Samtlige testpersoner gav udtryk for, at de tre videoer illustrerede gestikkerne godt og at det var godt, at fokus var på gestikkerne og at der ingen forstyrrelse var, eksempelvis i form af at demonstratoren ikke var iklædt anderledes afhængigt af video og at demonstratoren formåede at gengive det samme ansigtsudtryk. Gestikkerne optages, for at testpersonerne dels, kan få et indtryk af hvilke gestikker, der er mulige at anvende og dels, kan få bedre indtryk, af hvordan gestikkerne udføres sammenlignet med at gestikkerne præsenteres ved stilbilleder. Ydermere optages gestikkerne for at testpersonerne, alle får præsenteret gestikkerne fuldstændig ens. I den forbindelse påpeges det, at testpersonerne derfor kun får det indtryk af gestikkerne, som det illustreres i videoerne. I tilfælde af, at demonstratoren havde svært ved at gengive gestikken efter hensigten, resulterer det i, at det er sådan testpersonerne oplever gestikken. Dette kom særligt til udtryk ved TP16, som synes at bevægelsen i GP2 til justering af lydstyrken er ukomfortabel. 

Der er fordele og ulemper ved at testpersonerne præsenteres for gestikkerne i en klinisk udført sammenhæng. Fordelen er at testpersonerne kan koncentrere sig om gestikkerne, uden at blive forstyrret eller påvirket af eksterne elementer, hvor ulempen er, at testpersonerne ikke oplever gestikken i den tiltænkte kontekst; dagligstuesituation. Af de årsager, vælges det at foretage en ny videooptagelse af de tre udvalgte gestik-par til testen i den sociale kontekst, \autoref{app:VideooptagelseSocialAccept}. I denne videooptagelse illustreres gestikkerne i den opstillede dagligstue, hvor demonstratoren positioneres der hvor værten ligeledes positioneres og gengiver gestikkerne mere afslappet end i de tre foregående videoer.\blankline 
%
Testen i den sociale kontekst blev udført som et \textit{Wizard of Oz}-eksperiment, da der ikke er udarbejdet en funktionel prototype af Bang $\&$ Olufsens fremtidige musikanlæg. 


For at påvirke værterne til at interagere med musikanlægget, blev der anvendt hints, da der ellers er risiko for at værterne glemmer at interagere med musikanlægget, fordi de fordyber sig i samtalen med gæsten. 



 Ud fra \autoref{fig:KiggerImodAnlaeg} er det ikke muligt at konkludere, at værterne kigger mere ved nogle funktioner end andre, hvilket i dette tilfælde er positivt, da det kan betyde at der ikke er én gestik, som kræver mere opmærksomhed and andre. 

Fokuseres der generelt på de forskellige hints, kan disse have været med til at gøre situationen mindre økologisk, da det hverken er naturligt at ens musik ændrer lydstyrke midt i et musiknummer, at musiknumrene på ens egen playliste er forvrængede eller at telefonen ringer så mange gange på så kort tid. Derimod virkede det som om, at testpersonerne blev mere forstyrret af de forskellige hints end af selve interaktionen. Specielt hintet med det forvrængede stykke musik voldte problemer hos værterne, da nogle havde svært ved at høre forvrængningen, mens andre skiftede musiknummer, selvom der ikke var forvrægning. Alternativt til at forvrænge musikken kunne værterne introduceres til, at hver gang et dansk musiknummer eller en reklame blev afspillet, skulle der skiftes videre. Problemet med det er, at testsessionerne enten bliver meget længere eller at et testleder 2 skal skifte musiknummer midt i et andet nummer, hvilket kan skabe forvirring. Endnu et alternativ vil så være at korte sangene ned, så det danske musiknummer eller reklamen kunne afspilles hyppigere. 

Testen vil højst sandsynligt blive mere økologisk, hvis værten får længere tid under en familiarisering til at lære interaktionsformen at kende. Dette kan eventuelt ske over flere dage. I tilfælde af at værten lærer gestikkerne over længere tid, kan det samtidig være nemmere at påvise en perifer interaktion.

 Ydermere kan det overvejes, om det har en effekt at gæsterne slet ikke ved hvad der skal foregå med musikanlægget op imod at de både kender musikanlægget og interaktionsformen på forhånd. Det forventes, at gæster der kender både musikanlæg og interaktionsform er mindre interesseret i interaktionen end personer, der aldrig har set det før. \blankline
 