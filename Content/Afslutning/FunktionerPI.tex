\section{Valg af funktioner til perifer interaktion}
\label{DiskussionFunktionerCasualInteraction}
%
En af forudsætningerne for at interaktion kan foregå i den perifere opmærksomhed er dels at interaktionen repeteres gentagende gange så det bliver en indøvet rutine, hvorefter interaktionen ikke bebyrder de kognitive ressourcer i lige så høj grad, som før. Derudover bør den perifere interaktion ikke være afhængig af den visuelle opmærksomhed eller stemmestyring. Ydermere er en af forudsætningerne at den perifere interaktion kan udføres samtidig med eller med minimal indflydelse på en primær opgave. Som belyst i \fullref{CasualOgCalm}, så bygger perifer interaktion på to principper; \textit{casual interaction} og \textit{calm technology}, hvor førstnævnte relaterer til mængden af kontrol brugeren er villig til at overlade til et elektronisk apparat og hvor sidstnævnte relaterer sig til at hvis dele af interaktionen allokeres til den perifere opmærksomhed, så er det muligt at håndtere flere information ad gangen og samtidig forstærke følelsen af kontrol. Sammenholdes forudsætningerne og de to principper med antallet af funktioner Bang $\&$ Olufsen implementerer i sine produkter; 18 i alt, så er det ikke hensigtsmæssigt at allokere dem alle til den perifere opmærksomhed. Det skyldes flere ting; funktioner såsom \textit{add to..}, \textit{shift source} og \textit{shift experiences} kræver både mere kontrol og mere opmærksomhed sammenlignet eksempelvis med at pause musikken. Derudover er det ikke alle funktioner, som benyttes hver gang der høres musik. Da det er forholdvist nyt at interaktionen foregår via semaforiske gestikker er det heller ikke hensigtmæssigt at forvente, at brugeren skal lære 18 forskellige gestikker for at interagere med sit musikanlæg. Som nævnt er en forudsætning for perifer interaktion, at interaktionen bliver en rutine, hvorfor det igen indikerer at det er uhensigtmæssigt at tildele samtlige funktioner en semaforisk gestik. 

Selvom det i denne sammenhæng ikke har været muligt at påvise en decideret perifer interaktion, så tyder det på, at de valgte funktioner; pause, start, skift musiknummer samt justering af lydstyrken, med fordel kan allokeres til den perifere opmærksomhed. Det skyldes formentlig at de valgte funktioner er universelle funktioner i et musikanlæg og dermed er nødvendige for at høre musik, men derudover så kan funktionerne sammensættes i par. Så istedet for at designe seks forskellige semaforiske gestikke, så er det nok at designe tre, som dækker over modsætningerne; pause og start, skift sang frem og tilbage samt skru og ned for musikken. Derudover så er det forsøgt at udvælge semaforiske gestikker, som testpersonerne relaterede til den specifikke funktion men uden at gestikken indgår naturligt i kropssporget.  

I forhold til antallet af funktioner kommenterer TP1, som deltog i udvælgelsen af de semaforiske gestikker at: 
%
\begin{quotation}
	\noindent
	\textit{Der skal ikke være alt for mange funktioner. Jeg synes det her er sådan tilpas, i hvert fald med det der ligger naturligt.} TP1, \autoref{app:NoterValgAfGestikker}.
\noindent
\end{quotation}
%
I henhold til \textit{casual interaction}, mængden af kontrol brugeren er villig til at overdrage, så afhænger det af funktionen. Under udvælgelsen af hvilken semaforisk gestik, der skal knyttes til justering af lydstyrke er der flere testpersonerne, som kommenterer niveauet af kontrol i forhold til de forskellige foreslag. Ifølge TP5, som deltog i udvælgelsen af de semaforiske gestikker, så tillader to hænder mere kontrol end én hånd. Testpersonen understreger gentagende gange vigtigheden af at have kontrol over sin interaktion, hvilket eksempelvis udtrykkes ved: 
%
\begin{quotation}
	\noindent
	\textit{Det kommer an på hvor godt det virker og hvor meget kontrol man har over det. Jeg vil gerne have 100\% kontrol.} TP5, \autoref{app:NoterValgAfGestikker}.
\noindent
\end{quotation}
%
Hvor meget kontrol brugeren kræver at have selv i forhold til hvor meget kontrol brugeren er villig til at overdrage, afhænger formentlig af følgende faktorer; hvilken semaforisk gestik der anvendes, produktets reaktionsevne, funktionen og særligt af brugerens subjektive holdning. Derudover forventes det, at såfremt brugeren bliver fortrolig med interaktionsformen, så vil de vænne sig til hvordan kontrollen distribueres mellem dem selv og produktet. Selvom testpersonerne i ved nogle funktioner ønsker mere kontrol end ved andre, så er det ikke ens betydende med, at de ikke er villige til at overdrage noget kontrol til et elektronisk apparat. 

En gennemgående tendens blandt samtlige testpersoner er, at de betragter interaktionsformen som et godt alternativ til den fysiske interaktion, eksempelvis med en mobiltelefon eller fjernbetjening, som normalvist foregår i den centrale opmærksomhed. Dette understøtter dels at interaktionen med et musikanlæg kan allokeres fra den centrale til den perifere opmærksomhed og dels at der en villighed til at overdrage kontrol, hvilket relaterer sig til både \textit{casual interaction} og \textit{calm technology}. \blankline
%
At de valgte funktioner udgøre de seks mest gængse funktioner i et musikanlæg og at de tilhørende udvalgte semaforiske gestikker, direkte forbindes til funktionen, har formentligt forstærket testpersonerne, som agerede værter i undersøgelsen af interaktionsformen i en social kontekst, i at udføre begge opgaver; interaktionen med musikanlægget samtidig med at føre samtalen med gæsten. Dette udledes blandt andet af det lave antal af fejl værterne begik samt både værternes og gæsternes egne vurderinger om hvorvidt værterne var i stand til at udføre begge opgaver. Om værterne har været i stand til at håndtere flere informationer ad gangen kan ikke fastslåes, men de har været i stand til både at registrere og reagere på diverse hints uden markante fokuseringsskift fra samtalen med gæsten. Dette opfylder derfor principperne bag \textit{calm technology} og indikerer dermed at interaktionen potentielt kan foregå i den perifere opmærksomhed.

I tilfælde af at der istedet fokuseres på andre funktioner, eksempelvis dem listet som sekundære funktioner i \fullref{SamspilMedBO}, så er det uvist, hvorvidt særligt værterne vil være i stand til at løse begge opgaver. 

Tilstræbes der en perifer interaktion vurderes det på baggrund af foregående, at forudsætningerne er mødt, hvilket i denne sammenhæng relatere sig til at vælge funktioner, som ikke er unikke for et specifik produkt eller mærke, men som derimod er universelle og hyppigst anvendt. 
%
