\chapter{Kategoriseringer af gestikker}
\label{app:KategoriseringerAfGestikker}
%}
I følgende kapitel undersøges det, hvilke typer af gestikker der eksisterer. Det skal nævnes, at der i litteraturen fremgår mange forskellige termer for de samme typer af gestikker, \parencite[s. 3]{PDF:ATaxonomyOfGestures}. Eksempelvis bliver semaforiske gestikker også angivet som frihåndsgestikker og gestikker. Der vil derfor tages udgangspunkt i, hvordan de forskellige typer af gestikker defineres, baseret på definitionerne fremsat af \textcite[ss. 4-9]{PDF:ATaxonomyOfGestures}. Ifølge \textcite[s. 4]{PDF:ATaxonomyOfGestures} kan gestikker kategoriseres i fem grupper: \textit{Deictic Gestures}, \textit{Manipulative Gestures}, \textit{Semaphoric Gestures}, \textit{Gesticulation} og \textit{Language Gestures}. \blankline
%
Da der tidligere er afgrænset fra at arbejde med stemmestyring, vil \textit{Gesticulation} ikke blive nærmere beskrevet, da denne type af gestikker involverer både håndbevægelser og tale, \parencite[s. 7]{PDF:ATaxonomyOfGestures}. Dog kan det nævnes, at gestikulerende gestikker er den form for gestik, som anses for være den mest naturlige og samtidig mest udfordrende form for gestik, \parencite[s. 7]{PDF:ATaxonomyOfGestures}. Til trods for de store designmæssige udfordringer de gestikulerende gestikker forårsager, forudser \textcite[s. 28]{PDF:ATaxonomyOfGestures}, at denne form for gestik vil finde et større indpas i, hvordan der interageres med elektroniske apparater. \blankline
%
\textit{Language Gestures} er de tegn, som anvendes i tegnsprog. Denne type af gestikker er bygget op omkring en grammatisk struktur og anvendes primært til kommunikation fremfor at give kommandoer, \parencite[s. 8]{PDF:ATaxonomyOfGestures}. Ifølge \textcite[s. 8]{PDF:ATaxonomyOfGestures} er det lige så krævende for et system at processere tegnsprog, som det er at processere tale. For at anvende tegnsprog, særligt i den perifere opmærksomhed, vil det først og fremmest kræve, at brugeren lærer tegnsprog og at kræve det af brugeren, vurderes til at være for omstændigt. Der afgrænses derfor fra at arbejde med tegnsprog. \blankline
%  
\textit{Deictic Gestures} er en kategori af gestikker, som forudsætter at der peges på objekter med det formål at ændre deres egenskaber og deres spatiale lokation, \parencite[s. 4]{PDF:ATaxonomyOfGestures}. Ifølge \textcite[ss. 4-5]{PDF:ATaxonomyOfGestures} anvendes deiktiske gestikker, blandt andet, til at allokere objekter på en stor skærm, identificere objekter i \textit{Virtual Reality} og desktop- samt kommunikationsbaserede applikationer. \blankline
%
\textit{Manipulative Gestures}, som denne kategori antyder, vedrører de manipulerende gestikker hvorved objekter, i en eller anden grad, manipuleres, \parencite[s. 5]{PDF:ATaxonomyOfGestures}. Det kan både foregå ved 2- og 3-dimensionelle interaktioner. Manipulerende gestikker ved 2-dimensionelle interaktioner vedrører manipulering af objekter på en 2-dimensionel skærm, som for eksempel en cursor eller et vindue. Ifølge \textcite[s. 5]{PDF:ATaxonomyOfGestures} anses det ikke som værende en manipulerende gestik, hvis objektet trækkes eller klikkes på, da dette ikke vil ændre objektets egenskaber. For at det kan klassificeres som en manipulerende gestik, skal systemet, ifølge \textcite[s. 5]{PDF:ATaxonomyOfGestures}, modtage følgende parametre; brugerens anmodning om at flytte eller på anden vis ændre objektet. Ved 3-dimensionelle interaktioner kan de manipulerende gestikker enten afspejle en fysisk manipulering af et objekt, en fysisk manipulering af en computer eller ved at manipulere et fysisk objekt, som afspejles i et virtuelt objekt på en touchskærm, \parencite[s. 6]{PDF:ATaxonomyOfGestures}. Derudover kan 3-dimensionelle interaktioner inkorporeres, så de manipulerer 2-dimensionelle objekter, eksempelvis ved hjælp af tryksensorer, hvorved det er muligt at tilføre ekstra dimensioner til en 2-dimensionel interaktion, \parencite[s. 5]{PDF:ATaxonomyOfGestures}. 

De manipulerende gestures benyttes typisk til navigation i den virtuelle verden, fordi det, ved brug forskellige sensorer placeret i det virtuelle rum, er muligt at interagere med virtuelle objekter, \parencite[ss. 14-15]{PDF:ATaxonomyOfGestures}. I forhold til perifer interaktion tyder det på, at det er de manipulerende gestikker, som indtil videre er de mest anvendte, da testpersonerne manipulerer et fysisk objekt, \parencite[s. 164]{PDF:ComparingInputModalities}. I undersøgelsen foretaget af \textcite[ss. 164-165]{PDF:ComparingInputModalities} gør de, blandt andet, brug af to forskellige modaliteter inden for manipulerende gestikker; et fysisk, håndgribeligt knop-baseret håndtag og en touchskærm.\blankline
%
Den sidste, af de fem kategorier, er \textit{Semaphoric Gestures}, som, ifølge \textcite[s. 6]{PDF:ATaxonomyOfGestures}, er den mest udbredte form for gestik på trods af, at det anses for at være en unaturlig måde, at interagere med elektroniske apparater på, \parencite[s. 1961]{PDF:AStudyOnTheUseOfSemaphoricGestures}. Semaforiske gestikker relaterer sig til at benytte tegn til at kommunikere information, hvor disse tegn dannes med kropsbevægelser, særligt ved brug af hænderne, hvorfor semaforiske gestikker ofte angives som frihåndsgestikker. Ifølge \textcite[s. 1961]{PDF:AStudyOnTheUseOfSemaphoricGestures}, er det en unaturlig måde at interagere med en computer på og derudover gengiver disse gestikke kun en begrænset del af menneskets kommunikationsevne. Dog tyder det på, at netop semaforiske gestikke med fordel kan anvendes som en interaktions mulighed ved sekundære opgaver, \parencite[s. 1961]{PDF:AStudyOnTheUseOfSemaphoricGestures}. Det skyldes, ifølge \textcite[s. 1964]{PDF:AStudyOnTheUseOfSemaphoricGestures}, at denne form for gestik vil reducere restitutionstiden mellem den sekundære og primære opgave. Årsagen til det skyldes, blandt andet, at semaforiske gestikker ikke afhænger af den visuelle opmærksomhed, hvorfor opmærksomheden forsat kan være på den primære opgave. Ydermere reduceres restitutionstiden, fordi de semaforiske gestikker afhænger af den proprioceptive sans, som formegentlig ikke stimuleres i den primære, visuelle, opgave. Derudover skal der, ifølge \textcite[s. 1964]{PDF:AStudyOnTheUseOfSemaphoricGestures}, ved semaforiske gestikker et færre antal interaktioner til, for at den sekundære opgave bliver løst, hvilket ligeledes er med til at reducere restitutionstiden. En anden fordel ved at anvende semaforiske gestikke er, at de tillader en større afstand mellem brugeren og det elektroniske apparat, \parencite[s. 6]{PDF:ATaxonomyOfGestures}.    

%
Selvom det er muligt kun at fokusere på én af de fem kategorier af gestikker, kan det være en fordel at udnytte flere typer af gestikker, når der skal interageres med elektroniske apparater. Dog tyder det på, at det ikke er alle fem kategorier, der er lige eftertragtede at kombinerer. Som det fremgår af \textcite[s. 8]{PDF:ATaxonomyOfGestures}, er det kun; deiktiske gestikker, semaforiske gestikker og manipulerende gestikker, som kombineres. De to resterende gestikulerende gestikker og tegnsprog nævnes slet ikke i forbindelse med at bruge mere end én type gestik. 
