\subsection{Pilottest til valg af gestikker}
\label{PilottestValgAfGestikker}
Før enhver test kan det være en god idé at køre en pilottest, for på den måde at finde og eliminere fejl ved testdesignet. Formålet med denne pilottest var netop at undersøge, hvorvidt testopstillingen fungerede og om der opstod nogle komplikationer ved at filme testpersonerne. Til pilottesten blev der testet på fire studerende, der ellers ville falde uden for kriterierne for testpersoner, da disse var tilgængelige og at hovedfokus med pilottesten var at testkøre testopstillingen. Resultaterne fra pilottesten fremgår af det \textbf{elektroniske bilag}. 

Selvom det ikke var formålet, blev kravet omkring hvilke testpersoner der skulle bruges, undersøttet af pilottesten. Det viste sig nemlig, at testpersonerne svarede på hvilke gestikker, de bedst kunne lide udelukkende på baggrund af deres teknologiske viden og ikke ud fra hvad de bedst kunne lide.  

Under pilottesten blev der filmet med et Canon Powershot s110, som hurtigt viste sig ikke at kunne holde strøm tilstrækkeligt lang tid til. Optagelserne fra pilottesten blev derfor enten stoppet for tidligt eller slet ikke optaget, da kameraet løb tør for strøm, hvilket også er derfor disse vidoer ikke fremgår af det \textbf{elektroniske bilag}. På baggrund af dette blev der valgt fremadrettet at filme med et GoPro Hero4 Silver kamera. 

Når testpersonerne skulle prøve selv at styre musikken med gestikker blev iTunes på computeren avendt til at ændre musikken, så illusionen om at gestikkerne kunne ændre musikken blev opretholdt. Heri lå der et problem, når testpersonen skiftede sang og musiknummeret havde en stille intro, da testpersonerne ikke var sikre på om de havde udført gestikken ordentligt. Det blev derfor valgt at opsætte en playliste med musiknumre, der alle sammen startede med det samme, så der ikke herskede tvivl om hvorvidt der var blevet skiftet nummer eller ej. Ydermere blev positionen af observatøren ændret til at sidde bag testleder og testperson, så testpersonerne ikke kunne se observatøren, når de lavede de forskellige gestikker. 

Sidst blev spørgeskemaet ændret to gange undervejs, så det blev klart for testpersonerne, at det netop var brugen af gestikker, hvor der ikke skal røres ved en skærm, der blev spurgt ind til. 

Med disse forbedringer findes testen klar til at blive udført på de rigtige testpersoner. 