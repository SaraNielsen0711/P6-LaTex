\chapter{Testresultater}
\label{TestresultaterSocialAccept}
%
Igennem de følgende afsnit analyseres og diskuteres testresultaterne, for at vurdere om testpersonerne er i stand til, via de udvalgte semaforiske gestikker, at interagere med et musikanlæg som en sideløbende opgave til den primær opgave; samtalen med gæsten. Derefter analyseres og diskuteres det, om de udvalgte semaforiske gestikker påvirker den sociale accept samt om testpersonerne har lyst til at interagere med gestikker.\blankline
%
Der er i alt indsamlet data fra 12 testpersoner; fire kvinder og otte mænd. To af kvinderne var venstrehåndede, hvorfor der kun er to kvindlige værter. Der deltog sammenlagt tre venstrehåndede testpersoner. Testpersonernes alder spænder fra 21 år til 27 år med en gennemsnitalder på 23 år. Fire ud af seks par kender hinanden igennem studiet, et par er kærester og et par er forlovet. Ud af de 12 testpersoner er der én, som ikke er studerende på Aalborg Universitet, men er under uddannelse andetsteds. Testpersonerne er fra Kemiteknologi, Software, Erhvervsjura, Vej og Trafik, Energi og Indeklima, Vand og Miljø, Lærerseminariet og Konstruktion. Der henvises kun til det indsamlede data i tilfælde af, at testpersonerne citeres, ellers fremgår alt indsamlet data af bilag. Testpersonernes respons i exit-interviewet forefindes i \autoref{app:ResultaterSocialAccept} og i tilfælde af, at der er behov for at inddrage testpersonernes adfærd, henvises der til følgende \autoref{app:app:VideooptagelseVaertOgGaest}, som indeholder videomaterialet til de pågældende situationer. Medmindre der refereres til samtlige testpersoner, så angives testpersonerne ud fra hvilket par nummer samt hvilken rolle; vært eller gæst de havde. Refereres der eksempelvis til testperson 1 vil dette angives som vært 1 og testperson 2 vil angives som gæst 1. 



