\section{De udvalgte semaforiske gestikker}
\label{DiskussionUdvalgteGestikker}
%
For at besvare den første problemstilling: \textit{Hvilke specifikke semaforiske gestikker skal knyttes til hver af de seks mest gængse funktioner i Bang $\&$ Olufsens fremtidige musikanlæg, for at interaktionen kan foregå i den perifere opmærksomhed?}, deltog 18 testpersoner i udvælgelsen af hvilke semaforiske gestikker, der skal knyttes til henholdvis pause og start, skift musiknummer samt justering af lydstyrken, på baggrund af projektgruppens foreslag. Da fokus for den del af undersøgelsen var at udvælge de semaforiske gestikker, blev der ikke undersøgt perifer interaktion. 

En gennemgående tendens for testpersonernes begrundelser til de udvalgte gestikker er, at testpersonerne forbinder dem med noget velkendt; krokodillenæbet til at pause og starte musikken forbindes med "ti stille", swipe-bevægelsen med to strakte fingre forbindes med hvordan der normalvist interageres på en smartphone eller tablet, hvor gestikken til at justere lydstyrken forbindes direkte med hvordan lydstyrken justeres på et musikanlæg via en drejeknap. Udover at testpersonerne har tendens til at vælge gestikker, som de forbinder med noget velkendt, så er der ydermere en tendens til at vælge gestikker, der ikke normalvist indgår som en naturlig del af ens kropssprog. De udvalgte semaforiske gestikker medfører at testpersonerne ikke bekymre sig om, hvorvidt de ubevidst gengiver gestikkerne i situationer, hvor hensigten ikke er at interagere med musikanlægget. Med udgangspunkt i at de udvalgte semaforiske gestikker, normalvist ikke er en naturlig del af kropssproget reduceres sandsynligheden for et \textit{Midas touch problem}, både i forhold til om brugeren decideret gengiver en forkert gestik og i forhold til om produktet reagerer på en gestik, som ikke er henvendt til produktet. 

Særligt i forhold til at skifte musiknummer hvor valget stod mellem GP1 og GP5, blev der taget højde for \textit{Midas touch problem}. Fælles for de to gestik-par er, at de begge er bygget op omkring en swipe-bevægelse, hvor der i GP1 swipes med hele hånden og GP5 swipes med to udstrakte fingre. Selvom de fleste testpersoner favoriserede GP1, så vurderes det, at risikoen for at opleve et \textit{Midas touch problem} er langt større for GP1 sammenlignet med G5. Vurderingen bygger på at der ved GP5, foruden swipe-bevægelsen, kræves et specifikt håndtegn som systemet skal reagere på, hvilket ikke er tilfældet for GP1.\blankline
%
I henhold til hvilken semaforisk gestik, der anvendes så er den gennemgående tendens, at statiske gestikker er utilregnelige fordi de ikke tillader kontrol over, eksempelvis hvor meget lydstyrken justeres eller hvilket musiknummer der skiftes til. Derudover så tyder det på, at testpersonerne foretrækker gestikker, hvor det er muligt at danne en form for visueltbillede af hvor meget lydstyrken justeres. Dette udledes blandt andet fra testpersonernes begrundelser for hvorfor de fravælger de statiske gestikker, men kan også udledes fra følgende:  
%
\begin{quotation}
	\noindent
	\textit{Og så 5'eren fordi at det virker sådan at man har mere kontrol der i forhold til 9'eren, hvor du bare står sådan og vifter opad. Det kunne jo være hvilken som helst volumen du var kommet op på.} TP7, \autoref{app:NoterValgAfGestikker}.
\noindent
\end{quotation}
%
GP9 gengives ved en "kom så"- eller "ro på"-bevægelse, hvor det sammenlignet med de resterende gestik foreslag er svære at danne et visueltbillede af hvor meget lydstyrken justeres. Dertil er der ved GP9 og de statiske gestikker ingen fysisk begrænsning testpersonerne kan forholde sig til. Ved GP2 som består af en cirkulærbevægelse dannet ud fra albueleddet, håndleddet og fingrene, så er oplever testpersonerne en naturlig fysisk begrænsning ved, at det på et tidpunkt ikke længere er muligt at roterer hånden. Lignende er gældende GP3 og GP4 hvor testpersonernes begrænses af hvor meget de er i stand til at hæve og sænke armen.

Som det blev nævnt i \fullref{TestresultaterValgAfGestikkerValgVolumen} var det ikke muligt at ekskludere hverken GP2, GP3 eller GP4, hvorfor projektgruppen foretog beslutningen dels med udgangspunkt i at testpersonerne forbinder GP2 med en fysisk drejeknap og fordi det antages at risikoen for et \textit{Midas touch problem} er mindre ved GP2 end både GP3 og GP4. Det er derfor ikke muligt at fastslå hverken hvordan GP3 eller GP4 påvirker den sociale kontekst når værterne interagerer med musikanlægget samtidig med at de fører samtalen med gæsten.  
 





Diskuter de udvalgte gestikker, der var andre muligheder til at skrue op og ned, i forhold til bevægelsesmængde og afstand til kroppen, midas touch problem, maskering af gestikker, retningsbestemt fremfor aktiveringsgestik (perifert, socialt accept)\blankline
%
Midas touch problem. Der er allerede under komplikationer ved gestik åbnet op for at gestikkerne skal rettes mod anlægget. Så skygger man heller ikke for gestikken.\blankline
%
Retning på swipe\blankline
%
Standardisering af gestikker
%

