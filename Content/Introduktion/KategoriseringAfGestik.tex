\subsection{Kategoriseringer af gestikker}
\label{KategoriseringerAfGestikker}
%
Inden der dykkes ned i forskellige typer af gestikker skal det nævnes, at der i litteraturen fremgår mange forskellige termer for de samme typer af gestikker, \parencite[s. 3]{PDF:ATaxonomyOfGestures}. Eksempelvis bliver semaforiske gestikker også angivet, som frihåndsgestikker og gestikker. Der vil derfor tages udgangspunkt i, hvordan de forskellige typer af gestikker defineres, baseret på definitionerne fremsat af \textcite[ss. 4-9]{PDF:ATaxonomyOfGestures}. Ifølge \textcite[s. 4]{PDF:ATaxonomyOfGestures} kan gestikker kategoriseres i fem grupper: \textit{Deictic Gestures}, \textit{Manipulative Gestures}, \textit{Semaphoric Gestures}, \textit{Gesticulation} og \textit{Language Gestures}. \blankline
%
Da der tidligere er afgrænset fra at arbejde med stemmestyring, så vil \textit{Gesticulation} ikke blive nærmere beskrevet, da denne type af gestikker involverer både håndbevægelser og tale, \parencite[s. 7]{PDF:ATaxonomyOfGestures}. Dog kan det nævnes at gestikulerende gestikker er den form for gestik, som anses for være den mest naturlige og samtidig mest udfordrende form for gestik, \parencite[s. 7]{PDF:ATaxonomyOfGestures}. Tiltrods for de store designmæssige udfordringer de gestikulerende gestikker bringer med sig, så forudser \textcite[s. 28]{PDF:ATaxonomyOfGestures}, at denne form for gestik vil finde et større indpas i hvordan der interageres med elektroniske apparater. \blankline
%
\textit{Language Gestures} er de tegn, som bruges i tegnsprog. Denne type af gestikker er bygget op omkring en grammatisk struktur og anvendes primært til kommunikation fremfor at give kommandoer, \parencite[s. 8]{PDF:ATaxonomyOfGestures}. Ifølge \textcite[s. 8]{PDF:ATaxonomyOfGestures} så er det lige så krævende for et system at processere tegnsprog, som det er at processere tale. For at anvende tegnsprog, særligt i det perifere, vil det først og fremmest kræve at brugeren lærer tegnsprog og at kræve det af brugeren, vurderes til at være for omstændigt. Der afgrænses derfor fra at arbejde med tegnsprog. \blankline
%  
\textit{Deictic Gestures} er en kategori af gestikker, som forudsætter at der peges på objekter med det formål at ændre deres egenskaber og deres spatiale lokation, \parencite[s. 4]{PDF:ATaxonomyOfGestures}. Ifølge \textcite[ss. 4-5]{PDF:ATaxonomyOfGestures} anvendes deiktiske gestikker, blandt andet, til at allokere objekter på en stor skærm, identificere objekter i \textit{Virtual Reality} og desktop- samt kommunikationsbaserede applikationer. \blankline
%
\textit{Manipulative Gestures}, som denne kategori antyder, vedrører de manipulerende gestikker hvorved objekter, i en eller anden grad, manipuleres, \parencite[s. 5]{PDF:ATaxonomyOfGestures}. Det kan både foregå ved 2- og 3-dimensionelle interaktioner. Manipulerende gestikker ved 2-dimensionelle interaktioner vedrører manipulering af objekter på en 2-dimensionel skærm, som for eksempel en cursor, et vindue eller noget helt tredje. Ifølge \textcite[s. 5]{PDF:ATaxonomyOfGestures} anses det ikke som værende en manipulerende gestik hvis objektet blot trækkes eller klikkes på, da dette ikke vil ændre objektets egenskaber. For at det kan være en manipulerende gestik skal systemet, ifølge \textcite[s. 5]{PDF:ATaxonomyOfGestures}, modtage nogle parametre såsom; brugerens anmodning om at flytte eller på anden vis ændre objektet. Ved 3-dimensionelle interaktioner kan de manipulerende gestikker enten afspejle en fysisk manipulering af et objekt, en fysisk manipulering af en computer eller ved at manipulere et fysisk objekt, som afspejles i et virtuelt objekt på en touchskærm, \parencite[s. 6]{PDF:ATaxonomyOfGestures}. Derudover kan 3-dimensionelle interaktioner også inkorporeres så de manipulerer 2-dimensionelle objekter, for eksempel ved hjælp af tryksensorer, hvorved det er muligt at tilføre ekstra dimensioner til en 2-dimensionel interaktion, \parencite[s. 5]{PDF:ATaxonomyOfGestures}. 

De manipulerende gestures benyttes typisk til navigation i den virtuelle verden, fordi det, ved brug forskellige sensorer placeret i det virtuelle rum, er muligt at interagere med de virtuelle objekter, \parencite[ss. 14-15]{PDF:ATaxonomyOfGestures}. I forhold til perifer interaktion tyder det på, at det er de manipulerende gestikker, som indtil videre er de mest brugte, i form af at testpersonerne manipulerer et fysisk objekt, \parencite[s. 164]{PDF:ComparingInputModalities}. I undersøgelsen foretaget af \textcite[ss. 164-165]{PDF:ComparingInputModalities} gør de, blandt andet, brug af to forskellige modaliteter inden for manipulerende gestikker; et fysisk, håndgribeligt knop-baseret håndtag og en touchskærm.\blankline
%
Den sidste, af de fem kategorier, er \textit{Semaphoric Gestures}, som, ifølge \textcite[s. 6]{PDF:ATaxonomyOfGestures}, er den meste udbredte form for gestik på trods af, at det anses for at være en unaturlig måde, at interagere med elektroniske apparater på, \parencite[s. 1961]{PDF:AStudyOnTheUseOfSemaphoricGestures}. Semaforiske gestikker relaterer sig til at benytte tegn til at kommunikere information, hvor disse tegn dannes med kropsbevægelser, særligt ved brug af hænderne, hvorfor semaforiske gestikker ofte kaldes frihåndsgestikker. Ifølge \textcite[s. 1961]{PDF:AStudyOnTheUseOfSemaphoricGestures} er det en unaturlig måde at interagere med en computer på og derudover gengiver disse gestikke kun en begrænset del af menneskets kommunikationsevne. Dog tyder det på, at netop semaforiske gestikke med fordel kan anvendes som en interaktions mulighed ved sekundære opgaver, \parencite[s. 1961]{PDF:AStudyOnTheUseOfSemaphoricGestures}. Det skyldes, ifølge \textcite[s. 1964]{PDF:AStudyOnTheUseOfSemaphoricGestures}, at denne form for gestik vil reducere restitutionstiden mellem den sekundære og primære opgave. Årsagen til det skyldes, blandt andet, at semaforiske gestikker ikke afhænger af den visuelle opmærksomhed, hvorfor opmærksomheden forsat kan være på den primære opgave. Ydermere reduceres restitutionstiden, fordi de semaforiske gestikker afhænger af den proprioceptive sans, som formegentlig ikke stimuleres i den primære, visuelle, opgave. Derudover skal der, ifølge \textcite[s. 1964]{PDF:AStudyOnTheUseOfSemaphoricGestures}, ved semaforiske gestikker et færre antal interaktioner til, for at den sekundære opgave bliver løst, hvilket ligeledes kan være med til at reducere restitutionstiden. En anden fordel ved at anvende semaforiske gestikke er, at det tillader en større afstand mellem brugeren og det elektroniske apparat, \parencite[s. 6]{PDF:ATaxonomyOfGestures}.    

Der er to former for semaforiske gestikker; statiske og dynamiske, \parencite[s. 7]{PDF:ATaxonomyOfGestures}. De statiske kræver at gestikken fastholdes, hvor de dynamiske tillader bevægelse. At fastholde en gestik forudsætter, at brugeren holder sine fingre i ro, det vil, eksempelvist, ikke være muligt at lave en statisk gestik, som ændre lydstyrken i musikken. En statisk gestik kan eksempelvist være, at danne et OK-tegn med fingrene. Begge former for semaforiske gestikker kan udføres ved at anvende hænderne, fingrene, armene, fødderne, hovedet eller andre passive elektroniske apparater, \parencite[s. 7]{PDF:ATaxonomyOfGestures}. Ifølge \textcite[s. 823]{PDF:UnderstandingNaturalness} så bør gestikkerne være dynamiske, hvis formålet er at manipulere et objekt. \blankline
%
Selvom det er muligt kun at fokusere på én af de fem kategorier af gestikker, så kan det være en fordel at udnytte flere typer af gestikker, når der skal interageres med en elektroniske apparaters brugergrænseflader. Dog tyder det på at det ikke er alle fem kategorier, der er lige eftertragtet at blande med de andre, som det fremgår af \textcite[s. 8]{PDF:ATaxonomyOfGestures}, er det nemlig kun; deiktiske gestikker, semaforiske gestikker og manipulerende gestikker, som blandes. De to resterende gestikulerende gestikker og tegnsprog nævnes slet ikke i forbindelse med at bruge mere end én type gestik. 
%
\subsubsection{Udførelsen af gestik}
\label{UdfoerelseAfGestik}
%
I dette afsnit fokuseres der på to overordnet former for input-gestikker, som et elektronisk apparat kan reagere på. Dernæst fokuseres der på fordelene og ulemperne ved bevægelsesmængden i gestikker.\blankline
%
Udover de foregående fem klassificeringer af gestikker, så definerer \textcite[s. 9]{PDF:ATaxonomyOfGestures} to former for inputs et elektronisk apparat kan modtage. Det ene input, som et elektronisk apparat kan modtage, forekommer enten ved fysisk kontakt med et elektronisk apparat eller et andet objekt, \parencite[s. 10]{PDF:ATaxonomyOfGestures}. Denne form for input lægger godt op til at interaktionen foregår via manipulerende gestikker, da de netop vedrører fysisk manipuleringen af et objekt. Ifølge \textcite[s. 12]{PDF:ATaxonomyOfGestures} er den anden form for input, som et elektronisk apparat kan modtage, baseret på gestikker, som apparatet så kan genkende og reagere på ved hjælp af lys-, lyd- eller bevægelsessensorer. Ved denne type input er det derfor muligt helt at undgå at skulle interagere med eller lokalisere endnu et elektronisk apparat eller iklæde sig en form for elektronik, som for eksempel en elektronisk handske, \parencite[s. 12]{PDF:ATaxonomyOfGestures}. Sidst nævnte inputform lægger derfor godt op til at interaktionen kan foregå via semaforiske gestikker, gestikulerende gestikker og/eller deiktiske gestikker. \blankline
%
Bevægelsesmængde indenfor gestik kommer til udtryk igennem mikro- og makrogestikker, \parencite[s. 6]{PDF:UsabilityofMicroVsMacroGestures}. Mikrogestikker er små bevægelser, der ikke nødvendigvis kræver, at hele hånden bevæger sig og da de kan udføres samtidig med andre manuelle opgaver, er de specielt lovende indenfor perifer interaktion, \parencite[s. 95]{PDF:PIMicrogesturesKap5}. Derudover er mikrogestikker hurtige at udføre, hvilket, ifølge \textcite[s. 96]{PDF:PIMicrogesturesKap5}, gør dem til en god kandidat for perifer interaktion. For eksempel så tillader mikrogestikker kommunikation, som en sekundær opgave, samtidig med at fokus holdes på en samtale, som er den primære opgave, uden at samtalepartneren finder en uhøflig, \parencite[s. 97]{PDF:PIMicrogesturesKap5}.

Makrogestikker er derimod store bevægelser, der kræver signifikant bevægelse, som eksempelvis at løfte en arm eller rejse sig fra en siddende position, \parencite[s. 6]{PDF:UsabilityofMicroVsMacroGestures}. Ifølge \textcite[s. 9]{PDF:UsabilityofMicroVsMacroGestures} kan makrogestikker udføres meget mindre præcist end mikrogestikker, da en løftet arm vil tolkes som en løftet arm, uanset om den er løftet 60$^{\circ}$ eller 90$^{\circ}$. En af ulemperne ved makrogestikker kan dog være, at der skal bruges et stort område til at udføre de pågældende gestikker og at en sensor, der opfanger makrogestikker ofte vil dække et helt lokale, \parencite[s. 9]{PDF:UsabilityofMicroVsMacroGestures}. Der kan derfor opstå et problem hvis en anden træder ind i interaktionsområdet og ved et uheld laver en bevægelse, som genkendes af systemet. Da det kan være svært at træde ud af interaktionsområdet, kræver det at brugeren rent fysisk flytter sig langt nok væk, så en ubevidst interaktion undgåes. Til gengæld er en af fordelene ved makrogestikker, at de specifikke gestikker er så tilpas forskellige, at sandsynligheden for at systemet forveksler dem er lille, \parencite[s. 9]{PDF:UsabilityofMicroVsMacroGestures}.  

Mikrogestikker behøver, modsat makrogestikker, ikke et stort interaktionsområde, men skal til gengæld udføres meget mere præcist, \parencite[s. 10]{PDF:UsabilityofMicroVsMacroGestures}. En af fordelene ved mikrogestikker er, at de tillader en større diversitet end makrogestikker, med andre ord; det er muligt at designe flere mikrogestikker end makrogestikker, \parencite[s. 10]{PDF:UsabilityofMicroVsMacroGestures}. En anden fordel ved mikrogestikker er, at interaktionen med systemet sjældent sker ved en fejl, da det er mindre sandsynligt at specifikke gestikker udføres ubevidst, \parencite[s. 10]{PDF:UsabilityofMicroVsMacroGestures}. Ulemperne ved at bruge mikrogestikker er dels, at desto større afstanden til systemet er, desto større er usikkerheden for, hvorvidt systemet kan registrerer gestikkerne, \parencite[s. 10]{PDF:UsabilityofMicroVsMacroGestures}. Derudover skal der ydermere kompenseres for situationer, hvor brugere kommer til at skygge for hele eller dele af gestikken, så systemet ikke længere er i stand til at registrer og genkende den specifikke gestik. En anden ulempe ved mikrogestikker er brugerens evne til at gengive dem korrekt og uden at forveksle dem. Ifølge \textcite[s. 10]{PDF:UsabilityofMicroVsMacroGestures} så egner mikrogestikker sig bedst til situationer, hvor der foretrækkes et mere professionelt udtryk, for eksempel på arbejdspladsen. Derudover egner mikrogestikker sig også til underholdnings apparater.\blankline
%
I undersøgelsen foretaget af \textcite[s. 823]{PDF:UnderstandingNaturalness} fremgår det, at det ikke er hensigtmæssigt at benytte sine egne kropsdele til at simulere værktøjer i gestikker, da det vil virke unaturligt. Med værktøjer refereres der til det værktøj, som ellers ville være blevet anvendt såfremt det havde været en naturlig situation. Baseret på resultaterne, fremsat i \textcite[s. 823]{PDF:UnderstandingNaturalness}, fremgår det at 77.5\% af gangene så opstår der fejl, når testpersonerne skal anvende en kropsdel som værktøj.

Derimod anbefaler \textcite[s. 823]{PDF:UnderstandingNaturalness} at gestikkerne udføres som pantomime, hvor brugere udfører den tiltænkte aktivitet og forestiller sig, at have værktøjet i hånden. Derudover anbefaler \textcite[s. 824]{PDF:UnderstandingNaturalness}, at hvis gestikkerne skal udføres i rummet, altså væk fra det elektroniske apparat, så bør gestikkerne udføres med begge hænder. Hvor den ikke-dominante hånd skal danne en form for ramme eller udgangspunkt for den dominante hånd, som så udfører gestikken.      
%
