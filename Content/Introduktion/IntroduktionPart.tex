\newcommand{\Intorduktionpartname}{Introduktion}
\newcommand{\Introduktionparttext}{Volumenkontrol er en essentiel funktionalitet for ethvert lydsystem, men trods værende en veletableret teknologi, har selv de bedste lydsystemer, som er projektgruppen bekendt, en tendens til at mangle bas ved lav lydstyrke og have for meget ved høj lydstyrke. Fænomenet er mest udpræget ved musikafspilning ved et lavt lydtryksniveau hvor musik kan lyde fladt, men det kan også opleves når der eksempelvis ses film ved et højt lydtryksniveau, i hvilket tilfælde bassen til tider overdøver alt andet. Denne problematik skyldes menneskets frekvensafhængige opfattelse af lyd, kaldt \textit{Loudness}. Uanset årsag er det et problem som kræver en løsning. Film og musik er indspillet til at lyde godt ved et fast lydtryksniveau og hvis denne ændres, perciperes det anderledes. Formålet med dette projekt er at lave en analog løsning som kompenserer for menneskets frekvensafhængige hørelse ved at justere lydbilledet afhængigt at lydtryksniveauet, særligt ved lave frekvenser.
%
\newpage 
}
%
\part[\Intorduktionpartname]{\Intorduktionpartname
\label{\Intorduktionpartname}
\vspace{8mm}
	\begin{center}
		\begin{minipage}[l]{14cm}
			\textnormal{\normalsize\noindent\Introduktionparttext}
		\end{minipage}
	\end{center}
}