\section{Fremgangsmåde}
\label{FremgangsmaadeSocialAccept}
%
Da formålet, foruden at undersøgen af den sociale accept, er at undersøge, hvorvidt testpersonerne er i stand til at interagere med et musikanlægget sideløbende med en primær opgave, er der nødvendigt at definere den primære opgave. Den primære opgave for værten er at føre en samtale med gæsten og for at sørge for, at samtalen ikke ophører under testen, så instrueres gæsten i hvad samtalen skal omhandle. Imens værten gennemgår en familiarisering, som uddybes i \fullref{FamiliariseringSocialAccept}, bliver gæsten introduceret til samtaleemnet; ferieplanlægning, hvor gæsten får til opgave at planlægge en to ugers ferie i tre forskellige prisklasser; budget, standard og luksus, med henholdsvis 10.000 kr, 20.000 kr og +30.000 kr til rådighed. Det forventes ikke, at gæsten når at færdiggøre planlægningen af de tre ferie forslag inden værtens familiarisering er overstået, da det er hensigten at de to testpersoner i fællesskab skal planlægge videre. Ydermere forventes det, at opgaven er så tidskrævende, at de to testpersonerne i fællesskab heller ikke når at planlægge ferierne færdig. Ferieplanlægningen skal dække over, hvor testpersonernes skal hen, hvordan de kommer derhen, hvor de skal overnatte, hvad de skal opleve samt et madbudget afhængigt af rådighedsbeløbet. For at afgøre hvorvidt værten er i stand til både at planlægge ferie med gæsten og samtidig interagere med et musikanlæg, vil der foretages separate exit-interviews. Spørgsmålene til de to separate exit-interviews forefindes i \autoref{app:ExitInterviewSocialAccept}. På den måde er det muligt at få både værten og gæstens vurdering af om værten var i stand til at løse begge opgaver, uden at testpersonernes individuelle vurdering påvirker hinanden. \blankline
%
Nu hvor den primære opgave er specificeret, rettes fokus til hvordan værten skal interagere med musikanlægget. Da det antages, at det vil være for krævende for værten selv at skulle vurdere, hvornår musikken skal pauses, startes, hvornår der skal skiftes musiknummer og hvornår der skal skrues op eller ned, besluttes det, at værten skal reagere på nogle hints, som skal fremprovokere en bestemt gestik. Der udvælges derfor fem specifikke hints til pause, start, skift musiknummer frem, skru op og skru ned. At der ikke vælges et hint til at skifte til det forrige musiknummer skyldes dels at det er den samme bevægelse; frem og tilbage og dels fordi det vurderes, at det ikke kan lade sig gøre at fremprovokere værten til naturligt at skifte til det forrige musiknummer. Værten bliver præsenteret for de fem forskellige hints under familiariseringen og får forklaret, hvordan der skal reageres på dem. For at få værten til at sætte musikken på pause vælges det, at testleder 2 ringer til en mobiltelefon, som placeres på et sofabord inde i lokalet. Når der ringes skal værten pause musikken og tage mobiltelefonen op til øret, som hvis det var en virkelig kontekst. Testleder 2 har ved opringningen mulighed for, at give værten yderligere instruktioner eller påmindelser, hvis der er behov for det, samt give rosende feedback på værtens gestikker og efterfølgende bede værten om at lægge på igen, hvorefter værten har lært, at musikken skal startes igen. Hvis værten ikke reagerer på opringningen, så ringer testleder 2 igen, i tilfælde af at værten ikke reagerer på anden opringning noteres det som en fejl. Det tager omkring 15 sekunder for en opringning at gå til telefonsvaren, så med to opringninger har værten omkring 30 sekunder til at reagere på et hint, hvilket gør sig gældende for samtlige hints. 

For at få værten til at skifte til det næste musiknummer, vælges det at forvrænge en del af musiknummeret. For at forvrænge musiknummeret anvendes programmet \textit{Audacity 2.1.3}, hvori musiknummeret importeres og det vælges hvilken del, der skal forvrænges. Forvrængningen påføres i to forskellige niveauer, hvor første niveau af forvrængningen er af typen \textit{medium overdrive}, hvorefter andet niveau af forvrængningen er \textit{hård klipning}. Det tilstræbes, at de to niveauer varer omkring 15 sekunder og såfremt værten ikke reagerer inden forvrængningen ophører, så noteres det som en fejl. I \autoref{app:ForvraengningHint} forefindes et eksempel på hvordan forvrængningen er, musiknummeret indgår i den afspilningsliste, som både vært og gæst lytter til når de planlægger ferie. 

Hintet for at skrue op for musikken er, at testleder 2 skruer ned for musikken, modsat hvis værten skal skrue ned for musikken så skruer testleder 2 op for musikken. I tilfælde af at værten ikke reagerer inden 15 sekunder vil hintet blive yderligere forstærket enten ved at skrue længere ned eller højere op for musikken. Igen noteres det som en fejl, hvis værten ikke formår at reagere inden 30 sekunder. Hvordan de forskellige hints præsenteres uddybes i \fullref{PraesentationsraekkefoelgeSocialAccept}.\blankline
% 
Testen vil forløbe på følgende måde: Først byder testlederne testpersonerne velkommen, hvorefter én af de to testledere spørger de to testpersoner, om der er en af dem, der er venstre håndet. I tilfældet af at en af testpersonerne er venstrehåndet, vil denne testperson agere gæst og den anden vil agere vært. Hvis begge testpersoner er højre håndet vil det tilstræbes, at fordele testpersonerne så begge køn er repræsenteret både som vært og som gæst. Derefter følger testleder 1 testpersonen, som skal agere vært ind i et lokale, hvor testleder 2 og testpersonen, som skal agere gæst bliver ude foran lokalet. Testleder 1 har til ansvar, at familiarisere testpersonen i forhold til hvilke semaforiske gestikker, der knyttes de valgte funktioner samt hvilke hints testpersonen skal reagere på. Begge testledere sørger for at instruerer testpersonerne i deres respektive opgaver samt udlevere samtykkeerklæringer, som testpersonerne opfordres til at læse og underskrive. Samtykkeerklæringen fremgår af \autoref{app:SamtykkeerklaeringSocialAccept}, instruktionerne til testleder 1 fremgår af \autoref{app:InstruktionerVaert} og instruktionerne til testleder 2 fremgår af \autoref{app:InstruktionerGaest}. Når værten har underskrevet samtykkeerklæringen starter testleder 1 videooptagelsen. Så snart gæsten har forstået opgaven, banker testleder 2 på døren ind til testleder 1 og værten, og går derefter ind i kontrolrummet. At testleder 2 banker på døren indikerer overfor testleder 1, at testlederen går ind i kontrolrummet og dermed er klar til at reagere på gestikker og styre musikken derefter. Når værten er familiariseret følger testleder 1 gæsten ind i samme lokale og opsummerer opgaven, hvorefter testlederen forlader lokalet og går ind i kontrolrummet til testleder 2.
 
Der afsættes omkring fire minutter til, at de to testpersoner falder ind i rollerne og der vil derfor ikke præsenteres nogle hints i disse minutter. Derefter vil de forskellige hints blive præsenteret. Kort tid efter det sidste hint er præsenteret banker testleder 1 på døren og afbryder ferieplanlægning. Gæsten bliver bedt om at gå ud til testleder 2, hvor gæstens exit-interview afvikles samtidig med at værten og testleder 1 afvikler værtens exit-interview inde i lokalet.\blankline
%
Testens varighed afhænger dels af om testpersonen, som agerer vært, har brug for en længere familiarisering end planlagt og dels af testpersonernes respons i exit-interviewet, men testen anslåes til at vare mellem 35 minutter og 45 minutter. 
%
\subsection{Præsentationsrækkefølge for hints}
\label{PraesentationsraekkefoelgeSocialAccept}
%
Præsentationsrækkefølgen følger tilnærmelsesvist principperne bag et \textit{Counterbalanced design} hvor alle rækkefølge kombinationer er repræsenteret. At det kun tilnærmelsesvist følger et \textit{Counterbalanced design} skyldes dels at hintet til at pause og starte musikken angives som et samlet hint, dels at der kun er et hint til at skifte til det næste musiknummer og dels at der er et hint til at skrue op og et hint til at skrue ned. For at præsentere de forskellige hints skal der i alt anvendes syv musiknumre.
%
\begin{table}[H]
	\centering
	\begin{tabular}{ | p{2.8cm}| p{1.85cm}| p{1.85cm} | p{1.85cm} | p{1.85cm} | p{1.85cm}|}
		\hline
		Musiknummer 1 & Opringning & Forvræng & & & \\ \hline
		Musiknummer 2 & Skru ned & Skru op & Opringning & Forvræng & \\ \hline
		Musiknummer 3 & Forvræng & & & &  \\ \hline
		Musiknummer 4 & Skru ned & Opringning & Opringning & Skru op & Forvræng\\ \hline
		Musiknummer 5 & Skru ned & Forvræng & & & \\ \hline
		Musiknummer 6 & Opringning & Forvræng & & & \\ \hline
		Musiknummer 7 & Opringning & Skru op & & & \\ \hline
	\end{tabular}
	\caption{Oversigt over præsentationsrækkefølgen for de forskellige hints og hvordan de indgår i musiknumrene.}
	\label{tab:PraesentationsraekkefoelgeHints}
\end{table}
\noindent
%
I \autoref{tab:PraesentationsraekkefoelgeHints} fremgår præsentationsrækkefølgen for de forskellige hints. Denne rækkefølge anvendes både til familiariseringen af værten og når værten får besøg af gæsten, dog med forskellige musiknumre. At antallet af hints per musiknummer ikke er ligeligt fordelt skyldes, at så snart forvrængningen kommer så skal værten skifte til det næste musiknummer. Varigheden mellem hvert hint varierer mellem 10 sekunder til 40 sekunder, men afhænger også af hvor hurtigt værten reagerer på hintet.   

