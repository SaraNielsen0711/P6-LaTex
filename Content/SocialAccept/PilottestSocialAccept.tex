\section{Pilottest}
\label{PilottestSocialAccept}
%
Formålet med pilottesten er at undersøge, hvorvidt testdesignet fungerer og om værten kan interagere med musikanlægget, mens der planlægges ferie med gæsten. Derudover udføres pilottesten, for at begge testledere får styr på deres opgaver og for at undersøge om den afsatte tid stemmeroverens med hvor lang tid testen reelt tager. \blankline
%
Til pilottesten testes der på seks studerende, fordelt på tre par, der ellers ikke lever op til kriterierne for at medvirke i undersøgelsen, jævnfør \fullref{TestpersonerSocialAccept}. De kriterier som testpersonerne levede op til er, at de kender hinanden, de har ikke deltaget i den foregående undersøgelse og mindst én i hvert par er højrehåndet. Årsagen til at disse testpersoner blev testet er, at de var tilgængelige og at formålet med pilottesten, som nævnt, er at teste om testdesignet fungerer. Resultaterne fra pilottesten fremgår af \autoref{app:ResultaterPilottestSocialAccept}. 

Baseret på resultaterne fra pilottesten bekræftes det, at testdesignet fungerer og at værten er i stand til at interagere med musikanlægget via semaforiske gestikker mens samtalen med gæsten opretholdes. Derudover fik testlederne ud fra pilottestene præciseret deres instruktioner samt hvilke elementer, der skal lægges mere vægt på overfor både vært og gæst. Heriblandt skal testleder 1 understrege, at værten kun skal reagere på de hints, der bliver præsenteret og ikke andet, da det potentielt kan resultere i, at værterne ikke får lige mange hints. Ud fra pilottestens varighed blev den afsatte tid på mellem 35 minutter og 45 minutter til både familiarisering, selve testen og exit-interviews overholdt.

Ud fra de tre par, som deltog i pilottesten konkluderes det, at de forskellige dele af testdesignet fungerer efter hensigten og det er muligt at eksekvere den rigtige test. 

