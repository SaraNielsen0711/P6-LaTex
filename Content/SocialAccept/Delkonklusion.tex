\section{Delkonklusion}
\label{SocialAcceptDelkonklusion}
%
På baggrund af foregående analyse og diskussion kan der konkluderes på forskellige aspekter vedrørende interaktion med semaforiske gestikker. Først og fremmest fandt testpersonerne de udvalgte gestikker nemme at lære og de gav udtryk for, at gestikkerne passede godt til deres funktioner. Fælles for de seks gæster er, at de alle bemærkede værternes gestikker, men det var ikke alle, der var i stand til at gengive de udvalgte gestikker. Hvorvidt gæsterne formåede at gengive værtens gestikker fremgår af \autoref{fig:GengiverGestikker}, hvor gæsternes gengivelse af gestikkerne til henholdvis pause, start, skift musiknummer samt skru op og ned blev inddelt i tre kategorier; korrekt, delvist korrekt og ikke gengivet. Ud fra dette konkluderes det dels at gestikkerne kan udføres diskret, de er nemme at lære og de er nemme at relatere til den specifikke funktion, som de tilhører. 

Den eneste gestik, der ud fra testpersonernes holdning kan være problemer med, er volume-justeringen, da testpersonerne ikke altid følte de havde helt styr på hvor meget de skruede op eller ned. 

Af restultaterne kan det ses, at ingen af testpersonerne gestikulere således, at deres samtale-gestikker kan misforstås som kommandoer til musikanlægget. Dette kan dog skyldes at samtaleemnet er ferieplanlægning, hvor interaktionen foregår meget over computeren og med mindre direkte kontakt end det kunne forestilles at have i en almindelig samtale.

Af resultaterne ses det, at der laves få fejl, og at fejlene kun laves i forbindelse med at skifte sang, hvor værterne ved enkelte tilfælde bruger en enkelt finger eller hele hånden til at swipe. Ydermere kan det ses, at nogle af værternes bevægelser bliver mere dovne, jo bedre kendskab de får til gestikken. Når to af værterne spiser cookies med højre hånd, skifter de sang med venstre hånd, hvilket de to værter gør modsat af hinanden. Dette lægger op til diskussion om, hvordan interaktionen egentlig skal foregå, når der bruges venstre hånd.

Når værtens fokus vendes tydeligt mod musikanlægget, kan det have en indflydelse på hvor meget gæsten forstyrres og hvordan samtalen forløbet. Dog virker det ikke til at interaktionen påvirker samtalen negativt og der opstod ingen akavede pauser mellem vært og gæst, når hints eller interaktion afbrød samtalen midlertidigt. 

Ud fra resultaterne kan der konkluderes, at der var positivt at have telefonopkald med som en del af testen, da værterne på den måde kan få feedback på om de gør det rigtigt og hvilke forbedringer de eventuelt kan lave.\blankline
%
Ud fra testpersonernes egen respons, synes ingen af dem at interaktionen med musikanlægget var forstyrrende og alle synes at begge opgaver kunne udføres sideløbende. To testpersonerne nævner endda, at det er mindre forstyrrende at interagere på denne måde end ved brug af sin mobiltelefon. Selvom det ikke er muligt at påvise en perifer interaktion ud fra denne test, er det vist, at interaktionen kan foregå delvist sideløbende og med korte afbrydelser fra hovedopgaven, hvilket kan indikere at de semaforiske gestikker ved gentagende brug højst sandsynligt kan benyttes til perifer interaktion.\blankline
%
Ud fra resultaterne kan det fastslås, at gestikkerne højst sandsynligt er socialt acceptable, da ingen af testpersonerne vil have noget problem med at udføre dem eller se andre udføre dem i forskellige forsamlinger. Flere af testpersonerne nævner dog, at der kan være et problem ved større forsamlinger, hvor systemet kan reagere på flere personers gestikulationer. Hertil foreslår nogle af testpersonerne, at det kunne være smart at kunne slå interaktionsformen fra midlertidigt, når man havde gæster, som ikke kunne håndtere det. 

Sidst tyder testpersonernes respons på, at de gerne vil bruge interaktion med gestikker til at styre et musikanlæg, så længe det er under de rigtig omstændigheder.\blankline
%
Ud fra observationer samt testpersonernes respons fastslås det, at interaktion med et musikanlæg sagtens kan foregå ved brug af semfaoriske gestikker.