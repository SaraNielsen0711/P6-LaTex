%\fxwrite{Synopsis}
Projektet har til formål at designe semaforiske gestikker til de mest gængse funktioner i et musikanlæg. Det tilstræbes, at disse semaforiske gestikker kan benyttes, som interaktionsform i den perifere opmærksomhed samtidig med, at interaktionsformen er social acceptabel. 

En kvalitativ undersøgelse udføres på 18 testpersoner, hvorfra det fastlægges, hvilke semaforiske gestikker, der skal knyttes til de valgte funktioner; et krokodillenæb, der lukker og åbner for henholdsvis at pause og starte musikken, en swipe-bevægelse med to fingre, for at skifte musiknummer og et greb om en fiktiv drejeknap, for at justere lydstyrken. Efterfølgende udføres en kvalitativ undersøgelse af, hvorvidt gestikkerne egner sig til interaktion med et musikanlæg i en social kontekst, hvortil der deltog seks par bestående af en vært og en gæst. Af resultaterne fra undersøgelsen i den sociale kontekst, konkluderes det, at de valgte gestikker egner sig til interaktion med et musikanlæg sideløbende med en primær opgave; samtale med en gæst, hvilket indikerer, at interaktion med gestikkerne ved gentaget brug har potentiale for at foregå i den perifere opmærksomhed. Ydermere konkluderes det, at gestikkerne er socialt acceptable, da de ikke påvirker samtalen negativt. 