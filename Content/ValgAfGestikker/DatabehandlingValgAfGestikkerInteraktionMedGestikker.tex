\section{Interaktion med gestikker}
\label{TestresultaterInteraktioner}
%
I starten af testen blev testpersonerne bedt om at udfylde et spørgeskema, hvori der, blandt andet, blev stillet spørgsmål til deres erfaring med semaforiske gestikker. Hertil responderede 12 testpersoner, at de før har brugt gestikker, hvor de ikke skulle røre ved en skærm. De testpersoner, der responderede at de har brugt gestikker før, blev efterfølgende spurgt ind til med hvilke produkter, de havde brugt semaforiske gestikker. Af de 12 testpersoner, som har tidligere erfaringer med semaforiske gestikker, har tre brugt Xbox Kinect, ni har brugt Nintendo Wii, seks har brugt smartphones, fem har brugt tablets og to har brugt Virtual Reality briller. Tre testpersoner afkrydsede andet-kategorien og angav at de har brugt henholdsvis et Myo armbånd eller EyeToy, som er et PlayStation 2 spil. Ud af de 12 testpersoner følte én sig meget erfaren med semaforiske gestikker, fem følte sig erfarne, fem følte sig uerfarne og én følte sig meget uerfaren. Fra spørgeskemaet blev der ydermere indsamlet data om, hvordan testpersonerne betragter sig selv i forhold til hvor hurtigt de anskaffer sig ny teknologi. Hertil responderede ni, at de betragter sig som værende lige så hurtig som alle andre til at anskaffe sig ny teknologi og ni responderede at de betragter sig som værende de sidste til at anskaffe sig ny teknologi. Information som denne analyseres efterfølgende i forbindelse med testpersonernes udtalelser om hvad de synes om, at interagere med en brugergrænseflade ved brug af semaforiske gestikker. \blankline
%
Efter at have udvalgt hvilke semaforiske gestikker, der skal knyttes til henholdvis pause og start, skift musiknummer samt skru op og ned, blev testpersonerne, i exit-interviewet, spurgt indtil hvordan de forholder sig til denne interaktionsform. 14 testpersoner giver udtryk for, at de godt kan lide idéen om at interagere via semaforiske gestikker, hvoraf nogle kommenterede, at det selvfølgelig er med forbehold for, at det virker hver gang og at systemet reagerer på ens bevægelser. Ud af de 14 testpersoner giver ni testpersoner udtryk for, at de godt forestille sig at bruge denne slags interaktion. 


Testperson 3 og testperson 17 giver udtryk for, at de ikke vil bruge semaforiske gestikker, hvortil testperson 17 pointerer at det på nuværende tidspunkt ikke er aktuelt, da det er lige så nemt at trykke på en telefon. Dog afviser testpersonen ikke at det på sigt kan blive aktuelt. Testperson 3 føler sig lidt for gammeldags til at bruge den slags interaktion, men kan godt se det smarte i ikke at skulle rejse sig eller lede efter fjernbetjeningen. Begge testpersoner betragter sig selv som værende de sidste til at anskaffe sig ny teknologi og testperson 17, der har prøvet at interagere med semaforiske gestikker før, føler sig meget uerfaren. Testperson 15 giver udtryk for ikke at have et behov for at interaktionen skal foregå på det niveau, hvor niveau referer til brugen af semaforiske gestikker. I tillæg kommenterer testperson 15, at det ikke er et produkt testpersonen vil købe. Testperson 15's synspunkter stemmer overens med hvad testpersonen responderer i spørgeskemaet, hvor testepersonen betragter sig som værende den sidste til at anskaffe sig ny teknologi og føler sig derudover uerfaren i brugen af semaforiske gestikker. 
 
 
 Testperson 8, der føler sig erfaren i brug af semaforiske gestikker og betragter sig som den sidste til at anskaffe sig ny teknologi, giver også udtryk for, at vedkommende i hvert fald lige skulle vende sig til at bruge gestikker til interaktion.

Resultaterne viser altså at de testpersoner, der er skeptiske overfor interaktionsformen, også er personer, der ikke anskaffer sig den nyeste teknologi hurtigst. Dog viser resultaterne også at fem testpersoner synes det vil være fint at interagere med et musikanlæg ved brug af semaforiske gestikker, selvom de selv betragter sig som værende langsomme til at anskaffe sig ny teknologi. Af de ni testpersoner, der godt kunne forestille sig at bruge interaktionsformen, anser tre af dem sig som værende de sidste til at anskaffe sig ny teknologi. Derudover har otte af dem prøvet at interagere ved brug af gestikker før, hvoraf én føler sig meget erfaren, fire føler sig erfarne og tre føler sig uerfaren. Det kan ud fra resultaterne ses, at selvom ingen af de 18 testpersoner betragter sig som værende de hurtigste til at anskaffe sig ny teknologi, er testpersonerne positivt stillet overfor en interaktionsform som denne, der alligevel ikke er så udbredt. Derudover er der ikke nødvendigvis en hårdfin grænse for, om testpersonerne betragter sig som værende langsomme eller lige så hurtige som alle andre til at anskaffe sig ny teknologi og om de ønsker at eje eller interagere med et produkt ved brug af semaforiske gestikker. Det kan altså betyde, at der er en købelyst til netop sådan et produkt, om man er interesseret i den nyeste teknologi eller ej. De testpersoner, der ikke kan se sig selv interagere med et musikanlæg ved brug af semaforiske gestikker, kan alligevel godt se fordelen i ikke at skulle lede efter sin fjernbetjening eller røre ved en skærm med fedtede fingre. I alt er der ni testpersoner, der selv nævner fordelen ved enten ikke at skulle rejse sig, lede efter fjernbetjeningen, røre ved en skærm med fedtede fingre eller sidde og kigge på sin telefon, hver gang man vil interagere med musikanlægget. \blankline
 
Testpersonerne udtrykker, udover en positiv indstilling, lidt bekymringer, der skal tages med i overvejelserne, når semaforiske gestikulering skal udvikles som interaktionsform. Testperson 1 nævner, at der ikke skal være for mange funktioner og i samme stil nævner testperson 17 og testperson 18, at det ville være en fordel yderligere at kunne interagere med musikanlæggets skærm ved at trykke på den. Da det fra starten af har været målet kun at udvælge semaforiske gestikker til de mest gængse funktioner på et musikanlæg, vil det give mening at have en tilknyttet applikation, hvori de mest gængse funktioner samt de resterende funktioner vil befinde sig. Testperson 2 kommer med en vigtig pointe, da vedkommende selv udtrykker: \textsl{"Hvis det fungerer lige så godt som at trykke på en knap, så ja, så kunne jeg godt finde på at bruge det."} Det er netop vigtigt, at det fungerer lige så godt eller bedre at lave en semaforisk gestik, hvorefter musikken ændrer sig, for ellers kan brugeren lige så godt bruge sin telefon eller fjernbetjening til at styre musikanlægget med, da de ved dette plejer at virke. Hvis ikke interaktion med de semaforiske gestikker fungerer lige så godt eller bedre end alternativet, så vil brugeren højst sandsynligt holde op med at bruge produktet og i værste fald fraråde andre at købe det.

Med udgangspunkt i at interaktion ved brug af semaforiske gestikker til at styre et musikanlæg vil falde i god jord er det relevant at undersøge, hvordan testpersonerne egentlig har det med altid at blive opfanget af musikanlægget.

