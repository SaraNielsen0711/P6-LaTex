\chapter{Konklusion}
\label{Konklusion}
%
Det kan på baggrund af foregående analyse og diskussion konkluderes, at semaforiske gestikker fungerer til at styre de mest gængse funktioner på et musikanlæg. Herunder egner et krokodillenæb der lukker sammen og åbner op sig godt til henholdsvis at pause og starte musikken. Ydermere konkluderes det, at en swipe bevægelse med pege-langefinger egner sig godt til at skifte musiknummer, mens det, at tage fat i en fiktiv drejeknap og dreje enten med eller mod uret egner sig godt til at justere lydstyrken. Ud fra foregående undersøgelser konkluderes det, at interaktionen med musikanlægget kan foregå sideløbende med en primær opgave, hvorfor disse semaforiske gestikker ved gentagende brug danner grundlag for en potentiel perifer interaktion. 

Ved undersøgelse af interaktionsformen en social kontekst påvises det, at brugen af semaforiske gestikker ikke har negativ effekt på en samtale mellem to personer. Ydermere kan det ud fra testpersonernes respons konkluderes, at de valgte gestikker er socialt acceptable. 

Positiv indstilling til semaforiske gestikker

Det skal være muligt at deaktivere 

gestik-funktionen (grund overvågnings-problemer, mange mennesker i huset osv) 

Såfremt at det fungerer optimalt og er nemmere end at trykke på en knap

De er nemme at lære – egner sig godt (måske er det sagt)

Problemer ved lydjustering med GP2 – kan man lave noget aktivering her 

I stand til at udføre begge opgaver
