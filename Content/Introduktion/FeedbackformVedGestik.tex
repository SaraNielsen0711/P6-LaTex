\section{Feedbackformer}
\label{Feedbackformer}
%
Der er en del udbredte meninger om, hvorvidt der skal være feedback i alle interaktionssituationer samt hvilken type feedback, der skal anvendes. Diskussionen omkring feedbackformer vil primært omhandle semaforiske og manipulerende gestikker i forbindelse med at interagere med et musikanlæg, hvilket er oplagt fordi projektet er et samarbejde med Bang $\&$ Olufsen, \fullref{SamspilMedBO}. \blankline
%
\textcite[s. 10]{PDF:NaturalUserInterfaces} argumenterer for, at der skal være feedback, for på den måde at hjælpe brugeren til at forstå årsagen til eventuelle fejl og dertil lære den korrekte adfærd. Hvorimod \textcite[s. 16]{PDF:PIEmbeddingHCIOnTheRelevance} argumenterer for, at hvis gestikkerne er semaforiske, så er det ikke nødvendigt med, hvad \textcite[s. 16]{PDF:PIEmbeddingHCIOnTheRelevance} kategorisere som værende augmenteret feedback, da brugeren automatisk får feedback fra den proprioceptive sans, når der sker en aktivering af musklerne. På baggrund af det argumenterer \textcite[s. 16]{PDF:PIEmbeddingHCIOnTheRelevance} ydermere for, at det er derfor, at semaforiske gestikker egner sig til perifer interaktion. I henhold til semaforiske gestikker, så gav testpersonerne, i undersøgelsen foretaget af \textcite[ss. 172-173]{PDF:ComparingInputModalities}, udtryk for, at de manglede haptisk feedback. Årsagen til det skyldes formegentligt, at testpersonerne først og fremmest skal vænne sig til og stole på de semaforiske gestikker, hvorefter der formegentlig ikke længere er behov for haptisk feedback, \parencite[s. 174]{PDF:ComparingInputModalities}. 

Derudover argumenterer \textcite[s. 3]{PDF:FacilitatingPIDesignAndEvaluation} for, at det er problematisk at give brugeren feedback igennem den samme sensoriske modalitet, hvor opgaven befinder sig, da der kan opstå interferens. Derfor er det uhensigtmæssigt og mindre brugbart at anvende auditiv feedback, når brugeren samtidig lytter til musik, \parencite[s. 3]{PDF:FacilitatingPIDesignAndEvaluation}. I tilfælde hvor brugeren perifert skal interagere med et musikanlæg, kommenterer \textcite[s. 19]{PDF:PIEmbeddingHCIOnTheRelevance}, at brugeren i forvejen modtager feedback i form af funktionel feedback, hvor brugeren kan høre at musikken pauses, at der justeres på lydstyrken eller at der skiftes musiknummer. \textcite[s. 3]{PDF:InteractionFrogger} definerer funktionel feedback, som værende den information, som generes når en funktion initieres, svarende til at hvis brugeren gengiver gestikken, som pauser musikken så vil den funktionelle feedback være, at musikken stopper med at spille. 

Hvis perifere interaktion foregår igennem manipulerende gestikker, så har ikke-visuel feedback potentiale for at minimere belastningen af de mentale ressourcer, \parencite[s. 3]{PDF:FacilitatingPIDesignAndEvaluation}. Dog giver testpersonerne, i undersøgelsen foretaget af \textcite[s. 173]{PDF:ComparingInputModalities}, udtryk for, at de ikke manglede yderligere feedback end det de fik fra den funktionelle feedback. Hvor den funktionelle feedback i dette tilfælde vedrører den proprioceptive sans, fordi testpersonerne manipulerer et fysisk objekt, et knop-baseret håndtag og touchskærmen, og de kan høre at systemet reagerer på inputtet.

Udover funktionel feedback definerer \textcite[s. 3]{PDF:InteractionFrogger}, yderligere to feedbackformer; augmenteret og iboende feedback, hvor førstnævnte gengiver at feedbacken kommer fra en ekstern kilde, modsat iboende feedback, som gengiver at feedbacken kommer direkte fra, hvor interaktionen foregår. \blankline
%
\textcite[ss. 1263-1268]{PDF:ComparingModFeedback} undersøger om forskellige feedback typer, fra visuel feedback i den perifere opmærksomhed til feedback direkte på skærmen, har en effekt på testpersonernes præstationsevne. Ifølge \textcite[ss. 1267-1268]{PDF:ComparingModFeedback} har feedback ingen effekt på præstationsevnen i den sekundære opgave, men testpersonerne efterspørger alligevel feedback, så de kan få information omkring deres handlinger. Lignende resultater forefindes i en undersøgelse foretaget af \textcite[s. 8]{PDF:DoThatThere}, som ligeledes ikke finder at feedback har en effekt på præstationsevnen. Dog tyder det på, at feedback er med til at gøre det nemmere for testpersonerne at interagere med gestikker, \parencite[s. 8]{PDF:DoThatThere}. Ligesom \textcite[s. 174]{PDF:ComparingInputModalities} argumenterer \textcite[s. 1268]{PDF:ComparingModFeedback} for, at det ikke nødvendigvis vil være tilfældet i en feltundersøgelse da testpersonerne får længere tid til at vænne sig til den perifere interaktion. 

Det lader derfor til at testpersoner, som testes i et laboratorium, har andre behov, end hvis de blev testet i en feltundersøgelse. Det kan derfor være svært at afgøre, hvorvidt der skal være en form for feedback, når en bruger interagerer med et musikanlæg i den perifere opmærksomhed. Dette gør sig særligt gældende, når interaktionsformen enten bygger på semaforiske og/eller manipulerende gestikker. \blankline 
%
For at undgå nogle af de komplikationer, der blev adresseret i \fullref{KomplikationerGestikker}, foreslår \textcite{PDF:DoThatThere}, at det ved hjælp af feedback er muligt at inddele interaktionen i to; \textit{Do that} og \textit{There}. Formålet med \textit{Do that} er, at hjælpe brugeren til at udføre de korrekte handlinger, hvortil \textcite[s. 4]{PDF:DoThatThere} undersøger, hvad de definerer som rytmiske gestikker. Rytmiske gestikker vedrører at få brugeren til at udføre bevægelsen i takt med at systemet eksempelvis lyser i et bestemt mønster, som brugeren skal efterligne. Brugeren får dermed bekræftet, at en specifik gestik er registreret og ligeledes undgår brugeren, at systemet reagerer på gestikker, som ikke er henvendt til systemet, \parencite[s. 4]{PDF:DoThatThere}. \textcite[s. 10]{PDF:DoThatThere} anbefaler, at brugeren får feedback fra de rytmiske gestikker fra det tidspunkt interaktionen begynder. Ydermere anbefaler \textcite[s. 10]{PDF:DoThatThere}, at så snart det er muligt så skal de rytmiske gestikker gengives ved enten horisontale eller vertikale bevægelser. 

Formålet med \textit{There} er, ved hjælp af feedback, at vejlede brugeren til, hvor gestikken skal udføres, for at den kan registreres. \textcite[s. 10]{PDF:DoThatThere} anbefaler, at vejlede brugeren til det korrekte sted for at udføre gestikken, ved hjælp af lys. 
