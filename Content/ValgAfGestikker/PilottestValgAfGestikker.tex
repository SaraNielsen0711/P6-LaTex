\subsection{Pilottest til valg af gestikker}
\label{PilottestValgAfGestikker}
%
Før enhver test kan det være en fordel at køre en pilottest, for på den måde at finde og eliminere fejl i testdesignet. Formålet med denne pilottest er netop at undersøge, hvorvidt de forskellige aspekter i testen fungerer og om der opstår nogle komplikationer, eksempelvist ved at optage testpersonerne. Baseret på pilottesten vil det være muligt at foretage ændringer såfremt der opstår komplikationer eller andre problemer.\blankline
%  
Til pilottesten blev der testet på fire studerende, der ellers ikke lever op til kriterierne for at være testperson, jævnfør \fullref{TestpersonerValgAfGestikker}. Årsagen til at disse testpersoner blev testet er både at de var tilgængelige og at formålet med pilottesten var at teste de forskellige aspekter eksempelvis videoerne, spørgeskemaet, instruktionerne og testlederen samt observatørens position. Resultaterne fra pilottesten fremgår af det \textbf{elektroniske bilag}. Baseret på resultaterne fra pilottesten bekræftes antagelsen om at studerende med en forhåndsviden om design, brugertest og lignende samt studerende hvis hovedfokus er på elektronik og software, ikke er i stand til at give upåvirket respons. To af testpersonerne angav hvilke gestikker de bedst og mindst kunne lide på baggrund af deres teknologiske viden og ikke nødvendigvis ud fra hvad de faktisk ville foretrække.\blankline
% 
Under pilottesten blev der optaget med et Canon Powershot s110, som viste sig ikke at kunne holde strøm tilstrækkeligt lang tid til. Optagelserne fra pilottesten blev derfor enten afbrudt for tidligt eller slet ikke optaget, da kameraet løb tør for strøm, hvorfor optagelserne ikke fremgår i noget bilag. Det vælges derfor at optagelserne fremadrettet vil foregå med et GoPro Hero4 Silver kamera.\blankline
% 
Når testpersonerne opfordres til at gengive deres forbedringer til musik både ved de individuelle videoer og afslutningsvist, så blev musikken afspillet igennem iTunes på den computer, hvor observatøren tager noter. Heri opstod der et problem da det ikke var alle musiknumre i iTunes, som startede med det samme hvilket forårsagede forvirring for testpersonerne, som ikke vidste om de havde gjort noget forkert eller ej. Det blev derfor valgt at opsætte en afspilningsliste med musiknumre, som alle starter med det samme så der fremover ikke opstår tvivl om, hvorvidt der bliver skift musiknummer eller ej. 

Da testpersonerne blev opfordret til at gengive forbedringerne rettede de sig mod observatøren, formentligt fordi testpersonerne kunne se at det var observatøren, som styrede musikken. Observatørens position ændres derfor så testpersonen ikke længere har direkte mulighed for at se observatøren tage noter og styre musikken. 

Ydermere blev spørgeskemaet redigeret to gange, så det blev klart for testpersonerne, at det var brugen af gestikker, hvor der ikke skal røres ved en skærm, der blev spurgt ind til. \blankline
%
Med disse forbedringer vurderes det at testens forskellige aspekter fungerer.  
 

