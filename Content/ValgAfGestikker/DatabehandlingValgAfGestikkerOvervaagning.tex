\section{Holdning til gestik genkendelse}
\label{TestresultaterOvervaagning}
%
Under testen blev de 18 testpersoner spurgt, hvordan de ville have det med at systemet altid kunne opfange deres gestikker. Hertil svarede 14 af testpersonerne, at de ville have det fint med det, nogle dog med flere forbehold end andre. De resterende fire testpersoner var ikke afvisende overfor idéen, men sagde ikke direkte, at de havde det fint med systemet gerne altid måtte kunne opfange dem. Nogle af de bekymringer testpersonerne udtrykte var blandt andet testperson 4, der var bange for at det ville blive grimt med kameraer rundt omkring i hjemmet. Testperson 3, 7 og 16 følte sig bedre tilpas, hvis deres gestikker ikke blev registreret af et kamera, men sensorer af anden art og testperson 7 udtrykte bekymring om hvis man kunne hacke sig ind i systemet og se om der var nogen hjemme. For at have det bedre med at systemet altid kan opfange vedkommende siger testperson 6 og 18 at det ville være fint at få noget information fra udbyderens side om hvordan informationen blev bearbejdedt og eventuelt lagret. 

Testpersonerne kom ydermere med idéer til hvordan og hvornår systemet skulle kunne opfange bevægelser, så der ikke opstod problemer. Testperson 12 siger \textsl{"Jeg synes det kunne være super fedt, hvis det virker ordentligt. Det skal helt sikkert ikke være sådan, at der lige pludselig er et eller andet der bliver tændt af at man gør et eller andet, men hvis det fungerer, så synes jeg ikke at der som sådan vil være noget negativt i det."}, hvilket indikerer at idéen om at interagere med et interface ved brug af gestikker kunne falde i god jord, så længe systemet fungerer optimalt. Et optimalt fungerende system vil i denne sammenhæng være er system, der ikke opfanger almindelige samtale gestikker som værende gestikker henvendt til musikanlægget, hvilket udover testperson 12 også bekymrer testperson 8, 9 og 14. For at undgå dette siger testperson 8, at det ville være en fordel, hvis gestikkerne kun blev registreret i en bestemt zone, så gestikker uden for denne zone ikke blev opfanget som komandoer. I forbindelse med både trygheden i at have et system, der altid kan opfange én og måske fejltolke gestikker nævner testperson 1, 6, 10 og 18 at man skal kunne slukke det, når man ikke bruger det eller selv fortælle det, hvornår det skal starte op og begynde at registrere ens gestikker. \blankline

Ud fra testpersonernes holdninger virker det ikke til at være et problem at have et slags kamera eller anden sensor indbygget i et musikanlæg, som på den måde kan opfange og tolke på brugerens gestikuleringer. Ved at have det indbygget på den måde vil der automatisk skabes en enkelt zone, hvori gestikker kan opfanges og tolkes på. Ydermere kan, alt efter de teknologiske begrænsninger, være nødvendigt at rette gestikken mod anlægget, hvorfor risikoen for at gestikkerne bliver fejltolket bliver mindre. For at komme problematikken i at systemet altid vil være tændt til livs kan opsættes krav om, at registreringen af gestikker først vil kunne ske efter musik på anlægget er sat i gang. På den måde vil brugeren skulle tænde for det ønskede musik, hvorefter vedkommende kan sætte sig til rette i sofaen og interagere med systemet efter behov. Når brugeren slukker for musikken igen vil registreringen af bevægelser ligeledes lukke ned, og brugeren vil kunne bevæge sig frit, uden bekymringer om at musik det pludselig tænder eller et anlæg der følger enhver bevægelse. Skulle det ske at brugeren pauser sin musik i længere tid uden at starte den igen, ville det være en mulighed at have en timer på, der efter en bestemt tid uden at blive startet igen lukker ned for registeringen af bevægelser.  
