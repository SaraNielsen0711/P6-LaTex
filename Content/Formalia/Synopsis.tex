%\fxwrite{Synopsis}
Dette 5. Semester projekt har til formål at udvikle en analogelektronisk løsning, som kompenserer for menneskets frekvensafhængige perception af lyd, som beskrevet i \textcite{STD:ISO226}, ved at justere lydoplevelsen afhængigt at lydtryksniveauet, særligt ved lave frekvenser. Dette opnås ved at udvikle fraktalordensfiltre af passende størrelse og skifte mellem dem, afhængigt af en stereovolumenkontrol. Både måleteknisk og efter gruppens egen lyttetest, fungerer produktet i store træk som tiltænkt. I særdeleshed ved lave lydtryksniveauer høres en tydelig forbedring af lydbilledet, hvor musik synes at lyde mere livligt med filteret tilkoblet. Gruppen er meget tilfreds med resultatet.