\section{At lære og at gengive gestikker}
\label{TestresultaterSocialAcceptGestikker}
%
Følgende analyse og diskusion foretages på baggrund af værternes respons til spørgsmålene: \textit{Hvordan syntes du det var at lære gestikkerne?}, \textit{Havde du problemer med at huske gestikkerne?}, \textit{Hvor rigtigt syntes du, at du fik gengivet gestikkerne til de enkelte funktioner?} og \textit{Var der nogle gestikker, som du følte dig mere ukomfortabel ved at udføre end andre?} samt gæsternes respons til spørgsmålene: \textit{Bemærkede du, at din ven(inde)/kæreste lavede gestikker?}, \textit{Kan du gengive de gestikker som din ven(inde)/kæreste lavede?} og \textit{Hvad synes du om dem?}.\blankline
%
I forhold til hvordan værterne responderede på spørgsmålet: \textit{Hvordan syntes du det var at lære gestikkerne?}, er den gennemgående tendens, at det var nemt. Vært 2, vært 3 og vært 6 giver udtryk for, at gestikkerne kan associeres med, hvordan der normalvist interageres på en telefon, i forhold til swipe-bevægelsen, samt hvordan der normalvist interageres med et musikanlæg. I det henseende pointerer vært 3 og vært 4, at det er de rigtige gestikker, der er valgt til hver funktion og ifølge vært 3 er det nærmest ikke muligt at vælge nogle gestikker, som passer bedre. Vært 5 påpeger, at det var en fordel, at gestikkerne præsenteres via video og ikke på papirform, da det gjorde gestikkerne mere levende. Generelt er der ingen af de seks værter, som giver udtryk for at have problemer med at huske de udvalgte gestikker. Dog kommenterer vært 1, at værten normalvist ikke pauser musikken, når telefonen ringer og derfor var værten, ifølge værten selv, nødsaget til at minde sig selv om det, når telefonen ringede. Vært 3 gav udtryk for at være mere opmærksom på forvrængningen, fordi det er blevet pointeret, af andre, at værten har problemer med hørelsen. Første gang vært 5 skiftede til det næste musiknummer, blev det gjort med hele hånden og da det gik op for værten, at det var forkert, blev det rettet. Det skal dog påpeges, at fejlen opstod under familiariseringen, hvorefter værten fejlfrit gengav den korrekte gestik for at skifte til det næste musiknummer. Vært 2 påpeger, at fordi gestikkerne ligger i håndleddet, er der ingen problemer med at huske dem. 

Til spørgsmålet: \textit{Hvor rigtigt syntes du, at du fik gengivet gestikkerne til de enkelte funktioner?} giver de seks værter udtryk for, at gestikkerne blev gengivet korrekt. Vært 3 påpeger, at værten kom til at bruge venstre hånd istedet for højre, hvilket ud fra optagelserne skyldes, at værten spiser cookies med højre hånd. Derudover kommenterer vært 3, at værten glemte at starte musikken igen efter den sidste opringning i familiariseringsrunden, fordi værten troede, at familiariseringen var færdig grundet en længere samtale med testleder 2. Vært 4 pointerer, at værten ikke tænkte over det, da der blev reageret på gestikkerne hver gang, så der opstod ikke situationer, hvor værten var nødsaget til at repetere gestikkerne. \blankline
%
De seks gæster har alle bemærket, at værten lavede gestikker, men det er ikke alle, der formår at gengive gestikkerne korrekt, hvilket illustreres på \autoref{fig:GengiverGestikker}. Gæst 1 gengiver gestikken til at justere lydstyrken korrekt, men er i tvivl om, hvilken vej hånden skal dreje for enten at skrue op eller ned. I tillæg bemærker gæst 1, at vært 1 gengiver en swipe-bevægelse for at skifte musiknummer, men formår ikke at gengive gestikken korrekt. Til gengæld bemærker gæsten ikke, hvilken gestik værten gengiver for at pause eller starte musikken. Gæst 2 kan gengive de forskellige gestikker, dog lukker gæsten hånden i en knytnæve for at pause musikken og swiper med hele hånden istedet for med to fingre. Gæst 3 formår ligeledes gengive gestikkerne korrekt med forbehold for, at gæsten alternerer mellem to fingre og hele hånden i swipe-bevægelsen, hvorimod gæst 4 formår at gengive gestikkerne korrekt og uden forbehold. Gæst 5 formår at gengive, hvordan musikken pauses, men ikke hvordan den startes igen, hvilket formentlig skyldes, at gæsten slet ikke kigger på værten, når musikken startes igen. Derudover formår gæst 5 at gengive gestikken til at justere lydstyrken korrekt og ligesom tilfældet ved de tidligere gæster, gengiver gæst 5 swipe-bevægelsen med hele hånden. Gæst 6 gengiver gestikkerne korrekt, men formår kun at gengive gestikkerne til at pause musikken, skrue ned for musikken og til at skifte musiknummer. 
%
\begin{figure}[H]
	\centering
	\includegraphics[resolution=300,width=0.9\textwidth]{Test2/GengiverGestikker}
	\caption{Søjlediagram over antallet af gæster, der enten helt, delvist eller slet ikke formår at gengive værternes gestikker korrekt.}
	\label{fig:GengiverGestikker}
\end{figure}
\noindent
%
Baseret på \autoref{fig:GengiverGestikker} fremgår det, hvorvidt gæsterne helt, delvist eller slet ikke formår at gengive værternes gestikker korrekt. At gæsterne ikke formår at gengive værternes gestikker, kan indikere, at gestikkerne er naturlige og diskrete, hvorimod hvis gæsterne er i stand til helt eller delvist at gengive værternes gestikker korrekt, kan det indikere at gestikkerne er relaterbare til den specifikke funktion samt nemme at lære. Med delvist korrekt relateres der til, at gæsterne eksempelvis er i stand til at gengive swipe-bevægelsen, men ikke nødvendigvis gengiver det med de to fingre, som gestikken foreskriver eller at gæsterne lukker hånden helt i frem for at gengive et krokodillenæb, som gestikken til at pause musikken foreskriver. Ifølge gæst 2 og gæst 5 gav gestikkerne god meningen i forhold til, hvilken funktion de tilhører. Gæst 1 tænkte ikke over dem, men giver udtryk for, at de var diskrete og fungerede. Ifølge gæst 3 er gestikkerne intuitive og derudover kiggede gæsten kun enkelt gang på gestikkerne og vurderede, at de gav god mening og at de var lige til. Gæst 5 pointerer, at gestikkerne var nemme og at gæsten selv var i stand til at bruge dem. Den eneste gæst, der udviser en negativ holdning er gæst 6, som giver udtryk for gestikkerne i begyndelsen er blæret, hvorefter gæsten vil blive træt af dem. Gæsten giver udtryk for, at det er et fedt festtrick men forestiller sig, at det er træls, når der er flere mennesker.
%
\subsection{Komfortabilitet ved gestikker}
\label{TestresultaterSocialAcceptGestikkerUkomfortabelt}
%
I exit-interviewet påpeger vært 5, at der kan være noget kulturelt ved gestikken til at pause musikken, som kan misforståes, da det forbindes med en \enquote{klap i}-bevægelse. Dog pointerer vært 5, at ingen i værtens omgangskreds vil misforstå gestikken, hvorfor der ikke vil opstå et problem. \blankline
%
Ud af de seks værter giver vært 1 og vært 6 direkte udtryk for, at gestikken til at justere lydstyrken føltes mere ukomfortabel at udføre sammenlignet med de andre. Tre af de fire værter, som ikke oplevede at nogen gestikker er mere ukomfortable at udføre end andre, giver udtryk for bekymringer relateret til den specifikke gestik. Vært 3 og vært 4 giver begge udtryk for, at der kan opstå problemer i forhold til, hvor meget der skal roteres, for at justere lydstyrken en bestemt mængde. Vært 3 kommenterer ydermere, at gestikken kræver mere finmotorik sammenlignet med de andre udvalgte gestikker. Derudover påpeger værten, at det kan være et problem, hvis bevægelsen kræver en helt rolig og præcis hånd for at kontrollere lydstyrken. I tillæg kommenterer værten, at ved fysisk kontakt, eksempelvis med en drejeknap, er det muligt at gengive en mindre bevægelse sammenlignet med semaforiske gestikker, fordi den fysiske kontakt tillader en større kontrol over justeringen af lydstyrken. 

Der er to årsager til at vært 1 finder gestikken til justering af lydstyrken mere ukomfortabel end de andre gestikker; 1) der er risiko for, at der pludseligt skrues for højt op og 2) hvis rotationen begyndes et dårligt sted, så håndleddet fysisk ikke tillader mere rotation. Værten påpeger dog, at hvis grænsen for, hvor meget ens håndled kan roteres nås, kan grebet slippes, hvorefter det er muligt at gribe fat igen. Grebet referer til, at der tages fat i en fiktiv drejeknap. Vært 6 giver udtryk for, at der var noget underligt med responsen, når værten justerede lydstyrken. Dette kan formentlig skyldes flere ting; dels at det er testleder 2, som justerer lydstyrken afhængigt af værtens gestikulering, så i tilfælde af, at det ikke blev gjort i overensstemmelse med værtens forventning, er det forståeligt, at responsen kan virke underlig. Derudover kan det skyldes, at testleder 2 havde svært ved præcist at tolke bevægelsesmængden i vært 6's gestikulering, da fingrenes position ændrede sig i takt med rotationen. Derudover påpeger vært 6, at det kræver tilvænning i forhold til hvor meget der skal roteres og hvornår rotationen skal stoppe. Dog pointerer værten, at lydstyrken normalvist ikke ændres fra 0 til 100. 
%
\section{Observationer af vært og gæst}
\label{TestresultaterSocialAcceptGestikkerObservationer}
%
I følgende afsnit analyseres og diskuteres observationer fra videooptagelserne af de seks par. Observationerne dækker blandt andet over om gæsterne spørger ind til gestikkerne undervejs, hvordan værternes interaktion med musikanlægget forløber og hvordan samtalen mellem vært og gæst påvirkes af værtens interaktion med musikanlægget.
%
\subsection{Gæsterne interesse i gestikkerne}
\label{TestresultaterSocialAcceptGaestSPGGestikker}
%
Baseret på videooptagelserne er der ingen af gæsterne, der decideret spørger ind til værtens gestikker. Gæst 3 kommenterer, når værten bliver bedt om at indstille lydstyrken til et passende samtale niveau ved det første musiknummer. Kommentaren retter sig ikke direkte mod selve gestikken, men mod det, at værten anvender gestikker, hvor gæsten giver udtryk for, at det er fedt. I musiknummer 2 reagerer gæst 3 på, at musikken bliver skruet højt op og forsøger selv at skrue ned. Dette gøres dog i takt med at værten skruer ned, hvorfor det er værtens gestik testleder 2 reagerer på. Gæst 3 giver udtryk for gerne, at vil prøve gestikkerne, men det realiseres ikke, da samtalen drejes tilbage på ferieplanlægningen. Gæst 4 kommenterer ligeledes, når værten i starten skal indstille lydstyrken til et passende samtale niveau. Igen rettes kommentaren ikke direkte mod selve gestikken, men mod det, at der anvendes gestikker, hvorefter gæsten selv får lov til at prøve at indstille lydstyrken og spørger efterfølgende ikke ind til gestikkerne igen. De fire resterende gæster hverken prøver eller kommenterer på værtens gestikker. At gæsterne ikke direkte spørger ind til gestikker, skyldes formentligt, at det er en testsituation og at gæsterne informeres om, at værten har fået et musikanlæg, som de hører musik fra, hvor værten i den forbindelse skal interagere med anlægget. Det antages, at gæsterne i en virkelig situation formentlig vil være nysgerrige og spørge ind til interaktionsformen. Med udgangspunkt i det foregående samt gæsternes respons i exit-interviewet, er der ingen indikation af, at interaktionsformen tiltrækker negativ opmærksomhed. 
%
\subsection{Værtens interaktionsforløb med musikanlægget}
\label{TestresultaterSocialAcceptVaertsGestikker}
%
I følgende afsnit fokuseres der på, hvordan værterne generelt gestikulerer, når de ikke reagerer på hints, om værterne begår fejl og hvis de gør, hvornår og hvilke fejl begåes. Derudover fokuseres der på, om værterne interagerer med musikken, uden der har været et hint, om nogen af værterne bruger venstre hånd frem for højre, hvor mange gange værterne retter fokus mod enten anlægget, kommunikationskameraet eller selve gestikken og hvor mange gange gæsterne retter fokus mod værternes gestikker eller hints.\blankline
%
I forhold til hvordan værterne generelt gestikulerer, når de ikke reagerer på hints, er værterne generelt meget stillesiddende og ingen af værterne gengiver gestikker, som misforståes af testleder 2. Dog gengiver vært 4 en \enquote{pyt med det}-bevægelse, som tilnærmelsesvist minder om et swipe med hele hånden, bortset fra at værtens bevægelsen er vertikal fremfor horisontal. Værtens \enquote{pyt med det}-bevægelse forekommer kun én gang og kommer i forbindelse med, at værten og gæsten taler om budgetterne til de tre ferier. Ud fra observationerne af vært 2, tyder det på, at værten koncentrerer sig om at høre de forskellige hints, eksempelvis retter værten ofte fokus mod mobiltelefonen, selvom den ikke ringer.\blankline
%
Fokuseres der på, om værterne begår fejl under udførelsen af gestikkerne, er det vært 1, vært 3, vært 4 og vært 5, der begår direkte fejl i gestikken. Ved vært 4 og 5 drejer det sig om to enkeltstående tilfælde, hvor værterne skifter musiknummer ved at gengive en swipe-bevægelse med én finger i stedet for to. Vært 1 glemmer ved den sidste opringning at pause musikken og formår dermed heller ikke at starte musikken igen. Som tidligere nævnt kommenterer vært 1, at værten normalvist ikke pauser sin musik ved et opkald. Vært 3 skifter, i musiknummer 4, musiknummer med hele hånden fremfor de to fingre gestikken foreskriver. Generelt skifter vært 3 musiknummer med en doven hånd, der ofte starter swipe-bevægelsen med to fingre og ender med hele hånden. Et eksempel på den dovne hånd forefindes i videomaterialet vedlagt i \autoref{app:app:VideooptagelseVaertOgGaest}. For at undersøge om vært 3 under familiariseringen har lært at gengive den korrekte gestik til at skifte musiknummer, foretages der observationer af værtens familiarisering. Baseret på disse observationer vurderes det, at værten rent faktisk har lært at gengive den korrekte gestik og gengiver den korrekt under hele familiariseringen, dog swiper værten, som nævnt, med en doven hånd. Der er derfor ikke mistanke om, at værten har misforstået hvordan gestikken skal udføres, men derimod tyder det på, at værten udfører gestikken mere dovent desto mere fortrolig værten bliver med gestikken. Derudover bygger observationerne på optagelserne foretaget med GoPro-kameraet, som er placeret modsat kommunikationskameraet, hvilket potentielt kan have indflydelse på, hvordan værtens gestikker vurderes. Baseret på testledernes observationer inde fra kontrolrummet, er værten kun noteret for en enkelt fejl, hvor værten swipede med hele hånden, det tyder derfor på at GoPro-kameraets vinkel, har gjort det svært at genkende gestikken efterfølgende. 

Da vært 1, vært 4 og vært 5 hver især kun begår en enkelt fejl vurderes det, at det ikke er nødvendigt at undersøge deres familiarisering nærmere.\blankline
%
Når værterne interagerede med musikanlægget uden hints, blev det ikke noteret som en fejl, da brugere i en virkelig situation selvfølgelig skal kunne interagere med sit musikanlæg på ethvert tidspunkt. Interaktionerne uden hints relaterer sig nærmest udelukkende til at justere lydstyrken i forbindelse med, at værten tidligere enten har skruet for meget ned eller for højt op. 

Ydermere skifter vært 1, vært 2 og vært 6 musiknummer, uden at der har været et hint, hvilket påvirker præsentationsrækkefølgen af de efterfølgende hints og resulterer i, at værterne ikke præsenteres for lige mange hints. Vært 1 skiftede musiknummer i slutningen af det første musiknummer, som skal få samtalen ordentlig i gang, før de planlagte hints efterfølgende præsenteres. Årsagen til at værten skiftede musiknummer begrundes med, at værten oplevede at outtroen var irriterende og da det ikke har en betydning for rækkefølgen, hvorved de forskellige hints præsenteres, greb testleder 2 ikke ind. I den første opringningen fik værten dog at vide, at der fremover kun skal reageres på hints, hvilket værten, med undtagelse af en justering af lydstyrken, overholdte. Ligesom tilfældet med vært 1 skifter vært 2 ligeledes musiknummer i det første musiknummer, men denne gang sker det tidligt i musiknummeret, hvilket resulterer i at testleder 2 ringer til værten og spørger ind til om højtalerne skratter. Derudover beder testleder 2 værten om, kun at reagere på hints. Efter opkaldet startes musiknummeret forfra, og vært og gæst forstætter ferieplanlægningen. Efterfølgende skifter vært 2 ikke musiknummer uden et hint.    

Ved musiknummer 1 skifter vært 6 til det næste musiknummer, uden et hint, hvilket testleder 2 reagerer på, da det næste hint er forvrængningen, som netop skal få værterne til at skifte musiknummer og da vært 6 skifter musiknummer få sekunder før hintet, vurderes det til ikke at være en fejl. I musiknummer 4 prøver vært 6 at skifte musiknummer og da det sker samtidig med, at testleder 2 er i gang med at ringe op til værten, reageres der ikke på værtens gestik. Under opringningen får værten at vide, at der kun skal reageres på hints, hvor vært 6 begrunder sin interaktion med, at musikken var af dårlig kvalitet. Efterfølgende reagerer vært 6 kun på hints.\blankline
%
Som forklaret er det kun højrehåndede testpersoner, der vælges til at agere værter, hvorfor det forventes, at gestikkerne kun udføres med højre hånd. Baseret på observationerne er der dog to værter; vært 3 og vært 6, som benytter sig af venstre hånd til at interagere med musikken, hvilket skyldes at begge værter spiser cookie med højre hånd. I \autoref{app:app:VideooptagelseVaertOgGaest} forefindes videomateriale for de pågældende situationer. Det er bemærkelsesværdigt, at vært 3 swiper fra venstre mod højre, når værten gestikulerer med venstre hånd, hvorimod vært 6 swiper fra højre mod venstre med venstre hånd. Da der kun er opsat krav til, hvordan interaktionen ved swipe skal foregå med højre hånd, er det ikke muligt at konkludere om værterne gør det korrekt med venstre hånd. Der bør derfor foretages en undersøgelse af, hvordan interaktionen reelt skal foregå med venstre hånd, når brugeren enten er højrehåndet eller venstrehåndet, da der uundgåeligt er venstrehåndede brugere i Bang $\&$ Olufsens kundekreds.     
%
\begin{figure}[H]
	\centering
	\includegraphics[resolution=300,width=0.9\textwidth]{Test2/KiggerImodAnlaeg}
	\caption{Søjlediagram over antallet af gange værterne kigger hen mod anlægget ved hver funktion. Data bygger på observationerne fra videooptagelserne.}
	\label{fig:KiggerImodAnlaeg}
\end{figure}
\noindent
% 
Når værterne interagerer med musikanlægget, er det forskelligt hvem og hvor mange gange værternes blik rettes i den retning, jævnfør \autoref{fig:KiggerImodAnlaeg}. Vært 1 interagerer oftest med musikanlægget uden at kigge i den retning, modsat vært 3 og vært 4, som næsten udelukkende kigger mod musikanlægget, hver gang de interagerer med det. Afhængigt af hvordan værten interagerer med musikanlægget, er det derfor interessant, at undersøge dels, om det påvirker samtalen og dels, om det påvirker gæstens indtryk af interaktionsformen. For at undersøge det nærmere sammenholdes observationerne af hvordan værterne interagerer med musikanlægget samt hvor antallet af gange de kigger i den retning med om de respektive gæster er i stand til at gengive værtens gestikker. 

Gæst 1 reagerer to gange på, at værten interagerer med musikanlægget, hvilket særligt kommer til udtryk, da gæsten kun delvist kan gengive gestikkerne til henholdvis at justere lydstyrken og til at skifte musiknummer, men slet ikke formår at gengive pause og start. Ved par 2 fremgår det tydeligt ud fra observationerne, at så snart værten får at vide, at gestikkerne gerne må gengives diskret, kigger værten færre gange mod musikanlægget, hvilket mindsker gæstens reaktion på interaktionen. Situationen forefindes i videomaterialet vedlagt \autoref{app:VideooptagelseVaertOgGaest}. Som tidligere nævnt kan gæst 2 gengive samtlige gestikker korrekt eller delvist korrekt, hvilket højst sandsynligt skyldes, at gæsten reagerede brat på værtens tydelige interaktion med musikanlægget i starten af testen. 

For par 3 og par 4 er situationen ens; selvom værterne kiggede hen mod musikanlægget næsten hver gang de interagerede med det, var gæsterne primært fascinerede af interaktionsformen i starten, hvorefter fokus blev rettet mod ferieplanlægningen. Begge gæster kommenterede på gestikkerne og gæst 4 prøvede selv at justere lydstyrken, hvilket kan have indflydelse på den dalende interesse. Jævnfør \autoref{fig:KiggerImodAnlaeg} kigger vært 5 ofte hen imod musikanlægget ved interaktion, og ud fra observationerne af optagelsen af par 5 fremgår det, at værten oftere kigger hen imod anlægget i slutningen af testen end i starten, årsagen er dog uvis. Selvom værten ofte vender blikket mod anlægget, er der ingen indikation af, at gæsten påvirkes, da gæsten ikke på noget tidspunkt virker til at interessere sig for eller reagere på gestikkerne. Gæsten retter derimod sin opmærksomhed mod værtens ansigt eller ferieplanlægningen. 

At disse tre gæster; gæst 3, gæst 4 og gæst 5 bliver mindre påvirket af interaktionen, selvom den respektive vært ofte vender blikket mod anlægget og/eller gestikken, kan skyldes, at værterne gestikulerer diskret og samtidig interagerer naturligt med musikanlægget.

Ligesom ved vært 2, kigger vært 6 ofte hen i mod musikanlægget og gestikulerer tydeligt på hints, indtil værten i opkaldet i musiknummer 4 får at vide, at gestikkerne gerne må gengives diskret, hvorefter værten ikke kigger lige så ofte mod anlægget. Inden værten får opkaldet, fremgår det tydeligt at gæst 6 reagerer brat på interaktionen og at det forstyrrer samtalen. Så snart værten gengiver gestikkerne diskret, reagerer gæsten stort set ikke på værtens resterende gestikker. Det antages, at såfremt vært 6 tidligere i testen havde fået besked på at gøre gestikkerne diskrete, så ville værten ikke kigge hen imod anlægget lige så mange gange, som det er tilfældet, jævnfør \autoref{fig:KiggerImodAnlaeg}. 

Der vil i følgende afsnit fokuseres yderligere på, hvordan samtalen mellem vært og gæst påvirkes af værtens interaktion med musikanlægget.
%
\subsection{Påvirkningen af samtalen mellem vært og gæst}
\label{TestresultaterSocialAcceptSamtale}
%
I følgende afsnit fokuseres der på, hvordan samtalen mellem vært og gæst påvirkes af værtens interaktion med musikanlægget og om værten selv benytter sig af computeren eller papirerne, som gæsten har medbragt. \blankline
%
Baseret på observationerne af de seks par, er det primært ved par 2 og par 6, at samtalen påvirkes. Ved par 2 tyder det på, at samtalen afbrydes, fordi værten koncentrerer sig om at registrere og reagere på hints, hvilket får gæsten til at stoppe op og vente til værten er færdig før samtalen genoptages. Dette ændres dog efter værten får at vide dels, at værten gerne må benytte sig af computeren, hvilket sker i opringningen i musiknummer 1 og dels, når værten bliver opfordret til at gengive gestikkerne diskret, hvilket sker i opringningen i musiknummer 2. Efterfølgende tyder det på, at samtalen mellem vært og gæst ikke påvirkes i lige så høj grad som før og at værten ikke koncentrerer sig lige så meget om at registrere diverse hints. Ud fra både observationer og værtens egen respons i exit-interviewet, kan det fastslås, at værten havde problemer med at registre forvrængningen. Med udgangspunkt i observationerne af par 6, fremgår det, at hver gang værten reagerer på et hint, tiltrækker det gæstens opmærksomhed, hvilket afbryder samtalen. Dog er de begge i stand til at vende hurtigt tilbage til samtaleemnet. Årsagen til at værtens interaktion med musikanlægget tiltrækker gæstens opmærksomhed er formentlig, fordi værten både retter sit fokus og sin krop mod anlægget, som er placeret modsat gæsten, hvilket derfor resulterer i, at gæsten ligeledes retter fokus fra samtalen til værtens interaktion. I opringningen i musiknummer 4 opfordres vært 6, som nævnt, til at gengive gestikkerne diskret, hvorefter det tyder på, at samtalen ikke længere påvirkes i lige så høj grad. Dette indikerer, at både opringningen og det, at værten for alvor benytter sig af gæstens medbragte computer, er med til dels, at samtalen ikke afbrydes lige så brat som før og dels, at værten interagerer med musikanlægget mere afslappet og uden at rette sit fokus og sin kroppen mod anlægget.

Ud fra observationerne af de resternede fire par; par 1, par 3, par 4 og par 5, tyder det ikke på, at værtens interaktion med musikanlægget har en negativ påvirkning på samtalen med gæsten. Når vært 1 reagerer på en opringning, arbejder gæsten som regel videre med ferieplanlægningen, hvorefter værten hurtigt vender tilbage til samtalen. Baseret på observationerne af par 3 fremgår det, at den eneste gang et hint påvirker samtalen er ved musiknummer 4, hvor der skrues højt op for musikken, hvilket får gæsten til at reagere. Generelt er der ingen indikationer af, at samtalen mellem vært 3 og gæst 3 påvirkes negativt, da de holder sig til samtaleemnet og samtidig inkluderer opkaldende som en naturlig del af deres samtale. Lignende er gældende for par 5, hvis samtale ikke påvirkes af diverse hints og hvor værten ligeledes inkluderer opkaldende som en naturlig del af samtalen. Et eksempel på samtalen og interaktionen med musikanlægget forefindes i \autoref{app:app:VideooptagelseVaertOgGaest}. Tendens er gennemgående for par 4, hvor værtens interaktion med musikanlægget ikke påvirker samtalen negativt, da både vært og gæst formår at opretholde samtalen og gæsten arbejder videre, når værten reagerer på et opkald. \blankline
%
I forhold til om værterne benytter sig af gæstens medbragte computer eller de medbragte papirer, så benytter halvdelen af værterne sig af computeren og ingen benytter papirene. Vært 2 benytter sig af computeren allerede efter den første opringning i musiknummer 1 og benytter den stort set resten af tiden. Den givne situation forefindes i videomaterialet vedlagt i \autoref{app:app:VideooptagelseVaertOgGaest}. Vært 5 benytter computeren i starten af musiknummer 3, hvor gæsten hurtigt overtager inden værten reagerer på hintet om at skifte musiknummer og kortvarigt i musiknummer 6. Vært 6 benytter computeren kortvarigt i musiknummer 1 og igen efter opkaldet i musiknummer 4, hvorefter værten benytter sig af computeren stort set resten af tiden. Selvom brugen af computeren er med til, foruden opringningerne, at vært 2 og vært 6 ændrer adfærd, både i forhold til deltagelsen i samtalen og i forhold til, hvordan gestikkerne udføres, er det ikke en gennemgående tendens blandt værterne.   

På baggrund af observationerne vurderes det, med forbehold for at vært 2 og vært 6 først skal opfordres til at gestikulere diskret, at værternes interaktion med musikanlægget ikke påvirkede samtalen negativt. Ydermere opstod der ikke akavede pauser mellem vært og gæst, hvor de ikke vidste, hvad de skulle sige til hinanden eller hvor de kom fra. 

