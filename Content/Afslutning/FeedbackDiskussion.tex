\section{Feedback}
\label{DiskussionFeedback}
%
I \fullref{Feedbackformer} er 

man skal vænne sig til det, lære at stole på det
Feedback kan være godt i starten, hvorefter man måske kan slå det fra
funktionel feedback i form af musikken starter/stopper osv. 
Der kommer nok ikke nogen iboende feedback, da dette kræver man rører ved noget. 
Skal der være augmenteret feedback?
Det kommer også an på hvor hurtigt systemet kan registrere bevægelserne, specielt ved at justere på lydstyrken, hvor der er større problemer ved forsinkelser end ved pause/play og skift sang. 
Man bliver nok mere sikker i sine interaktioner, som der ikek er fuld kontrol over, hvis der er feedback. 
forskel på enkelt nu-og-her forsøg og en længervarende feltundersøgelse
Det lader derfor til at testpersoner, som testes i et laboratorium har nogle andre behov end hvad tilfældet ville være, hvis de blev testet i en feltundersøgelse, da de automatisk vil få længere tid til at vænne sig til den perifere interaktion. Det kan derfor være svært at afgøre, hvorvidt der skal være en form for feedback eller ej, når en bruger interagere med et musikanlæg i den perifere opmærksomhed. Dette gør sig særligt gældende når interaktionsformen enten bygger på semaforiske og/eller manipulerende gestikker.


Feedback
AChairAsUbiquitours... understreger vigtigtheden af et godt genkendelsessystem, da testpersonerne ikke får andet end funktional feedback fra musikken - manglen på feedback under tests - hvd kan man foreslå af feeback - generelt teori vi har om feedback