\section{Gestik}
\label{Gestik}
%
I følgende afsnit vil forskellige problemstillinger, vedrørende gestik, belyses, heriblandt komplikationer forårsaget af gestik, hvilken form for feedback der er nødvendig samt den sociale accept af gestik. Inden da vil der først og fremmest blive undersøgt hvilke former for gestikker, der kan gøres brug af.  
%
\begin{itemize}
	\item Nu har vi så beskrevet perifer interaktion, men hvordan bruger man det så? Hvilke gestures skal der til?
	\item objektbaseret gestures
	\begin{itemize}
		\item social accept
	\end{itemize}
\end{itemize}\blankline
%
\subsection{Kategoriseringer af gestikker}
\label{KategoriseringerAfGestikker}
%
Inden der dykkes ned i forskellige typer af gestikker skal det nævnes, at der i litteraturen fremgår mange forskellige termer for de samme typer af gestikker, \parencite[s. 3]{PDF:ATaxonomyOfGestures}. Der vil derfor tages udgangspunkt i, hvordan de forskellige typer af gestikker defineres, baseret på definitionerne fremsat af \textcite[ss. 4-9]{PDF:ATaxonomyOfGestures}. Ifølge \textcite[s. 4]{PDF:ATaxonomyOfGestures} kan gestikker kategoriseres i fem grupper: \textit{Deictic Gestures}, \textit{Manipulative Gestures}, \textit{Semaphoric Gestures}, \textit{Gesticulation} og \textit{Language Gestures}. \blankline
%
Da der tidligere er afgrænset fra at arbejde med stemmestyring, så vil \textit{Gesticulation} ikke blive nærmere beskrevet, da denne type af gestikker involverer både håndbevægelser og tale, \parencite[s. 7]{PDF:ATaxonomyOfGestures}. \textit{Language Gestures} er de tegn, som bruges i tegnsprog. Denne type af gestikker er bygget op omkring en grammatisk struktur og anvendes primært til kommunikation fremfor at give kommandoer, \parencite[s. 8]{PDF:ATaxonomyOfGestures}. Ifølge \textcite[s. 8]{PDF:ATaxonomyOfGestures} så er det lige så krævende at processere tegnsprog, som det er at processere tale. For at anvende tegnsprog, særligt i det perifere, vil det først og fremmest kræve at brugeren lærer tegnsprog og at kræve det af brugeren, vurderes til at være for omstændigt. Der afgrænses derfor fra at arbejde med tegnsprog. \blankline
%  
\textit{Deictic Gestures} er en kategori af gestikker, som forudsætter at der peges på objekter med det formål at ændre deres egenskaber og deres spatielle lokation, \parencite[s. 4]{PDF:ATaxonomyOfGestures}. Ifølge \textcite[ss. 4-5]{PDF:ATaxonomyOfGestures} anvendes denne form for gestik, blandt andet, til at allokere objekter på en stor skærm, identificere objekter i \textit{Virtual Reality} og desktop- samt kommunikationsbaserede applikationer. \blankline
%
\textit{Manipulative Gestures}, som denne kategori antyder, vedrører de manipulerende gestikker hvorved objekter, i en eller anden grad, manipuleres, \parencite[s. 5]{PDF:ATaxonomyOfGestures}. Det kan både foregå ved 2- og 3-dimensionelle interaktioner. Manipulerende gestikker ved 2-dimensionelle interaktioner vedrører manipulering af objekter på en 2-dimensionel skærm, som for eksempel en cursor, et vindue eller noget helt tredje. Ifølge \textcite[s. 5]{PDF:ATaxonomyOfGestures} anses det ikke som værende en manipulerende gestik hvis objektet blot trækkes, flyttes eller klikkes på. For at det kan være en manipulerende gestik skal systemet, ifølge \textcite[s. 5]{PDF:ATaxonomyOfGestures}, modtage nogle parametre såsom; brugerens anmodning om at flytte eller på anden vis ændre objektet. Ved 3-dimensionelle interaktioner kan de manipulerende gestikker enten afspejle en fysisk manipulering af et objekt, en fysisk manipulering af en computer eller ved at manipulere et fysisk objeket, som afspejles i et virtuelt objekt på en touchskærm, \parencite[s. 6]{PDF:ATaxonomyOfGestures}. Derudover kan 3-dimensionelle interaktioner også inkorporeres så de manipulerer 2-dimensionelle objekter, for eksempel ved hjælp af tryksensorer, hvorved det er muligt at tilføre ekstra dimensioner til en 2-dimensionel interaktion, \parencite[s. 5]{PDF:ATaxonomyOfGestures}. 

De manipulerende gestures benyttes typisk til navigation i den virtuelle verden, fordi det, ved brug forskellige sensorer placeret i det virtuelle rum, er muligt at interagere med de virtuelle objekter, \parencite[ss. 14-15]{PDF:ATaxonomyOfGestures}. I forhold til perifer interaktion tyder det på, at det er de manipulerende gestikker, som indtil videre er de mest brugte, i form af at testpersonerne manipulerer et fysisk objekt, \parencite[s. 164]{PDF:ComparingInputModalities}. I undersøgelsen foretaget af \textcite[ss. 164-165]{PDF:ComparingInputModalities} gør de, blandt andet, brug af to forskellige modaliteter inden for manipulerende gestikker; et fysisk, håndgribeligt knop-baseret håndtag og en touchskærm. Den tredje modalitet \textcite[s. 165]{PDF:ComparingInputModalities} undersøger er frihåndsgestikker, også kaldt semaforiske gestikker.  \blankline
%
Den sidste, af de fem kategorier, er \textit{Semaphoric Gestures}, som, ifølge \textcite[s. 6]{PDF:ATaxonomyOfGestures}, er den meste udbredte form for gestik på trods af, at den findes unaturlig. Semaforiske gestikker relaterer sig til at benytte tegn til at kommunikere information, hvor disse tegn dannes med kropsbevægelser. Ifølge \textcite[s. 1961]{PDF:AStudyOnTheUseOfSemaphoricGestures} er det en unaturlig måde at interagere med en computer på og derudover gengiver disse gestikke kun en begrænset del af menneskets kommunikationsevne. Dog tyder det på, at netop semaforiske gestikke med fordel kan anvendes som en interaktions mulighed ved sekundære opgaver, \parencite[s. 1961]{PDF:AStudyOnTheUseOfSemaphoricGestures}. Det skyldes, ifølge \textcite[s. 1964]{PDF:AStudyOnTheUseOfSemaphoricGestures}, at denne form for gestik vil reducere restitutionstiden mellem den sekundære og primære opgave. Årsagen til det skyldes, blandt andet, at semaforiske gestikker ikke afhænger af den visuelle opmærksomhed og at der skal et færre antal interaktioner til, for at den sekundære opgave er løst, \parencite[s. 1964]{PDF:AStudyOnTheUseOfSemaphoricGestures}. En anden fordel ved at anvende semaforiske gestikke er, at det tillader en større afstand mellem brugeren og det elektroniske apparat, \parencite[s. 6]{PDF:ATaxonomyOfGestures}.    

Der er to former for semaforiske gestikker; statiske og dynamiske, \parencite[s. 7]{PDF:ATaxonomyOfGestures}. De statiske kræver at gestikken fastholdes, hvor de dynamiske tillader bevægelse. Begge former for semaforiske gestikker kan udføres ved at anvende hænderne, fingrene, armene, fødderne, hovedet eller andre passive elektroniske apparater, \parencite[s. 7]{PDF:ATaxonomyOfGestures}. Ifølge \textcite[s. 823]{PDF:UnderstandingNaturalness} så bør gestikkerne være dynamiske, hvis formålet er at manipulere et objekt. \blankline
%
Selvom det er muligt kun at fokusere på én af de fem kategorier af gestikker, så kan det være en fordel at udnytte flere typer af gestikker når der skal interageres med en elektroniske apparaters brugergrænseflader. Dog tyder det på at det ikke er alle fem kategorier, der er lige eftertragtet at blande med de andre, som det fremgår af \textcite[s. 8]{PDF:ATaxonomyOfGestures}, er det nemlig kun; \textit{Deictic Gestures}, \textit{Manipulative Gestures} og \textit{Semaphoric Gestures}, som blandes. De to resterende \textit{Gesticulation} og \textit{Language Gestures} nævnes slet ikke i forbindelse med at bruge mere end én type gestik.
%
\subsubsection{Udførelsen af gestik}
\label{UdfoerelseAfGestik}
%
I dette afsnit fokuseres der på to overordnet former for input-gestikker, som et elektronisk apparat kan reagere på. Dernæst fokuseres der på fordelene og ulemperne ved bevægelsesmængden i gestikker.\blankline
%
Udover de foregående fem klassificeringer af gestikker, så definerer \textcite[s. 9]{PDF:ATaxonomyOfGestures} to former for inputs et elektronisk apparat kan modtage: \textit{Non-Perceptual Input} og \textit{Perceptual Input}. De går i alt sin enkelthed ud på, at hvis et elektronisk apparat modtager et \textit{Non-Perceptual Input}, så foregår det enten igennem fysisk kontakt med et elektronisk apparat eller et andet objekt, \parencite[s. 10]{PDF:ATaxonomyOfGestures}. Hvis det elektroniske apparat derimod modtager et \textit{Perceptual Input}, så foregår det ved at udføre en bestemt gestik, som apparatet så kan genkende og reagere på, ved hjælp af lys-, lyd- eller bevægelsessensorer, \parencite[s. 12]{PDF:ATaxonomyOfGestures}. Det er derfor muligt ved \textit{Perceptual Input} helt at undgå, at skulle interagere med endnu et elektronisk apparat eller iklæde sig en form for elektronik, som for eksempel en elektronisk handske, \parencite[s. 12]{PDF:ATaxonomyOfGestures}.\blankline
%
Bevægelsesmængde indenfor gestik kommer til udtryk igennem mikro- og makrogestikker, \parencite[s. 6]{PDF:UsabilityofMicroVsMacroGestures}. Mikrogestikker er små bevægelser, der ikke nødvendigvis kræver at hele hånden bevæger sig og da de kan udføres samtidig med andre manuelle opgaver, er de specielt lovende indenfor perifer interaktion, \parencite[s. 95]{PDF:PeripheralInteraction}. Derudover er mikrogestikker hurtige at udføre, hvilket, ifølge \textcite[s. 96]{PDF:PeripheralInteraction}, gør dem til en god kandidat for perifer interaktion. For eksempel så tillader mikrogestikker kommunikation, som en sekundær opgave, samtidig med at fokus holdes på en samtale, som er den primære opgave, uden at samtalepartneren finder en uhøflig, \parencite[s. 97]{PDF:PeripheralInteraction}.

Makrogestikker er derimod store bevægelser, der kræver signifikant bevægelse, som eksempelvis at løfte en arm eller rejse sig fra en siddende position, \parencite[s. 6]{PDF:UsabilityofMicroVsMacroGestures}. Ifølge \textcite[s. 9]{PDF:UsabilityofMicroVsMacroGestures} kan makrogestikker udføres meget mindre præcist end mikrogestikker, da en løftet arm vil tolkes som en løftet arm, uanset om den er løftet 60$^{\circ}$ eller 90$^{\circ}$. En af ulemperne ved makrogestikker er dog, at der skal bruges et stort område til at udføre de pågældende gestikker og at en sensor, der opfanger makrogestikker ofte vil dække et helt lokale, \parencite[s. 9]{PDF:UsabilityofMicroVsMacroGestures}. Der kan derfor opstå et problem hvis en anden træder ind i interaktionsområdet og ved et uheld laver en bevægelse, som genkendes af systemet. Da det kan være svært at træde ud af interaktionsområdet, kræver det at brugeren rent fysisk flytter sig langt nok væk, så en ubevidst interaktion undgåes. Til gengæld er en af fordelene ved makrogestikker, at de specifikke gestikker er så tilpas forskellige, at sandsynligheden for at systemet forveksler dem er lille, \parencite[s. 9]{PDF:UsabilityofMicroVsMacroGestures}.  

Mikrogestikker behøver, modsat makrogestikker, ikke et stort interaktionsområde, men skal til gengæld udføres meget mere præcist, \parencite[s. 10]{PDF:UsabilityofMicroVsMacroGestures}. En af fordelene ved mikrogestikker er, at de tillader en større diversitet end makrogestikker, med andre ord; det er muligt at designe flere mikrogestikker end makrogestikker, \parencite[s. 10]{PDF:UsabilityofMicroVsMacroGestures}. En anden fordel ved mikrogestikker er, at interaktionen med systemet sjældent sker ved en fejl, da det er mindre sandsynligt at specifikke gestikker udføres ubevidst, \parencite[s. 10]{PDF:UsabilityofMicroVsMacroGestures}. Ulemperne ved at bruge mikrogestikker er dels, at desto større afstanden til systemet er, desto større er usikkerheden for, hvorvidt systemet kan registrerer gestikkerne, \parencite[s. 10]{PDF:UsabilityofMicroVsMacroGestures}. En anden ulempe ved mikrogestikker er brugerens evne til at gengive dem korrekt og uden at forveksle dem. Ifølge \textcite[s. 10]{PDF:UsabilityofMicroVsMacroGestures} så egner mikrogestikker sig bedst til situationer, hvor der foretrækkes et mere professionelt udtryk, for eksempel på arbejdspladsen. Derudover egner mikrogestikker sig også til underholdnings apparater.\blankline
%
I undersøgelsen foretaget af \textcite[s. 823]{PDF:UnderstandingNaturalness} fremgår det, at det ikke er hensigtmæssigt at benytte kropsdele, som værktøjer i gestikker da de virker unaturlige. Med værktøjer refereres der til det værktøj, som ellers ville være blevet anvendt såfremt det havde været en naturlig situation. Baseret på resultaterne, fremsat i \textcite[s. 823]{PDF:UnderstandingNaturalness}, fremgår det at 77.5\% af gangene så opstår der fejl, når testpersonerne skal anvende en kropsdel som værktøj.

Derimod anbefaler \textcite[s. 823]{PDF:UnderstandingNaturalness} at gestikkerne udføres som pantomime, hvor brugere udføre den tiltænkte aktivitet og forestiller sig at have værktøjet i hånden. Derudover anbefaler \textcite[s. 824]{PDF:UnderstandingNaturalness} at hvis gestikkerne skal udføres i rummet, altså væk fra det elektroniske apparat, så bør gestikkerne udføres med begge hænder. Hvor den ikke-dominante hånd skal danne en form for ramme eller udgangspunkt for den dominante hånd, som så udføre gestikken.      
%

\subsection{Komplikationer forårsaget af gestikker}
\label{KomplikationerGestikker}
%
Selvom gestik virker lovende inden for perifer interaktioner, så er det stadig en forholdvis ny måde at interagere med elektroniske apparater på, \parencite[s. 163]{PDF:ComparingInputModalities}, hvorfor der naturligvis forekommer nogle komplikationer. I følgende afsnit vil nogle af disse komplikationer blive belyst, først i forhold til teknologiske komplikationer og derefter i forhold til komplikationer inden for det menneskeligeaspekt. Fokus vil primært være på komplikationer inden for det menneskeligeaspekt. \blankline
%
Nogle af de mest åbenlyse teknologiske komplikationer, der kan opstå relaterer sig til problemer med at genkende specifikke gestikker, \parencite[s. 27]{PDF:ATaxonomyOfGestures}. Det gør sig både gældende for mikrogestikker, som er svære at registrere og for makrogestikker hvor risikoen for at en anden træder ind i det interaktiveområde og skygger for eller ubevidst udføre gestikker, som genkendes af systemet, \parencite[s. 9]{PDF:UsabilityofMicroVsMacroGestures}. Derudover kan systemet have problemer med at registrere og genkende semaforiske gestikker, hvis de udføres samtidig med andre bevægelser, \parencite[s. 27]{PDF:ATaxonomyOfGestures}. Problemet med de semaforiske gestikker opstår fordi systemet kan have svært ved at registrere både hvornår interaktionen starter og slutter og hvornår det er en sekvens af gestikker fremfor en enkelt gestik, \parencite[s. 27]{PDF:ATaxonomyOfGestures}. 

Derudover kan det også opstå tekniske begrænsninger, hvor systemet simpelthen ikke er i stand til at registrere og genkende bestemte typer gestikker og hvis det er tilfældet skal designerne finde en måde at kommer uden om det problem, \parencite[s. 27]{PDF:ATaxonomyOfGestures}. Det kan resultere i at en simpel interaktion bliver kompleks og besværlig at udføre med gestik. Nogen af disse teknologiske komplikationer er forsøgt løst ved at bruge markører, som brugeren eksempelvist kan have på fingrene, og handsker, som indeholdere sensorer. Ulemperne ved de to løsninger er at de virker unaturlige og besværlige, \parencite[s. 26]{PDF:ATaxonomyOfGestures}. En anden teknologisk komplikation, der kan opstå foregår helt tilbage i designfasen, hvor systemet testes via en Lo-Fi prototype. Der skal nemlig tages højde for, at resultaterne fra en Lo-Fi prototype formegentligt vil afvige fra resultaterne fundet ved en feltundersøgelse. \textcite[s. 176]{PDF:ComparingInputModalities} oplevede denne afvigelse, hvor testpersonerne fortrak de semaforiske gestikker fremfor de to typer af manipulerende gestikker i testen med Lo-Fi prototypen, hvor det i feltundersøgelsen var de to typer af manipulerende gestikker, som testpersonerne fortrak. Årsagen til det skyldes, ifølge \textcite[s. 176]{PDF:ComparingInputModalities} tre ting; 1) tekniske problemer i forbindelse med semaforiske gestikker i feltundersøgelsen, 2) haptiske problemer og 3) manglende interaktivitet ved Lo-Fi prototypen.\blankline
%
Komplikationer inden for det menneskelige aspekt kan opstå ved noget så åbenlyst, som manglende evne til at udføre den specifikke gestik. I den forbindelse kan der også opstå problemer ved udmattelse, hvis det er fysisk udmattende at udføre gestikken eller hvis gestikken udføres gentagende gange, \parencite[s. 28]{PDF:ATaxonomyOfGestures}. Derudover så indeholder gestikker ikke visuelle hints, hvilket ,ifølge \textcite[s. 6]{PDF:NaturalUserInterfaces}, kan give problemer hvis systemet ikke reagerer på brugerens bevægelse eller hvis systemet reagere forkert. Det skyldes at brugeren ikke har mulighed for at evaluere den information systemet har registreret og finde ud af hvad fejlen var. En måde at undgå denne komplikation er ved at sørger for at brugeren modtager en form for feedback, så de dels kan lære af deres fejl og dels ved hvordan de skal gebærde sig, \parencite[s. 10]{PDF:NaturalUserInterfaces}. I forhold til hvilken type feedback og om der overhovedet skal være feedback vil blive belyst i \fullref{Feedbackformer}. \blankline
%
En anden komplikation der skal tages højde for, særligt når gestikker anvendes som en interaktions mulighed til perifer interaktion er, at gestikkerne ikke er perifere før de er lært, inkorporeret og husket, \parencite[s. 16]{PDF:PIEmbeddingHCIOnTheRelevance}. Derudover så kræver det, ifølge \textcite[s. 16]{PDF:PIEmbeddingHCIOnTheRelevance}, også at opmærksomheden ikke fjernes fra den primære opgave når der skal interageres i den perifære opmærksomhed, hvilket kun kan ske hvis gestikkerne holdes naturlige. I det henseende så pointere både \textcite[s. 8]{PDF:NaturalUserInterfaces}, \textcite[s. 26]{PDF:ATaxonomyOfGestures} og \textcite[s. 19]{PDF:PIEmbeddingHCIOnTheRelevance}, at der på nuværende tidspunkt både mangler simple og intuitive sets af gestikker og en form for standardisering af gestikker. Hvis det kan efterkommes så vil det betyde at de samme gestikker repræsenterer og aktiverer de samme funktioner på tværs af de elektroniske apparater. En af årsagerne til at der ikke allerede er en form for standard og fælles forståelse for hvilke gestikker, der skal bruges til hvad skyldes, formegentlig, både at det er relativt nyt at bruge gestik til interaktion med allestedsværende elektroniske apparater og at det er nyt at bruge gestik i forbindelse med perifer interaktion. Derudover er det, ifølge \textcite[s. 28]{PDF:ATaxonomyOfGestures}, stadig usikkert hvilke gestikker, særligt semaforiske gestikker, der passer bedst til bestemte scenarier.

Derudover er der en gennemgående diskussion vedrørende semaforiske gestikker, da de betragtes som værende unaturlige, \parencite[s. 1961]{PDF:AStudyOnTheUseOfSemaphoricGestures}, men samtidig er det den form for gestik, der er mest udbredt indenfor interaktion med allestedsværende elektroniske apparater, \parencite[s. 28]{PDF:ATaxonomyOfGestures}. Ydermere understøtter semaforiske gestikker også interaktion, der ikke afhænger af den visuelle opmærksomhed, \parencite[s. 1964]{PDF:AStudyOnTheUseOfSemaphoricGestures}, hvilket passer godt med perifer interaktion. Der kan være en fordel i at de semaforiske gestikke ikke nødvendigvis er fuldstændig naturlige, da helt naturlige gestikker også kan skabe problemer, \parencite[s. 9]{PDF:NaturalUserInterfaces}. \textcite[s. 9]{PDF:NaturalUserInterfaces} belyser problemet med de naturlige gestikker i forbindelse med bowling spillet til Nintendo Wii, som forårsagede en del ødelagte fjernsyn. Spillet skal gengive et helt almindeligt spil bowling hvor kontrollen svinges, ligesom bowlingkuglen. Så når spilleren normaltvist ville have sluppet bowlingkuglen, så kom spilleren naturligt til at slippe Nintendo Wii kontrollen, som derefter røg direkte ind i fjernsynet og ødelagde skærmen. Så det kan være værd at genoverveje hvor naturlige gestikkerne ønskes, da der altså kan være fordel ved at gøre dem mindre naturlige, som ved semaforiske gestikker.

En komplikation, der vil blive gået mere i detaljer med i HENVISNING TIL AFSNIT OM SOCAIL ACCEPT, vedrører at selvom systemer, der gør brug af gestikker kan anses for at have en mere naturlig interaktionsform fremfor den normale interaktion med tastatur og mus, så skal der tages højde for social accept, \parencite[s. 1]{PDF:WouldYouDoThat}. 
%
\subsection{Feedbackformer}
\label{Feedbackformer}
%
Det lader til at der er en del udbredte meninger om hvorvidt der skal være feedback i alle interaktionssituationer samt hvilken type feedback der skal anvendes. \textcite[s. 10]{PDF:NaturalUserInterfaces} argumentere for at der skal være feedback for på den måde at hjælpe brugeren til at forstå årsagen til eventuelle fejl og dertil lære den korrekte adfærd. Hvorimod \textcite[s. 16]{PDF:PIEmbeddingHCIOnTheRelevance} argumentere for at hvis gestikkerne er semaforiske, så er det ikke nødvendigt med, hvad \textcite[s. 16]{PDF:PIEmbeddingHCIOnTheRelevance} kategorisere som værende kunstig feedback, da brugeren automatisk får feedback fra den proprioceptive sans når der sker en aktivering af musklerne. På baggrund af det argumenterer \textcite[s. 16]{PDF:PIEmbeddingHCIOnTheRelevance} ydermere for, at det er derfor, at semaforiske gestikker egner sig til perifer interaktion. I forhold til de semaforiske gestikker så gav testpersonerne, i undersøgelsen foretaget af \textcite[ss. 172-173]{PDF:ComparingInputModalities}, udtryk for at de manglede haptisk feedback. Årsagen til det skyldes formegentligt at testpersonerne først og fremmest skal vænne sig til og stole på de semaforiske gestikker, hvorefter der formegentlig ikke længere vil være et behov for haptisk feedback, \parencite[s. 174]{PDF:ComparingInputModalities}. 

Derudover argumentere \textcite[s. 3]{PDF:FacilitatingPIDesignAndEvaluation} for, at det er problematisk at give brugeren feedback igennem den samme sensoriske modalitet, som hvor opgave befinder sig, da der kan opstå interferens. Derfor vil det være uhensigtmæssigt og mindre brugbart at anvende auditiv feedback når brugeren samtidig lytter til musik, \parencite[s. 3]{PDF:FacilitatingPIDesignAndEvaluation}. I tilfælde hvor brugeren perifert skal interagere med et musikanlæg eller en musikafspiller, kommenterer \textcite[s. 19]{PDF:PIEmbeddingHCIOnTheRelevance} at brugeren i forvejen modtager feedback i form af funktionel feedback, hvor brugeren kan høre at musikken stopper, at der skrues op eller ned for lyden eller at der skiftes musiknummer. Hvorimod hvis den perifere interaktion foregår igennem manipulerende gestikker så har ikke-visuel feedback potentiale for at være med til at minimere belastningen af de mentale ressourcer, \parencite[s. 3]{PDF:FacilitatingPIDesignAndEvaluation}. Dog giver testpersonerne, i undersøgelsen foretaget af \textcite[s. 173]{PDF:ComparingInputModalities}, udtryk for at de ikke manglede yderligere feedback end det de fik ved den funktionelle feedback. Hvor den funktionelle feedback i dette tilfælde vedrører den proprioceptive sans i og med at de manipulerer et fysisk objekt (knop-baseret håndtag og touchskærmen) og de kan høre at systemet reagerer på deres input ved enten at stoppe eller starte musikken, skrue op eller ned for lyden eller skifte musiknummer. 

\textcite[ss. 1263-1268]{PDF:ComparingModFeedback} undersøger om forskellige feedback typer, fra visuel feedback i det perifere til feedback direkte på skærmen, har en effekt på testpersonernes præstationsevne. Ifølge \textcite[ss. 1267-1268]{PDF:ComparingModFeedback} har feedback ingen effekt på præstationsevnen i den sekundære opgave, men testpersonerne efterspørger alligevel feedback så de kan få information omkring deres handlinger. Ligesom \textcite[s. 174]{PDF:ComparingInputModalities} så argumenterer \textcite[s. 1268]{PDF:ComparingModFeedback} for at det ikke nødvendigvis ville være tilfældet i en feltundersøgelse da testpersonerne da ville have mere tid til at vænne sig til den perifere interaktion.   
            








