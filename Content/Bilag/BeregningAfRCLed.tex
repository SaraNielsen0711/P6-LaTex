\chapter{Beregning af RC-led}
\label{app:BeregningAfRCLed}
 %  ______ _     _          _ 
 % |  ____| |   | |        | |
 % | |__  | |_  | | ___  __| |
 % |  __| | __| | |/ _ \/ _` |
 % | |____| |_  | |  __/ (_| |
 % |______|\__| |_|\___|\__,_|
\subsection{1 RC-led}
Når vi regner med et RC-led, skal vi sikre at vi får en forstærkning på 1, svarende til 0dB. 
%
\begin{equation}
	G_1 = 1 = \frac{10K\Omega}{10K\Omega}
\end{equation}
\noindent
%
Så finder vi tilbagekoblingsmodstanden ved $G_{1}$
%
\begin{equation}
	R_f' = R_i*G_1 = 10K\Omega*1 = 10K\Omega
\end{equation}
\noindent
%
Så kan vi beregne $R_{1}$
%
\begin{equation}
	R_1 = \frac{R_f*R_f'}{R_f-R_f'} = \frac{13.335K\Omega*10K\Omega}{13.335K\Omega-10K\Omega} = 39.983K\Omega
\end{equation}
\noindent
 %  _______      _          _ 
 % |__   __|    | |        | |
 %    | | ___   | | ___  __| |
 %    | |/ _ \  | |/ _ \/ _` |
 %    | | (_) | | |  __/ (_| |
 %    |_|\___/  |_|\___|\__,_|
\subsection{2 RC-led}
%
Når vi regner med to RC-led, så skal vi sikre at forstærkningen $G_{0}$ fordeles ligeligt mellem de to led, 
%
\begin{equation}
	G_0 = 2.5dB
\end{equation}
\noindent
%
Så kan vores forstærkning i db omregnes til almindelig tal
%
\begin{equation}
	G_0 = 10^{\frac{2.5}{20}} = 1.3335
\end{equation}
\noindent
%
Fordelingen af forstærkning regnes i dB
%
\begin{equation}
	G_1 = 2.5dB*\frac{1}{2} = 1.25dB
\end{equation}
\noindent
%
Så regner vi forstærkningen om fra dB til almindelige tal
%
\begin{equation}
	G_1 = 10^{\frac{1.25}{20}} = 1.1548
\end{equation}
\noindent
%
Så regner forstærkningen for det næste led, som skal gå mod 0
%
\begin{equation}
	G_2 = 2.5dB*0 = 0dB
\end{equation}
\noindent
%
Vores forstærkning kan så omregnes fra dB til almindelige tal
%
\begin{equation}
	G_2 = 10^{\frac{0}{20}} = 1
\end{equation}
\noindent
%
Resultatet angiver hvor meget hvert af de to RC-led skal forstærke. Det er nu muligt at beregne tilbagekoblingsmodstanden for det første led
%
\begin{equation}
	R_f' = R_i*G_1 = 10K\Omega*1.1548 = 11.548K\Omega
\end{equation}
\noindent
%
Så kan vi beregne tilbagekoblingsmodstanden for det andet led
%
\begin{equation}
	R_f'' = R_i*G_2 = 10K\Omega*1 = 10K\Omega
\end{equation}
\noindent
%
Så kan vi regne $R_{1}$
%
\begin{equation}
	R_1 = \frac{R_f*R_f'}{R_f-R_f'} = \frac{13.335K\Omega*11.548K\Omega}{13.335K\Omega-11.548K\Omega} = 86.155K\Omega
\end{equation}
\noindent
%
Så kan vi regne $R_{2}$
%
\begin{equation}
	R_2 = \frac{R_f'*R_f''}{R_f'-R_f''} = \frac{11.548K\Omega*10K\Omega}{11.548K\Omega-10K\Omega} = 74.607K\Omega
\end{equation}
\noindent
 %  _______         _          _ 
 % |__   __|       | |        | |
 %    | |_ __ ___  | | ___  __| |
 %    | | '__/ _ \ | |/ _ \/ _` |
 %    | | | |  __/ | |  __/ (_| |
 %    |_|_|  \___| |_|\___|\__,_|
\subsection{3 RC-led}
%
Når vi regner med tre RC-led, så skal vi sikre at forstærkningen $G_{0}$ fordeles ligeligt mellem de tre led, 
%
\begin{equation}
	G_0 = 2.5dB
\end{equation}
\noindent
%
Så kan vores forstærkning i db omregnes til almindelig tal
%
\begin{equation}
	G_0 = 10^{\frac{2.5}{20}} = 1.3335
\end{equation}
\noindent
%
Fordelingen af forstærkning regnes i dB
%
\begin{equation}
	G_1 = 2.5dB*\frac{2}{3} = 1.667dB
\end{equation}
\noindent
%
Så regner vi forstærkningen om fra dB til almindelige tal
%
\begin{equation}
	G_1 = 10^{\frac{1.667}{20}} = 1.2115
\end{equation}
\noindent
%
Så regner vi forstærkningen for det næste led
%
\begin{equation}
	G_2 = 2.5dB*\frac{1}{3} = 0.833dB
\end{equation}
\noindent
%
Vores forstærkning kan så omregnes fra dB til almindelige tal
%
\begin{equation}
	G_2 = 10^{\frac{0.833}{20}} = 1.1007
\end{equation}
\noindent
%
Så regner vi forstærkningen fra det tredje led mod 0
%
\begin{equation}
	G_3 = 2.5dB*0 = 0dB
\end{equation}
\noindent
%
Vores forstærkning kan så omregnes fra dB til almindelige tal
%
\begin{equation}
	G_3 = 10^{\frac{0}{20}} = 1
\end{equation}
\noindent
%
Det er nu muligt at beregne tilbagekoblingsmodstanden for det første led
%
\begin{equation}
	R_f' = R_i*G_1 = 10K\Omega*1.212 = 12.115K\Omega
\end{equation}
\noindent
%
Så kan vi beregne tilbagekoblingsmodstanden for det andet led
%
\begin{equation}
	R_f'' = R_i*G_2 = 10K\Omega*1.1007 = 11.007K\Omega
\end{equation}
\noindent
%
Så kan vi beregne tilbagekoblingsmodstanden for det sidste led
%
\begin{equation}
	R_f''' = R_i*G_3 = 10K\Omega*1 = 10K\Omega
\end{equation}
\noindent
%
Så kan vi regne $R_{1}$
%
\begin{equation}
	R_1 = \frac{R_f*R_f'}{R_f-R_f'} = \frac{13.335K\Omega*12.115K\Omega}{13.335K\Omega-12.115K\Omega} = 132.433K\Omega
\end{equation}
\noindent
%
Så kan vi regne $R_{2}$
%
\begin{equation}
	R_2 = \frac{R_f'*R_f''}{R_f'-R_f''} = \frac{12.115K\Omega*11.007K\Omega}{12.115K\Omega-11.007K\Omega} = 120.318K\Omega
\end{equation}
%
Så kan vi regne $R_{3}$
%
\begin{equation}
	R_3 = \frac{R_f''*R_f'''}{R_f''-R_f'''} = \frac{11.007K\Omega*10K\Omega}{11.007K\Omega-10K\Omega} = 109.311K\Omega
\end{equation}
 %  ______ _            _          _ 
 % |  ____(_)          | |        | |
 % | |__   _ _ __ ___  | | ___  __| |
 % |  __| | | '__/ _ \ | |/ _ \/ _` |
 % | |    | | | |  __/ | |  __/ (_| |
 % |_|    |_|_|  \___| |_|\___|\__,_|
\subsection{4 RC-led}
%
Når vi regner med fire RC-led, så skal vi sikre at forstærkningen $G_{0}$ fordeles ligeligt mellem de fire led, 
%
\begin{equation}
	G_0 = 2.5dB
\end{equation}
\noindent
%
Så kan vores forstærkning i db omregnes til almindelig tal
%
\begin{equation}
	G_0 = 10^{\frac{2.5}{20}} = 1.3335
\end{equation}
\noindent
%
Fordelingen af forstærkning regnes i dB
%
\begin{equation}
	G_1 = 2.5dB*\frac{3}{4} = 1.875dB
\end{equation}
\noindent
%
Så regner vi forstærkningen om fra dB til almindelige tal
%
\begin{equation}
	G_1 = 10^{\frac{1.875}{20}} = 1.2409
\end{equation}
\noindent
%
Så regner vi forstærkningen for det næste led
%
\begin{equation}
	G_2 = 2.5dB*\frac{2}{4} = 1.250dB
\end{equation}
\noindent
%
Vores forstærkning kan så omregnes fra dB til almindelige tal
%
\begin{equation}
	G_2 = 10^{\frac{1.250}{20}} = 1.1548
\end{equation}
\noindent
%
Så regner vi forstærkningen fra det tredje led 
%
\begin{equation}
	G_3 = 2.5dB*\frac{1}{4} = 0.625dB
\end{equation}
\noindent
%
Vores forstærkning kan så omregnes fra dB til almindelige tal
%
\begin{equation}
	G_3 = 10^{\frac{0.625}{20}} = 1.0746
\end{equation}
\noindent
%
Så regner vi forstærkningen fra det fjerde led, mod 0 
%
\begin{equation}
	G_4 = 2.5dB*0 = 0dB
\end{equation}
\noindent
%
Vores forstærkning kan så omregnes fra dB til almindelige tal
%
\begin{equation}
	G_4 = 10^{\frac{0}{20}} = 1
\end{equation}
\noindent
%
Det er nu muligt at beregne tilbagekoblingsmodstanden for det første led
%
\begin{equation}
	R_f' = R_i*G_1 = 10K\Omega*1.2409 = 12.409K\Omega
\end{equation}
\noindent
%
Så kan vi beregne tilbagekoblingsmodstanden for det andet led
%
\begin{equation}
	R_f'' = R_i*G_2 = 10K\Omega*1.1548 = 11.548K\Omega
\end{equation}
\noindent
%
Så kan vi beregne tilbagekoblingsmodstanden for det tredje led
%
\begin{equation}
	R_f''' = R_i*G_3 = 10K\Omega*1.0746 = 10.746K\Omega
\end{equation}
\noindent
%
Så kan vi beregne tilbagekoblingsmodstanden for det sidste led
%
\begin{equation}
	R_f'''' = R_i*G_4 = 10K\Omega*1 = 10K\Omega
\end{equation}
\noindent
%
Så kan vi regne $R_{1}$
%
\begin{equation}
	R_1 = \frac{R_f*R_f'}{R_f-R_f'} = \frac{13.335K\Omega*12.409K\Omega}{13.335K\Omega-12.409K\Omega} = 178.737K\Omega
\end{equation}
\noindent
%
Så kan vi regne $R_{2}$
%
\begin{equation}
	R_2 = \frac{R_f'*R_f''}{R_f'-R_f''} = \frac{12.409K\Omega*11.548K\Omega}{12.409K\Omega-11.548K\Omega} = 166.328K\Omega
\end{equation}
%
Så kan vi regne $R_{3}$
%
\begin{equation}
	R_3 = \frac{R_f''*R_f'''}{R_f''-R_f'''} = \frac{11.548K\Omega*10.746K\Omega}{11.548K\Omega-10.746K\Omega} = 154.780K\Omega
\end{equation}
%
Så kan vi regne $R_{4}$
%
\begin{equation}
	R_4 = \frac{R_f'''*R_f''''}{R_f'''-R_f''''} = \frac{10.746K\Omega*10K\Omega}{10.746K\Omega-10K\Omega} = 144.034K\Omega
\end{equation}
 %  ______               _          _ 
 % |  ____|             | |        | |
 % | |__ ___ _ __ ___   | | ___  __| |
 % |  __/ _ \ '_ ` _ \  | |/ _ \/ _` |
 % | | |  __/ | | | | | | |  __/ (_| |
 % |_|  \___|_| |_| |_| |_|\___|\__,_|
 \subsection{5 RC-led}
%
Når vi regner med fire RC-led, så skal vi sikre at forstærkningen $G_{0}$ fordeles ligeligt mellem de fire led, 
%
\begin{equation}
	G_0 = 2.5dB
\end{equation}
\noindent
%
Så kan vores forstærkning i db omregnes til almindelig tal
%
\begin{equation}
	G_0 = 10^{\frac{2.5}{20}} = 1.3335
\end{equation}
\noindent
%
Fordelingen af forstærkning regnes i dB
%
\begin{equation}
	G_1 = 2.5dB*\frac{4}{5} = 2dB
\end{equation}
\noindent
%
Så regner vi forstærkningen om fra dB til almindelige tal
%
\begin{equation}
	G_1 = 10^{\frac{2}{20}} = 1.2589
\end{equation}
\noindent
%
Så regner vi forstærkningen for det næste led
%
\begin{equation}
	G_2 = 2.5dB*\frac{3}{5} = 1.5dB
\end{equation}
\noindent
%
Vores forstærkning kan så omregnes fra dB til almindelige tal
%
\begin{equation}
	G_2 = 10^{\frac{1.5}{20}} = 1.1885
\end{equation}
\noindent
%
Så regner vi forstærkningen fra det tredje led 
%
\begin{equation}
	G_3 = 2.5dB*\frac{2}{5} = 1dB
\end{equation}
\noindent
%
Vores forstærkning kan så omregnes fra dB til almindelige tal
%
\begin{equation}
	G_3 = 10^{\frac{1}{20}} = 1.122
\end{equation}
\noindent
%
Så regner vi forstærkningen fra det fjerde led, mod 0 
%
\begin{equation}
	G_4 = 2.5dB*0 = 0.5dB
\end{equation}
\noindent
%
Vores forstærkning kan så omregnes fra dB til almindelige tal
%
\begin{equation}
	G_4 = 10^{\frac{0.5}{20}} = 1.059
\end{equation}
\noindent
%
Så regner vi forstærkningen fra det femte led, mod 0 
%
\begin{equation}
	G_4 = 2.5dB*0 = 0dB
\end{equation}
\noindent
%
Vores forstærkning kan så omregnes fra dB til almindelige tal
%
\begin{equation}
	G_5 = 10^{\frac{0}{20}} = 1
\end{equation}
\noindent
%
Det er nu muligt at beregne tilbagekoblingsmodstanden for det første led
%
\begin{equation}
	R_f' = R_i*G_1 = 10K\Omega*1.2409 = 12.589K\Omega
\end{equation}
\noindent
%
Så kan vi beregne tilbagekoblingsmodstanden for det andet led
%
\begin{equation}
	R_f'' = R_i*G_2 = 10K\Omega*1.1548 = 11.885K\Omega
\end{equation}
\noindent
%
Så kan vi beregne tilbagekoblingsmodstanden for det tredje led
%
\begin{equation}
	R_f''' = R_i*G_3 = 10K\Omega*1.0746 = 11.22K\Omega
\end{equation}
\noindent
%
Så kan vi beregne tilbagekoblingsmodstanden for det fjerde led
%
\begin{equation}
	R_f'''' = R_i*G_4 = 10K\Omega*1 = 10.593K\Omega
\end{equation}
\noindent
%
%
Så kan vi beregne tilbagekoblingsmodstanden for det sidste led
%
\begin{equation}
	R_f''''' = R_i*G_5 = 10K\Omega*1 = 10K\Omega
\end{equation}
\noindent
%
Så kan vi regne $R_{1}$
%
\begin{equation}
	R_1 = \frac{R_f*R_f'}{R_f-R_f'} = \frac{13.335K\Omega*12.589K\Omega}{13.335K\Omega-12.589K\Omega} = 225.053K\Omega
\end{equation}
\noindent
%
Så kan vi regne $R_{2}$
%
\begin{equation}
	R_2 = \frac{R_f'*R_f''}{R_f'-R_f''} = \frac{12.589K\Omega*11.885K\Omega}{12.589K\Omega-11.885K\Omega} = 212.464K\Omega
\end{equation}
%
Så kan vi regne $R_{3}$
%
\begin{equation}
	R_3 = \frac{R_f''*R_f'''}{R_f''-R_f'''} = \frac{11.885K\Omega*11.22K\Omega}{11.885K\Omega-11.22K\Omega} = 200.578K\Omega
\end{equation}
%
Så kan vi regne $R_{4}$
%
\begin{equation}
	R_4 = \frac{R_f'''*R_f''''}{R_f'''-R_f''''} = \frac{11.22K\Omega*10.593K\Omega}{11.22K\Omega-10.593K\Omega} = 189.358K\Omega
\end{equation}
%
Så kan vi regne $R_{5}$
%
\begin{equation}
	R_5 = \frac{R_f''''*R_f'''''}{R_f''''-R_f'''''} = \frac{10.593K\Omega*10K\Omega}{10.593K\Omega-10K\Omega} = 178.766K\Omega
\end{equation}