\section{Interaktion med gestikker}
\label{TestresultaterInteraktioner}
%
Inden testpersonerne præsenteres for videooptagelserne af de foreslåede semaforiske gestikker, bliver de bedt om at udfylde et spørgeskema, hvori de, blandt andet, spørges ind til deres erfaring med semaforiske gestikker. Hertil responderede 12 testpersoner, at de før har brugt gestikker, hvor de ikke skulle røre ved en skærm. De testpersoner, der responderede at de tidligere har brugt gestikker, blev efterfølgende spurgt ind til hvilke produkter, de har brugt semaforiske gestikker. Af de 12 testpersoner, som har tidligere erfaringer med semaforiske gestikker, har tre brugt Xbox Kinect, ni har brugt Nintendo Wii, seks har brugt smartphones, fem har brugt tablets og to har brugt Virtual Reality briller. Tre testpersoner afkrydsede andet-kategorien og angav at de har brugt henholdsvis et Myo armbånd eller EyeToy. Ud af de 12 testpersoner følte én sig meget erfaren med semaforiske gestikker, fem følte sig erfarne, fem følte sig uerfarne og én følte sig meget uerfaren. Efterfølgende skal testpersonerne angive hvordan de betragter sig selv i forhold til hvor hurtigt de anskaffer sig den nyeste teknologi. Hertil responderede ni, at de betragter sig som værende lige så hurtig som alle andre til at anskaffe sig den nyeste teknologi og ni responderede at de betragter sig som værende de sidste til at anskaffe sig den nyeste teknologi. Information som denne sammenholdes efterfølgende med testpersonernes holdning til hvordan det er, at interagere med en brugergrænseflade via semaforiske gestikker.\blankline
%
Efter udvælgelsen af hvilke semaforiske gestikker testpersonerne foretrækker at knytte til henholdvis pause og start, skift musiknummer samt justering af lydstyrken, blev testpersonerne, i exit-interviewet, udspurgt om hvordan de forholder sig til denne interaktionsform. 14 testpersoner giver udtryk for, at de godt kan lide idéen om at interagere via semaforiske gestikker, hvoraf nogle kommenterede, at det selvfølgelig er med forbehold for, at det virker hver gang og at systemet kun reagerer på ens gestikker. Ud af de 14 testpersoner forestiller ni testpersoner sig, at de vil gøre brug af denne slag interaktion.

TP3 og TP17 giver begge udtryk for, at de ikke vil bruge semaforiske gestikker, hvortil TP17 pointerer at det på nuværende tidspunkt ikke er aktuelt, da det er lige så nemt at trykke på en telefon. Dog afviser testpersonen ikke, at det på sigt kan blive aktuelt. TP3 føler sig lidt for gammeldags til at bruge den slags interaktion, men kan godt se det smarte i ikke at skulle rejse sig eller lede efter fjernbetjeningen. Begge testpersoner betragter sig selv som værende de sidste til at anskaffe sig den nyeste teknologi og TP17, der har prøvet at interagere med semaforiske gestikker før, føler sig meget uerfaren. TP15 giver udtryk for ikke at have behov for at interaktionen skal foregå på det niveau, hvor niveau referer til brugen af semaforiske gestikker. I tillæg kommenterer TP15, at det ikke er et produkt testpersonen vil købe. TP15's synspunkter stemmer overens med hvad testpersonen responderer i spørgeskemaet, hvor testepersonen betragter sig som værende den sidste til at anskaffe sig den nyeste teknologi og føler sig derudover uerfaren i brugen af semaforiske gestikker. TP8, der føler sig erfaren i brugen af semaforiske gestikker samt betragter sig som værende den sidste til at anskaffe sig den nyeste teknologi, giver udtryk for, at have brug for en tilvænningsperiode, hvis interaktionen skal foregå med semaforiske gestikker.\blankline
%
Ud af de ni testpersoner, der godt kan forestille sig at bruge interaktionsformen, anser tre af dem sig som værende de sidste til at anskaffe sig den nyeste teknologi. Derudover har otte ud af ni testpersoner tidligere erfaringer med semaforiske gestikker, hvoraf én føler sig meget erfaren, fire føler sig erfarne og tre føler sig uerfaren. Selvom ingen af de 18 testpersoner betragter sig som værende de første til at anskaffe sig den nyeste teknologi, så er størstedelen af testpersonerne åbne omkring denne interaktionsform. I tillæg giver ni testpersoner udtryk for, at de godt kan forestille sig at eje et produkt, hvor interaktionen foregår via semaforiske gestikker.

Det tyder derfor på, at der er en interesse for at erhverve sig denne type produkter, selvom brugeren ikke nødvendigvis betragter sig selv som værende den første til at anskaffe sig den nyeste teknologi. Selv de testpersoner, som giver udtryk for ikke at bryde sig om, at skulle interagere via semaforiske gestikker, giver udtryk for, at det er en fordel ikke at skulle lede efter sin fjernbetjening eller røre ved en skærm.\blankline 
%
Foruden den positive indstilling rettet mod de semaforiske gestikker, så giver testpersonerne dog udtryk for bekymringer, der bør tages til eftertragtning. Ifølge TP1 skal der kun være et begrænset antal funktioner, som er tilknyttet semaforiske gestikker. TP17 og TP18 efterspørger begge, at der stadig er mulighed for at interagere med produktet, eksempelvis via fysiske knapper eller anden form for nærbetjening.  

Da det er besluttet kun at udvælge semaforiske gestikker til de seks mest gængse funktioner på et musikanlæg, vil det være en fordel at have en tilhørende applikation, som både indeholder de seks valgte funktioner; pause, start, næste musiknummer, forrige musiknummer, justering af lydstyrke samt de resterende funktioner Bang $\&$ Olufsen gør brug af, jævnfør \fullref{SamspilMedBO}.\blankline
% 
TP2 har en vigtig pointe i forhold til, at anvende semaforiske gestikker, som interaktionsmulighed: 
%
\begin{quotation}
\noindent
\textit{Hvis det fungerer lige så godt som at trykke på en knap, så ja, så kunne jeg godt finde på at bruge det.} TP2, \autoref{app:NoterValgAfGestikker}. 
\end{quotation}
%
Det er en vigtig pointe fordi det er essentielt, at de semaforiske gestikker enten fungerer lige så godt eller bedre end hvordan interaktionen foregår nu. Er det ikke tilfældet kan brugeren lige så godt bruge sin telefon eller sin fjernbetjeningen til at styre musikanlægget, hvilket potentielt kan resulterer i at brugeren ikke længere anvender produktet og måske føler at produktet har været et fejlkøb og i værste fald fraråder andre at købe det. 

Det er en gennegående tendens, at testpersonerne kun vil anvende denne type produkt såfremt, at det fungerer hver gang og der ikke opstår tvivl om hvorvidt en gestik er registreret eller ej og om gestikken overhovedet er en gestik, som er henvendt til produktet fremfor en gestik i kropssproget.
