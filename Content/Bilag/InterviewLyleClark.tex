\chapter{Interview med Bang $\&$ Olufsen}
\label{app:InterviewLyleClarke}
%
Følgende notater er foretaget under et møde med Lars Bo Larsen, Lyle Clarke, Kashmiri Stec, Lucca Julie Nellemann og Sara Nielsen. Mødet blev afholdt den 27. Februar 2017 i Bang $\&$ Olufsens egne mødelokaler. \blankline
% 
\textbf{Har du et dokument, som vi kan henvise til i forhold til B$\&$O’s funktioner, og som forklare hvad de er?}\\
Henviser til Lyle Clarke, interview med B$\&$O den 27/2. OBS: Se længere nede i dokumentet.\blankline
%
\textbf{Har du noget generelt om målgrupper?}\\ 
Ja, men I må ikke referer til det alligevel. \blankline
%
\textbf{Har du præferencer til hvem vi tester på?}\\
I skal ikke teste på nogen, som går mere op i at have fuld kontrol over hvad der sker. Med andre ord skal i ikke teste på nogen fra elektronik, da de fokuserer mere på teknologien bagved end på oplevelsen. 

Derudover synes Lyle, at det er rigtig fint at teste på studerende.\blankline
%  
\textbf{Har du noget om målgruppe specifikt til det her produkt?}\\
Folk som køber musik og går op i design og som ikke vil have at teknologi kommer i vejen\blankline
%
\textbf{Hvem regner I med vil bruge B$\&$O og det produkt vi arbejder med?}\\
Se ovenfor 
Folk der køber musik, så hver billede er en plade. \blankline
%
\textbf{Hvor mange af B$\&$O’s funktioner forestiller I jer skal være perifere?}\\ 
Top elementer uden like og dislike \blankline
%
Prioritering:\\
Start\\
Pause \\
Volumen op/ned (cirkel bevægelse, men en finger - kontinuert fremfor håndledet der har begrænsninger)\\
Skift frem/tilbage\blankline
%
Man opgiver kontrol ved at lave justeringen/interaktionen med gestures. 
Gestures skal være tillærte, så man ikke laver dem ved en fejl. 
Vi kommer nok til at ødelægge interaktionen med et andet menneske, men vi skal kigge på hvad alternativerne er. Det er bedre at kigge op på musikanlægget og lave en gesture end at bruge stemmestyring, en app eller at gå hen til den. \blankline
%
\textbf{Lyle spørger: Er den always on eller er den state styret?}\\
Der skal vi lige finde ud af hvad vi vil. \\
Det kan man også lave brugertest over. \\
B$\&$O vil gerne have den er state styret.\blankline
%
\textbf{Får den en app?}\\
Det gør den sikkert\blankline
%
\textbf{Kan vi låne en BeoSound essence?}\\
Ja.\blankline
%
\textbf{Andre noter:} \\
To niveauer - langt fra gestures, tæt på - brug skærmen \\
Noget der kommer frem på skærmen, når man er langt fra - feedback. \\
Hvordan skal den vide at det er den vi henvender os til? \\
Feedback - highlights, der er jo delays. \\

