\chapter{Projekt afgrænsing}
\label{Afgraensning}
%
I følgende afsnit opsummeres og specificeres, hvilken tilgang til perifer interaktion med et musikanlæg, der fremadrettet vil blive fokuseret på i samarbejdet med Bang $\&$ Olufsen. Dertil vil der dels blive afgrænset fra nogle af de føromtalte problemstillinger, som fremgår af de forgående afsnit, og dels præciseres hvilke der arbejdes videre med, til det vil der blive udarbejdet en problemformulering.\blankline
% 
Da der ikke findes én universal måde, at interagere med produkter i den perifere opmærksomhed, så er det fordelagtigt, at interaktionen med et specifikt produkt undersøges nærmere. Der vil derfor kun blive fokuseret på den perifere interaktion med et musikanlæg fra Bang $\&$ Olufsen. På baggrund af \fullref{PeriferInteratkion} fremgår det, at for bedst at understøtte perifer interaktion, er det er en fordel at undgå, at interaktionen skal foregå i den visuelle opmærksomhed. Ved at afgrænse sig fra interaktion i den visuelle opmærksomhed, vil det i højere grad være muligt, at designe en interaktionsform, som ikke længere kræver at brugeren allokere sin centrale opmærksomhed og dermed afbryder sin primære opgave midlertidigt. Derudover så tyder det på, at der til perifer interaktion kan drages stor nytte af den proprioceptive sans, som netop tillader interaktion uafhængigt af den visuelle opmærksomhed, jævnfør \fullref{Potentiale}. Der bliver ydermere afgrænset fra stemmestyring, dels fordi det anses for at være for krævende og derfor ikke kan foregå i det perifere, \parencite[s. 41]{PDF:PIEmbeddingHCIMicroManageMe}, og dels fordi det betragtes, som værende mere pinligt end at udføre gestikker, \parencite[s. 4]{PDF:AnExploratoryStudy}.        

Som belyst i \fullref{KategoriseringerAfGestikker} findes der forskellige kategorier af gestikker, hvoraf nogen egner sig bedre til perifer interaktion end andre. Til interaktion med Bang $\&$ Olufsens fremtidige musikanlæg tyder det på, at semaforiske gestikker vil være gunstige. Det understøttes, blandt andet, af at det er en af de interaktionsformer, som andre har haft stor gavn af, jævnfør \fullref{RelateretUndersoegelser}. Derudover understøtter semaforiske gestikker den proprioceptive sans, da de ikke kræver, at brugeren skal være i besidelse af endnu en form for fjernbetjening og det er muligt at interaktionen kan foregå i den perifere opmærksomhed. Ydermere blev det uddybet, at gestikker kan kategoriseres efter bevægelsemængde i henholdsvis mikro- og makrogestikker. Da det forventes, at interaktionen med musikanlægget kan foregå ved forskellige afstande og op mod fire meter til brugeren, er det ikke på nuværende tidspunkt muligt at afgrænse sig til udelukkende at arbejde med enten mikro- eller makrogestikker. Der kan dog være en fordel i, særligt i forhold til social accept og hvorvidt brugeren føler sig komfortabel med at udføre de semaforiske gestikker, at designe de semaforiske gestikker så de kan foregå tæt på brugerens egen krop. Det tilstræbes derfor, at komme frem til en form for kompromis mellem decideret mikro- og makrogestikker, da de hver især bidrager med dels noget positivt og dels med noget negativt, jævnfør \fullref{UdfoerelseAfGestik}.\blankline
%
I \fullref{KomplikationerGestikker} præsenteres nogle af de komplikationer, som gestikker kan forårsage, det gælder både teknologiske såvel som menneskelige komplikationer. Der afgrænses i dette projekt fra, at undersøge det tekniske aspekt nærmere, da formålet med projektet primært retter sig mod, at finde en måde hvorpå brugerene kan interagere med Bang $\&$ Olufsens fremtidige musikanlæg perifert. Da Bang $\&$ Olufsens fremtidige musikanlæg på nuværende tidspunkt befinder sig på konceptbasis, er det ikke muligt at foretage feltundersøgelser ej heller udføre Hi-Fi tests, dog vil det blive forsøgt, at tilstræbe et så naturligt testscenarie, som muligt. Selvom der afgrænses fra det teknologiske aspekt tilstræbes det dog, at de semaforiske gestikker ikke designes til stor hinder for teknologien. Det forventes derfor ikke at teknologien kan registre semaforiske gestikker, som enten er delvist eller fuldstændig dækket af andre objekter.  

At anvende semaforiske gestikker, som interaktionsform i den perifere opmærksomhed er stadig forholdvist nyt, så der findes endnu ikke et standard sæt af gestikker, som henvender sig til musikkontrol, i hvert fald ikke hvad der er projektgruppen kendt. Det vil derfor være nødvendigt, at foretage en dybere undersøgelse af netop hvilke semaforiske gestikker, der egner sig bedst til nogen af de mest gængse funktioner et musikanlæg indeholder; start, pause, skru op og ned og skift musiknummer frem og tilbage. Ydermere er det interessant at undersøge, om der findes semaforiske gestikker, der trods deres unaturlighed kan læres og genkaldes og på den måde benyttes til perifer interaktion. Der skal dog tages højde for, at disse gestikker ikke skal afspejle almindelige, naturlige gestikulerende gestikker, der eksempelvis opstår ved samtale, da risikoen for at udføre en gestik ved en fejl vil stige. Det findes derfor interessant at undersøge lige netop det område, hvor semaforiske gestikker fejlfrit kan benyttes til HCI i den perifere opmærksomhed. \blankline
%
Undersøges behovet for feedback ved perifer interaktion, er der, som beskrevet i \fullref{Feedbackformer}, delte meninger. Der bliver både argumenteret for, at det ikke er nødvendigt med augmenteret feedback ved semaforiske gestikker, grundet den funktionelle feedback fra den proprioceptive sans og fordi brugeren ydermere får funktionel feedback i og med at musikken ændre sig afhængigt af inputtet. Der bliver også argumenteret for at den funktionelle feedback ikke er nok, for selvom feedback ikke nødvendigvis fremmer brugerens præstationsevne, så er feedbacken med til, at hjælpe dem igennem interaktionen. Fokus for dette projekt vil i første omgang ikke være rettet direkte mod at designe en bestemt form for feedback, men feedback vil undervejs blive holdt in mente, såfremt det skulle blive relevant. Det skyldes, blandt andet, at \textcite[s. 21]{PDF:FacilitatingPIDesignAndEvaluation} hævder at multimodal feedback, som er feedback der registreres af flere sanser, styrker perifer interaktion. Da der ikke afgrænses fuldstændigt fra at arbejde med feedback og det samtidig er et betydningsfyldt aspekt inden for interaktions design, så vil feedback blive yderligere diskuteret fremadrettet.  

Det er ikke muligt at forudbestemme præcist, hvordan gestikkerne skal udføres, da der skal tages højde for dels, om brugeren føler sig komfortable ved at udføre dem og dels, om mulige tilskuer oplever denne nye form for interaktion, jævnfør \fullref{Socialaccept}. På nuværende tidspunkt kan der med sikkerhed afgrænses fra at bruge spændingsfyldte gestikker, da disse ikke anses for at være socialt acceptable, \parencite[s. 277]{PDF:WouldYouDoThat} og hemmelighedsfyldte gestikker, da det ikke vil give meget mening ved interaktion med et musikanlæg, fordi det vil være svært og uhensigtmæssigt at skjule effekten af en gestik. Tilbage er der de udtryksfyldte- og magiske gestikker, som begge egner sig til interaktion med et musikanlæg og til perifer interaktion. Derudover lægger Bang $\&$ Olufsen vægt på, at brugeroplevelsen føles magisk, så det vil give mening, at undersøge om de magiske gestikker, som er defineret af \textcite[s. 276]{PDF:WouldYouDoThat} til at være gestikker hvor manipulationen er skjult og effekten er synlig, egner sig til at fremme denne brugeroplevelse. \blankline
%
Projektet er nu afgrænset til at undersøge, hvordan perifer interaktion kan foregå med semaforiske gestikker, når et musikanlæg skal styres fra en vilkårlig afstand, hvilket danner grundlag for en problemformulering.
