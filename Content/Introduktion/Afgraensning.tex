\section{Afgrænsing}
\label{Afgraensning}
%
Når der skal designes produkter, som skal interageres med i den perifere del af opmærksomheden, kan det være relevant at undersøge hvordan denne interaktion, i form af en sideopgave, kan udføres sideløbende med en primær opgave. Som nævnt i \fullref{PeriferInteratkion} afgrænses der derfor til kun at arbejde med én specifik sideopgave. Derudover afgrænses der fra at interaktionen vil foregå i den visuelle opmærksomhed eller ved brug af stemmestyring. Fremadrettet vil det altså blive undersøgt, hvordan den proprioceptive sans kan bruges til interaktion i det perifere. 

Som nævnt i \fullref{KategoriseringerAfGestikker} findes der flere typer af gestikker, hvoraf nogen af dem ikke egner sig til interaktion i det perifere. Til interaktion med Bang $\&$ Olufsens interaktive kunstværk virker det logisk at bruge semaforiske gestikker til interaktionen, da interaktionen på den måde ikke er bestemt af, om det pågældende interaktionsredskab er i nærheden. I den sammenhæng bliver der også afgrænset til at bruge \textit{Perceptaul Input}, da denne slags input lægger godt op til interaktion med semaforiske gestikker.

Som beskrevet kan gestikulationer inddeles i mikro- og makrogestikker. Til styring af Bang $\&$ Olufsens produkt afgrænses der fra at bruge makrogestikker, da disse hurtigt kan blive voldsomme til formålet. Da brugssituationen ofte vil være et stykke væk fra det interaktive kunstværk kan det som nævnt i \fullref{KategoriseringerAfGestikker} være en ulempe at bruge helt små mikrogestikker. Det er dog ønsket at bruge nogle mere præcise gestikulationer, som hører ind under mikrogestikker og da mikrogestikker egner sig godt til et mere professionelt udtryk, \parencite[s. 10]{PDF:UsabilityofMicroVsMacroGestures} kategoriseres de anvendte gestikker fremadrettet som mikrogestikker. 

I \fullref{KomplikationerGestikker} er der beskrevet hvilke tekniske og menneskelige komplikationer der kan opstå ved brug af gestikulationer til perifer interaktion. Der afgrænses i dette projekt fra at undersøge på det tekniske aspekt og fokuseres derimod på det menneskelige aspekt og de komplikationer der hertil kan opstå. Derudover afgrænses der fra at køre Hi-Fi- og in-situ-forsøg, da det interaktive kunstværk stadig er på konceptbasis og teknologien til et in-situ-forsøg ikke er til rådighed. Da det er relativt nyt at bruge gestikker til interaktion med produkter, findes der ikke nødvendigvis et endegyldigt svar på hvilke gestikker brugeren ønsker at bruge. Der afgrænses derfor til at kigge på hvilke gestikulationer brugeren gerne vil have til at styre musik gennem et interaktivt kunstværk, herunder gestikulationer knyttet til de funktioner det giver mening at styre perifert. 

Da der af afgrænset til at bruge semaforiske gestikker, der som nævnt i \fullref{KomplikationerVedroerendeDetMenneskelige} kan opfattes unaturlige, er det interessant at undersøge om der kan findes semaforiske gestikker, der trods deres unaturlighed kan læres og genkaldes i den perifere del af opmærksomheden. Dog skal gestikulationerne ikke lægge sig så meget op af almindelige menneske-til-menneske gestikker eller andre naturlige gestikker, at de bliver lavet ved en fejl. Det findes altså interessant at undersøge lige netop det område, hvor semaforiske gestikker kan bruges til fejlfri HCI i den perifere del af hukommelsen. 

tag det slavisk, opsummere det hele.
 vi bruger semaforiske gestikker, der kan være unaturlige, men det er måske meget godt, fordi vi vil gerne finde de gestikker kan man lave til et altid tændt anlæg, som man ikke skal bruge energi på at genkalde eller kommer til at lave uprovokeret. 

%Ifølge \textcite[s. 21]{PDF:FacilitatingPIDesignAndEvaluation} er feedback fra produktet med til at forbedre interaktionen. Ved et interaktivt kunstværk, der styrer musikken vil der naturligvis være feedback i form af musikken, der spiller og ændrer sig, men da systemet ikke nødvendigvis kan reagere med det samme, kan brugeren eventuelt have brug for feedback fra en kombination af modaliteter. 

%Der lægges op til flere undersøgelser af perifer interaktion, hvilke præcise semaforiske gestikker, der skal bruges og hvilken feedback brugeren har brug for, for at skabe en interaktion, der er både lige til, socialt acceptabelt og magisk.


afgræns feedback - diskuter hvorvidt der skal være feedback. Fordi vi arbejder med musik på den måde, så kan vi ikke give feedback gennem det auditive, vi afgrænser os fra visuelt, så måske er det her også dumt?
social accept - det giver ikke så meget mening at lave spændingsfyldte gestikker. i sammenhæng med de andet, så giver det mening at brugei de her gestikker. 
relaterede produkter - de fleste henviser til, når man sidder ved computerskærmen. Det er noget andet vi vil undersøge. 