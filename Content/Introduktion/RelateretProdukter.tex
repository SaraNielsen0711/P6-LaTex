\section{Relateret produkter og studier}
\label{RelateretProdukterOgStudier}

I det følgende afsnit er der kigget på hvilke relaterede produkter der findes, samt hvordan perifer interaktion tidligere er blevet studeret og brugt til at styre musik eller lignende.

Tages der udgangspunkt i konceptet omkring et interaktivt kunstværk til væggen findes der flere elektroniske billedrammer på markedet. Nogle af disse billedrammer kan endda styres med gestikker, herilandt Merual \parencite{WEB:Meural} og Framed 2.0 \parencite{WEB:Framed2.0}. Begge disse er billedrammer er en platform, hvor brugeren selv kan vælge, hvad der skal ske på displayet. Den viste kunst kan findes i en database, hvor kunstnere kan uploade og sælge kunst. Brugeren kan på den måde selv vælge hvilket maleri eller andet kunstværk skal vises i billedrammen. Interaktionen med billedrammerne kan ske ved hjælp af en tilknyttet applikation, men begge billedrammer har også bevægelsesensorer knyttet til, så brugeren på den måde kan skifte billedet på skærmen til det næste eller forrige ved hjælp af gestikker. Interaktion ved hjælp af gestikker forudsætter dog, at brugeren er tæt på billedrammen for at sensorene kan opfange bevægelsen. Vendes fokus mod B$\&$O's ønske om at kunne styre et interaktivt kunstværk fra en større afstand, kan der ikke drages mange paralleller mellem det, Merual og Framed 2.0. Dog kan det ud fra billedrammerne forstås, at brugerne er i stand til at interagere med billedrammerne kun ved brug af frihåndsgestikker, hvilket kan medtages til interaktionen med B$\&$O's interaktive kunstværk.

 Udover produkter, der minder om B$\&$O's interaktive kunstværk findes der studier, der har kigget på perifer interaktion til styring af musik og lys. \textcite{PDF:FacilitatingPIDesignAndEvaluation} har undersøgt, hvordan styring af musik, mens brugeren sidder ved en computer, kan finde sted i den perifere del af opmærksomheden. Her blev det først undersøgt hvilke bevægelser der skulle bruges til funktionerne play, pause, skift sang frem/tilbage og volumen op/ned ved henholdsvis brugen af en gribelig knop, en touchflade og fri hånd \parencite[ss. 165-166]{PDF:FacilitatingPIDesignAndEvaluation}. Herefter kørtes et in-situ eksperiment, hvor forsøgspersonerne over otte uger skulle bruge hver af de tre redskaber samt tastaturets egne lydjusteringsknapper til at skifte musikken, mens de brugte deres computer. Ifølge \textcite[ss. 172-173]{PDF:FacilitatingPIDesignAndEvaluation} var styring af musik uden at kigge på musikafspilleren nemmest ved brug af knoppen og touchfladen. Derimod kiggede forsøgspersonerne mere på musikafspilleren ved brug af frihåndsgestikker, da de dels manglede haptisk feedback og dels udtalte at frihåndsgestikkerne føltes magiske, hvilket gjorde at de nød at kigge på musikafspilleren mens de udførte gestikkerne. Ved brug af tasteturets egne lydjusteringsknapper kiggede forsøgspersonerne ifølge data ikke på musikafspilleren, men sagde efter forsøget at de havde kigget på musikafspilleren ved brug af disse. Selvom forsøgspersonerne finder det nemmest at interagere perifert ved brug af knoppen og touchfladen konkluderer \textcite[s. 177]{PDF:FacilitatingPIDesignAndEvaluation} at frihåndsgestikker også egner sig godt til perifer interaktion, i situationer hvor dette er mest passende.
 
 Alternativt til styring af musik har \textcite[s. 1]{PDF:FacilitatingPIDesignAndEvaluation} undersøgt, hvordan interaktion kan finde sted i den perifere del af opmærksomheden, når interaktionen består af lysstyrkejustering på en lampe. Undersøgen viste, at forsøgspersoner godt kunne interagere med både en opmærksomhedskrævende opgave og lampen på samme tid, uden at nogen af opgaverne blev tilsideset \parencite[ss. 20-21]{PDF:FacilitatingPIDesignAndEvaluation}. Dog blev ydeevnen nedsat ved begge opgaver i forhold til at udføre begge opgaver hver for sig. Det viste sig at det meste af den visuelle opmærksomhed rettede sig mod den opmærksomhedskrævende opgave, hvilket kan indikere, at lysjusteringen blev gjort i den perifere del af opmærksomheden. I forbindelse med perifer interaktion blev det også undersøgt hvilken feedback fra produktet brugeren har brug for, for at udføre den perifere opgave bedst muligt. Undersøgelsen viste, at ved udførelse af en opmærksomhedskrævende opgave samtidig med lysjusteringsopgaven, gav det de bedste resultater, når forsøgsersonen fik feedback fra flere modaliteter.  Undersøgelsen testede også, hvor godt opgaven med at justere på lyset perifert blev udført sammenlignet med at drikke vand, som i forvejen er en nem perifer opgave \parencite[s. 20]{PDF:FacilitatingPIDesignAndEvaluation}. Her viste det sig, at selvom lysjusteringen kunne gøres perifert, så forstyrrede det den primære opgave mere end når der blev drukket vand som sekundær opgave. Dette bliver dog forklaret med, at forsøgspersonerne ikke har haft så meget tid til at lære at styre lampen, som de har haft til at lære at drikke vand. 
 
Vendes fokus igen mod B$\&$O ses det altså fra andre produkter og studier, at idéen om at kunne styre et interaktivt kunstværk perifert og ved brug af gestikker virker mulig. Der er flere måder at bruge gestikker på, både ved touch, frihånd og ved brug af et objekt, der måler bevægelserne. Til interaktion med et interaktivt kunstværk virker det logisk at bruge frihåndsgestikker til interaktionen, da interaktionen på den måde ikke er bestemt af, om det pågældende interaktionsredskab er i nærheden. Ifølge \textcite[s. 21]{PDF:FacilitatingPIDesignAndEvaluation} er feedback fra produktet med til at forbedre interaktionen. Ved et interaktivt kunstværk, der styrer musikken vil der naturligvis være feedback i form af musikken, der spiller og ændrer sig, men da systemet ikke nødvendigvis kan reagere med det samme kan brugeren have brug for feedback fra en kombination af modaliteter. 

 Der lægges op til flere undersøgelser af perifer interaktion, hvilke præcise frihåndsgestikker der skal bruges og hvilken feedback brugeren har brug for, for at skabe en interaktion, der er både lige til, socialt acceptabelt og magisk.


%
\begin{itemize}
  \item Meural \parencite{WEB:Meural}
  \item Framed 2.0 
  \item Musikkontrol (kilde med 3 forskellige måder at styre det perifer)
  \item kilde om perifer styring af lysintentises/lysstyrke
\end{itemize}
