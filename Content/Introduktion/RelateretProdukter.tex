\section{Relateret produkter og studier}
\label{RelateretProdukterOgStudier}

I det følgende afsnit er der kigget på hvilke relaterede produkter der findes, samt hvordan perifer interaktion tidligere er blevet studeret og brugt til at styre musik eller lignende.

Tages der udgangspunkt i konceptet omkring et interaktivt kunstværk til væggen findes der flere elektroniske billedrammer på markedet. Nogle af disse billedrammer kan endda styres med gestures, herilandt Merual \parencite{WEB:Meural} og Framed 2.0 (kilde til framed 2.0)

%Hvordan skal vi få flowet til at passe? Vi beskriver først canvas nede i B$\&$O afsnittet, men vi skal nævne gestures før. Måske skal dette afsnit handle mere om hvordan man har brugt perifer interaktion til styring af produkter - jeg ved ikke hvor meget meural og framed passer ind? Måske kan der forklares om gesturebaseret interaktion med meural og framed uden at snakke om den her bestemte billedramme. Så kan afslutningen på B$\&$O være at at det kunne være fedt at arbejde videre med styring af et kunstværk med gestures langt fra. 


%
\begin{itemize}
  \item Meural \parencite{WEB:Meural}
  \item Framed 2.0 
  \item Musikkontrol (kilde med 3 forskellige måder at styre det perifer)
  \item kilde om perifer styring af lysintentises/lysstyrke
\end{itemize}
