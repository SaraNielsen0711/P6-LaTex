\section{Fremgangsmåde}
\label{FremgangsmaadeSocialAccept}
%
Foruden at undersøge hvordan de udvalgte semaforiske gestikker påvirker den social accept, så er formålet med denne test også at undersøge hvorvidt testpersonerne er i stand til at interagere med et musikanlæg via de udvalgte semaforiske gestikker samtidig med at de udføre en primære opgave, hvorfor det er nødvendigt at definere den primære opgave. Den primære opgave for værten er, at føre en samtale med gæsten og for at sørge for, at samtalen ikke ophører under testen, så instrueres gæsten i hvad samtalen skal omhandle. Imens værten gennemgår en familiarisering bliver gæsten introduceret til samtale emnet; ferieplanlægning, hvor gæsten får til opgave at planlægge en to ugers ferie i tre forskellige prisklasser; budget, standard og luksus. Til budgetferien har testpersonerne 10000kr. til rådighed, til standardferien har de 20000kr. til rådighed og til luksusferien har de +30000kr. til rådighed. Det forventes ikke at gæsten når at færdiggøre planlægningen af de tre ferie forslag inden værtens familiarisering er overstået, da det er hensigten at de to testpersoner i fællesskab skal planlægge videre. Ydermere forventes det, at opgaven er så tidskrævende, at de to testpersonerne i fællesskab heller ikke når at planlægge ferien færdig. Ferieplanlægningen skal nemlig dække over hvor testpersonernes skal hen, hvordan de kommer derhen, hvor de skal overnatte, hvad de skal opleve samt et madbudget afhængigt af rådighedsbeløbet. For at afgøre hvorvidt værten er i stand til både at planlægge ferie med gæsten og samtidig interagere med et musikanlæg, vil der foretages separate exit-interviews. På den måde er det muligt, at få både værten og gæstens vurdering af om værten var i stand til at løse begge opgaver, uden at testpersonernes individuelle vurdering påvirker hinanden. 

Nu hvor den primære opgave er specificeret rettes fokus til hvordan værten skal interagere med musikanlægget. Da det antages, at det vil være for krævende for værten selv at skulle vurdere hvornår musikken skal pauses, startes, hvornår der skal skiftes musiknummer og hvornår der skal skrues op eller ned, besluttes det at værten skal reagere på nogle hints, som skal fremprovokere en bestemt gestik. Der udvælges derfor fem specifikke hints til; pause, start, skift musiknummer frem, skru op og skru ned. At der ikke vælges et hint til at skifte til det forrige musiknummer skyldes dels at det er den samme bevægelse; frem og tilbage og dels fordi det vurderes, at det ikke kan lade sig gøre at fremprovokere værten til naturligt at skifte til det forrige musiknummer. Værten bliver præsenteret for de fem forskellige hints under familiariseringen og bliver forklaret hvordan der skal reageres på dem. For at få værten til at sætte musikken på pause vælges det, at ringe til en mobiltelefon, som placeres på et sofabord inde i lokalet, værten pauser musikken og tager mobiltelefonen op til øret, som hvis det var en virkelig kontekst.   






Testen vil forløbe på følgende måde: Først byder testlederne testpersonerne velkommen, hvorefter én af de to testledere spørger de to testpersoner, om der er en af dem, der er venstre håndet. I tilfældet af at en af testpersonerne er venstre håndet vil denne testperson agere gæst og den anden vil agere vært. Hvis begge testpersoner er højre håndet vil det tilstræbes, at fordele testpersonerne så begge køn er repræsenteret både som vært og som gæst. Derefter følger testleder 1 testpersonen, som skal agere vært ind i et lokale, hvor testleder 2 og testpersonen, som skal agere gæst bliver ude foran lokalet. Testleder 1 har til ansvar, at familiarisere testpersonen i forhold til hvilke semaforiske gestikker der knyttes de valgte funktioner. Familiariseringen af værten samt testleder 1's instruktioner uddybes i \fullref{FamiliariseringSocialAccept}.  





De to testledere instruerer testpersonerne i hver deres opgave og udleverer samtykkeerklæringer, som testpersonerne opfordres til at læse og underskrive. 




Samtykkeerklæringen fremgår af \autoref{app:SamtykkeerklaeringSocialAccept}, instruktionerne til testleder 1 fremgår af \autoref{app:InstruktionerVaert} og instruktionerne til testleder 2 fremgår af \autoref{app:InstruktionerGaest}. 




Så snart testpersonen, som agerer gæst, har forstået opgaven, banker testleder 2 på døren ind til testleder 1 og værten, og går derefter ind i kontrolrummet. At testleder 2 banker på døren indikerer overfor testleder 1, at testlederen går ind i kontrolrummet og dermed er klar til at reagere på gestikker og styre musikken derefter. Når testpersonen, som agerer vært, er familiariseret følger testleder 1 gæsten ind i samme lokale og opsummerer opgaven, hvorefter testlederen forlader lokalet og går ind i kontrolrummet til testleder 2.
 
 
  

Der afsættes omkring fire minutter til, at de to testpersoner falder ind i rollerne og der vil derfor ikke præsenteres nogle hints i disse minutter. Derefter vil de forskellige hints blive præsenteret. Kort tid efter det sidste hint er præsenteret banker testlederen, som har ansvar for familiarisering af værten, på døren og afbryder ferieplanlægning. Gæsten bliver bedt om at forlade lokalet og gå ud til den testleder, som instruerede gæsten, hvor gæstens exit-interview afvikles samtidig med at værten og testlederen, som har ansvar for familiariseringen, afvikler værtens exit-interview inde i lokalet. 


FRA ROLLEFORDELING 
% 
For at udføre testen skal der udpeges to roller; en testleder, som har ansvaret for familiariseringen af værten og en testleder, som har ansvaret for at instruere gæsten. Derudover har de to testledere begge ansvaret for, at afvikle og optage hver deres exit-interview, som er tilpasset til henholdvis vært og gæst. Ydermere har begge testledere til ansvar at introducere testen og i den forbindelse opfordre testpersonerne til at læse og underskrive samtykkeerklæringen, som er vedlagt i \autoref{app:SamtykkeerklaeringSocialAccept}. Testlederen, som har ansvaret for familiariseringen af værten, har desuden til opgave at starte optagelsen og hente gæsten, så snart værten er familiariseret. Når både vært og gæst er samlet så har testlederen ydermere ansvar for at opsummere ferieplanlægningsopgaven. Testlederen, som derimod har ansvaret for at instruere gæsten, har desuden ansvar for både at præsentere de hints som værten skal reagere på og for at reagere på værtens gestikker.







Testens varighed afhænger dels af om testpersonen, som agerer vært, har brug for en længere familiarisering end planlagt og dels af testpersonernes respons i exit-interviewet, men testen anslåes til at vare mellem 35 minutter og 45 minutter. 
