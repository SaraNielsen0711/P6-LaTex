\section{Feedback}
\label{DiskussionFeedback}
%
I \fullref{Feedbackformer} åbnes der op for en diskussion af feedbackformer, herunder hvorvidt feedback er nødvendig ved brug af semaforiske gestikker. Foregår interaktionen via semaforiske gestikker, forekommer der ingen haptisk iboende feedback i forbindelse med interaktionen. Med udgangspunkt i definitionen af iboende feedback fremsat af \textcite[s. 3]{PDF:InteractionFrogger}, vurderes det, at fordi semaforiske gestikker afhænger af den proprioceptive sans, kan den iboende feedback relateres til den motoriske udførelse af interaktionen; udførelsen af gestikkerne. Derudover forekommer der funktionel feedback, da musikken ændres afhængigt af brugerens gestik, hvorfor den proprioceptive sans ligeledes forårsager feedback. Den sidste af de tre feedback former, som  \textcite[s. 3]{PDF:InteractionFrogger} definerer, er augmenteret feedback, som er feedback fra en ekstern kilde i forhold til hvor interaktionen opstår.\blankline
%
Når der købes produkter, som styres via semaforiske gestikker, i denne sammenhæng et musikanlæg, kan det være nødvendigt med en tilvænningsperiode, hvor brugeren lærer at interagere med produktet i henhold til hvor og hvornår diverse gestikker kan udføres. I tilvænningsperioden kan det være fordelagtigt at implementere augmentet feedback, for at kommunikere til brugeren hvornår en gestik er registreret korrekt. Det kan eksempelvis være en indikation af, at brugeren har rettet gestikken mod anlægget og uden at denne er maskeret. Efter tilvænningsperioden bør det være muligt aktivt at deaktivere feedbacken, hvis det ikke længere er nødvendigt for at interaktionen kan foregå, alternativt deaktiverer produktet feedbacken uden brugerens indblanding. Ved at deaktivere den augmenteret feedback, vil produktet i højere grad opfører sig passivt, da produktet ikke vil tiltrække unødvendig opmærksomhed forårsaget af feedbacken.

Hvorvidt augmenteret feedback skal implementeres samt hvordan det implementeres i Bang $\&$ Olufsens fremtidig musikanlæg afhænger dels af, hvordan produktet designmæssigt fremstår og dels, hvor hurtig systemet reagerer på gestikkerne. I den forbindelse understreger \textcite[s. 8]{PDF:AChairAsUbiquitousInputDevice} vigtigheden i et effektivt genkendelsessytem, der er med til af undgå forvirring ved interaktion. 

Da både pause, start og skift musiknummer følger et enten-eller princip, hvor produktet ikke skal registrere en specifik mængde, men om gestikken er udført, bør det vurderes hvor tydelig den augmenteret feedback er. Derimod kan der være behov for augmenteret feedback, når brugeren tager fat i den fiktive drejeknap og efterfølgende justerer lydstyrken.\blankline
%
Under udvælgelsen af semaforiske gestikker er det kun TP18, som nævner feedback, men det er i forbindelse med, at testpersonen foretrækker at have en fysisk brugergrænseflade i tillæg til interaktionen med de semaforiske gestikker. Når interaktionsformen undersøges i en social kontekst er der ingen af testpersonerne, der nævner noget om manglende feedback fra systemet. At ingen af testpersonerne pointerer, at de manglede feedback, skyldes formentlig at testleder 2 har reageret tilfredsstillende hurtigt på værternes gestikker og dels, at den funktionelle feedback fra systemet samt den iboende feedback fra den proprioceptive sans, har leveret den nødvendige feedback.

Da der er forskel på reelle brugeres behov sammenlignet med testpersonernes behov, er det ikke muligt, at afgører hvorvidt brugere vil opleve at de manglede feedback. Ydermere er der ikke foretaget undersøgelser omkring hvordan en potentitel augmenteret feedback skal implementeres, hvorfor det ikke er muligt, at fastlægge hvordan det skal implementeres, da det ligeledes afhænger af hvordan produktet designes. Det tilstræbes at interaktionen på sigt skal foregå i den perifere opmærksomhed, hvorfor det frarådes at anvende kraftige lys som feedback, da det formentlig vil trække uønsket opmærksomhed. Derimod vil en diskret feedback i form af et svagt oplyst område eller en animation, der indikerer at musikanlægget har registreret gestikke være understøttende for den perifere interaktion.
%

