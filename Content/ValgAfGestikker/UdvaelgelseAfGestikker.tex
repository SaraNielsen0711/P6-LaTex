\section{Udvælgelse af gestikker}
\label{UdvaelgelseAfGestikker}
%
I det følgende afsnit vil fokus være på at udvælge hvilke semaforiske gestikker, der egner sig til at interagere med et musikanlæg. De udvalgte semaforiske gestikker er enten fundet i relaterede undersøgelser, inspiration fra gestikker i Bang $\&$ Olufsens egne produkter eller fra projektgruppen selv. Gestikkerne udvælges parvist, så for hver gang der eksempelvis introduceres en ny gestik til at pause musikken så vil en modsvarende gestik, til at starte musikken, ligeledes introduceres. Hvis gestikken er fundet i en relateret undersøgelse vil der henvises til den specifikke undersøgelse, henvises der derimod til Bang $\&$ Olufsen indikerer det, at den specifikke gestik er	inspireret af et eller flere produkter fra $\&$ Olufsen. Fremgår der ingen henvisning indikerer det, at det er projektgruppen selv, der har designet gestikken.  

Formålet med at udvælge flere forskellige semaforiske gestikker til hver af de seks funktioner er dels for at illustrere hvordan interaktionen potentielt kan foregå, dels for at undersøge hvilke egenskaber testpersonerne foretrækker og dels for at inspirere testpersonerne til at komme med et forbedringsforslag. \blankline
%
I \autoref{tab:IndsamledeGestikkerPause} introduceres syv forskellige gestik-par, som enten pauser musikken eller starter musikken. At starte musikken er i projektsammenhæng forbundet med at musikken først er blevet sat på pause, hvorefter musikken startes igen. Start musikken retter sig derfor ikke i mod at tænde for musikanlægget, hvilket skyldes antagelsen om, at hvis en person gerne vil høre musik er det et aktivt valg at tænde for anlægget og det kan derfor ikke foregå i den perifere opmærksomhed.   
%
\begin{table}[H]
	\centering
	\begin{tabular}{| p{6cm} | p{6cm} | }
		\hline
		\textbf{Pause} & \textbf{Start} \\ \hline
		Stationær vertikal hånd, \parencite[s. 166]{PDF:ComparingInputModalities} & Stationær vertikal hånd, \parencite[s. 166]{PDF:ComparingInputModalities} \\ \hline
		Dynamisk vertikal hånd, der starter ved brystet og ender i en udstrakt horisontal arm  & Dynamisk vertikal hånd, der starter ved brystet og ender i en udstrakt horisontal arm  \\ \hline
		Kryds i luften med pegefingeren, \parencite[s. 48]{PDF:UserDefinedGesturesTV} & Flueben i luften med pegefingeren\\ \hline
		Pegefingeren peger og trykker i luften hen mod anlægget, \parencite{WEB:BeosoundMoment, WEB:Beosound2} & Pegefingeren peger og trykker i luften hen mod anlægget, \parencite[s. 48]{WEB:BeosoundMoment, WEB:Beosound2, PDF:UserDefinedGesturesTV} \\ \hline
		Krokodillenæb med fingrene, der lukker sammen, \parencite[s. 48]{PDF:UserDefinedGesturesTV} & Krokodillenæb med fingrene, der åbner op \\ \hline
		Hånden lukker sammen i en dynamisk horisontal bevægelse & Hånden åbner op i en dynamisk horisontal bevægelse \\ \hline
		Hånden lukker sammen i dynamisk vertikal bevægelse nedad & Hånden åbner op i dynamisk vertikal bevægelse opad  \\ \hline
	\end{tabular}
	\caption{Oversigt over forslag til semaforiske gestikker, der pauser og starter musikken.}
	\label{tab:IndsamledeGestikkerPause}
\end{table}
\noindent
%
Det fremgår af \autoref{tab:IndsamledeGestikkerPause}, at de syv forskellige gestik-par ikke nødvendigvis er angivet med den samme henvisning. \textcite[s. 48]{PDF:UserDefinedGesturesTV} anvender eksempelvis et tryk hen imod skærmen til at åbne for noget, mens der anvendes et kryds til at lukke igen. Det tilstræbes at gestikken til at pause og starte musikken enten skal være den samme eller være i tæt relation, hvorfor det vælges at afskille det foreslåede åben/luk-par og introducere en modpart til hver; tryk hen imod skærmen anvendes både til at tænde og slukke, krydset bibeholdes og modparten vil være at tegne et flueben i luften, jævnfør \autoref{tab:IndsamledeGestikkerPause}.   

Ydermere anvender \textcite[s. 48]{PDF:UserDefinedGesturesTV} et krokodillenæb, dannet med fingrene, til at mute lyden. Det er højst usandsynligt, at der er behov for at mute lyden når der høres musik, da det naturlige vil være at sætte musikken på pause. Det vælges derfor at anvende krokodillenæbet som en pause-gestik og da der skal være en modpart vil start-gestikken ligeledes være et krokodillenæb, jævnfør \autoref{tab:IndsamledeGestikkerPause}. 
%
\begin{table}[H]
	\centering
	\begin{tabular}{| p{6cm} | p{6cm} |}
		\hline
		\textbf{Skift musiknummer frem} & \textbf{Skift musiknummer tilbage} \\ \hline
		Swipe bevægelse fra højre mod venstre, \parencite[s. 48]{WEB:Beosound2, WEB:BeosoundMoment, PDF:UserDefinedGesturesTV} & Swipe bevægelse fra venstre mod højre, \parencite[s. 48]{WEB:Beosound2, WEB:BeosoundMoment, PDF:UserDefinedGesturesTV} \\ \hline
		Swipe bevægelse fra venstre mod højre, \parencite[s. 166]{PDF:ComparingInputModalities}  & Swipe bevægelse fra højre mod venstre, \parencite[s. 166]{PDF:ComparingInputModalities}  \\ \hline
		Peg tommelfingeren mod højre, mens de andre fingre bøjes, \parencite[s. 166]{PDF:ComparingInputModalities} & Peg tommelfingeren mod venstre, mens de andre fingre bøjes, \parencite[s. 166]{PDF:ComparingInputModalities} \\ \hline
		Swipe bevægelse fra højre mod venstre med tommel- og lillefinger strakt, mens de andre fingre bøjes & Swipe bevægelse fra venstre mod højre med tommel- og lillefinger strakt, mens de andre fingre bøjes \\ \hline
		Swipe bevægelse fra højre mod venstre med tommel-, pege- og langefinger strakt, mens de andre fingre bøjes & Swipe bevægelse fra venstre mod højre med tommel-, pege- og langefinger strakt, mens de andre fingre bøjes \\ \hline
		Tommel-, pege- og langefinger holdes strakt, de andre fingre bøjes samtidig med hånden køres i en bue fra venstre mod højre og ender med at pege til højre & Tommel-, pege- og langefinger holdes strakt, de andre fingre bøjes samtidig med hånden køres i en bue fra højre mod venstre og ender med at pege til højre\\ \hline
		Pegefinger peger mod højre & Pegefinger peger mod venstre\\ \hline
	\end{tabular}
	\caption{Oversigt over forslag til gestikker, der skifter musiknummer frem og tilbage.}
	\label{tab:IndsamledeGestikkerSkift}
\end{table}
\noindent
%
Med udgangspunkt i relaterede undersøgelser samt Bang $\&$ Olufsens produkter er det tydeligt, at en swipe bevægelse med hånden eller fingrene forbindes med videre eller næste, eksempelvis ved at skifte sang. Derudover er der inkorporeret en eller anden form for swipe bevægelse i de fleste smartphones og bærbare computere med et pegefelt, hvorfor en swipe bevægelse kan anses for at være en velkendt og velanvendt gestik.

Dog tyder det på, at der er forskel på hvilken retning swipe bevægelsen er for eksempelvis at skifte til det næste musiknummer. Ifølge \textcite[s. 166]{PDF:ComparingInputModalities} skal swipe bevægelsen foregå i den retning musikken skal skiftes, så en bevægelse fra venstre mod højre resulterer i at det er det næste musiknummer, som afspilles. Hvorimod \textcite[s. 48]{PDF:UserDefinedGesturesTV}, \textcite{WEB:Beosound2} og \textcite{WEB:BeosoundMoment} anvender en swipe bevægelse fra højre mod venstre for at skifte til det næste musiknummer. Da resultatet fra den samme swipe bevægelse kan være forskelligt, blandt andet afhængigt af hvilken smartphone eller bærbar computer bevægelsen udføres på, så medtages begge muligheder. For at minimere risikoen for at der opstår et \textit{Midas touch problem}, viderebygges der på swipe bevægelserne, så de designes med forskellige håndtegn, eksempelvis med tommel- og lillefinger strakt mens de resterende fingre bøjes, jævnfør \autoref{tab:IndsamledeGestikkerSkift}.

\textcite[s. 166]{PDF:ComparingInputModalities} foreslår at tommelfingeren skal i pege den retning, som det ønskes at musikken skal skifte i, hvilket inspirerer til en viderebygning hvor pegefingeren ligeledes peger i den retning musikken skal skifte, jævnfør \autoref{tab:IndsamledeGestikkerSkift}.            
%
\begin{table}[H]
	\centering
	\begin{tabular}{| p{6cm} | p{6cm} |}
		\hline
		\textbf{Skru op for musikken} & \textbf{Skru ned for musikken} \\ \hline
		Pegefingeren køres i en cirkulær bevægelse med uret, \parencite{WEB:BeosoundMoment, WEB:BMW7} & Pegefingeren køres i en cirkulær bevægelse mod uret, \parencite{WEB:BeosoundMoment, WEB:BMW7} \\ \hline
		Hold hånden og drej håndleddet med uret som ved kontakt med en volumen drejeknap, \parencite{WEB:Beosound2} & Hold hånden og drej håndleddet mod uret som ved kontakt med en volumen drejeknap, \parencite{WEB:Beosound2} \\ \hline
		Horisontal hånd løftes vertikalt med håndfladen nedad, \parencite[s. 166]{PDF:ComparingInputModalities} & Horisontal hånd sænkes vertikalt med håndfladen nedad, \parencite[s. 166]{PDF:ComparingInputModalities} \\ \hline
		Horisontal hånd løftes vertikalt med håndfladen opad & Horisontal hånd sænkes vertikalt med håndfladen nedad \\ \hline
		Horisontal ikke-dominant hånd med håndfladen opad holdes stationær, mens den dominante hånd holdes horisontal med håndfladen nedad løftes vertikalt væk den ikke-dominante hånd, \parencite[s. 48]{PDF:UserDefinedGesturesTV} & Horisontal ikke-dominant hånd med håndfladen opad holdes stationær, mens den dominante hånd holdes horisontal med håndfladen nedad sænkes vertikalt ned mod den ikke-dominante hånd, \parencite[s. 48]{PDF:UserDefinedGesturesTV} \\ \hline
		Vertikal ikke-dominant hånd holdes stationær, mens den dominante hånd ligeledes holdes vertikal og bevæger sig væk fra den ikke-dominante hånd i en horisontal bevægelse, \parencite[s. 48]{PDF:UserDefinedGesturesTV} & Vertikal ikke-dominant hånd holdes stationær, mens den dominante hånd ligeledes holdes vertikal og bevæger sig hen mod den ikke-dominante hånd i en horisontal bevægelse, \parencite[s. 48]{PDF:UserDefinedGesturesTV} \\ \hline
		Krum hånd bevæges i en bue fra venstre mod højre, \parencite{WEB:BeoplayA9} & Krum hånd bevæges i en bue fra højre mod venstre, \parencite{WEB:BeoplayA9} \\ \hline
		Pegefingeren peger opad & Pegefingeren peger nedad \\ \hline
		\enquote{Kom så}-bevægelse med fingrene & \enquote{Ro på}-bevægelse med fingrene \\ \hline		
	\end{tabular}
	\caption{Oversigt over forslag til gestikker, der ændrer skruer op og ned for musikken.}
	\label{tab:IndsamledeGestikkerVolumen}
\end{table}
\noindent
%
At skrue op og ned for musikken kendetegnes ofte ved en cirkulær bevægelse, med uret for at skrue op og mod uret for at skrue ned, eller ved at trække en slider vertikalt eller horisontalt. Disse kendetegn går derfor igen i størstedelen af de udvalgte gestikker, jævnfør \autoref{tab:IndsamledeGestikkerVolumen}. I henhold til at skrue op og ned for musik foreslår \textcite[s. 48]{PDF:UserDefinedGesturesTV}, at begge hænder skal indgå, hvor den ikke-dominante hånd gengiver en reference som den dominante hånd enten føres mod eller væk fra.

Udover forslag, som ofte forbindes med at skrue op og ned for musik, introduceres der yderligere to gestik-par hvoraf den ene er et statisk peg med pegefingeren opad eller nedad afhængigt af om der skal skrues op eller ned. Det andet gestik-par designes ud fra en \enquote{kom så}-bevægelse for at skrue op og en \enquote{ro på}-bevægelse for at skrue ned, jævnfør \autoref{tab:IndsamledeGestikkerVolumen}. \blankline
%
Til at styre de seks mest gængse funktioner på et musikanlæg er der i alt udvalgt 23 par af semaforiske gestikker, syv par til pause og start, syv til at skifte musiknummer frem og tilbage og ni til at skrue op og ned for musikken. Da gestikkerne skal præsenteres for testpersoner er det favorabelt at gestikkerne gengives visuelt og på en måde, der tillader at gengive de dynamiske bevægelse størstedelen af gestikkerne indeholder. Det er derfor oplagt at optage gestikkerne, hvilket det følgende afsnit vil omhandle. 
%
