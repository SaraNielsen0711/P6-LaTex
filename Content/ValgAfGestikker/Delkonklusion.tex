\section{Delkonklusion}
\label{ValgAfGestikkerDelkonklusion}
%
På baggrund af foregående analyse er semaforiske gestikker til de seks mest gængse funktioner på et musikanlæg valgt. Jævnfør \fullref{TestresultaterPauseStart} er gestik-par 5 valgt til at pause og starte musikken, hvor hånden lukker som et krokodillenæb for at pause musikken og åbner op igen for at starte musikken, \autoref{fig:GestikPar5Pause}. For at skifte musiknummer frem og tilbage er gestik-par 5 ligeledes valgt, jævnfør \fullref{TestresultaterSkiftMusiknummer}, hvor der her skiftes til det næste musiknummer med en swipe bevægelse fra højre mod venstre samtidig med at tommel-, pege- og langefinger holdes strakt, som illustreret på \autoref{fig:GestikPar5Skift}. Til sidst er gestik-par 2 valgt til at skrue op og ned for musikken, jævnfør \fullref{TestresultaterVolumen}, hvor hånden gengiver at have fat i en drejeknap, hvilket illustreres på \autoref{fig:GestikPar2Volumen}. Størstedelen af de testpersoner, der har deltaget i undersøgelsen har en positiv indstilling til at interaktionen med et musikanlæg kan foregå via semaforiske gestikker. Det på trods af, at ingen af testpersonerne betragter sig selv som værende de første til at anskaffe sig den nyeste teknologi, hvilket er beskrevet i \fullref{TestresultaterInteraktioner}. Ydermere har størstedelen af testpersonerne ikke et problem med, at blive opfanget af deres musikanlæg, dog med de forebehold at det skal være muligt, at slukke for registreringen og at naturlige samtale-gestikker ikke fejlagtigt skal registreres af musikanlægget. For at komme testpersonernes bekymringerne til livs, kan det være en fordel at de semaforiske gestikker skal rettes direkte mod musikanlægget samt at denne interaktionsform kun fungerer, når musikanlægget allerede er tændt. \blankline
%
Når de semaforiske gestikker til interaktionen med de mest gængse funktioner på et musikanlæg er udvalgt, er det relevant at teste disse i en social sammenhæng. I den sociale sammenhæng kan både den sociale accept af gestikkerne samt muligheden for, at interagere med musikanlægget sideløbende med en primær opgave undersøges. 