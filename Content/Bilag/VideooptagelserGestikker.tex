\chapter{Videooptagelser af gestik forslag}
\label{app:VideooptagelserValgAfGestikker}
%
I følgende kapitel vil de visuelle repræsentationer af de udvalgte gestikker udarbejdes. Der udarbejdes tre videoer; én hvor de udvalgte semaforiske gestikker til pause og start fremgår, én hvor de udvalgte semaforiske gestikke til at skifte musiknummer fremgår og én hvor de udvalgte semaforiske gestikker til at justere lydstyrken fremgår. Videooptagelserne redigeres i Windows Movie Maker version 2012 og de tre færdigredigerede videooptagelser fremgår i af \autoref{app:VideooptagelseGestikForslag}.\blankline
%
De tre videoer er optaget med et Canon Powershot s110 kamera på et stativ. Kameraet optager både billede og lyd, optagelserne er foretaget med en 45$^{\circ}$'s vinkel i et halvnært perspektiv samt en neutral hvid baggrund og et neutralt ansigtsudtryk, hvilket illustreres på \autoref{fig:Tryk}. Ved at optage med en 45$^{\circ}$'s vinkel i et halvnært perspektiv er det muligt at tydeliggøre størrelse, bevægelsesretning og dybden for hver af de foreslåede semaforiske gestikker. Ydermere tillader en 45$^{\circ}$'s vinkel, at forstyrrelser i form af øjenkontakt mellem testperson og demonstratoren i optagelserne minimeres. 
%
\begin{figure}[H]
	\centering
	\includegraphics[resolution=300,width=0.9\textwidth]{Flowdiagram/Tryk}
	\caption{Illustration af hvordan optagelserne foretages, med 45$^{\circ}$'s vinkel i et halvnært perspektiv samt en neutral hvid baggrund og et neutralt ansigtsudtryk.}
	\label{fig:Tryk}
\end{figure}
\noindent
%
Under optagelserne afspilles der musik fra en computer tilkoblet en Sonos PLAY:5 højtaler. Musiknummeret til at pause og starte musikken er \enquote{Stay} fra Zedd og Alessia Cara (2017). Musiknumrene til at skifte musiknummer er \enquote{Closer} fra The Chainsmokers ft. Halsey (2016) og \enquote{Galway Girl} fra Ed Sheeran (2017). Musiknummeret til at justere lydstyrken er \enquote{You Don't Know Me} fra Jax Jones ft. RAYE (2017). For at videooptagelserne bedst muligt gengiver det auditive resultat af gestikkerne, er det en fordel, at de pågældende musiknumre ikke har en stille intro, da dette kan være svært efterfølgende at høre på videooptagelserne. Ingen af de førnævnte musiknumre vurderes til at have en stille intro, hvorfor de inkluderes i videooptagelserne. Udover det er der ikke nogen særlige overvejelser bag musikvalget. 

Når der eksempelvis laves en gestik for at pause musikken, pauses musikken på computeren, for på den måde at demonstrere hvilken gestik, der skal til for at musikken pauses. De resterende funktioner optages på tilsvarende måde.\blankline
%
Når de udvalgte semaforiske gestikker er optaget, bliver de som tidligere nævnt redigeret i Windows Movie Maker version 2012. På \autoref{fig:FlowdiagramPause} illustreres rækkefølgen på de skærmbilleder, som videoerne består af. Eksemplet på \autoref{fig:FlowdiagramPause} er fra videoen, der gengiver de foreslåede semaforiske gestikker til at pause og starte musikken, de to andre videoer er bygget op omkring den samme struktur, dog indeholder videoen for at justering af lydstyrke ydereligere fire skærmbilleder, hvor to angiver hvilket gestik-par nummer der præsenteres og to gengiver gestik-parret.       
%
\begin{figure}[H]
	\centering
	\includegraphics[resolution=300,width=\textwidth]{Flowdiagram/FlowdiagramPauseogStart.pdf}
	\caption{Illustration af hvilke skærmbilleder, der indgår i videoen til at pause og starte musikken.}
	\label{fig:FlowdiagramPause}
\end{figure}
\noindent
%
Det første skærmbillede testpersonerne præsenteres for består af teksten: \textit{Er du klar?} og præsenteres i to sekunder, før skærmbilledet med teksten: \textit{Sæt musikken på pause og start den igen} præsenteres i tre sekunder. Teksten på det andet skærmbillede tilpasses afhængigt af hvilket videoklip, som afspilles, men varigheden er fast. De nummerede skærmbilleder, eksempelvis med teksten: \textit{Gestik 1}, præsenteres med en to sekunders varighed. Det andet sidste skærmbillede med teksten: \textit{Hvordan vil du rangere de tre bedste gestikker?} præsenteres med fire sekunders varighed, før skærmbilledet med opsummeringen af de foreslåede gestikker præsenteres. Opsummeringen angives med en 30 sekunders varighed, da dette er den maksimale varighed i Windows Movie Maker, dog forsvinder skærmbilledet ikke efter de 30 sekunder. Det skal understreges, at med undtagelse af opsummeringen, indeholder videoerne ikke statiske skærmbilleder af gestikkerne, da de er dynamiske og hvert gestik-par indeholder en modpart, som i dette tilfælde er at starte musikken, hvilket ikke illustreres på \autoref{fig:FlowdiagramPause}.

Varigheden af de tre videooptagelser er for pause og start 1 minut og 51 sekunder, for at skifte musiknummer er varigheden 2 minutter og 14 sekunder og for justering af lydstyrken er varigheden 2 minutter og 25 sekunder.\blankline
%
Som illustreret på \autoref{fig:FlowdiagramPause} indeholder videoerne et skærmbillede, hvor de udvalgte semaforiske gestikker opsummeres. På disse skærmbilleder er bevægelsen i gestikkerne forsøgt gengivet med retningspile, som er til for, at testpersonerne kan adskille gestikkerne  og så de ikke behøver at huske gestikkerne, efterhånden som de introduceres i videoen. Derudover vil testpersonerne få mulighed for at få genspillet videoen, hvis de ønsker. Ydermere er gestikkerne nummerede afhængigt af den rækkefølge, de er blevet optaget i samt rækkefølgen de præsenteres i for testpersonerne. På \autoref{fig:OversigtPauseStart} gengives opsummeringsskærmbilledet for at pause og starte musikken, igen er det kun pause-gestikken, der indgår i opsummeringen. Modparten beskrives i \autoref{tab:IndsamledeGestikkerPause}.  
% 
\begin{figure}[H]
	\centering
	\includegraphics[resolution=300,width=\textwidth]{Test1/collage_start_pause_tekst}
	\caption{Opsummering af de syv semaforiske gestikker, der pauser og starter musikken, inklusiv nummerering og pile, som indikerer bevægelsesretning.}
	\label{fig:OversigtPauseStart}
\end{figure}
\noindent
%
Ligesom illustreret på \autoref{fig:OversigtPauseStart} gengives de udvalgte semaforiske gestikker til at skifte musiknummer ligeledes med retningspile, for at indikere bevægelsen ved hver gestik, jævnfør \autoref{fig:OversigtSkift}. Ydermere er gestikkerne på \autoref{fig:OversigtSkift} ligeledes nummeret efter samme princip, som ved pause og start. Gestikkerne, som illustreres på \autoref{fig:OversigtSkift}, gengiver alle, at der skiftes til det næste musiknummer, hvor modparten enten vil være den samme bevægelse den modsatte retning eller et peg i den modsatte retning af hvad der angives på \autoref{fig:OversigtSkift}. Modparten beskrives i \autoref{tab:IndsamledeGestikkerSkift}.
%
\begin{figure}[H]
	\centering
	\includegraphics[resolution=300,width=\textwidth]{Test1/collage_Skiftsang_tekst}
	\caption{Opsummering af de syv semaforiske gestikker, der skifter musiknummer, inklusiv nummerering og pile, som indikerer bevægelsesretning.}
	\label{fig:OversigtSkift}
\end{figure}
\noindent
%
Da både videoer og opsummeringsskærmbilleder er redigeret efter samme struktur, gengiver \autoref{fig:OversigtVolumen}, efter samme principper, opsummeringsskærmbilledet for justering af lystyrken. 
\newpage 
%
\begin{figure}[H]
	\centering
	\includegraphics[resolution=300,width=\textwidth]{Test1/collage_volumen_tekst}
	\caption{Opsummering af de ni semaforiske gestikker til justering af lydstyrken, inklusiv nummerering og pile, som indikerer bevægelsesretning.}
	\label{fig:OversigtVolumen}
\end{figure}
\noindent
%
