\section{Afgrænsing}
\label{Afgraensning}
%
I det følgende afsnit opsummeres og beskrives hvilke dele af teorien indenfor perifer opmærksomhed og gestikker, der vælges at arbejde videre med i samarbejde med Bang $\&$ Olufsen, for at få interaktionen med et interaktivt kunstværk til at fungere i den perifere del af opmærksomheden. 

Når der skal designes produkter, som skal interageres med i den perifere del af opmærksomheden, kan det være relevant at undersøge, hvordan denne interaktion, i form af en sideopgave, kan udføres sideløbende med en primær opgave. Som nævnt i \fullref{PeriferInteratkion} afgrænses der derfor til kun at arbejde med én specifik sideopgave. Derudover afgrænses der fra at interaktionen vil foregå i den visuelle opmærksomhed eller ved brug af stemmestyring. Fremadrettet vil det altså blive undersøgt, hvordan den proprioceptive sans kan bruges til interaktion i det perifere. 

Som nævnt i \fullref{KategoriseringerAfGestikker} findes der flere typer af gestikker, hvoraf nogen af dem ikke egner sig til interaktion i det perifere. Til interaktion med Bang $\&$ Olufsens interaktive kunstværk virker det logisk at bruge semaforiske gestikker til interaktionen, da interaktionen på den måde ikke er bestemt af, om det pågældende interaktionsredskab er i nærheden. I den sammenhæng bliver der også afgrænset til at bruge \textit{Perceptaul Input}, da denne slags input lægger godt op til interaktion med semaforiske gestikker.

Som beskrevet kan gestikulationer inddeles i mikro- og makrogestikker. Til styring af Bang $\&$ Olufsens produkt afgrænses der fra at bruge makrogestikker, da disse hurtigt kan blive for voldsomme til formålet. Da brugssituationen ofte vil være et stykke væk fra det interaktive kunstværk kan det som nævnt i \fullref{KategoriseringerAfGestikker} være en ulempe at bruge helt små mikrogestikker. Det er dog ønsket at bruge nogle mere præcise gestikulationer, som hører ind under mikrogestikker, og da mikrogestikker egner sig godt til et mere professionelt udtryk, \parencite[s. 10]{PDF:UsabilityofMicroVsMacroGestures}, kategoriseres de anvendte gestikker fremadrettet som mikrogestikker. 

I \fullref{KomplikationerGestikker} er der beskrevet hvilke tekniske og menneskelige komplikationer, der kan opstå ved brug af gestikulationer til perifer interaktion. Der afgrænses i dette projekt fra at undersøge på det tekniske aspekt og fokuseres derimod på det menneskelige aspekt og de komplikationer, der hertil kan opstå. Derudover afgrænses der fra at køre Hi-Fi test og feltundersøgelser, da det interaktive kunstværk stadig er på konceptbasis og teknologien til en feltundersøgelse ikke er til rådighed. Da det er relativt nyt at bruge gestikker til interaktion med produkter, findes der ikke nødvendigvis et endegyldigt svar på hvilke gestikker brugeren ønsker at bruge. Der afgrænses derfor til at undersøge, hvilke gestikulationer brugeren gerne vil have til at styre musik gennem et interaktivt kunstværk, herunder gestikulationer knyttet til de funktioner det giver mening at styre perifert. 

Da der er afgrænset til at bruge semaforiske gestikker, der som nævnt i \fullref{KomplikationerVedroerendeDetMenneskelige} kan opfattes unaturlige, er det interessant at undersøge, om der kan findes semaforiske gestikker, der trods deres unaturlighed kan læres og genkaldes i den perifere del af opmærksomheden. Dog skal gestikulationerne ikke lægge sig så meget op af almindelige menneske-til-menneske gestikker eller andre naturlige gestikker, at de bliver lavet ved en fejl. Det findes altså interessant at undersøge lige netop det område, hvor semaforiske gestikker kan bruges til fejlfri HCI i den perifere del af opmærksomheden. 

Undersøges behovet for feedback ved perifer interaktion er der, som beskrevet i \fullref{Feedbackformer}, delte meninger. Da det interaktive kunstværk skal styre musikken og brugeren ved interaktion med dette derfor kan høre ændringerne i musikken alt efter kommandoen og føle bevægelserne ved brug af semaforiske gestikker, findes det i første omgang ikke nødvendigt at designe feedback til at understøtte den perifere interaktion. Da der dog findes undersøgelser, som eksemelvis \textcite[s. 21]{PDF:FacilitatingPIDesignAndEvaluation}, der hævder at multimodal feedback styrker den perifere interaktion, vælges det at spørge brugerne om deres behov for feedback i forbindelse med undersøgelse af gestikker, uden i første omgang at lave en direkte undersøgelse af feedbackformer. 

Det er ikke muligt at afgrænse sig fuldstændig til, hvordan gestikkerne skal udføres jævnfør \fullref{Socialaccept}, da dette kommer an på brugerens ønske. Der kan dog allerede afgrænses fra at lave spændingsfyldte- og hemmelighedsfyldte gestikker, da der ud fra Bang $\&$ Olufsens ønske ikke skal laves gestikker til nogle funktioner, der ikke giver tydelig feedback i form af musik. Derudover giver det ikke mening at lave spændingsfyldte gestikker, da det allerede er set, at disse ikke er socialt acceptable, \parencite{PDF:WouldYouDoThat}. Tilbage står altså udtryktfyldte- og magiske gestikker, der begger egner sig godt til interaktion med et interaktivt kunstværk, der kan styre musikken. 

I \fullref{RelateretProdukterOgUndersoegelser} ses det, at lignende problemstillinger er set før, da der både er lavet undersøgelser omkring perifer interaktion, musikstyring og gestikker både hver for sig og i sammenspil. Den situation som projektet i samarbejde med Bang $\&$ Olufsen ønsker at undersøge er ikke fuldstændig sammenlignelig med tidligere undersøgelser. De relaterede undersøgelser kan derfor bruges som inspiration til hvilke semaforiske gestikker, der skal undersøges og hvordan undersøgelserne skal udføres, uden på forhånd at have bestemt, hvordan der skal skrues op og ned for musikken med semaforiske gestikker. 

Projektet er nu afgrænset til at undersøge, hvordan perifer interaktion kan foregå med semaforiske gestikker, når musik skal styres fra en afstand, hvilket danner grundlag for en problemformulering.
