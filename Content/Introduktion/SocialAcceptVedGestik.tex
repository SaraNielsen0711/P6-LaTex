\section{Er semaforiske gestikker social acceptable?}
\label{Socialaccept}
%            
Der fokuseres hverken på manipulerende eller deiktiske gestikker i forhold til social accept, hvilket skyldes at ved begge tilfælde er hele interaktionen og resultatet af interaktion synlig. Det er muligt for tilskuere, at se hvad en gestik medfører, uanset om det er ved at manipulere et objekt eller ved at pege på et objekt. Ydermere antages det, at manipulerende gestikker er så veletableret, at både tilskuere og brugeren selv ikke nødvendigvis tænker over, at det er en gestik. \blankline
%
Af de fire undersøgelser, der undersøger eller kommenterer på social accept, som det har været muligt at finde, fremgår det, at fokus har været på social accept af semaforiske gestikker, \parencite{PDF:AChairAsUbiquitousInputDevice, PDF:WouldYouDoThat, PDF:AreYouComfortableDoingThat, PDF:AnExploratoryStudy}. Af de fire undersøgelser fokuserer \textcite{PDF:AreYouComfortableDoingThat} og \textcite{PDF:WouldYouDoThat} på, hvordan semaforiske gestikker opfattes af andre samt i hvilke situationer, brugeren er villig til og komfortable ved at udføre bestemte former for semaforiske gestikker og hvornår de ikke er. I undersøgelsen foretaget af \textcite{PDF:AnExploratoryStudy} fokuseres der på, hvordan sociale elektroniske apparater kan være med til at fremme kontakten mellem mennesker. Hvor der i undersøgelsen, foretaget af \textcite{PDF:AChairAsUbiquitousInputDevice}, fokuseres på en interaktiv stol. Interaktionen undersøges i to scenarier; i den centrale opmærksomhed hvor testpersonerne interagere med en computer og i den perifere opmærksomhed hvor testpersonerne interagere med en musikafspiller, \parencite{PDF:AChairAsUbiquitousInputDevice}. Ved begge scenarier giver testpersonerne udtryk for, at de er bekymret for, hvorvidt deres interaktion er social acceptabel, \parencite[s. 8]{PDF:AChairAsUbiquitousInputDevice}. Bekymringerne vedrører, hvorvidt andre anser gestikkerne, som værende akavet, \parencite[s. 4]{PDF:AChairAsUbiquitousInputDevice}. Dog pointerer \textcite[s. 9]{PDF:AChairAsUbiquitousInputDevice}, at det kan hænge sammen med, at gestikken er synlig, men resultatet deraf er usynligt, det er derfor kun testpersonen, der kan høre musikken ændre sig. \textcite[s. 9]{PDF:AChairAsUbiquitousInputDevice} argumentere for, at hvis den interaktive stol implementeres i et i forvejen interaktivt miljø, så kan det potentielt reducere den akavede fornemmelse.\blankline
%
Der er overordet to måder at definere social accept; ud fra brugerens syn eller ud fra tilskuerens syn. Ud fra brugerens synspunkt handler social accept om det indtryk, brugeren får ved at udføre interaktionen, om oplevelsen var positiv eller negativ, \parencite[s. 276]{PDF:WouldYouDoThat}. Social accept ud fra tilskuerens synspunkt afhænger af, hvordan tilskueren vurderer brugerens adfærd som værende enten positiv eller negativ, \parencite[s. 276]{PDF:WouldYouDoThat}. Ifølge \textcite[s. 276]{PDF:WouldYouDoThat} påvirker følgende faktorer social accept: Brugertype, tid og sted samt manipulation kontra effekt. Brugertype henviser til, om brugeren hurtigt accepterer gestik, som interaktionsform eller om de stritter i mod. Tid og sted henviser til kultur og hvor længe den pågældende teknologi har været på markedet, hvilket kan være med til at øge den sociale accept. Manipulation kontra effekt henviser til de specifikke gestikker og hvad de resulterer i, hvilket vil blive uddybet yderligere.

Ifølge \textcite[s. 276]{PDF:WouldYouDoThat} kan gestikker inddeles i følgende fire kategorier; udtryksfyldte, spændingsfyldte, hemmelighedsfyldte og magiske. Hvor de udtryksfyldte gestikker både har synlig manipulation og synlig effekt modsat de hemmelighedsfyldte gestikker, som både har usynlig manipulation og usynlig effekt. De magiske gestikker defineres, som de gestikker, hvor manipulationen er usynlig, men effekten er synlig, modsat de spændingsfyldte, som har synlig manipulation og usynlig effekt. Baseret på resultaterne fremsat af \textcite[s. 277]{PDF:WouldYouDoThat} fremgår det, at de spændingsfyldte gestikker ikke anses for, at være socialt acceptable, det gælder både i offentlige og private sammenhænge. De hemmelighedsfyldte gestikker, derimod blev rangeret højt i forhold til social accept både i det offentlig og i det private. Ydermere fremgår det ligeledes af resultaterne, at både de udtryksfyldte og magiske gestikke generelt opleves, som værende social acceptable, \parencite[s. 277]{PDF:WouldYouDoThat}. 

Det tyder på, at gestikker hvor manipulationen er synlig nødvendigvis også må have en synlig effekt, for at blive socialt accepteret. Alternativt kan manipulationen være usynlig, hvor effekten enten kan være synlig og usynlig, for stadig at opnå social accept, \parencite[s. 278]{PDF:WouldYouDoThat}. Disse kategorier inden for social accept er uafhængige af kategoriseringen i \fullref{KategoriseringerAfGestikker}, i den forstand at de ikke er koblet til én specifik form for gestik men nærmere hænger sammen med hvordan gestikken udføres.\blankline
%KATEGORISERING AF GESTIKKE-HENVISNINGEN VIRKER IKKE LÆNGERE
I undersøgelsen foretaget af \textcite[s. 193]{PDF:AreYouComfortableDoingThat}, fokuseres der dels på den sociale accept og dels på, om brugeren føler sig komfortabel ved at udføre gestikker i området omkring et elektronisk apparat. Det fremgår, at der er en stærk forbindelse mellem gestikkens nødvendige bevægelsesmængde, varigheden hvormed gestikken udføres samt positionen, relativt til det elektroniske apparat, \parencite[s. 193]{PDF:AreYouComfortableDoingThat}. Derudover fremgår det ligeledes, at brugere er både selektive og bevidste om hvor interaktionen foregår, privat eller offentligt, samt hvem der kan se denne interaktion, \parencite[s. 193]{PDF:AreYouComfortableDoingThat}. Denne problemstilling undersøges i tre forskellige henseende; 1) området og afstanden fra det elektroniske apparat, 2) Bevægelsesmængden og varigheden af gestikken og 3) Tilskuere, \parencite[ss. 195-200]{PDF:AreYouComfortableDoingThat}. 

Ud fra den første undersøgelse fremgår det, at testpersonerne føler sig mest komfortable ved at udføre gestikkerne i området tæt på, lige over og helst til højre for det elektroniske apparat, \parencite[s. 197]{PDF:AreYouComfortableDoingThat}. Endvidere vurderer \textcite[s. 201]{PDF:AreYouComfortableDoingThat}, at grænsen for, hvornår en gestik er i risiko for, at blive anset som værende social uacceptabel, er når afstanden til det elektroniske apparat når 30 cm. I forhold til hvem testpersonerne føler sig komfortable ved, når de skal udføre gestikkerne, fremgår det, at det primært er personer, som de har et tæt forhold til, såsom venner, familie og deres partnere, \parencite[s. 196]{PDF:AreYouComfortableDoingThat}. Dertil føler testpersonerne sig mest ukomfortable overfor fremmede og kollegaer. Testpersonerne vurdere, ydermere, at de to mest ukomfortable lokationer, at udføre gestikkerne på er i offentligtransport og på et museum. 

Ved den næste undersøgelse, som vedrører bevægelsesmængde og varighed, defineres en bevægelsesmængde inden for et område på 15x15 cm, som værende lille, hvorimod hvis gestikken udføres inden for et område på 30x30 cm, så anses bevægelsesmængden for at være stor, \parencite[s. 198]{PDF:AreYouComfortableDoingThat}. Gestikkerne udføres ved tre forskellige varigheder; tre sekunder, seks sekunder og ni sekunder. På baggrund af resultaterne fra anden undersøgelse fremgår det, at gestikker med en varighed på mindre end seks sekunder klart er at fortrække, hvis testpersonerne skal føle sig komfortable, \parencite[s. 199]{PDF:AreYouComfortableDoingThat}. Ydermere foretrækker testpersonerne en lille bevægelsesmængde, dog pointere \textcite[s. 199]{PDF:AreYouComfortableDoingThat}, at såfremt en gestik med en stor bevægelsesmængde udføres i en favorabel lokation, tæt på det elektroniske apparat og har en kort varighed, så kan der kompenseres for den ellers ukomfortable oplevelse. 

Ved den tredje undersøgelse, som vedrører tilskuerne, fremgår det, at tilskuerne ikke fandt gestikkerne påtrængende og en del af tilskuerne tænkte ikke yderliger over de gestikker, som de lige havde overværet, \parencite[s. 200]{PDF:AreYouComfortableDoingThat}. Ifølge \textcite[s. 200]{PDF:AreYouComfortableDoingThat} så har tilskuerne en tendens til at vurdere den sociale accept en del højere end hvad tilfældet var ved de to foregående undersøgelser. \blankline
%
\textcite{PDF:AnExploratoryStudy} undersøger, hvordan semaforiske gestikker kan understøtte interaktionen med elektroniske apparater, ved at fremme den sociale kontakt mellem mennesker. Baseret på disse resultater tyder det på, at gestikker med en stor bevægelsesmængde kan være socialt acceptable og komfortable, hvis de udføres hjemme hos ens venner, \parencite[s. 4]{PDF:AnExploratoryStudy}. Endvidere finder testpersonerne det mindre pinligt, at udføre gestikker fremfor både stemmestyring og lydkontrol, \parencite[s. 4]{PDF:AnExploratoryStudy}. Det fremgår ydermere, at testpersonerne foretrækker, at gestikkerne benyttes i en positiv sammenhæng fremfor en negativ sammenhæng, hvor de foretrækker at gestikkerne da er mindre synlige, \parencite[s. 4]{PDF:AnExploratoryStudy}. \blankline
%
Så for at svare på, om semaforiske gestikker er social acceptable, så afhænger det af flere aspekter; dels hvilket miljø gestikken optræder i, forholdet mellem manipulation og effekt i forhold til hvad der er synligt kontra usynligt, gestikkens bevægelsesmængde og varighed, området og afstanden til det elektroniske apparat og hvilken type tilskuere, der overvære gestikken. For at opsummere nogle af de anbefalinger, der fremgår i de fire undersøgelser, så tyder det på, at brugerne i forhold til miljø er mere tilbøjelige til at føle sig komfortable i private omgivelser og såfremt gestikken udføres i det offentlige, så skal den være forholdvist diskret. En af årsageren til, at testpersonerne i undersøgelsen, foretaget af \textcite[s. 4]{PDF:AChairAsUbiquitousInputDevice}, gav udtryk for bekymringer i forbindelse med interaktionen med den interaktive stol, kan formegentlig skyldes, at i det tilfælde var manipulationen synlig og effekt var usynlig for tilskuerne, svarende til hvad \textcite[s. 276]{PDF:WouldYouDoThat}, definerer til at være en spændingsfyldt gestik.  

I forbindelse med manipulation og effekt, så anbefaler \textcite[s. 278]{PDF:WouldYouDoThat}, at der netop ikke bruges spændingsfyldte gestikker, hvilket efterlader de udtryksfyldte, hemmelighedsfyldte og magiske gestikker til fri afbenyttelse for at opnå social accept. I relation til bevægelsesmængden så anbefaler \textcite[s. 201]{PDF:AreYouComfortableDoingThat}, at såfremt gestikken udføres i det offentlige, så skal der være en forholdvist lille bevægelsesmængde i et område svarende til 15x15 cm. Ifølge \textcite[s. 278]{PDF:WouldYouDoThat}, så er både små og store bevægelsesmængder socialt acceptable, så længe effekten er synlig.

Ydermere skal varigheden være kortere end seks sekunder, da risikoen for at brugeren da føler sig ukomfortabel stiger. I forbindelse med området og afstanden vurderer \textcite[s. 201]{PDF:AreYouComfortableDoingThat}, at området til højre og foran det elektroniske apparat er at foretrække, særligt for højrehåndet brugere, ellers skal der tages højde for afstanden. Ifølge \textcite[s. 201]{PDF:AreYouComfortableDoingThat}, så er grænsen for, hvornår det er komfortabelt samt social acceptabelt, omkring 30 cm mellem det elektroniske apparat og gestikken.\blankline
% HENVISNINGEN TIL KATEGORISERING AF GESTIKKER VIRKER IKKE LÆNGERE
Igennem \fullref{Gestik} er der blevet belyst, hvilke kategorier forskellige former for gestik tilhører, komplikationer forårsaget af gestikker, feedbackformer og hvorvidt semaforiske gestikker anses for at være social acceptable. Der er i \fullref{KategoriseringerAfGestikker} blevet afgrænset fra, at arbejde videre med tegnsprog og gestikulerende gestikker. I forhold til hvilke komplikationer, der er forårsaget af gestik, er det primære fokus rettet mod det menneskelige aspekt, for på den måde at undersøge hvilke gestikker, der bedst egner sig til perifer interaktion med et musikanlæg eller en musikafspiller. Dertil skal det overvejes hvor meget fokus, der skal rettes mod feedbackformer, da der ikke er en fælles konsensus om, hvorvidt der skal være feedback i situationer, hvor der i forvejen fremgår funktionel feedback både i forhold til gestikkerne og i forhold til musikken. Såfremt den perifere interaktion udføres med semaforiske gestikker, så er der nogle faktorer, der skal tages højde for i forbindelse med brugernes og tilskuernes sociale accept.\blankline
% 
På baggrund af de foregående afsnit er det nu muligt, at konkretisere projektets fokus, igennem en afgrænsning, som efterfølgende munder ud i en problemformulering.  








