\section{De udvalgte semaforiske gestikker}
\label{DiskussionUdvalgteGestikker}
%
For at besvare den første problemstilling: \textit{Hvilke semaforiske gestikker skal knyttes til hver af de seks mest gængse funktioner i Bang $\&$ Olufsens fremtidige musikanlæg, for at interaktionen kan foregå i den perifere opmærksomhed?}, deltog 18 testpersoner i udvælgelsen af hvilke semaforiske gestikker, der skal knyttes til henholdvis pause og start, skift musiknummer samt justering af lydstyrken. Da fokus for den del af undersøgelsen var at udvælge de semaforiske gestikker, blev der ikke undersøgt perifer interaktion. 

En gennemgående tendens for testpersonernes begrundelser relateret til de udvalgte gestikker er, at testpersonerne forbinder dem med noget velkendt; krokodillenæbet til at pause og starte musikken forbindes med \enquote{ti stille}, swipe-bevægelsen med to strakte fingre forbindes med, hvordan der normalvis interageres med en smartphone eller tablet, hvor gestikken til at justere lydstyrken forbindes direkte med, hvordan lydstyrken justeres på et musikanlæg via en drejeknap. Udover at testpersonerne har tendens til at vælge gestikker, som forbindes med noget velkendt, er der ydermere en tendens til at vælge gestikker, som ikke normalvis indgår som en naturlig del af kropssproget. De udvalgte semaforiske gestikker medfører, at testpersonerne højst sandsynligt ikke bør bekymre sig om, hvorvidt de ubevidst gengiver gestikkerne i situationer, hvor hensigten ikke er at interagere med musikanlægget. Med udgangspunkt i, at de udvalgte semaforiske gestikker normalvis ikke er en naturlig del af kropssproget, reduceres sandsynligheden for et \textit{Midas touch problem}, både i forhold til om brugeren decideret gengiver en forkert gestik og i forhold til om produktet reagerer på en gestik, som ikke er henvendt til produktet. 

Særligt i forhold til at skifte musiknummer, hvor valget stod mellem GP1 og GP5, blev der taget højde for \textit{Midas touch problem}. Fælles for de to gestik-par er, at de begge er bygget op omkring en swipe-bevægelse, hvor der i GP1 swipes med hele hånden og GP5 swipes med to udstrakte fingre. Selvom de fleste testpersoner favoriserede GP1, vurderes det, at risikoen for at opleve et \textit{Midas touch problem} er større for GP1 sammenlignet med G5. Vurderingen bygger på, at der ved GP5, foruden swipe-bevægelsen, kræves et specifikt håndtegn som systemet skal reagere på, hvilket ikke er tilfældet for GP1.\blankline
%
I henhold til hvilken semaforisk gestik, der anvendes, er den gennemgående tendens, at statiske gestikker er utilregnelige, fordi de ikke tillader kontrol over eksempelvis, hvor meget lydstyrken justeres eller hvilket musiknummer, der skiftes til. Derudover tyder det på, at testpersonerne foretrækker gestikker, hvor det er muligt at danne en form for visuelt billede af, hvor meget lydstyrken justeres. Dette udledes blandt andet fra testpersonernes begrundelser for, hvorfor de fravælger de statiske gestikker, men kan også udledes fra følgende:  
%
\begin{quotation}
	\noindent
	\textit{Og så 5'eren fordi at det virker sådan at man har mere kontrol der i forhold til 9'eren, hvor du bare står sådan og vifter opad. Det kunne jo være hvilken som helst volumen du var kommet op på.} TP7, \autoref{app:NoterValgAfGestikker}.
\noindent
\end{quotation}
%
GP9 gengives ved en \enquote{kom så}- eller \enquote{ro på}-bevægelse, hvor det, sammenlignet med de resterende gestik-forslag, er svært at danne et visuelt billede af, hvor meget lydstyrken justeres. Dertil er der ved GP9 og de statiske gestikker ingen fysisk begrænsning, som testpersonerne kan forholde sig til. Ved GP2, som består af en cirkulær bevægelse dannet ud fra albueleddet, håndleddet og fingrene, oplever testpersonerne en naturlig fysisk begrænsning, når det på et tidpunkt ikke længere er muligt at rotere hånden. Lignende er gældende GP3 og GP4, hvor testpersonerne begrænses af, hvor meget de er i stand til at hæve og sænke armen.

Som det blev nævnt i \fullref{TestresultaterValgAfGestikkerValgVolumen} var det ikke muligt at ekskludere hverken GP2, GP3 eller GP4, hvorfor projektgruppen foretog beslutningen med udgangspunkt i, at testpersonerne forbinder GP2 med en fysisk drejeknap og fordi det antages, at risikoen for et \textit{Midas touch problem} er mindre ved GP2 end ved både GP3 og GP4. Det er derfor ikke muligt at fastslå hverken, hvordan GP3 eller GP4 påvirker den sociale kontekst, når værterne interagerer med musikanlægget, samtidig med at de fører samtalen med gæsten. Sammenlignes GP2 med både GP3 og GP4 er bevægelsesmængden større i GP3 og GP4, hvorfor det antages, at de to gestik-par i højere grad fanger gæstens opmærksomhed, hvilket resulterer i at risikoen for at samtalen afbrydes, forøges.\blankline
%
Da der under udvælgelsen af semaforiske gestikker blev fremsat forbedringsforslag relateret til, at gestikkerne retningsbestemmes, blev det besluttet at værterne i undersøgelsen af, hvordan interaktionsformen påvirker en social kontekst, skulle rette gestikkerne mod musikanlægget. Fordelen ved at gestikkerne er retningsbestemt er dels, at det minimerer risikoen for at gestikkerne maskeres af et eller flere objekter og dels, at det fra et teknologisk synspunkt formentlig vil være lettere at udvikle softwaren til at genkende gestikkerne. Baseret på både værter og gæsters udsagn forefindes der ingen indikation af, at det er forstyrrende eller på anden måde negativt, at gestikkerne er retningsbestemt. Et alternativ til at gestikkerne retningsbestemmes kan være at designe en aktiveringsgestik, som skal gengives, inden brugeren får adgang til at interagere med musikanlæggets funktioner. Det kan dog medføre andre problemer såsom; hvilken gestik det skal være, om det skal være den samme gestik for alle funktioner eller om hver funktion skal tilknyttes sin egen aktiveringsgestik. Det vurderes, at et af de største problemer ved at indføre en aktiveringsgestik er, hvor længe aktiveringen skal være aktiv og hvordan varigheden kommunikeres til brugeren. Det anses som værende et problem, fordi det vil kræve at brugeren konstant ved, hvor lang tid der er gået af aktiveringen for at være sikker på, at systemet er modtagelig overfor gestikker. Det forventes, at der fra brugerens synspunkt vil opstå et irriteretionsmoment, når brugeren gestikulerer til produktet, som ikke reagerer fordi aktiveringen er overskredet. Det vil medføre, at brugeren først skal gengive aktiveringsgestikken efterfulgt af gestikken til eksempelvis at pause musikken. Hvis problemet opstår, er det ikke muligt at holde interaktionen i den perifere opmærksomhed, da det, som belyst i \fullref{Potentiale}, vil resultere i at brugeren er nødt til at foretage interaktionen i den centrale opmærksomhed. 

Derudover, hvis argumentet for at indføre en aktiveringsgestik er, at produktet dermed kan registrere de efterfølgende gestikker, bør der ikke implementeres en aktiveringsgestik. For hvis produktet er i stand til at registrere en aktiveringsgestik for efterfølgende at registrere de gestikker, som styrer musikken, bør produktet være i stand til at registrere disse gestikker uden en aktivering. 

Flere af værterne kommenterer, at gestikkerne til henholdvis pause og start samt til at skifte musiknummer relaterer sig til et enten/eller-princip, enten er musikken sat på pause eller så er den ikke sat på pause. Det gør sig dog ikke gældende for justeringen af lydstyrken, da det kræver at produktet er i stand til at registrere en specifik bevægelsesmængde; hvor meget lydstyrken justeres. I tilfælde af, at produktet ikke er i stand til med det samme at reagere på justeringen af lydstyrken, kan det være en idé at tilføje denne gestik et nyt element. Formålet med en aktiveringsgestik kan overføres til formålet med det nye element i GP2; brugeren skal først gengive aktiveringen for at foretage yderligere interaktioner. I det henseende fremsætter TP1 et forbedringsforslag; inden rotationen initieres skal hånden lukkes sammen i en knytnævne og først når hånden åbnes kan den cirkulære bevægelse, som justerer lydstyrken, påbegyndes, jævnfør \fullref{TestresultaterValgAfGestikkerForbedringGP2Volumen}. I den forbindelse fremsætter projektgruppen et alternativ til knytnæven; brugeren skal først indikere, at der tages fat i en fiktiv drejeknap, hvorefter den cirkulære bevægelse, som justerer lydstyrken, påbegyndes. Da dette kun er et forslag forefindes der ingen evidens for, at det fungerer i praksis, hvilket derfor bør undersøges i forhold til, hvilken indflydelse det har på brugen af GP2 og i forhold til hvilke andre måder det kan gøres på.       


