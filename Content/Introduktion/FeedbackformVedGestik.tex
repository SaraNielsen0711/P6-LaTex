\subsection{Feedbackformer}
\label{Feedbackformer}
%
Det lader til, at der er en del udbredte meninger om, hvorvidt der skal være feedback i alle interaktionssituationer samt hvilken type feedback der skal anvendes. Diskussionen omkring feedbackformer vil primært omhandle semaforiske og manipulerende gestikker i forbindelse med, at interagere med et musikanlæg eller en musikafspiller, hvilket er oplagt fordi projektet er et samarbejde med Bang $\&$ Olufsen, jævnfør \fullref{SammenspilMedBO}. \blankline
%
\textcite[s. 10]{PDF:NaturalUserInterfaces} argumenterer for, at der skal være feedback for på den måde, at hjælpe brugeren til at forstå årsagen til eventuelle fejl og dertil lære den korrekte adfærd. Hvorimod \textcite[s. 16]{PDF:PIEmbeddingHCIOnTheRelevance} argumenterer for, at hvis gestikkerne er semaforiske, så er det ikke nødvendigt med, hvad \textcite[s. 16]{PDF:PIEmbeddingHCIOnTheRelevance} kategoriserer som værende kunstig feedback, da brugeren automatisk får feedback fra den proprioceptive sans, når der sker en aktivering af musklerne. På baggrund af det argumenterer \textcite[s. 16]{PDF:PIEmbeddingHCIOnTheRelevance} ydermere for, at det er derfor, at semaforiske gestikker egner sig til perifer interaktion. I forhold til de semaforiske gestikker så gav testpersonerne, i undersøgelsen foretaget af \textcite[ss. 172-173]{PDF:ComparingInputModalities}, udtryk for, at de manglede haptisk feedback. Årsagen til det skyldes formegentligt, at testpersonerne først og fremmest skal vænne sig til og stole på de semaforiske gestikker, hvorefter der formegentlig ikke længere vil være et behov for haptisk feedback, \parencite[s. 174]{PDF:ComparingInputModalities}. 

Derudover argumenterer \textcite[s. 3]{PDF:FacilitatingPIDesignAndEvaluation} for, at det er problematisk at give brugeren feedback igennem den samme sensoriske modalitet, som hvor opgave befinder sig, da der kan opstå interferens. Derfor vil det være uhensigtmæssigt, at anvende auditiv feedback når brugeren samtidig lytter til musik, \parencite[s. 3]{PDF:FacilitatingPIDesignAndEvaluation}. I tilfælde hvor brugeren perifert skal interagere med et musikanlæg eller en musikafspiller, kommenterer \textcite[s. 19]{PDF:PIEmbeddingHCIOnTheRelevance}, at brugeren i forvejen modtager feedback i form af funktionel feedback, hvor brugeren kan høre at musikken stopper, at der skrues op eller ned for lyden eller at der skiftes musiknummer. Hvorimod hvis den perifere interaktion foregår igennem manipulerende gestikker, så kan ikke-visuel feedback potentielt være med til at minimere belastningen af de mentale ressourcer, \parencite[s. 3]{PDF:FacilitatingPIDesignAndEvaluation}. Dog giver testpersonerne, i undersøgelsen foretaget af \textcite[s. 173]{PDF:ComparingInputModalities}, udtryk for, at de ikke manglede yderligere feedback end det de fik ved den funktionelle feedback. Hvor den funktionelle feedback i dette tilfælde vedrører den proprioceptive sans i og med at testpersonerne manipulerer et fysisk objekt, et knop-baseret håndtag og touchskærmen, og de kan høre at systemet reagerer på deres input ved enten at pause eller starte musikken, skrue op eller ned for lyden eller skifte musiknummer. \blankline
%
\textcite[ss. 1263-1268]{PDF:ComparingModFeedback} undersøger om forskellige feedback typer, fra visuel feedback i det perifere til feedback direkte på skærmen, har en effekt på testpersonernes præstationsevne. Ifølge \textcite[ss. 1267-1268]{PDF:ComparingModFeedback} har feedback ingen effekt på præstationsevnen i den sekundære opgave, men testpersonerne efterspørger alligevel feedback, så de kan få information omkring deres handlinger. Ligesom \textcite[s. 174]{PDF:ComparingInputModalities} så argumenterer \textcite[s. 1268]{PDF:ComparingModFeedback} for, at det ikke nødvendigvis ville være tilfældet i en feltundersøgelse, da testpersonerne da ville have mere tid til at vænne sig til den perifere interaktion. 

Det lader derfor til at testpersoner, som testes i et laboratorium har nogle andre behov end hvad tilfældet ville være, hvis de blev testet i en feltundersøgelse, da de automatisk vil få længere tid til at vænne sig til den perifere interaktion. Det kan derfor være svært at afgøre, hvorvidt der skal være en form for feedback eller ej, når en bruger interagere med et musikanlæg eller en musikafspiller i den perifere opmærksomhed. Dette gør sig særligt gældende når interaktionsformen enten bygger på semaforiske og/eller manipulerende gestikker.  
%
