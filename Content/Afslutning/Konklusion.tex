\chapter{Konklusion}
\label{Konklusion}
%
Der vil i konklusionen tilstræbes, at besvare den fremsatte problemformulering: 
%
\begin{quotation}
	\noindent
	\textit{Hvilke semaforiske gestikker skal knyttes til hver af de seks mest gængse funktioner i Bang $\&$ Olufsens fremtidige musikanlæg, for at interaktionen kan foregå i den perifere opmærksomhed?\blankline
		%
		Hvordan vil de udvalgte semaforiske gestikker påvirke den sociale accept?}\blankline
\end{quotation}
%
Da det er valgt, at definere de seks funktioner i par af tre, som udgør henholdvis; pause og start, skift musiknummer frem og tilbage, samt justering af lydstyrken, udvælges der tilhørende gestik-par. For at pause musikken knyttes et krokodillenæb, der lukker sammen og for at starte musikken knyttes et krokodillenæb, der åbner op, jævnfør \autoref{fig:GestikPar5PauseApp}. For at skifte musiknummer knyttes en swipe-bevægelse fra højre mod venstre med pege- og langefinger strakt, for at skifte til det næste musiknummer og fra venstre mod højre for at skifte til det forrige musiknummer, jævnfør \autoref{fig:GestikPar5SkiftApp}. Til at justere lydstyrken stod valget mellem tre gestik-par; GP2, GP3 og GP4. I denne undersøgelse vælges det at fokusere på GP2, som gengives ved en cirkulær bevægelse med halvbøjede fingre med uret for at skrue op og mod uret for at skrue ned, som hvis det var en drejeknap lydstyrken justeres på, jævnfør \autoref{fig:GestikPar2VolumenApp}. I henhold til diskussionen omkring aktiveringsgestikker, jævnfør \fullref{DiskussionUdvalgteGestikker}, konkluderes det, at såfremt produktet ikke er i stand til omgående at reagere på rotationen i GP2, bør der implementeres en form for aktivering. Aktiveringen kan eksempelvis være, at brugeren først skal gribe fat i en fiktiv drejeknap for at indikere, at lydstyrken efterfølgende justeres.

Derudover konkluderes det, at de tre gestik-par ikke forekommer naturligt i kropssproget, hvorfor risikoen for et \textit{Midas touch problem} minimeres. I tillæg konkluderes det, at fordi testpersonerne direkte forbinder de udvalgte gestikker med noget velkendt, skal de ikke først lære gestikken, hvorfor det ydermere konkluderes, at gestikkerne i forvejen ikke kræver en stor mængde kognitive ressourcer.\blankline 
%
På baggrund af undersøgelsen af interaktionsformen i en social kontekst konkluderes det, at de udvalgte semaforiske gestikker ikke påvirker den social accept negativt, jævnfør \fullref{SocialAcceptDelkonklusion}. Derimod konkluderes det, at de udvalgte semaforiske gestikker socialt accepteres i en social kontekst. Derudover forefindes der ingen indikationer af, at de udvalgte semaforiske gestikker er forstyrrende for samtalen, da værten er i stand til at interagere med musikanlægget samtidig med at føre samtalen med gæsten, jævnfør \fullref{SocialAcceptDelkonklusion}. Ydermere blev det besluttet, at retningsbestemme gestikkerne, hvorfor værterne skulle rette gestikkerne mod anlægget. Udover at retningsbestemme gestikkerne formentlig er den primære årsag til, at værterne kigger hen imod musikanlægget, når de reagerer på et hint, konkluderes det, at det hverken påvirker den sociale accept eller værternes evne til at udføre begge opgaver. Derfor konkluderes det, at gestikkerne bør retningsbestemmes, da det formentlig vil gøre det nemmere at registrere og reagere på brugerens gestikker.  

I \fullref{DiskussionFunktionerCasualInteraction} diskuteres hvilke forudsætninger, der skal være mødt før en interaktion kan foregå i den perifere opmærksomhed og selvom det ikke direkte kan påvises, hvorvidt interaktionen med musikanlægget foregår i den perifere opmærksomhed, konkluderes det, at forudsætningerne tilnærmelsesvist er mødt, hvorfor det ligeledes konkluderes, at interaktionen kan foregå i den perifere opmærksomhed. Den eneste forudsætning, der ikke imødekommes relateres til, at testpersonerne ikke nødvendigvis har repeteret interaktionen gentagende gange for, at det bliver en indøvet rutine.\blankline
%
Det er ikke muligt at afgøre, hvorvidt der bør implementeres en form for augmenteret feedback, som konsekvent forekommer ved interaktion eller som deaktiveres efter en forudbestemt tilvænningsperiode eller efter brugerens egne behov. Det vil derfor være nødvendigt at foretage en undersøgelse af augmenteret feedback og hvordan det påvirker interaktionen, både i forhold til brugeroplevelsen og i forhold til, om det fremmer eller hæmmer perifer interaktion. For at foretage denne undersøgelse anbefales det at udvikle en fuldtfunktion prototype af produktet for at teste interaktionen i en reel brugssituation.




