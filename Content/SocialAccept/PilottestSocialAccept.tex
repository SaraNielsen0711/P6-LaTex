\section{Pilottest}
\label{PilottestSocialAccept}
%
Formålet med denne pilottest er at undersøge, hvorvidt testdesignet fungerer og om den testperson, der agerer vært, kan interagere med musikanlægget, mens testpersonen planlægger ferie med gæsten. Derudover køres pilottestene for at begge testledere har styr på deres opgaver og for at undersøge om den afsatte tid passer med hvor lang tid testen tager. 
%
Til pilottesten blev der testet på seks studerende, fordelt på tre par, der ellers ikke lever op til kriterierne for at være testpersoner, jævnfør \fullref{TestpersonerSocialAccept}. Årsagen til disse testpersoner blev testet er at disse var tilgængelige og at formålet med pilottesten, som nævnt, var at teste om testdesignet fungerede. Resultaterne fra pilottesten fremgår af HENVISNING. 

Baseret på resultaterne fra pilottestes kan det bekræftes, at testdesignet fungerer og at værten sagtens kan finde ud af at styre musikanlægget med semaforiske gestikker mens en samtale med gæsten opretholdes. Derudover fik testlederne ud fra pilottestene præciseret, hvad der skulle siges og hvilke dele der var vigtige at lægge vægt på overfor værten. Heriblandt blev det valgt, at værten kun skulle reagere på de hints der blev præsenteret og ikke selv sidde og lege med anlægget. Efter pilottestene var kørt igennem viste det sig at de 35-45 minutter der var afsat til både familiarisering og selve testen blev overholdt. 

Ud fra de tre pilottest viste det sig altså at testdesignet fungerede, hvorefter den rigtige test kan eksekveres. 

