\chapter{Oversigt over elektronisk bilag}
\label{app:OversigtOverElektroniskBilag}
%
%
\section{Samtykkeerklæring til valg af gestikker}
\label{app:SamtykkeerklaeringValgAfGestikker}
%
Samtykkeerklæringen som testpersonerne blive opfordret til at læse og underskrive forefindes i (/ElektroniskBilag/UdvaelgelseAfGestikker/SamtykkeerklaeringValgAfGestikker.pdf)
%
\section{Videooptagelser af gestik forslag}
\label{app:VideooptagelseGestikForslag}
%
De tre færdigredigerede videoptagelser af de forskellige foreslag til semaforiske gestikker til henholdvis pause og start, skift musiknummer og skru op og ned for musikken forefindes i (/ElektroniskBilag/UdvaelgelseAfGestikker), hvor de angives som PauseStart.mp4, SkiftSang.mp4 og Volumen.mp4
%
\section{Resultater fra pilottest til valg af gestikker}
\label{app:ResultaterPilottestValgAfGestikker}
%
Data fra de fire testpersoner, som deltog i pilottesten forefindes i \\
(/ElektroniskBilag/DataValgAfGestikker/Pilottest-resultater.pdf)
%
\section{Rådata tilhørende spørgeskema}
\label{app:RaaDataSpoergeskema}
%
Testpersonernes respons forefindes i (/ElektroniskBilag/DataValgAfGestikker/DataSpoergeskema.csv)
%
\section{Testpersonernes top tre rangering}
\label{app:TestpersonernesTopTre}
%
Testpersonernes top tre rangering af de semaforiske gestikker både til pause og start, skift musiknummer og justering af lydstyrke, samt hvilke gestikker de fravælger forefindes i (/ElektroniskBilag/DataValgAfGestikker/DataTest1.xls)
%
\section{Observatørens notater til valg af gestikker}
\label{app:NoterValgAfGestikker}
%
Observatørens notater og testpersonernes respons til valg af gestikker forefindes i (/ElektroniskBilag/DataValgAfGestikker/Svarark fra test1.pdf)
%
\section{Videooptagelser af testpersoner}
\label{app:VideooptagelseValgAfGestikkerTestpersoner}
%
Udklip af videooptagelser, der er foretaget til valg af gestikker forefindes i (/ElektroniskBilag/DataValgAfGestikker/Videooptagelser), hvor de er angivet som Eksempel1Test1TP1PauseStart.mp4, Eksempel2Test1TP5PauseStart.mp4, Eksempel3Test1TP1Skift.mp4, Eksempel4Test1TP4Skift.mp4, Eksempel5Test1TP18Skift.mp4 og Eksempel6Test1TP1ForbedringsForslag.mp4
%
\section{Samtykkeerklæring til test interaktionsform af en social kontekst}
\label{app:SamtykkeerklaeringSocialAccept}
%
Samtykkeerklæringen som testpersonerne blive opfordret til at læse og underskrive forefindes i (/ElektroniskBilag/SocialAccept/SamtykkeerklaeringSocialSammenhaeng.pdf)
%
\section{Videooptagelsen til familiarisering af vært}
\label{app:VideooptagelseFam}
%
Den færdigredigerede videooptagelse brugt under familiariseringen af værten forefindes i (/ElektroniskBilag/SocialAccept/Test2.mp4)
%
\section{Eksempel på forvrængning}
\label{app:ForvraengningHint}
%
Et eksempel på forvrængningen, der bruges som hint, for at få værten til at skifte til næste musiknummer, forefindes i (/ElektroniskBilag/SocialAccept/Musiknummer3.wav). Musiknummer 3 er "I Feel It Comming" fra The Weekend ft. Daft Punk (2016)
%
\section{Rådata fra pilottest af interaktionsform i en social kontekst}
\label{app:ResultaterPilottestSocialAccept}
%
Data fra de seks testpersoner, som deltog i pilottesten forefindes i \\
(/ElektroniskBilag/DataSocialAccept/Pilottest2-resultater.pdf)
%
\section{Rådata fra test af interaktionsform i en social kontekst}
\label{app:ResultaterSocialAccept}
%
Respons fra de 12 testpersoner i de to individuelle exit-interviews samt observationer af den sociale kontekst forefindes i (/ElektroniskBilag/DataSocialAccept/Test2-resultater.pdf)
%
\section{Videooptagelse af vært og gæst}
\label{app:VideooptagelseVaertOgGaest}
%
Udklip af videooptagelser, der er foretaget i den sociale kontekst, hvor vært og gæst fører en samtale om ferieplanlægning forefindes i (/ElektroniskBilag/DataSocialAccept/Videooptagelser), hvor de er angivet som Eksempel7Test2TP2AendretAdfaerd.mp4, Eksempel8Test2TP3DovenHaand.mp4, Eksempel9Test2TP36VenstreHaand.mp4 og Eksempel10Test2TP5SamtaleOgGestikker.mp4
%

