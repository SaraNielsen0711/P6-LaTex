\section{Fremgangsmåde}
\label{FremgangsmaadeValgAfGestikker}
%
Testpersonerne får først en kort introduktion til hvad formålet med testen er; at undersøge hvilke semaforiske gestikker, der bedst egner sig til de mest gængse funktioner på et musikanlæg. Derefter får testpersonerne udleveret en samtykkeerklæring, som de opfordres til at læse og underskrive. Samtykkeerklæringen fremgår af \autoref{app:SamtykkeerklaeringValgAfGestikker}. Derefter bliver testpersonerne bedt om at udfylde et spørgeskema, hvori de, foruden at angive køn og alder, blandt andet skal besvare spørgsmål omkring deres erfaringer med gestikker og om de ejer produkter fra Bang $\&$ Olufsen. En oversigt over spørgsmålene i spørgeskemaet forefindes i \autoref{app:FlowdiagramSPG}, hvor svarmulighederne ligeledes fremgår. Når testpersonerne har udfyldt spørgeskemaet bliver de introduceret yderligere til hvad der kommer til at foregå; de bliver præsenteret for de tre videoer og får fortalt hvad deres opgaver er; efter videoen er slut skal de rangere de tre bedste gestikker og forklare hvorfor de har valgt netop dem og hvorfor de er rangeret som de er. Efterfølgende skal testpersonerne udpege den gestik som de syntes mindst om og forklare årsagen til hvorfor de har valgt den. Inden de præsenteres for den næste af de tre videoer, opfordres de til at komme med et forbedringsforslag til deres favorit gestik og i den forbindelse får de mulighed for at prøve med musik i baggrunden. Har de eksempelvis en forbedring til hvordan musikken pauses så bliver musikken tændt, hvorefter testpersonerne pauser den med deres forbedring. Årsagen til at der sættes musik på når testpersonerne skal forbedre deres favorit er for at lade dem prøve det i kontekst og så de kan vurdere hvordan det føles og foretage justeringer, hvis de har lyst. Både testleder og testperson står op når de tre videoer præsenteres og når testpersonerne skal rangere gestikkerne og udpege den de mindst kan lide. At testlederen opfordre testpersonen til at stå op, skyldes at det forventes, at det vil friste testpersonen til gengive bevægelserne så de kan fremsætte et forbedringsforslag. 

Når testpersonerne er blevet præsenteret for alle tre videoer afvikles der et exit-interview, hvor testlederen stiller følgende spørgsmål; \blankline
%
\begin{itemize}
  \item Hvordan synes du det gik? 
  \item Hvor nemt syntes du det var at udpege hvilke gestikker du bedst og mindst kunne lide? 
  \item Hvordan syntes du filmene illustrerede de forskellige gestikker? 
  \item Hvad synes du om, at skulle interagere med et interface ved brug af gestikker? 
  \item Hvis dette system skal reagere på dine gestikker hver gang du laver en, så skal systemet altid kunne opfange dine gestikker. Hvordan vil du have det med det? 
  \item Har du andre kommentarer?\blankline
\end{itemize}
\noindent
%
Exit-interviewet afvikles som et semi-struktureret interview, hvilket tillader en flydende dialog mellem testperson og testleder samt giver mulighed for at testlederen kan stille opfølgende og uddybende spørgsmål, hvis nødvendigt. Exit-interviewet afvikles siddende. Efter exit-interviewet bliver testpersonerne bedt om at gengive deres forbedringsforslag til de seks funktioner mens der afspilles musik. De bliver undervejs spurgt ind til hvordan det føles og om de ønsker at ændre noget. Grunden til at testpersonerne bliver bedt om gengive deres forbedringer igen er for at sammenholde det med hvad de gjorde, da de blev præsenteret for den specifikke video under testen. På den måde vil det være muligt at undersøge om de semaforiske gestikker ændre sig efter alle tre videoer er blevet præsenteret og om testpersonerne foretager yderligere ændringer eller om de har glemt hvad de forbedrede de individuelle gestikker til. \blankline   
%
Testens varighed afhænger af testpersonernes respons, men anslåes til at vare mellem et kvarter og en halv time, inklusiv testlederens instruktioner og det afsluttende exit-interview. Testlederens instruktioner fremgår af \autoref{app:InstruktionerValgAfGestikker}. Foruden testlederen er der en observatør til stede, hvis opgave er at notere testpersonernes respons samt sørger for styre musikken når testpersonerne opfordres til at forbedre deres fortrukne gestikker og når de afslutningsvist opfordres til at gengive gestikkerne igen. 
%
\subsection{Præsentationsrækkefølge}
\label{PraesentationsraekkefoelgeValgAfGestikker}
%
Da testen er bygget op omkring tre forskellige videoer; pause og start, skift musiknummer frem og tilbage og skru op og ned for musikken, er det favorabelt at ændre præsentationsrækkefølgen. Præsentationsrækkefølgen defineres ud fra et \textit{Counterbalanced design} hvor alle rækkefølge kombinationer er repræsenteret, med tre videoer resulterer det i seks forskellige kombinationer. Antallet af rækkefølge kombinationer udledes ved følgende udtryk;
%
\begin{equation}
	3! = 3*2*1 = 6
\end{equation}
%
Det tilstræbes at de seks forskellige kombinationer bliver præsenteret ligeligt, hvorfor det er nødvendigt med et antal testpersoner, som kan deles med seks.
%
\begin{table}[H]
	\centering
	\begin{tabular}{ |  p{4cm}  |  p{4cm}  |  p{4cm}  |}
		\hline
		Pause og start & Skift musiknummer & Skru op og ned \\ \hline
		Skru op og ned & Pause og start & Skift musiknummer\\ \hline
		Skift musiknummer & Skru op og ned & Pause og start \\ \hline
		Pause og start & Skru op og ned & Skift musiknummer\\ \hline
		Skru op og ned & Skift musiknummer & Pause og start\\ \hline
		Skift musiknummer & Pause og start & Skru op og ned \\ \hline
	\end{tabular}
	\caption{Oversigt over præsentationsrækkefølgen for de tre videoer; start og pause, skift musiknummer og skru op og ned.}
	\label{tab:PraesentationsraekkefoelgeValgAfGestikker}
\end{table}
\noindent
%
I \autoref{tab:PraesentationsraekkefoelgeValgAfGestikker} fremgår de seks forskellige kombinationer af hvordan de tre videoer bliver præsenteret for testpersonerne. Der foretages ikke ændringer på rækkefølgen hvorved gestikkerne til hver video præsenteres. Det skyldes at testpersonerne blandt andet har mulighed for at gense gestikkerne igen, de præsenteres for en opsamling hvor alle gestikkerne fremgår og der vil være en dialog mellem testleder og testperson omkring testpersonens præferencer.  
%
