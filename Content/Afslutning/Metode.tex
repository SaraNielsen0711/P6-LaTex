\section{Metodevalg}
\label{DiskussionMetodevalg}
%
I den første undersøgelse, hvor de semaforiske gestikker udvælges, præsenteres testpersonerne for forskellige forslag til semaforiske gestikker, hvilket foregår igennem tre videoer, jævnfør \autoref{app:VideooptagelserValgAfGestikker}. Herefter blev testpersonerne udspurgt om, hvor nemt de syntes det var at udpege, hvilke gestikker de bedst og mindst kunne lide samt hvordan de syntes videoerne illustrerede de forskellige gestikker.

Den gennemgående tendens er, at det er nemt for testpersonerne at udpege, hvilke gestikker de bedst og mindst kan lide, men at det i flere tilfælde er svært at tildele et gestik-par tredjepladsen. Årsagen til at det har været svært at tildele tredjepladsen, skyldes blandt andet, at testpersonerne gjorde brug af udelukkelsesmetoden, gestikken mindede mest om første- og andenpladsen eller så valgte testpersonerne det gestik-par, der egnede sig bedst til en tredjeplads. 

Samtlige testpersoner gav udtryk for, at de tre videoer illustrerede gestikkerne godt og at det var godt, at fokus var på gestikkerne og at der ingen forstyrrelse var, eksempelvis i form af, at demonstratoren påklædning ikke varierer afhængigt af video og at demonstratoren formåede at gengive det samme ansigtsudtryk. Gestikkerne optages, for at testpersonerne dels kan få et indtryk af hvilke gestikker, der er mulige at anvende og dels kan få et bedre indtryk af, hvordan gestikkerne udføres sammenlignet med hvis gestikkerne præsenteres som stilbilleder. Ydermere optages gestikkerne for, at testpersonerne alle får præsenteret gestikkerne fuldstændig ens. I den forbindelse påpeges det, at testpersonerne derfor kun får det indtryk af gestikkerne, som det illustreres i videoerne. I tilfælde af, at demonstratoren havde svært ved at gengive gestikkerne efter hensigten, resulterer det i, at det er sådan testpersonerne oplever gestikken. Dette kom særligt til udtryk ved TP16, som synes at bevægelsen i GP2 til justering af lydstyrken er ukomfortabel. 

Der er fordele og ulemper ved at testpersonerne præsenteres for gestikkerne i en klinisk udført sammenhæng. Fordelen er, at testpersonerne kan koncentrere sig om gestikkerne, uden forstyrrelser eller påvirkninger fra eksterne elementer, hvor ulempen er, at testpersonerne ikke oplever gestikken i den tiltænkte kontekst; dagligstuesituation. Af de årsager, vælges det at foretage en ny videooptagelse af de tre udvalgte gestik-par til testen i den sociale kontekst, jævnfør \autoref{app:VideooptagelseSocialAccept}. I denne videooptagelse illustreres gestikkerne i den opstillede dagligstue, hvor demonstratoren positioneres der, hvor værten ligeledes positioneres og gengiver gestikkerne mere afslappet end i de tre foregående videoer.\blankline 
%
Testen i den sociale kontekst blev udført som et \textit{Wizard of Oz}-eksperiment, da der ikke er udarbejdet en funktionel prototype af Bang $\&$ Olufsens fremtidige musikanlæg. For at påvirke værterne til at interagere med musikanlægget, anvendes der hints, da der ellers er risiko for, at værterne glemmer at interagere med musikanlægget, fordi de fordyber sig i samtalen med gæsten. 

Baseret på \autoref{fig:KiggerImodAnlaeg} samt tilhørende analyse, forefindes der ingen indikation af, at værterne oftere kigger mod anlægget ved nogle funktioner end andre. At det ikke er tilfældet anses for at være positivt, da det indikerer, at der ikke er én gestik, som er mere opmærksomhedskrævende end de andre. Derudover kan det ikke fastslås, hvorvidt værterne kigger mod anlægget, fordi de reagerer på et hint til at skifte musiknummer eller til at justere lydstyrken, da de typer af hints afpilles igennem de to højtalere, som er placeret på hver side af anlægget, jævnfør \autoref{fig:Testopstilling2}. Hvis dette er tilfældet, forklarer det ikke, hvorfor værterne kigger mod musikanlægget når musikken pauses, da dette hint er en opringning til mobiltelefonen, som er placeret på sofabordet og derfor ikke ændre på lyden fra højtalerne. At værterne skal rette gestikkerne mod anlægget, kan derfor være årsagen til, at værterne ligeledes kigger mod anlægget, hvilket indikerer, at det hverken er gestikkerne eller de forskellige hints, som provokerer værterne til at kigge mod anlægget.     

Med henblik på de forskellige hints, kan de forårsage en mindre økologisk testsituation, da det hverken er normalt at ens musik markant ændrer lydstyrke midt i et musiknummer, at musiknumrene på ens egen afspilningsliste er forvrængede eller at ens mobiltelefon ringer seks gange på relativt kort tid; testens varighed. I tillæg vurderes det, at testpersonerne blev mere forstyrret af de forskellige hints end interaktionen med musikanlægget. Specielt forvrængningen forårsagede forstyrrelser, da nogle af værterne havde svært ved at høre forvrængningen, hvor andre værter skiftede musiknummer, selvom der ikke var forvrængning. Alternativt til at værterne skifter musiknummer, fordi der er forvrængning, kan værterne istedet introduceres til, at hver gang et dansk musiknummer eller reklame afspilles, er det hintet til at skifte musiknummer.

Problemet ved at ændre hintet til enten et dansk musiknummer eller en reklame er, at testsessionen enten skal forløbe over en længere tidsperiode eller at testleder 2 skal skifte til et dansk musiknummer undervejs i et andet musiknummer, hvilket kan skabe forvirring for værterne eksempelvis i forhold til, om de begik en fejl. Endnu et alternativ er at forkorte musiknumrene, så det danske musiknummer eller reklamen kan afspilles hyppigere og samtidig passe ind i afspilningslisten.\blankline
%
Skal en perifer interaktion påvises, er det nødvendigt, foruden en fuldfunktionel prototype af produktet, at træne værterne i interaktionsformen for at gestikkerne og interaktionen bliver en rutine. Derudover er det favorabelt, at interaktionen med produktet foregår i testpersonernes egne hjem, for på den måde, at teste produktet i en reel brugssituation og ikke i en opsat dagligstue. At denne undersøgelse ikke er foretaget i forbindelse med projektet skyldes tre ting; det tidsmæssige aspekt projektet er underlagt, der er ikke udarbejdet en fuldtfunktionel prototype af produktet og at det vurderes, at de to undersøgelser foretaget i dette projekt, er nødvendige at foretage før der kan påvises en perifer interaktion.     
% 