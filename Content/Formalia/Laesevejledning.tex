\section*{Læsevejledning}
\label{Laesevejledning}
Rapporten bør læses kronologisk, da nogle afsnit antager, at læseren har kendskab til tidligere afsnit i rapporten. Derudover er rapporten struktureret således, at resultater og viden løbende diskuteres og konkluderes.
%
\subsection*{Kildehenvisninger}
Kildehenvisninger angives enten som en del af teksten eller i parentes. Et eksempel på de to kildehenvisningsmetoder: \textcite[s. 3]{PDF:PIIntroduction} eller \parencite[s. 1]{PDF:PIIntroduction}. Såfremt der refereres til en bestemt del af kilden, angives dette med sidetal, eksempeltvist; s.1 for en bestemt side eller ss. 1-3 for flere sider.
%
\subsection*{Afsnitshenvisning}
Afsnitshenvisninger angives med et afsnitsnummer efterfulgt af et afsnitsnavn. Et eksempel på en afsnitshenvisning: \fullref{PeriferInteratkion}. Samme gør sig gældende for kapitler.
%
\subsection*{Figurhenvisning}
Henvisninger til figurer angives med et decimaltal, som først gengiver kapitlets nummer efterfulgt af figurnummeret i det pågældende kapitel. Et eksempel på en figurhenvisning: \autoref{fig:InputModalitiesMusicControl}, der svarer til figur 2 i kapitel 1. 
%
\subsection*{Bilagshenvisninger}
Henvisninger til bilag angives med et bogstav. Et eksempel på en bilagshenvisning: \autoref{app:InterviewLyleClarke}. Udover bilaget, der forefindes til sidst i rapport, udarbejdes der et tilhørende elektronisk bilag. Henvisninger til det elektroniske bilag angives ved en afsnitshenvisning, hvor stien til det pågældende bilag fremgår i det pågældende afsnit. 
%
%\subsection*{Decimalseparator}
%Der benyttes "." (dot), som decimalseparator, af hensyn til diverse programmer benyttet til databehandling.
%
%\subsection*{Ordforklaring}
%Nedenfor findes en liste med ord og begreber, som benyttes i rapporten og synes vigtige at forklare for at opnå en fælles forståelse.
%\fxfit{Ordforklaring}
%%
%\blankline
%\begin{itemize}
%  \item En \textbf{almen bruger} er et vidt begreb, men benyttes igennem rapporten til at beskrive en bruger som hverken er ekspert indenfor, eller ubekendt med, et givet emne.
%  %\item Programme Loudness, er et udtryk for 
%  %\item Maximum True Peak Level er 
%  %\item Loudness Range
%  %\item \textbf{System Usability Scale} forkortes igennem rapporten til "SUS"
%  %\item \textbf{The Modified Cooper-Harper Rating Scale} forkortes igennem rapporten til "MCHRS"
%\end{itemize}