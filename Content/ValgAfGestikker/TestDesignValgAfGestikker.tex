\chapter{Testdesign}
\label{TestdesignValgAfGestikker}
%
I de følgende afsnit vil forskellige aspekter vedrørende den test, som skal klarlægge hvilke gestikker, der skal knyttes til de seks mest gængse funktioner på et musikanlæg, blive belyst. De forskellige aspekter vil omhandle testens omfang, fremgangsmåde, test sessionen, hvilken type testpersoner der vil blive testet på, rollefordelingen, materialer og hvor testen afvikles. Afslutningsvist vil erfaringerne fra en pilottest blive belyst.   
%

\section{Testens omfang}
\label{TestensOmfangValgAfGestikker}
%
Med udgangspunkt i og for at besvare problemstillingen: \textit{Hvilke specifikke semaforiske gestikker skal knyttes til hver af de seks mest gængse funktioner i Bang $\&$ Olufsens fremtidige musikanlæg, for at interaktionen kan foregå i den perifere opmærksomhed?} vil omfanget af denne test primært indebære en undersøgelse af hvilke specifikke gestikker, der skal knyttes til hver af de seks mest gængse funktioner, samt hvordan testpersonerne forholder sig til denne form for interaktion. Formålet med testen er derfor hverken at undersøge om interaktionen med et musikanlæg kan foregå i den perifere opmærksomhed eller at undersøge hvordan gestikkerne påvirker social accept. 

Der blev i \fullref{UdvaelgelseAfGestikker} udvalgt forskellige semaforiske gestikker til henholdvis pause og start, \autoref{tab:IndsamledeGestikkerPause}, skift musiknummer frem og tilbage, \autoref{tab:IndsamledeGestikkerSkift}, og til at skrue op og ned for musikken, \autoref{tab:IndsamledeGestikkerVolumen}. Formålet med at udvælge flere forskellige semaforiske gestikker til hver af de seks funktioner er dels for at illustrere hvordan interaktionen potentielt kan foregå, dels for at undersøge hvilke egenskaber testpersonerne foretrækker og dels for at inspirere testpersonerne til at komme med et forbedringsforslag. 
%
