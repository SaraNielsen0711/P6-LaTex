\chapter{Projektafgrænsning}
\label{Afgraensning}
%
På baggrund af de foregående kapitler er det muligt, at konkretisere projektets fokus igennem en afgrænsning, hvor der efterfølgende udarbejdes en problemformulering. I følgende afsnit opsummeres og specificeres hvilken tilgang til perifer interaktion med et musikanlæg, der fremadrettet vil blive fokuseret på i samarbejdet med Bang $\&$ Olufsen.\blankline
%
Da formålet er, at interaktionen med Bang $\&$ Olufsens musikanlæg foregår i den perifere opmærksomhed, vil denne interaktion anses som værende en sideopgave til en primær opgave. Af \fullref{RelateretUndersoegelser} fremgår det, at det er en fordel at anvende enten manipulerende eller semaforiske gestikker til perifer interaktion. Et af formålene ved Bang $\&$ Olufsens fremtidig musikanlæg er, at de valgte funktioner skal betjenes uden en form for fysisk fjernbetjening, hvorfor manipulerende gestikker udelukkes. Der vil fremadrette kun fokuseres på semaforiske gestikker, der understøtter den proprioceptive sans. Fordelen ved at inddrage den proprioceptive sans, er at den er uafhængig af den visuelle opmærksomhed, hvorfor det vurderes at være velegnet til at løse en sideopgave i den perifere opmærksomhed, jævnfør \fullref{Potentiale}.     

At anvende semaforiske gestikker, som interaktionsform i den perifere opmærksomhed er stadig forholdvist nyt, hvorfor der ikke findes et standardiserede sæt af gestikker, som henvender sig til musikkontrol, i hvert fald ikke hvad der er projektgruppen kendt. Det er derfor nødvendigt, at foretage en dybere undersøgelse af hvilke semaforiske gestikker, der egner sig bedst til de udvalgte funktioner. Da semaforiske gestikker ikke indgår som en naturlig del af kropssproget, er det muligt at denne type gestikker kan reducere risikoen for et \textit{Midas touch problem}.\blankline
%
Udførelsen af gestikker blev i \fullref{UdfoerelseAfGestik} kategoriseret i forhold til bevægelsesmængden i henholdsvis mikro- og makrogestikker. Da det forventes, at interaktionen med musikanlægget kan foregå ved forskellige afstande og op mod fire meter til brugeren, er det ikke på nuværende tidspunkt muligt at afgrænse sig til udelukkende at arbejde med enten mikro- eller makrogestikker. Der kan dog være en fordel i, særligt i forhold til social accept og hvorvidt brugeren føler sig komfortabel ved at udføre semaforiske gestikker, at designe gestikkerne så de kan gengives tæt på brugerens egen krop. Det tilstræbes derfor, at opnå et kompromis mellem mikro- og makrogestikker, da de hver især bidrager med positive og negative aspekter, jævnfør \fullref{UdfoerelseAfGestik}.

I relation til hvorvidt brugeren føler sig komfortabel, blev der i \fullref{Socialaccept} fremlagt forskellige typer af semaforiske gestikker. På baggrund af dette afgrænses der fra at anvende spændingsfyldte gestikker, da de ikke anses for at være socialt acceptable, \parencite[s. 277]{PDF:WouldYouDoThat}. Endvidere afgrænses der fra hemmelighedsfyldte gestikker, da de ikke giver mening i forhold til interaktion med et musikanlæg. De to resterende, udtryksfyldte- og magiske gestikker, egner sig både til interaktion med et musikanlæg samt til perifer interaktion.\blankline
%
I \fullref{KomplikationerGestikker} præsenteres nogle af de komplikationer, som gestikker kan forårsage, det gælder både teknologiske såvel som menneskelige komplikationer. Der afgrænses fra at undersøge det tekniske aspekt, da formålet med projektet primært retter sig mod at udvælge hvilke semaforiske gestikker, der skal knyttes til de valgte funktioner, så interaktionen med Bang $\&$ Olufsens fremtidige musikanlæg kan foregå i den perifere opmærksomhed.

Da Bang $\&$ Olufsens fremtidige musikanlæg på nuværende tidspunkt befinder sig på konceptbasis, er det ikke muligt at foretage test på et fuldt funktionelt system. Det tilstræbes dog at teste semaforiske gestikker i en social kontekst, for at undersøge om det er muligt at interagere med musikanlægget som en sideopgave til en ikke relateret primær opgave.\blankline
%
Undersøges behovet for feedback ved perifer interaktion, er der, som beskrevet i \fullref{Feedbackformer}, delte meninger. Der bliver både argumenteret for, at det ikke er nødvendigt med augmenteret feedback ved semaforiske gestikker, grundet den funktionelle feedback fra den proprioceptive sans og fordi brugeren får funktionel feedback fra musikken. Modsat argumenteres der for, at den funktionelle feedback ikke er tilstrækkelig. Selvom feedback ikke nødvendigvis fremmer brugerens præstationsevne, hjælper feedback brugeren igennem interaktionen. Derudover argumenterer \textcite[s. 21]{PDF:FacilitatingPIDesignAndEvaluation} for, at multimodal feedback, som er feedback der registreres af flere sanser, styrker perifer interaktion. 

Fokus for dette projekt rettes ikke mod at designe en bestemt form for feedback, men da det er et betydningsfyldt aspekt inden for interaktions design, vil feedback blive yderligere diskuteret fremadrettet.\blankline
%
På baggrund af foregående er projektet afgrænset til at undersøge, hvordan perifer interaktion kan foregå med semaforiske gestikker samt hvordan disse egner sig i en social kontekst, hvor interaktionen med musikanlægget er en sideopgave. 
