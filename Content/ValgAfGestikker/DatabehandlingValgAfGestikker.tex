\chapter{Testresultater}
\label{Testresultater}
%
Igennem de følgende afsnit vil testresultaterne blive analyseret og diskuteret. I de tre første sektioner vil fokus være på at udvælge hvilke semaforiske gestikker, der skal knyttes til henholdvis at pause og starte musikken, skifte musiknummer og til at justere lydstyrken. Udover at udvælge semaforiske gestikker, som testpersonerne foretrækker at udføre, er det et kriterium for udvælgelsen, at gestikkerne ikke er nogen, som ikke indgår naturligt i kropssproget. Efterfølgende belyses testpersonernes holdning til, hvordan det er at interagere med gestikker samt deres holdning til hvordan gestikker registreres af et produkt.\blankline
%
Der er i alt indsamlet data fra 18 testpersoner; 7 kvinder og 11 mænd, hvoraf én af de mandlige testpersoner er venstrehåndet. Testpersonernes alder spænder fra 22 år til 32 år med en gennemsnitalder på 24 år og ud af de 18 testpersoner er der én, som ikke er studerende på Aalborg Universitet. Testpersonerne, som er under uddannelse, er blandt andet fra Sundhedsteknologi, Matematik-teknologi, Matematik-økonomi, Kemiteknologi, Idræt, Medicin, Energi og Indeklima, Sociologi og Erhvervsjura. Ingen af de 18 testpersoner angiver, at de har et eller flere Bang $\&$ Olufsen produkter. Der henvises kun til det indsamlede data i tilfælde af at testpersonerne citeres, ellers fremgår alt indsamlet data af bilag. Data fra spørgeskemaet er vedlagt i \autoref{app:RaaDataSpoergeskema}, testpersonernes top tre rangering fremgår af \autoref{app:TestpersonernesTopTre} og observatørens notater er vedlagt i \autoref{app:NoterValgAfGestikker}. Fremadrettet vil testpersonerne angives som TP efterfulgt af nummer, eksempelvis vil testperson 1 forkortes til TP1. Tilsvarende er gældende for de forskellige gestik-par, som angives GP efterfulgt af nummer, eksempelvis vil gestik-par 1 forkortes til GP1.

I tilfæde af, at der er behov for at inddrage testpersonernes respons samt testpersonernes bevægelser, henvises der til følgende \autoref{app:VideooptagelseValgAfGestikkerTestpersoner}, som indeholder videomaterialet til de pågældende situationer.       