\section{Holdning til gestik genkendelse}
\label{TestresultaterOvervaagning}
%
I exit-interviewet blev de 18 testpersoner spurgt, hvordan de har det med at systemet altid kan opfange deres gestikker. Hertil er det kun TP6, TP7 og TP18, som ikke direkte udtrykker, at det er i orden, at produktet registrerer deres gestikker. Ifølge TP6 afhænger det af udbyderen; gemmes data i et datacenter og om de videregiver data, hvorimod TP7 foretrækker, at det er en form for sensor, som registrerer bevægelserne og ikke et kamera, da det ved et kamera potentielt kan være muligt for fremmede at tvinge sig adgang. Dog kommenterer testpersonen, at det er en meget lille ting. Ifølge testperson 18, så vil det være værre hvis produktet har internetforbindelse, da det så er muligt at sende data. Dog anerkender testpersonen, at det er en nødvendighed for at produktet både kan blive bedre og lære undervejs. TP18 fremsætter en lister af krav som producenten, i det her tilfælde Bang $\&$ Olufsen, skal kunne forsvare, før testpersonen vil have det fint med registreringen. På testpersonens liste over ting producenten skal forsvare indgår, blandt andet;\blankline 
%
\begin{itemize}
  \item Dokumentation for hvad data bruges til
  \item Hvor data sendes hen
  \item Hvem har adgang til data
  \item Hvilken kryptering anvendes der
  \item Optager produktet eller filmes der bare
  \item Hvem lagrer data
  \item Hvordan lagres data\blankline
\end{itemize}
%
Såfremt TP18 får information om dette, så vil testpersonen være villig til at gå på kompromis. Fælles for TP1, TP6, TP10 og TP18 er, at de begge giver udtryk for, at der skal være mulighed for, at slukke produktet så det ikke længere registrerer gestikker. Det antages at dette er en selvfølge og at produktet kun registrerer brugerens gestikker når de aktivt har tændt musikanlægget og at så snart det slukkes, så stopper registreringen. 

TP16 giver udtryk for, at medmindre data sendes op i en sky og kan tilgås af andre og at testpersonen ikke bliver filmet på den måde, så er det fint at produktet registrerer gestikkerne. Lignende tendens går igen ved TP11, som heller ikke vil have data vidersendes og ved TP3, som ikke vil have at data gemmes. TP14 kommenterer, at det er ubehageligt hvis registreringen foregår via CPR-nummer og at produktet på den måde kan registrerer testpersonens færden, men hvis det bare er testpersonens radio eller musikanlæg, så vil det være i orden. Udover TP14, så kommenterer fire andre testpersoner, at det er essentielt at produktet ikke fejlagtigt registrerer en tilfældig bevægelse som en semaforisk gestik, der er knyttet til en af funktionerne. Hvor TP8 foreslår at gestikkerne skal være retningsbestemt. I tillæg kommenterer TP5, at gestikkerne kun skal registreres i bestemte zoner, kommentaren kommer i forbindelse med at testpersonen kun vil have at registreringen foregår i bestemte rum.  

Det eneste forbehold TP4 giver udtryk for er, at testpersonen ikke vil have hele huset fyldt med kameraer i frygt for, at det bliver grimt. Derudover er det en gennemgående tendens at testpersonerne ikke betragter registreringen som et problem. Tendensen bygger blandt andet på følgende udsagn: 
%
\begin{quotation}
\noindent
\textit{Men at det registrerer og opfanger mine bevægelser det er jeg ligeglad med, det gør alle andre produkter alligevel, så det er jeg fuldstændig ligeglad med.} TP5, \autoref{app:NoterValgAfGestikker}. 
\end{quotation}
%
%
\begin{quotation}
\noindent
\textit{Det tror jeg ikke, at jeg vil have noget i mod, jeg kan jo bare lade være med at købe det hvis jeg ikke vil have det.} TP17, \autoref{app:NoterValgAfGestikker}. 
\end{quotation}
%
Størstedelen af testpersonerne giver direkte udtryk for, at de har det fint med at produktet registrerer dem, hvilket understøtter forventningen om, at denne type interaktion kan finde indpas i dagligdagen, i hvert fald så længe at produktet fungerer optimalt.  

Ud fra testpersonernes holdninger tyder det ikke på, at det er et problem at indbygge en form sensor eller kamera i musikanlægget, som derfra kan reagere på brugerens gestikker. Ydermere kan det, afhængigt af de teknologiske begrænsninger, være nødvendigt at gestikkerne rettes direkte mod anlægget, hvilket formentlig vil reducere risikoen for at produktet fejlfortolker gestikkerne. For at imødekomme ønsket fra de testpersoner, som efterspørger at produktet kun registrerer gestikker, såfremt de aktivt har tændt musikanlægget, så bør dette implementeres af Bang $\&$ Olufsen i deres fremtidige musikanlæg. Når brugeren slukker musikanlægget, så skal registreringssystemet ligeledes deaktiveres, så brugeren dermed ikke skal bekymre sig om, hvorvidt de registreres eller ej.  

