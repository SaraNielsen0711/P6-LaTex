\section{Gestikker}
\label{TestresultaterSocialAcceptGestikker}
%
Følgende analyse og diskusion foretages på baggrund af værternes respons til spørgsmålene: \textit{Hvordan syntes du det var at lære gestikkerne?}, \textit{Havde du problemer med at huske gestikkerne?}, \textit{Hvor rigtigt syntes du, at du fik gengivet gestikkerne til de enkelte funktioner?} og \textit{Var der nogle gestikker, som du følte dig mere ukomfortabel ved at udføre end andre?} samt gæsternes respons til spørgsmålene: \textit{Bemærkede du at din ven(inde)/kæreste lavede gestikker?}, \textit{Kan du gengive de gestikker som din ven(inde)/kæreste lavede?} og \textit{Hvad synes du om dem?}. Derudover inddrages observationer af hvordan værten reagerer på diverse hints. Førstedel vil omhandle dels, hvordan værterne vurderer det at lære gestikkerne, om de havde problemer med at huske gestikkerne samt hvor rigtigt de syntes de gengav gestikkerne og dels om gæsterne bemærkede det og var i stand til at gengive dem, samt deres vurdering af gestikkerne. Derefter vil fokus rettes mod om værterne fandt nogle af gestikkerne mere ukomfortable at udføre end andre. Slutteligt vil observationerne af værtens reaktion på de forskellige hints blive analyseret og diskuteret. \blankline
%
I forhold til hvordan værterne responderede på spørgsmålet: \textit{Hvordan syntes du det var at lære gestikkerne?}, så er den gennemgående tendens at det var nemt. Vært 2, vært 3 og vært 6 giver alle udtryk for, at gestikkerne kan associeres med hvordan der normaltvist interageres på en telefon, i forhold til swipe-bevægelsen, samt hvordan der normaltvist interageres med et musikanlæg. I det henseende pointerer vært 3 og vært 4, at det er de rigtige gestikker, der er blevet udvalgt til hver funktion og ifølge vært 3 er det nærmest ikke muligt at vælge nogle gestikker, som vil passe bedre. Vært 5 påpeger, at det var en fordel at gestikkerne blev præsenteret via video og ikke på papirform, da det gjorde gestikkerne mere levende. Generelt er der ingen af de seks værter, som giver udtryk for at have problemer med at huske de udvalgte gestikker. Dog kommenterer vært 1, at værten normalvist ikke pauser musikken når telefonen ringer og derfor var værten, ifølge værten selv, nødsaget til at minde sig selv om det når telefonen ringede. Vært 3 gav udtryk for, at være mere opmærksom på forvrængningen fordi det er blevet pointeret, af andre, at værten har problemer med hørelsen. Første gang vært 5 skiftede til det næste musiknummer blev det gjort med hele hånden og eftersom det gik op for værten at det var forkert, så blev det rettet. Det skal dog påpeges at fejlen opstod under familiariseringen, hvorefter værten fejlfrit gengav den korrekte gestik for at skifte til det næste musiknummer. Vært 2 påpeger at fordi gestikkerne ligger i håndleddet, så er der ingen problemer med at huske dem. 

Til spørgsmålet: \textit{Hvor rigtigt syntes du, at du fik gengivet gestikkerne til de enkelte funktioner?} giver de seks værter udtryk for, at de syntes at gestikkerne blev gengivet korrekt. Vært 3 påpeger, at værten kom til at bruge venstre hånd istedet for højre, hvilket ud fra optagelserne skyldes, at værten spiser cookies med højre hånd. Derudover kommenterer vært 3, at værten glemte at starte musikken igen efter den sidste opringning i familiariseringsrunden fordi værten troede at familiariseringen var færdig, grundet en længere samtale med testleder 2. Vært 4 pointerer at værten ikke tænkte over det, da der blev reageret på gestikkerne hver gang så der opstod ikke situationer hvor værten var nødsaget til at repetere gestikkerne. \blankline
%
De seks gæster har alle bemærket at værten lavede gestikker, men det er ikke alle, der er formår at gengive gestikkerne korrekt. Gæst 1 gengiver gestikken til at skrue op og ned korrekt, men er i tvivl om hvilken vej hånden skal dreje for enten at skrue op eller ned. I tillæg bemærker gæst 1, at vært 1 laver en swipe-bevægelse for at skifte musiknummer, men formår ikke at gengive gestikken korrekt. Til gengæld bemærker gæsten ikke hvilken gestik værten gengiver for at pause eller starte musikken. Gæst 2 kan gengive de forskellige gestikker, dog lukker gæsten hånden i en knytnæve for at pause musikken og swiper med hele hånden istedet for med to fingre. Gæst 3 kan ligeledes gengive gestikkerne korrekt med forbehold for at gæsten alternerer mellem to fingre og hele hånden i swipe-bevægelsen, hvorimod gæst 4 formår at gengive gestikkerne korrekt og uden nogen forbehold. Gæst 5 kan gengive hvordan musikken pauses men ikke hvordan den startes igen, hvilket formentlig skyldes, at gæsten slet ikke kigger på værten når musikken startes igen. Derudover formår gæst 5 at gengive gestikken til at skrue op og ned for musikken korrekt og ligesom tilfældet ved de tidligere gæster, gengiver gæst 5 swipe-bevægelsen med hele hånden. Gæst 6 gengiver gestikkerne korrekt, men formår kun at gengive gestikkerne til at pause musikken, skrue ned for musikken og til at skifte musiknummer. 

Ifølge gæst 2 og gæst 5 gav gestikkerne god meningen i forhold til hvilken funktion de tilhører. Gæst 1 tænkte ikke over dem men giver udtryk for at de var diskrete og fungerede. Derudover kommenterer gæst 1 at værten ikke skulle stå og fægte for at styre musikken. Ifølge gæst 3 er gestikkerne intuitive og derudover kiggede gæsten kun enkelt gang på gestikkerne og vurderede at de gav god mening og at de var lige til. Gæst 5 pointerer at gestikkerne var nemme og at gæsten selv var i stand til at bruge dem. Den eneste gæst, der udviser en negativ holdning er gæst 6, som giver udtryk for gestikkerne til at starte med er blæret, hvorefter gæsten vil blive træt af dem. Gæsten giver udtryk for at det er et fedt festtrick men forestiller sig, at det er træls når der er flere mennesker.
%
\subsection{Var der nogle gestikker, som du følte dig mere ukomfortabel ved at udføre end andre?}
\label{TestresultaterSocialAcceptGestikkerUkomfortabelt}
%
Ud af de seks værter er det kun vært 1 og vært 6, som direkte giver udtryk for at en af gestikkerne føltes mere ukomfortabel end de andre; den til at skrue op og ned for musikken. Tre af de fire værter, som ikke oplevede, at nogen gestikker var mere ukomfortable at udføre end andre, giver udtryk for bekymringer relateret til den specifikke gestik. Vært 5 kommenterer at der kan være noget kulturelt ved gestikken til at pause musikken, som kan misforståes, i og med at det forbindes med en "klap i"-bevægelse. Dog pointere vært 5 at ingen i værtens omgangskreds vil misforstå gestikker, hvorfor det ikke vil opstå et problem. Vært 3 og vært 4 giver begge udtryk for bekymringer relateret til gestikken, som er tilknyttet skru op og ned. Fælles for de to værter er, at de giver udtryk for, at der kan opstå problemer i forhold til hvor meget der skal drejes for at skrue en hvis mængde op eller ned. Hvor vært 3 ydermere kommenterer at gestikker kræver mere finmotorik sammenlignet med de andre udvalgte gestikker. Derudover påpeger værten, at det kan være et problem, hvis bevægelsen kræver en helt rolig og præcis hånd for at kontrollere volumen. I tillæg kommenterer værten at ved fysisk kontakt, eksempelvis med en drejeknap, så er det muligt at lave en mindre bevægelse sammenlignet med semaforiske gestikker, fordi den fysiske kontakt tillader bedre føling med den pågældende knap. 

Der er to årsager til at vært 1 finder gestikken til at skrue op og ned for musikken mere ukomfortabel end de andre gestikker; 1) der er risiko for at der pludseligt skrues for højt op for musikken og 2) hvis rotationen begyndes et dårligt sted, så håndleddet fysisk ikke tillader mere rotation. Værten påpeger dog at når grænsen for, hvor meget ens håndled kan roteres nåes, så kan grebet slippes, hvorefter det er muligt at gribe fat igen. Grebet referer til, at der tages fat i en imaginær drejeknap. Vært 6 giver udtryk for, at der var noget underligt med responsen når værten skruede op og ned. Dette kan formentlig skyldes flere ting; dels at det er testleder 2, som styrer volumen afhængigt af værtens gestikulering, så i tilfælde af at det ikke blev gjort i overensstemmelse med værtens forventning, så er det forståeligt, at responsen kan virke underlig. Derudover kan det skyldes, at testleder 2 havde svært ved præcis at tolke bevægelsesmængden i vært 6's gestikulering, da fingrenes position ændrede sig i takt med rotationen. Derudover påpeger vært 6, at det kræver tilvænning i forhold til hvor meget der skal drejes og hvornår rotationen skal stoppe. Dog pointerer værten, at volumen normaltvist ikke ændres fra 0 til 100. 
%
\subsection{Observationer af værtens gestikulering}
\label{TestresultaterSocialAcceptGestikkerObservationer}
%

Punkter der skal kigges efter i film: 
Hvad er en fejl? reaktion efter 30 sekunder, hvis de bruger én eller hele hånden til at swipe, hvis de glemmer at starte og pause musikken ved opringning.  

Hvem laver fejl, hvornår, hvorfor, hvor mange? - Hold det op med fami hvis de laver fejl.  

Hvordan er deres generelle gestikulation? Gør de nogen ting der kan tolkes som noget andet? 


Kigger gæst 5 overhovedet, når værten starter musikken?

Generelt hvor mange gange gæsterne kigger på værtens gestikker?

Kigger værten på anlægget eller gestikken, når de laver dem? 

Hvordan flyder samtalen? Bliver den helt afbrudt, skifter den lidt retning (i forhold til samtalen) eller sker der ikke noget, er de i stand til at falde ind i samtalen igen? 

Spørger gæsten ind til gestikkerne? 

Prøver gæsten gestikkerne?

Hvornår bruger værter venstre hånd - cookies? 

Notater om opkald sammenholdt med værters ændringer i adfærd. 

Værter interagere med anlæg, uden hint. 
