\chapter{Testresultater}
\label{TestresultaterSocialAccept}
%
Igennem de følgende afsnit vil testresultaterne blive analyseret og diskuteret, for at vurdere om testpersonerne var i stand til, via de udvalgte semaforiske gestikker, at interagere med et musikanlæg som en sideløbende opgave til den primær opgave; samtalen med gæsten. Derefter vil det blive undersøgt om de udvalgte semaforiske gestikker påvirker den sociale accept samt om testpersonerne vil have lyst til at interagere med gestikker.\blankline
%
Der er i alt indsamlet data fra 12 testpersoner; fire kvinder og otte mænd. To af kvinderne var venstrehåndet, hvorfor der kun er to kvindlige værter. Der deltog sammenlagt tre venstrehåndede testpersoner. Testpersonernes alder spænder fra 21 år til 27 år med en gennemsnitalder på 23,2 år. Fire ud af seks par kender hinanden igennem studiet, et par er kærester og et par er forlovet. Ud af de 12 testpersoner er der én, som ikke er studerende på Aalborg Universitet, men er under uddannelse andetsteds. Testpersonernes respons til spørgsmålene i exit-interviewet forefindes i \autoref{app:ResultaterSocialAccept}. Medmindre der refereres til samtlige testpersoner, så angives testpersonerne ud fra hvilket par nummer samt hvilken rolle; vært eller gæst de havde. Refereres der eksempelvis til testperson 1 vil dette angives som vært 1 og testperson 2 vil angives som gæst 1. 





