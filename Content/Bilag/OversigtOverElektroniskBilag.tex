\chapter{Oversigt over elektronisk bilag}
\label{app:OversigtOverElektroniskBilag}
%
%
\section{Samtykkeerklæring til valg af gestikker}
\label{app:SamtykkeerklaeringValgAfGestikker}
%
Samtykkeerklæringen som testpersonerne blive opfordret til at læse og underskrive forefindes i (/ElektroniskBilag/UdvaelgelseAfGestikker/SamtykkeerklaeringValgAfGestikker.pdf)
%
\section{Videooptagelser af gestik forsalg}
\label{app:VideooptagelseGestikForslag}
%
De tre færdigredigerede videoptagelser af de forskellige foreslag til semaforiske gestikker til henholdvis pause og start, skift musiknummer og skru op og ned for musikken forefindes i (/ElektroniskBilag/UdvaelgelseAfGestikker), hvor de angives som PauseStart.mp4, SkiftSang.mp4 og Volumen.mp4.
%
\section{Resultater fra pilottest til valg af gestikker}
\label{app:ResultaterPilottestValgAfGestikker}
%
Data fra de fire testpersoner, som deltog i pilottesten forefindes i \\
(/ElektroniskBilag/DataValgAfGestikker/Pilottest-resultater.pdf)
%
\section{Rådata tilhørende spørgeskema}
\label{app:RaaDataSpoergeskema}
%
Testpersonernes respons, forefindes i (/ElektroniskBilag/DataValgAfGestikker/DataSpoergeskema.csv)
%
\section{Testpersonernes top tre rangering}
\label{app:TestpersonernesTopTre}
%
Testpersonernes top tre rangering af de semaforiske gestikker både til pause og start, skift musiknummer og skru op og ned, samt hvilke gestikker de fravælger forefindes i (/ElektroniskBilag/DataValgAfGestikker/DataTest1)
%
\section{Observatørens notater til valg af gestikker}
\label{app:NoterValgAfGestikker}
%
Observatørens notater til valg af gestikker forefindes i (/ElektroniskBilag/DataValgAfGestikker/Svarark fra test1.pdf)
%
\section{Samtykkeerklæring til test interaktionsform i en social kontekst}
\label{app:SamtykkeerklaeringSocialAccept}
%
Samtykkeerklæringen som testpersonerne blive opfordret til at læse og underskrive forefindes i (/ElektroniskBilag/SocialAccept/SamtykkeerklaeringSocialSammenhaeng.pdf)
%
\section{Videooptagelsen til familiarisering af vært}
\label{app:VideooptagelseFam}
%
Den færdigredigerede videooptagelse brugt under familiariseringen af værten forefindes i (/ElektroniskBilag/SocialAccept/Test2.mp4)
%
\section{Eksempel på forvrængning}
\label{app:ForvraengningHint}
%
Hintet for at få værten til at skifte til det næste musiknummer er forvrængning. Forvrængning tilføjes til et musiknummer igennem programmet Audacity 2.1.3 og tilføjes i to forskellige niveauer; \textit{medium overdrive} og \textit{hård klipning}, hvor hvert niveau varer omkring 15 sekunder. Et eksempel på forvrængningen i et musiknummer, som anvendes når både vært og gæst er tilstede forefindes i (/ElektroniskBilag/SocialAccept/Musiknummer3.wav). Musiknummer 3 er "I Feel It Comming" fra The Weekend ft. Daft Punk (2016).
%
\section{Rådata fra pilottest af interaktionsform i en social kontekst}
\label{app:ResultaterPilottestSocialAccept}
%
Data fra de seks testpersoner, som deltog i pilottesten forefindes i \\
(/ElektroniskBilag/DataSocialAccept/Pilottest2-resultater.pdf)
%
\section{Rådata fra test af interaktionsform i en social kontekst}
\label{app:ResultaterSocialAccept}
%
Respons fra de 12 testpersoner i exit-interviewet samt observationer, som deltog i testen af interaktionsform i en social kontekst forefindes i (/ElektroniskBilag/DataSocialAccept/Test2-resultater.pdf)
%
\section{Videooptagelse af vært og gæst}
\label{app:VideooptagelseVaertOgGaest}
%
Videooptagelserne der er foretaget dels under værtens familiarisering og dels i den social kontekst hvor vært og gæst fører en samtale om ferieplanlægning forefindes i (/ElektroniskBilag/DataSocialAccept/Videooptagelser/SKRIV)


