\section{Gestik}
\label{Gestik}
%
I følgende afsnit vil forskellige problemstillinger, vedrørende gestik, belyses, heriblandt komplikationer forårsaget af gestik, hvilken form for feedback der er nødvendig samt den sociale accept af gestik. Inden da vil der først og fremmest blive undersøgt hvilke former for gestikker, der kan gøres brug af.  
%
\subsection{Kategoriseringer af gestikker}
\label{KategoriseringerAfGestikker}
%
Inden der dykkes ned i forskellige typer af gestikker skal det nævnes, at der i litteraturen fremgår mange forskellige termer for de samme typer af gestikker, \parencite[s. 3]{PDF:ATaxonomyOfGestures}. Eksempelvis bliver semaforiske gestikker også angivet, som frihåndsgestikker og gestikker. Der vil derfor tages udgangspunkt i, hvordan de forskellige typer af gestikker defineres, baseret på definitionerne fremsat af \textcite[ss. 4-9]{PDF:ATaxonomyOfGestures}. Ifølge \textcite[s. 4]{PDF:ATaxonomyOfGestures} kan gestikker kategoriseres i fem grupper: \textit{Deictic Gestures}, \textit{Manipulative Gestures}, \textit{Semaphoric Gestures}, \textit{Gesticulation} og \textit{Language Gestures}. \blankline
%
Da der tidligere er afgrænset fra at arbejde med stemmestyring, så vil \textit{Gesticulation} ikke blive nærmere beskrevet, da denne type af gestikker involverer både håndbevægelser og tale, \parencite[s. 7]{PDF:ATaxonomyOfGestures}. Dog kan det nævnes at gestikulerende gestikker er den form for gestik, som anses for være den mest naturlige og samtidig mest udfordrende form for gestik, \parencite[s. 7]{PDF:ATaxonomyOfGestures}. Tiltrods for de store designmæssige udfordringer de gestikulerende gestikker bringer med sig, så forudser \textcite[s. 28]{PDF:ATaxonomyOfGestures}, at denne form for gestik vil finde et større indpas i hvordan der interageres med elektroniske apparater. \blankline
%
\textit{Language Gestures} er de tegn, som bruges i tegnsprog. Denne type af gestikker er bygget op omkring en grammatisk struktur og anvendes primært til kommunikation fremfor at give kommandoer, \parencite[s. 8]{PDF:ATaxonomyOfGestures}. Ifølge \textcite[s. 8]{PDF:ATaxonomyOfGestures} så er det lige så krævende for et system at processere tegnsprog, som det er at processere tale. For at anvende tegnsprog, særligt i det perifere, vil det først og fremmest kræve at brugeren lærer tegnsprog og at kræve det af brugeren, vurderes til at være for omstændigt. Der afgrænses derfor fra at arbejde med tegnsprog. \blankline
%  
\textit{Deictic Gestures} er en kategori af gestikker, som forudsætter at der peges på objekter med det formål at ændre deres egenskaber og deres spatiale lokation, \parencite[s. 4]{PDF:ATaxonomyOfGestures}. Ifølge \textcite[ss. 4-5]{PDF:ATaxonomyOfGestures} anvendes deiktiske gestikker, blandt andet, til at allokere objekter på en stor skærm, identificere objekter i \textit{Virtual Reality} og desktop- samt kommunikationsbaserede applikationer. \blankline
%
\textit{Manipulative Gestures}, som denne kategori antyder, vedrører de manipulerende gestikker hvorved objekter, i en eller anden grad, manipuleres, \parencite[s. 5]{PDF:ATaxonomyOfGestures}. Det kan både foregå ved 2- og 3-dimensionelle interaktioner. Manipulerende gestikker ved 2-dimensionelle interaktioner vedrører manipulering af objekter på en 2-dimensionel skærm, som for eksempel en cursor, et vindue eller noget helt tredje. Ifølge \textcite[s. 5]{PDF:ATaxonomyOfGestures} anses det ikke som værende en manipulerende gestik hvis objektet blot trækkes, flyttes eller klikkes på, da dette ikke vil ændre objektets egenskaber. For at det kan være en manipulerende gestik skal systemet, ifølge \textcite[s. 5]{PDF:ATaxonomyOfGestures}, modtage nogle parametre såsom; brugerens anmodning om at flytte eller på anden vis ændre objektet. Ved 3-dimensionelle interaktioner kan de manipulerende gestikker enten afspejle en fysisk manipulering af et objekt, en fysisk manipulering af en computer eller ved at manipulere et fysisk objekt, som afspejles i et virtuelt objekt på en touchskærm, \parencite[s. 6]{PDF:ATaxonomyOfGestures}. Derudover kan 3-dimensionelle interaktioner også inkorporeres så de manipulerer 2-dimensionelle objekter, for eksempel ved hjælp af tryksensorer, hvorved det er muligt at tilføre ekstra dimensioner til en 2-dimensionel interaktion, \parencite[s. 5]{PDF:ATaxonomyOfGestures}. 

De manipulerende gestures benyttes typisk til navigation i den virtuelle verden, fordi det, ved brug forskellige sensorer placeret i det virtuelle rum, er muligt at interagere med de virtuelle objekter, \parencite[ss. 14-15]{PDF:ATaxonomyOfGestures}. I forhold til perifer interaktion tyder det på, at det er de manipulerende gestikker, som indtil videre er de mest brugte, i form af at testpersonerne manipulerer et fysisk objekt, \parencite[s. 164]{PDF:ComparingInputModalities}. I undersøgelsen foretaget af \textcite[ss. 164-165]{PDF:ComparingInputModalities} gør de, blandt andet, brug af to forskellige modaliteter inden for manipulerende gestikker; et fysisk, håndgribeligt knop-baseret håndtag og en touchskærm.\blankline
%
Den sidste, af de fem kategorier, er \textit{Semaphoric Gestures}, som, ifølge \textcite[s. 6]{PDF:ATaxonomyOfGestures}, er den meste udbredte form for gestik på trods af, at det anses for at være en unaturlig måde, at interagere med elektroniske apparater på, \parencite[s. 1961]{PDF:AStudyOnTheUseOfSemaphoricGestures}. Semaforiske gestikker relaterer sig til at benytte tegn til at kommunikere information, hvor disse tegn dannes med kropsbevægelser, særligt ved brug af hænderne, hvorfor semaforiske gestikker ofte kaldes frihåndsgestikker. Ifølge \textcite[s. 1961]{PDF:AStudyOnTheUseOfSemaphoricGestures} er det en unaturlig måde at interagere med en computer på og derudover gengiver disse gestikke kun en begrænset del af menneskets kommunikationsevne. Dog tyder det på, at netop semaforiske gestikke med fordel kan anvendes som en interaktions mulighed ved sekundære opgaver, \parencite[s. 1961]{PDF:AStudyOnTheUseOfSemaphoricGestures}. Det skyldes, ifølge \textcite[s. 1964]{PDF:AStudyOnTheUseOfSemaphoricGestures}, at denne form for gestik vil reducere restitutionstiden mellem den sekundære og primære opgave. Årsagen til det skyldes, blandt andet, at semaforiske gestikker ikke afhænger af den visuelle opmærksomhed, hvorfor opmærksomheden forsat kan være på den primære opgave. Ydermere reduceres restitutionstiden fordi de semaforiske gestikker afhænger af den proprioceptive sans, som formegentlig ikke stimuleres i den primære, visuelle, opgave. Derudover skal der, ifølge \textcite[s. 1964]{PDF:AStudyOnTheUseOfSemaphoricGestures}, ved semaforiske gestikker et færre antal interaktioner til, for at den sekundære opgave bliver løst, hvilket ligeledes kan være med til at reducere restitutionstiden. En anden fordel ved at anvende semaforiske gestikke er, at det tillader en større afstand mellem brugeren og det elektroniske apparat, \parencite[s. 6]{PDF:ATaxonomyOfGestures}.    

Der er to former for semaforiske gestikker; statiske og dynamiske, \parencite[s. 7]{PDF:ATaxonomyOfGestures}. De statiske kræver at gestikken fastholdes, hvor de dynamiske tillader bevægelse. At fastholde en gestik forudsætter, at brugeren holder sine fingre i ro, det vil, eksempelvist, ikke være muligt at lave en statisk gestik, som ændre lydstyrken i musikken. En statisk gestik kan eksempelvist være, at danne et OK tegn med fingrene. Begge former for semaforiske gestikker kan udføres ved at anvende hænderne, fingrene, armene, fødderne, hovedet eller andre passive elektroniske apparater, \parencite[s. 7]{PDF:ATaxonomyOfGestures}. Ifølge \textcite[s. 823]{PDF:UnderstandingNaturalness} så bør gestikkerne være dynamiske, hvis formålet er at manipulere et objekt. \blankline
%
Selvom det er muligt kun at fokusere på én af de fem kategorier af gestikker, så kan det være en fordel at udnytte flere typer af gestikker når der skal interageres med en elektroniske apparaters brugergrænseflader. Dog tyder det på at det ikke er alle fem kategorier, der er lige eftertragtet at blande med de andre, som det fremgår af \textcite[s. 8]{PDF:ATaxonomyOfGestures}, er det nemlig kun; deiktiske gestikker, semaforiske gestikker og manipulerende gestikker, som blandes. De to resterende gestikulerende gestikker og tegnsprog nævnes slet ikke i forbindelse med at bruge mere end én type gestik. 
%
\subsubsection{Udførelsen af gestik}
\label{UdfoerelseAfGestik}
%
I dette afsnit fokuseres der på to overordnet former for input-gestikker, som et elektronisk apparat kan reagere på. Dernæst fokuseres der på fordelene og ulemperne ved bevægelsesmængden i gestikker.\blankline
%
Udover de foregående fem klassificeringer af gestikker, så definerer \textcite[s. 9]{PDF:ATaxonomyOfGestures} to former for inputs et elektronisk apparat kan modtage. Det ene input, som et elektronisk apparat kan modtage, forekommer enten ved fysisk kontakt med et elektronisk apparat eller et andet objekt, \parencite[s. 10]{PDF:ATaxonomyOfGestures}. Denne form for input lægger godt op til at interaktionen foregår via manipulerende gestikker, da de netop vedrører fysisk manipuleringen af et objekt. Ifølge \textcite[s. 12]{PDF:ATaxonomyOfGestures} er den anden form for input, som et elektronisk apparat kan modtage baseret på gestikker, som apparatet så kan genkende og reagere på, ved hjælp af lys-, lyd- eller bevægelsessensorer. Ved denne type input er det derfor muligt helt at undgå, at skulle interagere med eller lokalisere endnu et elektronisk apparat eller iklæde sig en form for elektronik, som for eksempel en elektronisk handske, \parencite[s. 12]{PDF:ATaxonomyOfGestures}. Sidst nævnte inputform lægger derfor godt op til at interaktionen kan foregå via semaforiske gestikker, gestikulerende gestikker og/eller deiktiske gestikker. \blankline
%
Bevægelsesmængde indenfor gestik kommer til udtryk igennem mikro- og makrogestikker, \parencite[s. 6]{PDF:UsabilityofMicroVsMacroGestures}. Mikrogestikker er små bevægelser, der ikke nødvendigvis kræver, at hele hånden bevæger sig og da de kan udføres samtidig med andre manuelle opgaver, er de specielt lovende indenfor perifer interaktion, \parencite[s. 95]{PDF:PIMicrogesturesKap5}. Derudover er mikrogestikker hurtige at udføre, hvilket, ifølge \textcite[s. 96]{PDF:PIMicrogesturesKap5}, gør dem til en god kandidat for perifer interaktion. For eksempel så tillader mikrogestikker kommunikation, som en sekundær opgave, samtidig med at fokus holdes på en samtale, som er den primære opgave, uden at samtalepartneren finder en uhøflig, \parencite[s. 97]{PDF:PIMicrogesturesKap5}.

Makrogestikker er derimod store bevægelser, der kræver signifikant bevægelse, som eksempelvis at løfte en arm eller rejse sig fra en siddende position, \parencite[s. 6]{PDF:UsabilityofMicroVsMacroGestures}. Ifølge \textcite[s. 9]{PDF:UsabilityofMicroVsMacroGestures} kan makrogestikker udføres meget mindre præcist end mikrogestikker, da en løftet arm vil tolkes som en løftet arm, uanset om den er løftet 60$^{\circ}$ eller 90$^{\circ}$. En af ulemperne ved makrogestikker er dog, at der skal bruges et stort område til at udføre de pågældende gestikker og at en sensor, der opfanger makrogestikker ofte vil dække et helt lokale, \parencite[s. 9]{PDF:UsabilityofMicroVsMacroGestures}. Der kan derfor opstå et problem hvis en anden træder ind i interaktionsområdet og ved et uheld laver en bevægelse, som genkendes af systemet. Da det kan være svært at træde ud af interaktionsområdet, kræver det at brugeren rent fysisk flytter sig langt nok væk, så en ubevidst interaktion undgåes. Til gengæld er en af fordelene ved makrogestikker, at de specifikke gestikker er så tilpas forskellige, at sandsynligheden for at systemet forveksler dem er lille, \parencite[s. 9]{PDF:UsabilityofMicroVsMacroGestures}.  

Mikrogestikker behøver, modsat makrogestikker, ikke et stort interaktionsområde, men skal til gengæld udføres meget mere præcist, \parencite[s. 10]{PDF:UsabilityofMicroVsMacroGestures}. En af fordelene ved mikrogestikker er, at de tillader en større diversitet end makrogestikker, med andre ord; det er muligt at designe flere mikrogestikker end makrogestikker, \parencite[s. 10]{PDF:UsabilityofMicroVsMacroGestures}. En anden fordel ved mikrogestikker er, at interaktionen med systemet sjældent sker ved en fejl, da det er mindre sandsynligt at specifikke gestikker udføres ubevidst, \parencite[s. 10]{PDF:UsabilityofMicroVsMacroGestures}. Ulemperne ved at bruge mikrogestikker er dels, at desto større afstanden til systemet er, desto større er usikkerheden for, hvorvidt systemet kan registrerer gestikkerne, \parencite[s. 10]{PDF:UsabilityofMicroVsMacroGestures}. Derudover skal der ydermere kompenseres for situationer hvor brugere kommer til at skygge for hele eller dele af gestikken, så systemet ikke længere er i stand til at registrer og genkende den specifikke gestik. En anden ulempe ved mikrogestikker er brugerens evne til at gengive dem korrekt og uden at forveksle dem. Ifølge \textcite[s. 10]{PDF:UsabilityofMicroVsMacroGestures} så egner mikrogestikker sig bedst til situationer, hvor der foretrækkes et mere professionelt udtryk, for eksempel på arbejdspladsen. Derudover egner mikrogestikker sig også til underholdnings apparater.\blankline
%
I undersøgelsen foretaget af \textcite[s. 823]{PDF:UnderstandingNaturalness} fremgår det, at det ikke er hensigtmæssigt at benytte sine egne kropsdele til at simulere værktøjer i gestikker, da det vil virke unaturligt. Med værktøjer refereres der til det værktøj, som ellers ville være blevet anvendt såfremt det havde været en naturlig situation. Baseret på resultaterne, fremsat i \textcite[s. 823]{PDF:UnderstandingNaturalness}, fremgår det at 77.5\% af gangene så opstår der fejl, når testpersonerne skal anvende en kropsdel som værktøj.

Derimod anbefaler \textcite[s. 823]{PDF:UnderstandingNaturalness} at gestikkerne udføres som pantomime, hvor brugere udfører den tiltænkte aktivitet og forestiller sig, at have værktøjet i hånden. Derudover anbefaler \textcite[s. 824]{PDF:UnderstandingNaturalness}, at hvis gestikkerne skal udføres i rummet, altså væk fra det elektroniske apparat, så bør gestikkerne udføres med begge hænder. Hvor den ikke-dominante hånd skal danne en form for ramme eller udgangspunkt for den dominante hånd, som så udfører gestikken.      
%
\subsection{Komplikationer forårsaget af gestikker}
\label{KomplikationerGestikker}
%
Selvom gestik virker lovende inden for perifer interaktioner, så er det stadig en forholdvis ny måde at interagere med elektroniske apparater på, \parencite[s. 163]{PDF:ComparingInputModalities}, hvorfor der naturligvis forekommer nogle komplikationer. I følgende afsnit vil nogle af disse komplikationer blive belyst, først i forhold til teknologiske komplikationer og derefter i forhold til komplikationer vedrørende det menneskeligeaspekt. Fokus vil primært være på sidstnævnte.
%
\subsubsection{Teknologiske komplikationer}
\label{TeknologiskeKomplikationer}
%
Nogle af de mest åbenlyse teknologiske komplikationer, der kan opstå relaterer sig til problemer med at genkende specifikke gestikker, \parencite[s. 27]{PDF:ATaxonomyOfGestures}. Det gør sig både gældende for mikrogestikker, som er svære at registrere på lang afstand og for makrogestikker hvor risikoen for at en anden træder ind i det interaktiveområde og skygger for eller ubevidst udføre gestikker, som genkendes af systemet, \parencite[s. 9]{PDF:UsabilityofMicroVsMacroGestures}. Takket være \textit{LeapMotion} udbedres den teknologiske begrænsning, der kan forekomme ved mikrogestikker, så snart gestikken udføres tæt på det elektroniske apparat. \textit{LeapMotion} er en lille enhed, som er utrolig god til at registrere og genkende semaforiske gestikker på tæt hold, \parencite[s. 7]{PDF:UsabilityofMicroVsMacroGestures}. Til forskel fra \textit{LeapMotion} anvendes \textit{Microsoft Kinect} til makrogestikker, da dette system er i stand til at genkende helkropsbevægelser, \parencite[s. 4]{PDF:UsabilityofMicroVsMacroGestures}, hvor \textit{LeapMotion} primært relatere sig til mindre bevægelser, eksempelvist med fingrene. Dog er \textit{Microsoft Kinect} ikke lige så nøjagtigt som \textit{LeapMotion}, hvilket kan resulterer fejlkendelse, hvis to semaforiske gestikker minder om hinanden, \parencite[s. 3]{PDF:UsabilityofMicroVsMacroGestures}. \blankline
%
Særligt ved brug af semaforiske gestikker kan der opstå komplikationer, hvis en naturlig bevægelse fejlagtigt bliver genkendt af et system. Dette velkendte, problem kaldes \textit{Midas touch problem}, som for brugeren kan skabe stor forvirring, frustration og mistillid til det elektroniske apparat, \parencite[s. 109]{PDF:PIMicrogesturesKap5}. Ydermere kan systemet have problemer med at registrere og genkende semaforiske gestikker, hvis de udføres samtidig med andre bevægelser, \parencite[s. 27]{PDF:ATaxonomyOfGestures}. 

Problemet med de semaforiske gestikker opstår fordi systemet kan have svært ved at registrere både hvornår interaktionen starter og slutter og hvornår det er en sekvens af gestikker fremfor en enkelt gestik, \parencite[s. 27]{PDF:ATaxonomyOfGestures}. For at udbrede dette problem bør brugerne, ifølge \textcite[s. 1]{PDF:DoThatThere}, først og fremmest henvende sig direkte til det pågældende apparat. Det kan potentielt reducere sandsynligheden for et \textit{Midas touch problem} opstår, der systemet nu ved hvornår det modtager et korrekt input. I tillæg til at systemet ikke er i stand til at registrere og genkende bestemte gestikker, så er der, ifølge \textcite[s. 37]{PDF:ATaxonomyOfGestures}, meget få undersøgelser, som fokuserer på brugerens tolerance for at systemet begår fejl. Det kan være i forhold til at systemet slet ikke reagerer på brugerens input, men det kan også være i forhold til at systemet fejlfortolker et input eller registrerer et input fordi brugeren ubevist udfører en bestemt gestik. \blankline
%
I takt med at teknologi er i konstant udvikling og at virksomheder i elektronik industrien konstant skal fornye sig og finde nye revolutionerende interaktionformer, kan det med stor sikkerhed antages, at der er andre, som undersøger interaktion, perifer eller ej, med semaforiske gestikker. Hvis den antagelse holder stik, så vil der højst sandsynligt blive udviklet flere produkter, som kan interageres med via semaforiske gestikker. Når det sker og vi har flere af den type produkter, så vil der sandsynligvis opstå et problem, som tilnærmelsesvis minder om \textit{Midas touch problem}, hvor det ikke nødvendigvis er brugerens bevægelse, der bliver fejlfortolket af systemet, men systemerne der fejlfortolker hvorvidt gestikken er rettet mod dem. For at forbygge og helt undgå et potentielt stort problem foreslår \textcite[s. 2]{PDF:DoThatThere}, at et hvert system først skal addresseres med hver deres unikke gestik, hvorefter brugeren da kan interagerer med det specifikke system. \blankline
%        
En anden teknologisk komplikation, der kan opstå, foregår helt tilbage i designfasen, hvor systemet testes via en Lo-Fi prototype. Der skal nemlig tages højde for, at resultaterne fra en Lo-Fi prototype formegentligt vil afvige fra resultaterne fundet ved en feltundersøgelse. \textcite[s. 176]{PDF:ComparingInputModalities} oplevede denne afvigelse, hvor testpersonerne fortrak de semaforiske gestikker fremfor de to typer af manipulerende gestikker i testen med Lo-Fi prototypen, hvor det i feltundersøgelsen var de to typer af manipulerende gestikker, som testpersonerne fortrak. Årsagen til det skyldes, ifølge \textcite[s. 176]{PDF:ComparingInputModalities} tre ting; 1) tekniske problemer i forbindelse med semaforiske gestikker i feltundersøgelsen, 2) haptiske problemer og 3) manglende interaktivitet ved Lo-Fi prototypen. Derudover så er det sjældent at en Lo-Fi prototype indeholder den endelige og fuldtfunktionelle teknologi, hvorfor testpersonerne, i nogle situationer, nærmere evaluerer konceptet fremfor det endelige produkt. Hvis der ikke tages højde for det i udviklingsfasen, så er der risiko for at endelige produkt enten ikke lever op til brugerens ønske eller er for kompliceret at interagere med. Ydermere er det ikke sikkert, at testpersonerne oplever de samme problemer ved en Lo-Fi prototype, som de ville gøre ved det færdige produkt.   
%
\subsubsection{Komplikationer vedrørende det menneskelige aspekt}
\label{KomplikationerVedroerendeDetMenneskelige}
%
Komplikationer vedrørende det menneskelige aspekt kan opstå ved noget så åbenlyst, som manglende evne til at udføre den specifikke gestik. I den forbindelse kan der også opstå problemer ved udmattelse, hvis det er fysisk udmattende at udføre gestikken eller hvis gestikken udføres gentagende gange, \parencite[s. 28]{PDF:ATaxonomyOfGestures}. Derudover så indeholder gestikker ikke visuelle hints, hvilket, ifølge \textcite[s. 6]{PDF:NaturalUserInterfaces}, kan give problemer hvis systemet ikke reagerer på brugerens bevægelse eller hvis systemet reagerer forkert. Det skyldes at brugeren ikke har mulighed for at evaluere den information systemet har registreret og finde ud af, hvad fejlen var. De visuelle hints referer til selve udførelsen af gestikken, så når brugeren udfører en bestemt gestik er det ikke muligt for dem, at gå tilbage i tid for at få gengivet bevægelsen. En måde at undgå denne komplikation er ved at sørge for, at brugeren modtager en form for feedback, så de dels kan lære af deres fejl og dels ved, hvordan de skal gebærde sig, \parencite[s. 10]{PDF:NaturalUserInterfaces}. I forhold til hvilken type feedback og om der overhovedet skal være feedback vil blive belyst i \fullref{Feedbackformer}. \blankline
%
En anden komplikation der skal tages højde for, særligt når gestikker anvendes som en interaktions mulighed til perifer interaktion, er, at gestikkerne ikke er perifere før de er lært, inkorporeret og husket, \parencite[s. 16]{PDF:PIEmbeddingHCIOnTheRelevance}. Derudover så kræver det, ifølge \textcite[s. 16]{PDF:PIEmbeddingHCIOnTheRelevance}, også at opmærksomheden ikke fjernes fra den primære opgave når der skal interageres i den perifære opmærksomhed, hvilket kun kan ske hvis gestikkerne holdes naturlige. I det henseende så pointerer både \textcite[s. 8]{PDF:NaturalUserInterfaces}, \textcite[s. 26]{PDF:ATaxonomyOfGestures} og \textcite[s. 19]{PDF:PIEmbeddingHCIOnTheRelevance}, at der på nuværende tidspunkt både mangler simple og intuitive sets af gestikker og en form for standardisering af gestikker. Hvis det kan efterkommes, så vil det betyde, at de samme gestikker repræsenterer og aktiverer de samme funktioner på tværs af de elektroniske apparater. En af årsagerne til at der ikke allerede er en form for standard og fælles forståelse for hvilke gestikker, der skal bruges til hvad skyldes, formegentlig, både at det er relativt nyt at bruge gestik til interaktion med allestedsværende elektroniske apparater og at det er nyt at bruge gestik i forbindelse med perifer interaktion. Ifølge \textcite[s. 28]{PDF:ATaxonomyOfGestures}, er det stadig usikkert hvilke gestikker, særligt semaforiske gestikker, der passer bedst til bestemte scenarier. Selvom markedet har ændret sig i takt med, at nye teknologier finder indpas og selvom vores måde at interagere med disse teknologier ligeledes har ændret sig i forhold til hvordan det var i 2005, før smartphones florerede på markedet, så er det stadig yderst relevant at undersøge hvilke semaforiske gestikker, der egner sig bedst til specifikke scenarier, hvor interaktionen foregår i den perifere opmærksomhed.

Der er en gennemgående diskussion vedrørende semaforiske gestikker, da de betragtes som værende unaturlige, \parencite[s. 1961]{PDF:AStudyOnTheUseOfSemaphoricGestures}, men samtidig er det den form for gestik, der er mest udbredt indenfor interaktion med allestedsværende elektroniske apparater, \parencite[s. 28]{PDF:ATaxonomyOfGestures}. Ydermere understøtter semaforiske gestikker også interaktion, der ikke afhænger af den visuelle opmærksomhed, \parencite[s. 1964]{PDF:AStudyOnTheUseOfSemaphoricGestures}, hvilket passer godt med perifer interaktion. Der kan være en fordel i at de semaforiske gestikke ikke nødvendigvis er fuldstændig naturlige, da helt naturlige gestikker også kan skabe problemer, \parencite[s. 9]{PDF:NaturalUserInterfaces}. \textcite[s. 9]{PDF:NaturalUserInterfaces} belyser problemet med de naturlige gestikker i forbindelse med bowling spillet til Nintendo Wii, som forårsagede en del ødelagte fjernsyn. Spillet skal gengive et helt almindeligt spil bowling, hvor kontrollen svinges ligesom bowlingkuglen. Så når spilleren normaltvist ville have sluppet bowlingkuglen, så kom spilleren naturligt til at slippe Nintendo Wii kontrollen, som derefter røg direkte ind i fjernsynet og ødelagde skærmen. Så det kan være værd at genoverveje hvor naturlige gestikkerne ønskes, da der altså kan være fordel ved at gøre dem mindre naturlige, som ved semaforiske gestikker.

En komplikation, der vil blive gået mere i detaljer med i \fullref{Socialaccept}, fokuserer på at selvom systemer, der gør brug af semaforiske gestikker kan anses for at have en mere naturlig interaktionsform fremfor den normale interaktion med tastatur og mus, så skal der tages højde for social accept, \parencite[s. 275]{PDF:WouldYouDoThat}. 
%
\subsection{Feedbackformer}
\label{Feedbackformer}
%
Det lader til at der er en del udbredte meninger om hvorvidt der skal være feedback i alle interaktionssituationer samt hvilken type feedback der skal anvendes. Diskussionen omkring feedbackformer vil primært omhandle semaforiske og manipulerende gestikker i forbindelse med at interagere med et musikanlæg eller en musikafspiller, hvilket er oplagt fordi projektet er et samarbejde med Bang $\&$ Olufsen. \blankline
%
\textcite[s. 10]{PDF:NaturalUserInterfaces} argumenterer for, at der skal være feedback for på den måde at hjælpe brugeren til at forstå årsagen til eventuelle fejl og dertil lære den korrekte adfærd. Hvorimod \textcite[s. 16]{PDF:PIEmbeddingHCIOnTheRelevance} argumenterer for, at hvis gestikkerne er semaforiske, så er det ikke nødvendigt med, hvad \textcite[s. 16]{PDF:PIEmbeddingHCIOnTheRelevance} kategorisere som værende kunstig feedback, da brugeren automatisk får feedback fra den proprioceptive sans når der sker en aktivering af musklerne. På baggrund af det argumenterer \textcite[s. 16]{PDF:PIEmbeddingHCIOnTheRelevance} ydermere for, at det er derfor, at semaforiske gestikker egner sig til perifer interaktion. I forhold til de semaforiske gestikker så gav testpersonerne, i undersøgelsen foretaget af \textcite[ss. 172-173]{PDF:ComparingInputModalities}, udtryk for, at de manglede haptisk feedback. Årsagen til det skyldes formegentligt, at testpersonerne først og fremmest skal vænne sig til og stole på de semaforiske gestikker, hvorefter der formegentlig ikke længere vil være et behov for haptisk feedback, \parencite[s. 174]{PDF:ComparingInputModalities}. 

Derudover argumentere \textcite[s. 3]{PDF:FacilitatingPIDesignAndEvaluation} for, at det er problematisk at give brugeren feedback igennem den samme sensoriske modalitet, som hvor opgave befinder sig, da der kan opstå interferens. Derfor vil det være uhensigtmæssigt og mindre brugbart at anvende auditiv feedback når brugeren samtidig lytter til musik, \parencite[s. 3]{PDF:FacilitatingPIDesignAndEvaluation}. I tilfælde hvor brugeren perifert skal interagere med et musikanlæg eller en musikafspiller, kommenterer \textcite[s. 19]{PDF:PIEmbeddingHCIOnTheRelevance}, at brugeren i forvejen modtager feedback i form af funktionel feedback, hvor brugeren kan høre at musikken stopper, at der skrues op eller ned for lyden eller at der skiftes musiknummer. Hvorimod hvis den perifere interaktion foregår igennem manipulerende gestikker, så har ikke-visuel feedback potentiale for at være med til at minimere belastningen af de mentale ressourcer, \parencite[s. 3]{PDF:FacilitatingPIDesignAndEvaluation}. Dog giver testpersonerne, i undersøgelsen foretaget af \textcite[s. 173]{PDF:ComparingInputModalities}, udtryk for, at de ikke manglede yderligere feedback end det de fik ved den funktionelle feedback. Hvor den funktionelle feedback i dette tilfælde vedrører den proprioceptive sans i og med at testpersonerne manipulerer et fysisk objekt, et knop-baseret håndtag og touchskærmen, og de kan høre at systemet reagerer på deres input ved enten at pause eller starte musikken, skrue op eller ned for lyden eller skifte musiknummer. \blankline
%
\textcite[ss. 1263-1268]{PDF:ComparingModFeedback} undersøger om forskellige feedback typer, fra visuel feedback i det perifere til feedback direkte på skærmen, har en effekt på testpersonernes præstationsevne. Ifølge \textcite[ss. 1267-1268]{PDF:ComparingModFeedback} har feedback ingen effekt på præstationsevnen i den sekundære opgave, men testpersonerne efterspørger alligevel feedback, så de kan få information omkring deres handlinger. Ligesom \textcite[s. 174]{PDF:ComparingInputModalities} så argumenterer \textcite[s. 1268]{PDF:ComparingModFeedback} for at det ikke nødvendigvis ville være tilfældet i en feltundersøgelse da testpersonerne da ville have mere tid til at vænne sig til den perifere interaktion. 

Det lader derfor til at testpersoner, som testes i et laboratorium har nogle andre behov end hvad tilfældet ville være, hvis de blev testet i en feltundersøgelse, da de automatisk vil få længere tid til at vænne sig til den perifere interaktion. Det kan derfor være svært at afgøre hvorvidt der skal være en form for feedback eller ej, når en bruger interagere med et musikanlæg eller musikafspiller i den perifere opmærksomhed. Dette gør sig særligt gældende når interaktionsformen enten bygger på semaforiske og/eller manipulerende gestikker.  
%
\subsection{Er semaforiske gestikker social acceptable?}
\label{Socialaccept}
%            
Grunden til, at der ikke fokuseres på hverken manipulerende eller deiktiske gestikker i forhold til social accept skyldes, at ved begge tilfælde er hele interaktionen og resultatet af interaktion synlig. Det er muligt for tilskuere, at se hvad en gestik medfører, uanset om det er ved at manipulere et objekt eller ved at pege på et objekt. Ydermere antages det at manipulerende gestikker er så veletableret, at både tilskuere og brugeren selv ikke nødvendigvis tænker over, at det er en gestik. \blankline
%
Af de fire undersøgelser, som det har været muligt at finde, fremgår det, at fokus har været på social accept af semaforiske gestikker, \parencite{PDF:AChairAsUbiquitousInputDevice, PDF:WouldYouDoThat, PDF:AreYouComfortableDoingThat, PDF:AnExploratoryStudy}. Af de fire undersøgelser fokuserer \textcite{PDF:AreYouComfortableDoingThat} og \textcite{PDF:WouldYouDoThat} på hvordan semaforiske gestikker opfattes af andre samt i hvilke situationer brugeren er villig til og komfortable ved at udføre bestemte former for semaforiske gestikker og hvornår de ikke er. I undersøgelsen foretaget af \textcite{PDF:AnExploratoryStudy} fokuseres der på, hvordan sociale elektroniske apparater kan være med til at fremme kontakten mellem mennesker. Hvor der i undersøgelsen, foretaget af \textcite{PDF:AChairAsUbiquitousInputDevice}, fokuseres på en interaktiv stol. Interaktionen undersøges i to scenarier; i den centrale opmærksomhed hvor testpersonerne interagere med en computer og i den perifere opmærksomhed hvor testpersonerne interagere med en musikafspiller, \parencite{PDF:AChairAsUbiquitousInputDevice}. Ved begge scenarier giver testpersonerne udtryk for, at de er bekymret for hvorvidt deres interaktion er social acceptabel, \parencite[s. 8]{PDF:AChairAsUbiquitousInputDevice}. Bekymringerne vedrører, hvorvidt andre anser gestikkerne, som værende akavet, \parencite[s. 4]{PDF:AChairAsUbiquitousInputDevice}. Dog pointerer \textcite[s. 9]{PDF:AChairAsUbiquitousInputDevice}, at det kan hænge sammen med, at gestikken er synlig men resultatet deraf er usynligt, det er derfor kun testpersonen, der kan høre at musikken ændre sig. \textcite[s. 9]{PDF:AChairAsUbiquitousInputDevice} argumentere for, at hvis den interaktive stol implementeres i et i forvejen interaktivt miljø, så kan det potentielt reducere den akavede fornemmelse.\blankline
%
Der er overordet to måder at definere social accept; ud fra brugerens syn eller ud fra tilskuerens syn. Ud fra brugerens synspunkt handler social accept om det indtryk brugeren får ved at udføre interaktionen, var oplevelsen positiv eller negativ, \parencite[s. 276]{PDF:WouldYouDoThat}. Hvor social accept ud fra tilskuerens synspunkt afhænger af hvordan tilskueren vurderer brugerens adfærd som værende enten positiv eller negativ, \parencite[s. 276]{PDF:WouldYouDoThat}. Ifølge \textcite[s. 276]{PDF:WouldYouDoThat} påvirker følgende faktorer social accept; brugertype, tid og sted samt manipulation kontra effekt. Brugertype henviser til om brugeren hurtigt accepterer gestik, som interaktionsform eller om de stritter i mod. Tid og sted henviser til kultur og hvor længe den pågældende teknologi har været på markedet, hvilket kan være med til at øge den sociale accept. Manipulation kontra effekt henviser til de specifikke gestikker og hvad de resulterer i, hvilket vil blive uddybet yderligere.

Ifølge \textcite[s. 276]{PDF:WouldYouDoThat} kan gestikker inddeles i følgende fire kategorier; udtryksfyldte, spændingsfyldte, hemmelighedsfyldte og magiske. Disse kategorier er uafhængige af kategoriseringen i \fullref{KategoriseringerAfGestikker}, i den forstand at de ikke er koblet til én specifik form for gestik men nærmere hænger sammen med hvordan gestikken udføres.\blankline
%
I undersøgelsen foretaget af \textcite[s. 193]{PDF:AreYouComfortableDoingThat}, fokuseres der dels på den sociale accept og dels på om brugeren føler sig komfortabel ved, at udføre gestikker i området omkring et elektronisk apparat. Det fremgår, at der er en stærk forbindelse mellem gestikkens nødvendige bevægelsesmængde, varigheden hvormed gestikken udføres samt positionen, relativt til det elektroniske apparat, \parencite[s. 193]{PDF:AreYouComfortableDoingThat}. Derudover fremgår det ligeledes at brugere er både selektive og bevidste om hvor interaktionen foregår, privat eller offentligt, samt hvem der kan se denne interaktion, \parencite[s. 193]{PDF:AreYouComfortableDoingThat}. Denne problemstilling undersøges i tre forskellige henseende; 1) området og afstanden fra det elektroniske apparat, 2) Bevægelsesmængden og varigheden af gestikken og 3) Tilskuere, \parencite[ss. 195-200]{PDF:AreYouComfortableDoingThat}. 

Ud fra den første undersøgelse fremgår det at testpersonerne føler sig mest komfortable ved at udføre gestikkerne i området tæt på, lige over og helst til højre for det elektroniske apparat, \parencite[s. 197]{PDF:AreYouComfortableDoingThat}. Endvidere vurderer \textcite[s. 201]{PDF:AreYouComfortableDoingThat}, at grænsen for hvornår en gestik er i risiko for, at blive anset som værende social uacceptabel er når afstanden til det elektroniske apparat når 30cm. I forhold til hvem testpersonerne føler sig komfortable ved, når de skal udføre gestikkerne, fremgår det, at det primært er personer, som de har et tæt forhold til, såsom venner, familie og deres partnere, \parencite[s. 196]{PDF:AreYouComfortableDoingThat}. Dertil føler testpersonerne sig mest ukomfortable overfor fremmede og kollegaer, \parencite[s. 196]{PDF:AreYouComfortableDoingThat}. Testpersonerne vurdere, ydermere, at de to mest ukomfortable lokationer, at udføre gestikkerne på er i offentligtransport og på et museum, \parencite[s. 196]{PDF:AreYouComfortableDoingThat}. 

Ved den næste undersøgelse, som vedrører bevægelsesmængde og varighed, defineres en bevægelsesmængde inden for et område på 15x15cm., som værende lille, hvorimod hvis gestikken udføres inden for et område på 30x30cm., så anses bevægelsesmængden for at være stor, \parencite[s. 198]{PDF:AreYouComfortableDoingThat}. Gestikkerne udføres ved tre forskellige varigheder; tre sekunder, seks sekunder og ni sekunder. På baggrund af resultaterne fra undersøgelse 2) fremgår det, at gestikker med en varighed på mindre end seks sekunder klart er at fortrække, hvis testpersonerne skal føle sig komfortable, \parencite[s. 199]{PDF:AreYouComfortableDoingThat}. Ydermere foretrækker testpersonerne en lille bevægelsesmængde, dog pointere \textcite[s. 199]{PDF:AreYouComfortableDoingThat}, at såfremt en gestik med en stor bevægelsesmængde udføres i en favorabel lokation, tæt på det elektroniske apparat, og har en kort varighed så kan der kompenseres for den ellers ukomfortable oplevelse. 

Ved den tredje undersøgelse, som vedrører tilskuerne, fremgår det, at tilskuerne ikke fandt gestikkerne påtrængende og en del af tilskuerne tænkte ikke yderliger over de gestikker, som de lige havde overværet, \parencite[s. 200]{PDF:AreYouComfortableDoingThat}. Ifølge \textcite[s. 200]{PDF:AreYouComfortableDoingThat} så har tilskuerne en tendens til, at vurderer den sociale accept en del højere end hvad tilfældet var ved de to foregående undersøgelser. \blankline
%
\textcite{PDF:AnExploratoryStudy} undersøger, hvordan semaforiske gestikker kan understøtte interaktionen med elektroniske apparater, som fremme den sociale kontakt mellem mennesker. Baseret på disse resultater tyder det på, at gestikker med en stor bevægelsesmængde kan være social acceptable og komfortable, hvis de udføres hjemme hos ens venner, \parencite[s. 4]{PDF:AnExploratoryStudy}. Endvidere finder testpersonerne det mindre pinligt, at udføre gestikker fremfor både stemmestyring og lydkontrol, \parencite[s. 4]{PDF:AnExploratoryStudy}. Ydermere fremgår det, at testpersonerne foretrækker, at gestikkerne benyttes i en positiv sammenhæng fremfor en negativ sammenhæng, hvor de foretrækker at gestikkerne da er mindre synlige, \parencite[s. 4]{PDF:AnExploratoryStudy}. \blankline
%
Så for at svare på om semaforiske gestikker er social acceptable, så afhænger det af flere aspekter, dels hvilket miljø gestikken optræder i, forholdet mellem manipulation og effekt i forhold til hvad der er synligt kontra usynligt, gestikkens bevægelsesmængde og varighed, området og afstanden til det elektroniske apparat og hvilken type tilskuere, der overvære gestikken. For at opsummere nogle af de anbefalinger, der fremgår i de fire undersøgelser, så tyder det på, at i forhold til miljø så er brugerene mere tilbøjelige til at føle sig komfortable i private omgivelser og såfremt gestikken udføres i det offentlige, så skal den være forholdvist diskret. En af årsageren til, at testpersonerne i undersøgelsen, foretaget af \textcite[s. 4]{PDF:AChairAsUbiquitousInputDevice}, gav udtryk for bekymringer i forbindelse med interaktionen med den interaktive stol, kan formegentlig skyldes, at i det tilfælde var manipulationen synlig og effekt var usynlig for tilskuerne, svarende til hvad \textcite[s. 276]{PDF:WouldYouDoThat}, definerer til at være en spændingsfyldt gestik.  

I forbindelse med manipulation og effekt, så anbefaler \textcite[s. 278]{PDF:WouldYouDoThat}, at der netop ikke bruges spændingsfyldte gestikker hvor manipuleringen er synlig og effekten er usynlig, hvilket efterlader de udtryksfyldte, hemmelighedsfyldte og magiske gestikker til fri afbenyttelse for at opnå social accept. I relation til bevægelsesmængden så anbefaler \textcite[s. 201]{PDF:AreYouComfortableDoingThat}, at såfremt gestikken udføres i det offentlige, så skal der være en forholdvist lille bevægelsesmængde i et område svarende til 15x15cm. Ifølge \textcite[s. 278]{PDF:WouldYouDoThat}, så er både små og store bevægelsesmængder socialt acceptable, så længe effekten er synlig.

Ydermere skal varigheden være kortere end seks sekunder, da risikoen for at brugeren da føler sig ukomfortabel stiger. I forbindelse med området og afstanden vurderer \textcite[s. 201]{PDF:AreYouComfortableDoingThat}, at området til højre og foran det elektroniske apparat er at foretrække, særligt for højrehåndet brugere, ellers skal der tages højde for afstanden. Ifølge \textcite[s. 201]{PDF:AreYouComfortableDoingThat}, så er grænsen for, hvornår det er komfortabelt samt social acceptabelt  omkring 30cm mellem det elektroniske apparat og gestikken.\blankline
%
Igennem \fullref{Gestik} er der blevet belyst dels hvilke kategorier forskellige former for gestik tilhører, komplikationer forårsaget af gestikker, feedbackformer og hvorvidt semaforiske gestikker anses for at være social acceptable. Der er i \fullref{KategoriseringerAfGestikker} blevet afgrænset fra, at arbejde videre med tegnsprog og gestikulerende gestikker. I forhold til hvilke komplikationer, der er forårsaget af gestik, er det primære fokus rettet mod det menneskelige aspekt, for på den måde at undersøge hvilke gestikker, der bedst egner sig til perifer interaktion med et musikanlæg eller en musikafspiller. Dertil skal det overvejes hvor meget fokus, der skal rettes mod feedbackformer, da der ikke er en fælles konsensus om hvorvidt der skal være feedback i situationer, hvor der i forvejen fremgår funktionel feedback både i forhold til gestikkerne og i forhold til musikken. Såfremt den perifere interaktion udføres med semaforiske gestikker, så er der nogle faktorer, der skal tages højde for i forbindelse med brugernes og tilskuernes sociale accept. Før der blev gået i dybden med de forskellige problemstillinger, som gestik kan medføre, blev det nævnt, at der i projektet kun vil blive fokuseret på én sideløbende opgave, som ikke har forbindelse til den primære opgave, denne sideløbende opgave præsenteres i det følgende afsnit.








