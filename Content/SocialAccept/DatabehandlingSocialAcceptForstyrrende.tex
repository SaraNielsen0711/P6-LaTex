\section{Forstyrrende elementer ved interaktionsformen}
\label{TestresultaterSocialAcceptForstyrrende}
%
Følgende analyse og diskusion foretages på baggrund af værternes respons til spørgsmålene: \textit{Oplevede du, at det var forstyrrende at lave gestikkerne imens I planlagde ferie?} og \textit{Følte du, at du var i stand til både at planlægge ferien og styre musikken?} samt gæsternes respons til spørgsmålene: \textit{Oplevede du at det var forstyrrende at din ven(inde)/kæreste lavede gestikker?} og \textit{Følte du, at din ven(inde)/kæreste var i stand til både at planlægge ferien sammen med dig og styre musikken?}. \blankline
%
Ud af de 12 testpersoner synes ingen af dem, at gestikkerne var forstyrrende. Derimod synes seks testpersoner, herunder tre værter og tre gæster, at de forskellige hints forstyrrede, fordi de afbrød flowet i samtalen, når at værten enten blev ringet op, musikken skrattede eller musikken blev skruet op. Tre af værterne gav udtryk for, at de koncentrerede sig lidt om at lytte efter de forskellige hints, mens de planlagde ferien. Heriblandt giver vært 4 udtryk for, at vedkommende var opmærksom på om musikken begyndte at skratte, men nævner samtidig, at det i en virkelig situation ikke ville være noget problem, da der ikke ville komme hints på samme måde. Vært 6 sad nogle gange og lyttede til musikken for at finde ud af om det var kvaliteten af musiknummeret, der var dårlig, eller om der var direkte forvrængning af musikken, som vedkommende skulle reagere på. To af testpersonerne, vært 3 og gæst 3, giver udtryk for, at det efter kort tid blev mere naturligt at interaktionen foregik med gestikker. Gæst 3 giver udtryk for at vedkommende var meget fascineret af interaktionsformen i starten, hvorefter det blev mere naturligt. Vært 3 nævner derimod at vedkommende i første omgang havde i baghovedet at skulle reagere på hints, men at det med tiden blev naturligt at håndtere de forskellige hints, når de blev præsenteret. 

Når der spørges ind til om gestikkerne var forstyrrende at udføre nævner vært 5, at det ikke forstyrrer mere at interagere ved brug af gestikker, end når der interageres ved brug af sin telfon. Gæst 5 og vært 6 giver i denne sammenhæng udtryk for, at det er mindre forstyrrende at interagere ved brug af semaforiske gestikker end ved enten brug af sin telefon eller en fjernbetjening, hvilket tegner godt, hvis interaktionsformen med gentagende brug skal blive perifer. \blankline
%
Resultaterne viser ydermere, at alle testpersoner, både værter og gæster, følte at værten var i stand til både at deltage i ferieplanlægning og styre musikanlægget. Selvom gæst 1 synes vært 1 var i stand til at deltage i begge opgaver, gav testpersonen udtryk for at vært 1 nogle gange mistede fokus. Dog nævner gæst 1, at der selvfølgelig skulle interageres mere med musikanlægget under en test end ude i virkeligheden, hvilket kan have haft betydning for det mindskede fokus. Gæst 3 nævner, at begge parter blev lidt distraheret, når musikken eksempelvis blev skruet op, hvilket nødvendigvis fjernede fokus fra ferieplanlægningen. Dog giver gæst 3 udtryk for at vært 3 derefter bare skruede ned, hvorefter samtalen kunne fortsætte efter kortvarig afbrydelse. \blankline
%
Selvom det ikke nødvendigvis er muligt at påvise, at interaktionen kan foregå i den perifere interaktion viser resultaterne, at interaktionen kan foregå delvist sideløbende med og med korte afbrydelser fra hovedopgaven, hvilket kan indikere at de semaforiske gestikker ved gentagende brug kan danne grundlag for en perifer interaktion.


