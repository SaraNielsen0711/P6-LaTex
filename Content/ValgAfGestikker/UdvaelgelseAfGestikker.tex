\section{Udvælgelse af gestikker}
\label{UdvaelgelseAfGestikker}
%
I det følgende afsnit vil forskellige forslag til gestikker, der skal bruges til at interagere med et musikanlæg, blive fremstillet og forklaret. Herunder vil henholdsvis pause og start, skift musiknummer frem og tilbage og skru op og ned blive set som sammenhængende.
 
For at vælge hvilke gestikker, der skal bruges til at styre et musikanlæg undersøges det først og fremmest, hvilke gestikker der er blevet brugt i andre studier - enten til styring af musik eller eksempelvis et TV med lignende funktioner. Derudover blev det undersøgt, hvilke gestikker der i forvejen bruges i Bang $\&$ Olufsens produkter og hvorvidt disse kan overføres fra en 2D flade til 3D. Både gestikker fra tidligere studier og Bang $\&$ Olufsens egne produkter blev ydermere brugt som inspiration til forslag på andre gestikker. Studerende i de omkringliggende grupperum blev spurgt ind til deres input og der blev designet nogle gestikker af projektgruppen selv. Alle de indsamlede forslag fremgår af \autoref{tab:IndsamledeGestikkerPause}, \autoref{tab:IndsamledeGestikkerSkift} og \autoref{tab:IndsamledeGestikkerVolumen}. Er forslaget kommet fra tidligere studier eller Bang $\&$ Olufsens produkter er der blevet henvist. Er forslaget derimod kommet fra andre studerende eller projektgruppen selv er der ikke skrevet en henvisning. Ved enkelte gestikker er der skrevet flere henvisninger, da lige netop den type gestik er blevet brugt flere steder før. 


\begin{table}[H]
	\centering
	\begin{tabular}{| p{6cm} | p{6cm} | }
		\hline
		\textbf{Pause} & \textbf{Start} \\ \hline
		Stationær vertikal hånd, \parencite[s. 166]{PDF:ComparingInputModalities} & Stationær vertikal hånd, \parencite[s. 166]{PDF:ComparingInputModalities} \\ \hline
		Dynamisk vertikal hånd, der starter ved brystet og ender i udstrakt arm  & Dynamisk vertikal hånd, der starter ved brystet og ender i udstrakt arm  \\ \hline
		Kryds i luften, \parencite[s. 48]{PDF:UserDefinedGesturesTV} & Flueben i luften \\ \hline
		Peg og tryk i luften hen mod anlægget, \parencite{WEB:BeosoundMoment, WEB:Beosound2} & Peg og tryk i luften hen mod anlægget, \parencite[s. 48]{WEB:BeosoundMoment, WEB:Beosound2, PDF:UserDefinedGesturesTV} \\ \hline
		Krokodillenæb med hånden, der lukker sammen, \parencite[s. 48]{PDF:UserDefinedGesturesTV} & Krokodillenæb med hånden, der åbner op \\ \hline
		Hånden lukker sammen hen mod anlægget & Hånden åbner op hen mod anlægget \\ \hline
		Hånden lukker sammen vertikalt nedad & Hånden åbner op vertikalt opad  \\ \hline
	\end{tabular}
	\caption{Oversigt over forslag til gestikker, der pauser og starter musikken. Hvis forslaget stammer fra tidligere studier eller Bang $\&$ Olufsens produkter er der lavet en henvisning, ellers stammer forslagene fra andre studerende og projektgruppen selv.}
	\label{tab:IndsamledeGestikkerPause}
\end{table}
\noindent

Jævnfør \autoref{tab:IndsamledeGestikkerPause} har ikke alle pause og start gestikker samme henvisning. \textcite[s. 48]{PDF:UserDefinedGesturesTV} bruger eksempelvis et tryk hen imod skærmen til at åbne for noget, mens der bruges et kryds til at lukke igen. Det er valgt at pause og start gestikken enten skal være den samme eller være i tæt relation og det vælges derfor at adskille det foreslåede åben/luk-par og finde en modpart til hver af disse. Ydermere bruger \textcite[s. 48]{PDF:UserDefinedGesturesTV} hånden formet som et krokodillenæb, der lukker sammen, til at mute lyden. Der er ved interaktion med et musikanlæg højst sandsynligt ikke brug for at mute en sang, hvis den kan pauses, hvorfor krokodillenæbet foreslås som pause-gestik. Det findes passende at foreslå en åbne-bevægelse til at starte musikken igen, når der bruges en lukke-bevægelse til at pause musikken, hvilket går igen i forslagene for pause og start gestikker.

\begin{table}[H]
	\centering
	\begin{tabular}{| p{6cm} | p{6cm} |}
		\hline
		\textbf{Skift musiknummer frem} & \textbf{Skift musiknummer tilbage} \\ \hline
		Swipe bevægelse fra højre mod venstre, \parencite[s. 48]{WEB:Beosound2, WEB:BeosoundMoment, PDF:UserDefinedGesturesTV} & Swipe bevægelse fra venstre mod højre, \parencite[s. 48]{WEB:Beosound2, WEB:BeosoundMoment, PDF:UserDefinedGesturesTV} \\ \hline
		Swipe hånden mod højre, \parencite[s. 166]{PDF:ComparingInputModalities}  & Swipe hånden mod venstre, \parencite[s. 166]{PDF:ComparingInputModalities}  \\ \hline
		Peg tommelfingeren mod højre, \parencite[s. 166]{PDF:ComparingInputModalities} & Peg tommelfingeren mod venstre, \parencite[s. 166]{PDF:ComparingInputModalities} \\ \hline
		Swipe bevægelse fra højre mod venstre med tommel- og lillefinger strakt, mens de andre fingre bøjes & Swipe bevægelse fra venste mod højre med tommel- og lillefinger strakt, mens de andre fingre bøjes \\ \hline
		Swipe bevægelse fra højre mod venstre med tommel-, pege- og langefinger strakt, mens de andre fingre bøjes & Swipe bevægelse fra venstre mod højre med tommel-, pege- og langefinger strakt, mens de andre fingre bøjes \\ \hline
		Tommel-, pege- og langefinger holdes strakt, de andre fingre bøjes samtidig med hånden køres i en bue fra venstre mod højre og ender med at pege til højre & Tommel-, pege- og langefinger holdes strakt, de andre fingre bøjes samtidig med hånden køres i en bue fra højre mod venstre og ender med at pege til højre\\ \hline
		Peg mod højre & Peg mod venstre\\ \hline
		
	\end{tabular}
	\caption{Oversigt over forslag til gestikker, der skifter musiknummer. Hvis forslaget stammer fra tidligere studier eller Bang $\&$ Olufsens produkter er der lavet en henvisning, ellers stammer forslagene fra andre studerende og projektgruppen selv.}
	\label{tab:IndsamledeGestikkerSkift}
\end{table}
\noindent

Ved at undersøge tidligere studier og Bang $\&$ Olufsens produkter er det tydeligt, at en swipe bevægelse med hånden eller fingeren forbindes med at skifte videre, eksepmelvis ved at skifte sang. Dog ses der, at \textcite[s. 166]{PDF:ComparingInputModalities} swiper hånden i den retning musikken skal skifte mens \textcite[s. 48]{PDF:UserDefinedGesturesTV}, \textcite{WEB:Beosound2} og \textcite{WEB:BeosoundMoment} swiper fra højre mod venstre, som det gøres når musikken \enquote{trækkes} videre. Begge disse medtages, for at høre testpersonernes holdning til hvilken retning der skal swipes. For at prøve at undgå Midas touch problem viderebygges der på swipe bevægelserne, så disse får anderledes håndtegn. \textcite[s. 166]{PDF:ComparingInputModalities} foreslår at pege tommelfingeren den retning det ønskes at musikken skifter, hvilket inspirerer til at medtage gestikken, hvor der peges med pegefingeren i den retning musikken skal skifte. 

\begin{table}[H]
	\centering
	\begin{tabular}{| p{6cm} | p{6cm} |}
		\hline
		\textbf{Skru op for musikken} & \textbf{Skru ned for musikken} \\ \hline
		Pegefingeren køres i en cirkulær bevægelse med uret, \parencite{WEB:BeosoundMoment, WEB:BMW7} & Pegefingeren køres i en cirkulær bevægelse mod uret, \parencite{WEB:BeosoundMoment, WEB:BMW7} \\ \hline
		Hold hånden og drej håndleddet med uret som ved kontakt med en volumen drejeknap, \parencite{WEB:Beosound2} & Hold hånden og drej håndleddet mod uret som ved kontakt med en volumen drejeknap, \parencite{WEB:Beosound2} \\ \hline
		Horisontal hånd løftes vertikalt med håndfladen nedad, \parencite[s. 166]{PDF:ComparingInputModalities} & Horisontal hånd sænkes vertikalt med håndfladen nedad, \parencite[s. 166]{PDF:ComparingInputModalities} \\ \hline
		Horisontal hånd løftes vertikalt med håndfladen opad & Horisontal hånd løftes vertikalt med håndfladen nedad \\ \hline
		Horisontal ikke-dominant hånd holdes horisontalt, mens den anden hånd bevæger sig væk fra referencen/opad, \parencite[s. 48]{PDF:UserDefinedGesturesTV} & Horisontal ikke-dominant hånd bruges som reference, mens den anden hånd bevæger sig mod referencen/nedad, \parencite[s. 48]{PDF:UserDefinedGesturesTV} \\ \hline
		Vertikal ikke-dominant hånd bruges som reference, mens den anden hånd bevæger sig væk fra referencen/henad, \parencite[s. 48]{PDF:UserDefinedGesturesTV} & Vertikal ikke-dominant hånd bruges som reference, mens den anden hånd bevæger sig mod referencen/hendad, \parencite[s. 48]{PDF:UserDefinedGesturesTV} \\ \hline
		Åben hånd bevæges i en bue fra venstre mod højre, \parencite{WEB:BeoplayA9} & Åben hånd bevæges i en bue fra højre mod venstre, \parencite{WEB:BeoplayA9} \\ \hline
		Peg opad & Peg nedad \\ \hline
		\enquote{Kom så}-bevægelse med hånden & \enquote{Ro på}-bevægelse med hånden \\ \hline		
		
	\end{tabular}
	\caption{Oversigt over forslag til gestikker, der ændrer volumen på musikken. Hvis forslaget stammer fra tidligere studier eller Bang $\&$ Olufsens produkter er der lavet en henvisning, ellers stammer forslagene fra andre studerende og projektgruppen selv.}
	\label{tab:IndsamledeGestikkerVolumen}
\end{table}
\noindent

At ændre på volumen kendetegnes ofte ved en cirkulær bevægelse eller ved at trække en en slider vertikalt eller horisontalt. Disse bevægelser er derfor også medtaget som forslag på gestikker, der kan ændre volumen. \textcite[s. 48]{PDF:UserDefinedGesturesTV} foreslår i den forbindelse at bruge to hænder, hvor den ikke-dominante hånd kan bruges som reference. Udover forslag, der ofte kendetegner ændring af volumen, blev forslag som at pege opad/nedad og lave \enquote{kom så}/\enquote{ro på}-bevægelser medtaget, da disse bevægelser også kan symbolisere at øge og mindske intensiteten af noget. \blankline
%

Efter forslag til gestikker, der skal styre de mest gængse funktioner i et musikanlæg, er udvalgt, udarbejdes tre videoer, der henholdsvis viser gestikkerne for pause og start, skift musiknummer og ændring af volumen. 

