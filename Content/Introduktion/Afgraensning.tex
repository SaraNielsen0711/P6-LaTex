\section{Projekt afgrænsing}
\label{Afgraensning}
%
I følgende afsnit opsummeres og specificeres hvilken tilgang til perifer interaktion med et musikanlæg, der fremadrettet vil blive fokuseret på i samarbejdet med Bang $\&$ Olufsen. Dertil vil der dels blive afgrænset fra nogle af de føromtalte problemstillinger, som fremgår af de forgående afsnit, og dels præciseres hvilke der arbejdes videre med, til det vil der blive udarbejdet en problemformulering.\blankline
% 
Da der ikke findes én universal måde, at interagere med produkter i den perifere opmærksomhed, så er det fordelagtigt, at interaktionen med et specifikt produkt undersøges nærmere. Der vil derfor kun blive fokuseret på den perifere interaktion med et musikanlæg, som stadig befinder sig på konceptbasis, fra Bang $\&$ Olufsen, hvilket betragtes som en uafhængig sideløbende opgave til den primære opgave. På baggrund af \fullref{PeriferInteratkion}, fremgår det, at for bedst at understøtte perifer interaktion er det er en fordel, at undgå at interaktionen skal foregå i den visuelle opmærksomhed. Ved at afgrænse sig fra interaktion i den visuelle opmærksomhed vil det i højere grad være muligt, at designe en interaktionsform, som ikke længere kræver at brugeren allokere sin centrale opmærksomhed og dermed afbryder sin primære opgave midlertidigt. Derudover så tyder det på, at der til perifer interaktion kan drages stor nytte af den proprioceptive sans, som netop tillader interaktion uafhængigt af den visuelle opmærksomhed, jævnfør \fullref{Potentiale}. Der bliver ydermere afgrænset fra stemmestyring dels fordi det anses for at være for krævende og derfor ikke kan foregå i det perifere, \parencite[s. 41]{PDF:PIEmbeddingHCIMicroManageMe}, og dels fordi det betragtes, som værende mere pinligt end at udføre gestikker, \parencite[s. 4]{PDF:AnExploratoryStudy}.        

Som belyst i \fullref{KategoriseringerAfGestikker} findes der forskellige kategorier af gestikker, hvoraf nogen egner sig bedre til perifer interaktion end andre. Til interaktion med Bang $\&$ Olufsens fremtidige musikanlæg tyder det på, at semaforiske gestikker vil være gunstige. Det understøttes, blandt andet, af at det er en af de interaktionsformer, som andre har haft stor gavn af, jævnfør \fullref{RelateretUndersoegelser}. Derudover understøtter semaforiske gestikker den proprioceptive sans, de kræver ikke, at brugeren skal være i besidelse af endnu en form for fjernbetjening og det er muligt at interaktionen kan foregå i det perifere. Ydermere blev det uddybet at gestikker kan kategoriseres efter bevægelsemængde i henholdsvis mikro- og makrogestikker. Til interaktion med Bang $\&$ Olufsens fremtidige musikanlæg afgrænses der fra makrogestikker, da disse hurtigt kan blive for påtrængende og voldsomme til formålet. Da det forventes, at interaktionen med musikanlægget kan foregå ved forskellige afstande til brugeren, kan det være en ulempe at anvende helt små mikrogestikker, hvor der eksempelvis kun er en lille bevægelse i et fingerled, jævnfør \fullref{UdfoerelseAfGestik}. Dog tilstræbes der, at de semaforiske gestikker stadig vil høre under kategorien mikrogestikker.\blankline
%






I \fullref{KomplikationerGestikker} er der beskrevet hvilke tekniske og menneskelige komplikationer, der kan opstå ved brug af gestikulationer til perifer interaktion. Der afgrænses i dette projekt fra at undersøge på det tekniske aspekt og fokuseres derimod på det menneskelige aspekt og de komplikationer, der hertil kan opstå. Derudover afgrænses der fra at køre Hi-Fi test og feltundersøgelser, da det interaktive kunstværk stadig er på konceptbasis og teknologien til en feltundersøgelse ikke er til rådighed. Da det er relativt nyt at bruge gestikker til interaktion med produkter, findes der ikke nødvendigvis et endegyldigt svar på hvilke gestikker brugeren ønsker at bruge. Der afgrænses derfor til at undersøge, hvilke gestikulationer brugeren gerne vil have til at styre musik gennem et interaktivt kunstværk, herunder gestikulationer knyttet til de funktioner det giver mening at styre perifert. 

Da der er afgrænset til at bruge semaforiske gestikker, der som nævnt i \fullref{KomplikationerVedroerendeDetMenneskelige} kan opfattes unaturlige, er det interessant at undersøge, om der kan findes semaforiske gestikker, der trods deres unaturlighed kan læres og genkaldes i den perifere del af opmærksomheden. Dog skal gestikulationerne ikke lægge sig så meget op af almindelige menneske-til-menneske gestikker eller andre naturlige gestikker, at de bliver lavet ved en fejl. Det findes altså interessant at undersøge lige netop det område, hvor semaforiske gestikker kan bruges til fejlfri HCI i den perifere del af opmærksomheden. \\

Undersøges behovet for feedback ved perifer interaktion er der, som beskrevet i \fullref{Feedbackformer}, delte meninger. Da det interaktive kunstværk skal styre musikken og brugeren ved interaktion med dette derfor kan høre ændringerne i musikken alt efter kommandoen og føle bevægelserne ved brug af semaforiske gestikker, findes det i første omgang ikke nødvendigt at designe feedback til at understøtte den perifere interaktion. Da der dog findes undersøgelser, som eksemelvis \textcite[s. 21]{PDF:FacilitatingPIDesignAndEvaluation}, der hævder at multimodal feedback styrker den perifere interaktion, vælges det at spørge brugerne om deres behov for feedback i forbindelse med undersøgelse af gestikker, uden i første omgang at lave en direkte undersøgelse af feedbackformer. \\

Det er ikke muligt at afgrænse sig fuldstændig til, hvordan gestikkerne skal udføres jævnfør \fullref{Socialaccept}, da dette kommer an på brugerens ønske. Der kan dog allerede afgrænses fra at lave spændingsfyldte- og hemmelighedsfyldte gestikker, da der ud fra Bang $\&$ Olufsens ønske ikke skal laves gestikker til nogle funktioner, der ikke giver tydelig feedback i form af musik. Derudover giver det ikke mening at lave spændingsfyldte gestikker, da det allerede er set, at disse ikke er socialt acceptable, \parencite[s. 277]{PDF:WouldYouDoThat}. Tilbage står altså udtryktfyldte- og magiske gestikker, der begger egner sig godt til interaktion med et interaktivt kunstværk, der kan styre musikken. \\

I \fullref{RelateretProdukterOgUndersoegelser} ses det, at lignende problemstillinger er set før, da der både er lavet undersøgelser omkring perifer interaktion, musikstyring og gestikker både hver for sig og i sammenspil. Den situation som projektet i samarbejde med Bang $\&$ Olufsen ønsker at undersøge er ikke fuldstændig sammenlignelig med tidligere undersøgelser. De relaterede undersøgelser kan derfor bruges som inspiration til hvilke semaforiske gestikker, der skal undersøges og hvordan undersøgelserne skal udføres, uden på forhånd at have bestemt, hvordan der skal skrues op og ned for musikken med semaforiske gestikker. \\

Projektet er nu afgrænset til at undersøge, hvordan perifer interaktion kan foregå med semaforiske gestikker, når musik skal styres fra en afstand, hvilket danner grundlag for en problemformulering.
