\section{Delkonklusion}
\label{ValgAfGestikkerDelkonklusion}
%
Der en tendens til at testpersonerne helt eller delvist fravælger statiske gestikker, hvor de primære begrundelser er, at de føles akavede at gengive og de tillader ingen kontrol. De testpersonerne, som foretrækker de statiske gestikker begrunder det primært ud fra, at de er nemme og diskrete at udføre. Dog vurderes det at testpersonerne i højere grad føler sig komfortable ved at udføre dynamiske gestikker i forbindelse med interaktionen. 

Ydermere er der en tendens til at vælge gestik-par, hvor der, eksempelvis, er forskel på hvordan musikken pauses og startes. Hvor forskellen relaterer sig til bevægelsesretningen, når krokodillenæbet lukkes pauses musikken og når det åbnes startes musikken igen. Forskellen i bevægelsesretningen er gennemgående for de tre gestik-par; der swipes fra højre mod venstre skiftes der til det næste musiknummer, hvor et swipe fra venstre mod højre skifter til det forrige musiknummer og for at skrue ned for musikken roteres der mod uret og med uret for at skrue op.\blankline
%
På baggrund af foregående analyse er semaforiske gestikker til de seks mest gængse funktioner på et musikanlæg valgt. Jævnfør \fullref{TestresultaterPauseStart} er GP5 valgt til at pause og starte musikken, hvor hånden lukker som et krokodillenæb for at pause musikken og åbner op igen for at starte musikken, \autoref{fig:GestikPar5Pause}. For at skifte musiknummer er GP5 ligeledes valgt, jævnfør \fullref{TestresultaterSkiftMusiknummer}, hvor der skiftes til det næste musiknummer med en swipe-bevægelse fra højre mod venstre samtidig med at tommel-, pege- og langefinger holdes strakt, som illustreret på \autoref{fig:GestikPar5Skift}. GP2 er valgt til at justere lydstyrken, jævnfør \fullref{TestresultaterVolumen}, hvor der drejes på en fiktiv drejeknap, hvilket illustreres på \autoref{fig:GestikPar2Volumen}. Størstedelen af de testpersoner, der har deltaget i undersøgelsen har en positiv indstilling til at interaktionen med et musikanlæg kan foregå via semaforiske gestikker. Det på trods af, at ingen af testpersonerne betragter sig selv som værende de første til at anskaffe sig den nyeste teknologi, jævnfør \fullref{TestresultaterInteraktioner}. 

Ydermere har størstedelen af testpersonerne ikke et problem med, at blive registreret af deres musikanlæg, dog med de forebehold at det skal være muligt, at deaktivere registreringen og at naturlige samtale-gestikker ikke fejlagtigt registreres af musikanlægget. For at imødekomme testpersonernes bekymringer, kan det være en fordel at de semaforiske gestikker rettes direkte mod musikanlægget samt at denne interaktionsform kun fungerer, såfremt musikanlægget allerede er tændt. \blankline
%
Da de semaforiske gestikker til de seks mest gængse funktioner i et musikanlæg er udvalgt, er det relevant at teste disse i en social kontekst. I den sociale kontekst kan både den sociale accept af gestikkerne samt muligheden for, at interagere med musikanlægget sideløbende med en primær opgave undersøges. 