\subsection{Komplikationer forårsaget af gestikker}
\label{KomplikationerGestikker}
%
Selvom gestik virker lovende inden for perifer interaktioner, så er det stadig en forholdvis ny måde at interagere med elektroniske apparater på, \parencite[s. 163]{PDF:ComparingInputModalities}, hvorfor der naturligvis forekommer nogle komplikationer. I følgende afsnit vil nogle af disse komplikationer blive belyst, først i forhold til teknologiske komplikationer og derefter i forhold til komplikationer vedrørende det menneskeligeaspekt. Fokus vil primært være på sidstnævnte.
%
\subsubsection{Teknologiske komplikationer}
\label{TeknologiskeKomplikationer}
%
Nogle af de mest åbenlyse teknologiske komplikationer, der kan opstå relaterer sig til problemer med at genkende specifikke gestikker, \parencite[s. 27]{PDF:ATaxonomyOfGestures}. Det gør sig både gældende for mikrogestikker, som er svære at registrere på lang afstand og for makrogestikker, hvor risikoen for at en anden træder ind i det interaktiveområde og skygger for eller ubevidst udføre gestikker, som genkendes af systemet, \parencite[s. 9]{PDF:UsabilityofMicroVsMacroGestures}. Takket være \textit{LeapMotion} udbedres den teknologiske begrænsning, der kan forekomme ved mikrogestikker, så snart gestikken udføres tæt på det elektroniske apparat. \textit{LeapMotion} er en lille enhed, som er utrolig god til at registrere og genkende semaforiske gestikker på tæt hold, \parencite[s. 7]{PDF:UsabilityofMicroVsMacroGestures}. Til forskel fra \textit{LeapMotion} anvendes \textit{Microsoft Kinect} til makrogestikker, da dette system er i stand til at genkende helkropsbevægelser, \parencite[s. 4]{PDF:UsabilityofMicroVsMacroGestures}, hvor \textit{LeapMotion} primært relatere sig til mindre bevægelser, eksempelvist med fingrene. Dog er \textit{Microsoft Kinect} ikke lige så nøjagtigt som \textit{LeapMotion}, hvilket kan resulterer fejlkendelse, hvis to semaforiske gestikker minder om hinanden, \parencite[s. 3]{PDF:UsabilityofMicroVsMacroGestures}. \blankline
%
Særligt ved brug af semaforiske gestikker kan der opstå komplikationer, hvis en naturlig bevægelse fejlagtigt bliver genkendt af et system. Dette velkendte, problem kaldes \textit{Midas touch problem}, som for brugeren kan skabe stor forvirring, frustration og mistillid til det elektroniske apparat, \parencite[s. 109]{PDF:PIMicrogesturesKap5}. Ydermere kan systemet have problemer med at registrere og genkende semaforiske gestikker, hvis de udføres samtidig med andre bevægelser, \parencite[s. 27]{PDF:ATaxonomyOfGestures}. 

Problemet med de semaforiske gestikker opstår fordi systemet kan have svært ved at registrere både hvornår interaktionen starter og slutter og hvornår det er en sekvens af gestikker fremfor en enkelt gestik, \parencite[s. 27]{PDF:ATaxonomyOfGestures}. For at udbrede dette problem bør brugerne, ifølge \textcite[s. 1]{PDF:DoThatThere}, først og fremmest henvende sig direkte til det pågældende apparat. Det kan potentielt reducere sandsynligheden for et \textit{Midas touch problem} opstår, der systemet nu ved hvornår det modtager et korrekt input. I tillæg til at systemet ikke er i stand til at registrere og genkende bestemte gestikker, så er der, ifølge \textcite[s. 37]{PDF:ATaxonomyOfGestures}, meget få undersøgelser, som fokuserer på brugerens tolerance for at systemet begår fejl. Det kan være i forhold til at systemet slet ikke reagerer på brugerens input, men det kan også være i forhold til at systemet fejlfortolker et input eller registrerer et input fordi brugeren ubevist udfører en bestemt gestik. \blankline
%
I takt med at teknologi er i konstant udvikling og at virksomheder i elektronik industrien konstant skal fornye sig og finde nye revolutionerende interaktionformer, kan det med stor sikkerhed antages, at der er andre, som undersøger interaktion, perifer eller ej, med semaforiske gestikker. Hvis den antagelse holder stik, så vil der højst sandsynligt blive udviklet flere produkter, som kan interageres med via semaforiske gestikker. Når det sker og vi har flere af den type produkter, så vil der sandsynligvis opstå et problem, som tilnærmelsesvis minder om \textit{Midas touch problem}, hvor det ikke nødvendigvis er brugerens bevægelse, der bliver fejlfortolket af systemet, men systemerne der fejlfortolker, hvorvidt gestikken er rettet mod dem. For at forbygge og helt undgå et potentielt stort problem foreslår \textcite[s. 2]{PDF:DoThatThere}, at et hvert system først skal addresseres med hver deres unikke gestik, hvorefter brugeren da kan interagerer med det specifikke system. \blankline
%        
En anden teknologisk komplikation, der kan opstå, foregår helt tilbage i designfasen, hvor systemet testes via en Lo-Fi prototype. Der skal nemlig tages højde for, at resultaterne fra en Lo-Fi prototype formegentligt vil afvige fra resultaterne fundet ved en feltundersøgelse. \textcite[s. 176]{PDF:ComparingInputModalities} oplevede denne afvigelse, hvor testpersonerne fortrak de semaforiske gestikker fremfor de to typer af manipulerende gestikker i testen med Lo-Fi prototypen, hvor det i feltundersøgelsen var de to typer af manipulerende gestikker, som testpersonerne fortrak. Årsagen til det skyldes, ifølge \textcite[s. 176]{PDF:ComparingInputModalities} tre ting; 1) tekniske problemer i forbindelse med semaforiske gestikker i feltundersøgelsen, 2) haptiske problemer og 3) manglende interaktivitet ved Lo-Fi prototypen. 

Derudover så er det sjældent at en Lo-Fi prototype indeholder den endelige og fuldtfunktionelle teknologi, hvorfor testpersonerne, i nogle situationer, nærmere evaluerer konceptet fremfor det endelige produkt. Hvis der ikke tages højde for det i udviklingsfasen, så er der risiko for at endelige produkt enten ikke lever op til brugerens ønske eller er for kompliceret at interagere med. Ydermere er det ikke sikkert, at testpersonerne oplever de samme problemer ved en Lo-Fi prototype, som de ville gøre ved det færdige produkt.   
%
\subsubsection{Komplikationer vedrørende det menneskelige aspekt}
\label{KomplikationerVedroerendeDetMenneskelige}
%
Komplikationer vedrørende det menneskelige aspekt kan opstå ved noget så åbenlyst, som manglende evne til at udføre den specifikke gestik. I den forbindelse kan der også opstå problemer ved udmattelse, hvis det er fysisk udmattende at udføre gestikken eller hvis gestikken udføres gentagende gange, \parencite[s. 28]{PDF:ATaxonomyOfGestures}. Derudover så indeholder gestikker ikke visuelle hints, hvilket, ifølge \textcite[s. 6]{PDF:NaturalUserInterfaces}, kan give problemer, hvis systemet ikke reagerer på brugerens bevægelse eller hvis systemet reagerer forkert. Det skyldes at brugeren ikke har mulighed for at evaluere den information systemet har registreret og finde ud af, hvad fejlen var. De visuelle hints referer til selve udførelsen af gestikken, så når brugeren udfører en bestemt gestik er det ikke muligt for dem, at gå tilbage i tid for at få gengivet bevægelsen. En måde at undgå denne komplikation er ved at sørge for, at brugeren modtager en form for feedback, så de dels kan lære af deres fejl og dels ved, hvordan de skal gebærde sig, \parencite[s. 10]{PDF:NaturalUserInterfaces}. I forhold til hvilken type feedback og om der overhovedet skal være feedback vil blive belyst i \fullref{Feedbackformer}. \blankline
%
En anden komplikation der skal tages højde for, særligt når gestikker anvendes som en interaktions mulighed til perifer interaktion, er, at gestikkerne ikke er perifere før de er lært, inkorporeret og husket, \parencite[s. 16]{PDF:PIEmbeddingHCIOnTheRelevance}. Derudover så kræver det, ifølge \textcite[s. 16]{PDF:PIEmbeddingHCIOnTheRelevance}, også at opmærksomheden ikke fjernes fra den primære opgave, når der skal interageres i den perifere opmærksomhed, hvilket kun kan ske hvis gestikkerne holdes naturlige. I det henseende så pointerer både \textcite[s. 8]{PDF:NaturalUserInterfaces}, \textcite[s. 26]{PDF:ATaxonomyOfGestures} og \textcite[s. 19]{PDF:PIEmbeddingHCIOnTheRelevance}, at der på nuværende tidspunkt både mangler simple og intuitive sets af gestikker og en form for standardisering af gestikker. Hvis det kan efterkommes, så vil det betyde, at de samme gestikker repræsenterer og aktiverer de samme funktioner på tværs af de elektroniske apparater. Hvis sådan en standardisering bliver en realitet, så bliver der i den grad brug for, at systemerne kan differentiere mellem hvornår brugeren henvender sig specifikt til dem og hvornår det ikke er tilfældet, eksempelvis ved at tildele hvert system sin egen unikke aktiverings gestik, jævnfør \fullref{TeknologiskeKomplikationer}. En af årsagerne til, at der ikke allerede er en form for standard og fælles forståelse for hvilke gestikker, der skal bruges til hvad skyldes, formegentlig, både at det er relativt nyt at bruge gestik til interaktion med allestedsværende elektroniske apparater og at det er nyt at bruge gestik i forbindelse med perifer interaktion. Ifølge \textcite[s. 28]{PDF:ATaxonomyOfGestures}, er det stadig usikkert hvilke gestikker, særligt semaforiske gestikker, der passer bedst til bestemte scenarier. Selvom markedet har ændret sig i takt med, at nye teknologier finder indpas og selvom vores måde at interagere med disse teknologier ligeledes har ændret sig i forhold til hvordan det var i 2005, før smartphones florerede på markedet, så er det stadig yderst relevant at undersøge hvilke semaforiske gestikker, der egner sig bedst til specifikke scenarier, hvor interaktionen foregår i den perifere opmærksomhed.

Der er en gennemgående diskussion vedrørende semaforiske gestikker, da de betragtes som værende unaturlige, \parencite[s. 1961]{PDF:AStudyOnTheUseOfSemaphoricGestures}, men samtidig er det den form for gestik, der er mest udbredt indenfor interaktion med allestedsværende elektroniske apparater, \parencite[s. 28]{PDF:ATaxonomyOfGestures}. Ydermere understøtter semaforiske gestikker også interaktion, der ikke afhænger af den visuelle opmærksomhed, \parencite[s. 1964]{PDF:AStudyOnTheUseOfSemaphoricGestures}, hvilket passer godt med perifer interaktion. Der kan være en fordel i at de semaforiske gestikke ikke nødvendigvis er fuldstændig naturlige, da helt naturlige gestikker også kan skabe problemer, \parencite[s. 9]{PDF:NaturalUserInterfaces}. \textcite[s. 9]{PDF:NaturalUserInterfaces} belyser problemet med de naturlige gestikker i forbindelse med bowling spillet til Nintendo Wii, som forårsagede en del ødelagte fjernsyn. Spillet skal gengive et helt almindeligt spil bowling, hvor kontrollen svinges ligesom bowlingkuglen. Så når spilleren normaltvist ville have sluppet bowlingkuglen, så kom spilleren naturligt til at slippe Nintendo Wii kontrollen, som derefter røg direkte ind i fjernsynet og ødelagde skærmen. Så det kan være værd at genoverveje hvor naturlige gestikkerne ønskes, da der altså kan være fordel ved at gøre dem mindre naturlige, som ved semaforiske gestikker.

En komplikation, der vil blive gået mere i detaljer med i \fullref{Socialaccept}, fokuserer på at selvom systemer, der gør brug af semaforiske gestikker, kan anses for at have en mere naturlig interaktionsform fremfor den almindelige interaktion med tastatur og mus, så skal der tages højde for social accept, \parencite[s. 275]{PDF:WouldYouDoThat}. 
%
