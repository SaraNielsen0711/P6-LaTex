\chapter{Perspektivering}
\label{Perspektivering}
%
I følgende afsnit vil der fokuseres på perspektiverende overvejelser omkring projektet, herunder hvordan videre arbejde kan forløbe.

Da Bang $\&$ Olufsen på nuværende tidspunkt ikke har en færdig prototype, har det ikke været muligt at lave brugertests af prototypen, hvilket vil være at foretrække, hvis en perifer interaktion skal påvises. Til videre undersøgelse af perifer interaktion vil det være fordelagtigt at undersøge musikanlægget og interaktionsformen ved en feltundersøgelse, hvor testpersonerne kan lære og vænne sig til at styre musikanlægget med semaforiske gestikker over en længere periode. Ved gentagende interaktion med musikanlægget kan det, ud fra resultaterne fundet i dette projekt, højst sandsynligt påvises, at interaktionen kan foregå i den perifere del af opmærksomheden. Udover det fordelagtige ved at kunne teste brugerne over en længere periode, så er det også fordelagtigt at kunne teste en virkende prototype, frem for at udføre et \textit{Wizard of Oz}-eksperiment. Her kan netop aspekter som musikanlæggets reaktionstid og genkendelsessytem testes, for at undersøge, hvorvidt brugerne har brug for aumenteret feedback eller udfører de valgte semaforiske gestikker i anden kontekst end ved interaktion med musikanlægget.\blankline
%
På baggrund af de to tests udført i dette projekt kan det udledes, at flere af testpersonerne har problemer med at forstå, hvor meget den valgte semforiske gestik til at justere lydstyrken skruer op for lydstyrken, når de dreje hånden. Der kan derfor vælges at lave flere brugertest med denne gestik i fokus, så den bedste måde at justere lydstyrken på sit anlæg findes. Da der høsjt sandsynligt ikke er nogen, der skruer fra 0 til 100 på en gang, så kan der være en idé at lade en rotation med hånden, svarende til det maksimale hånden kan rotere, svare til eksempelvis 15$\%$'s lydjustering på et anlæg. På den måde skal grebet tages igen, hvis der skal skrues højere op, uden at der ved et enkelt tag kan skrues alt for højt op. Et andet forslag kan være at tage udgangspunkt i en drejeknap, der kan skrue fra 0 til 100 på en omgang. Her kan det at dreje eksempelvis 45$^{\circ}$ på drejeknappen svare til at dreje hånden 90$^{\circ}$, hvorfor man på samme måde vil have fat i musikken. At få musikken til at følge bevægelserne synkront på denne måde, kræver højst sandsynligt at anlægget 1) er klar over at det er nu, det skal reagere og 2) at sensorene er gode nok til at opfange små præcise bevægelser. For at meddele anlægget, at det skal reagere på den specifikke gestik, kan der som foreslået i \fullref{TestresultaterValgAfGestikkerForbedringGP2Volumen} være en form for indikation af, at brugeren tager fat i en fiktiv drejeknap.\blankline
%
Forhåbentlig kan \textit{Midas touch problem} afhjælpes ved at rette gestikkerne hen imod musikanlægget. Er dette ikke tilfældet, kan der laves yderligere brugerundersøgelser, hvor der bliver fokuseret på, hvilken slags aktivering af musikanlægget passer bedst ind, når interaktionen ellers er perifer. I projektet er der overvejet aktivering ved øjenkontakt med produktet eller ved udførelse af en aktiveringsgestik. Begge forslag strider dog imod den perifere interaktion, da hovedet enten skal vendes hver gang interaktionen skal foregå eller endnu en gestik skal huskes. En aktiveringsgestik kan være svær at huske, da denne ikke nødvendigvis føles naturlig, grundet at interaktion med gestikker ikke er særlig udbredt på nuværende tidspunkt. \blankline
%
I de foregående test er der hovedsaligt brugt unge studerende, da det dels har været et ønske fra Bang $\&$ Olufsens side at teste på denne målgruppe og dels været de testpersoner, der har været til rådighed. Ved videre arbejde kan det være interessant at undersøge, hvordan en ældre målgruppe dels vil have det med at bruge gestikker og dels vil udføre de valgte semaforiske gestikker. Derudover er det interessant at undersøge, om en ældre generation vil forstå gestikkerne anderledes og have en anden holdning til hvilke gestikker, der falder dem naturligt. 
