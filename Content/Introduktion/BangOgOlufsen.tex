\section{Sammenspil mellem perifer interaktion og B$\&$O}
\label{Sammenspil mellem perifer interaktion og BO}

Med viden om perifer interaktion er det muligt at forestille sig situationer, hvor ens musikanlæg kan styres perifert og uden at bryde for meget med de opgaver, der ellers udføres. B$\&$O har i den sammenhæng et ønske om at udvikle et interaktivt kunstværk, der kan styre musikken. Med dette interaktive kunstværk er ønsket ikke, at det skal fange opmærksomheden, hver gang brugeren går forbi. Ønsket med produktet er, at det kan bruges til at starte musikken, hvorefter det igen opfører sig som et passivt stykke kunst på væggen. Når der lægges vægt på, at dette interaktive kunstværk ikke skal forstyrre brugeren i tide og utide, er det også et ønske at kunne styre musikken ved hjælp af perifer interaktion, så illusionen om et passivt stykke kunst opretholdes. 

Produktet henvender sig til mennesker, der ønsker at have deres musik synligt, uden nødvendigvis at have flere reoler fyldt op med gamle CD'er. Derudover vil produktet henvende sig til mennesker, der ikke har et problem med at opgive noget af kontrollen ved at styre et musikanlæg. Netop ved brug af perifer interaktion fralægges den præcise kontrol, der normalt haves ved at bruge en fjernbetjening eller trykke direkte på interfacet og derfor henvender produktet sig ikke i lige så høj grad til brugere, der kan lide altid at være i kontrol. Da B$\&$O generelt bestræber sig efter at lave produkter, der virker magiske, er målgruppen ved dette produkt også herefter. Et magisk produkt lægger op til, at de gestikker der bruges til den perifere interaktion også fremstår magiske, så helhedsbilledet af magien bevares. 

Det forventes som nævnt at et interaktivt kunstværk til at styre musikken i denne sammenhæng vil blive hængt på væggen. Da det er usandsynligt at brugeren bliver stående foran kunstværket, men derimod flytter sig rundt i rummet og eventuelt sætter sig i en kontorstol eller sofa, skal der være mulighed for interaktion med produktet fra afstand. Her kommer de omtalte gestikker ind i billedet, da kunstværket gerne skulle kunne forstå, hvornår der skal skrues op og ned for musikken, uden at fokus skal brydes ved at en applikation eller fjernbetjening skal findes frem. Gestures til styring af produkter kan bestemmes på forskellige måder, hvor det er vigtigt at tage højde for de forskellige brugssituationer der findes til produktet.  

Et interaktivt kunstværk kan benyttes både  når brugeren ønsker at høre musik i eget selskab og i selskab med andre. Når brugeren sidder alene i stuen, kan de rigtige gestikker til at skrue op eller skifte sang højst sandsynligt laves uden større problemer, men hvordan ændrer det sig, når det er gæster? Ifølge \textcite[ss. 276-277]{PDF:WouldYouDoThat} findes der flere måder at udføre gestikker på, hvor både gestikker og effekter er enten skjult eller synlige. Ved undersøgelse af de forskellige gestikker findes der frem til, at magiske gestikker, hvor bevægelserne er gemt og effekten synlig, er meget socialt acceptable, mens spændingsfyldte gestikker, hvor bevægelserne er synlige og effekten gemt, ikke bliver modtaget så godt. Derudover viser undersøgelsen, at store som små gestikker er socialt acceptable, så længe effekten af gestikkerne er synlig \parencite[s. 278]{PDF:WouldYouDoThat}. Når den sociale accept kan svinge alt efter hvilken gestikker der bliver brugt, er det derfor vigtigt at et interaktivt kunstværk styres med de rigtige gestikker, så produktet også kan bruges med flere personer i rummet. Ydermere er det vigtigt, at det interaktive kunstværk ikke misforstår naturlige samtale-gestikker som værende kommandoer til eksempelvis at skrue op for musikken. 

Ved interview med Lyle Clarke hos B$\&$O findes der frem til både primære og sekundære funktioner, som B$\&$O's produkter skal indeholde. De primære funktioner kan ses i \autoref{tab:BogOsPrimaereFunktioner} og de sekundære funktioner i \autoref{tab:BogOsSekundaereFunktioner}.

%
\begin{table}[H]
	\centering
	\begin{tabular}{ | l | p{8cm} |}
		\hline
		\multicolumn{1}{|l|}{\textbf{Primære funktioner}} & \multicolumn{1}{l|}{\textbf{Formål}} \\ \hline
		Start & Start musikken \\ \hline
		Stop & Stop Musikken \\ \hline
		Standby & Musik sat på pause \\ \hline
		Forward/backward in time & Skift sang frem eller tilbage \\ \hline
		Intensity up/down & Justering af intensiteten op og ned (lyd, lys osv.) \\ \hline
		Like & Sæt en sang til farvoritter \\ \hline
		Dismiss & Fjern denne og lignende sange fra playlister \\ \hline
		Shift source & Skift musikkilde (radio, CD, AUX osv.) \\ \hline
		Shift experiences & Skift mellem hvilken højtaler der skal spilles sammen med (skift mellem zoner) \\ \hline
		Initiate interaction & Start interaktion med produktet \\ \hline
		Confirm & Bekræft interaktion med produktet \\ \hline
	\end{tabular}
	\caption{Primære funktioner i B$\&$O's produkter.}
	\label{tab:BogOsPrimaereFunktioner}
\end{table}
\noindent
%

%
\begin{table}[H]
	\centering
	\begin{tabular}{ | l | p{8cm} |}
		\hline
		\multicolumn{1}{|l|}{\textbf{Sekundære funktioner}} & \multicolumn{1}{l|}{\textbf{Formål}} \\ \hline
		Fastforward/fastbackward & Spol frem eller tilbage i musikken \\ \hline
		Stay in mood & Bliv i pågældende genre/stemning \\ \hline
		Send/expand & Flyt den afspillede musik til en anden højtaler \\ \hline
		Store & Gem sang/kunstner/playliste \\ \hline
		Add to.. & Tilføj til.. (eksempelvis playliste) \\ \hline
		Play next & Sæt i kø som det næste nummer \\ \hline
		Play later & Sæt i kø til sidst \\ \hline
	\end{tabular}
	\caption{Sekundære funktioner i B$\&$O's produkter.}
	\label{tab:BogOsSekundaereFunktioner}
\end{table}
\noindent
%

Af disse funktioner er det ikke alle, der giver mening have en tilknyttet gestik, da interaktionen ikke nødvendigvis behøver at være perifer for alle funktioner. Eksempelvis giver det god mening at skulle bruge den centrale opmærksomhed, når der skal spoles i et musikstykke eller en bestemt sang skal tilføjes en bestemt playliste. Det er hovedsaligt de sekundære funktioner, men også flere af de primære funktioner, der tænkes at kunne styres på selve produktet eller på en applikation. På den måde bliver det kun de vigtigste funktioner der kan styres med gestikker. Ud fra interviewet med Lyle Clarke ligges det fast, at de vigtigste funktioner, hvortil der ønskes en tilknyttet gestikker, er start, pause, intensity (i det her tilfælde volumen) up/down og forward/backward in time. De overskydende funktioner skal naturligvis være til stede, men egner sig højst sandsynligt bedst til direkte fysisk interaktion med kunstværket eller ved brug af en tilhørende applikation.

For at få en idé om hvilke gestikker, der egner sig til styring af et interaktivt kunstværk som dette og hvordan perifer interaktion ellers er blevet brugt til styring af produkter, undersøges relaterede produkter og studier. 

%\begin{itemize}
%  \item Brugssituationer 
%  \item Kobling
%  \item Hvem henvender produktet sig til?
%  \item Magi
%  \item Hvad er deres mål med projektet
%\end{itemize}


