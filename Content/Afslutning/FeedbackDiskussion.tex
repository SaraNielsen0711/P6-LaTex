\section{Feedback}
\label{DiskussionFeedback}
%
I \fullref{Feedbackformer} er der åbnet op for en diskussion af feedbackformer, herunder hvorvidt feedback er nødvendigt ved brug af semaforiske gestikker. Når interaktionen skal foregå på fremsatte måde, vil der ikke være en iboende feedback i forbindelse med interaktionen. Derudover vil der være funktionel feedback, da musikken pauses, når gestikken for pause udføres og lignende, hvilket efterlader aumenteret feedback til diskussion.

Når der købes nyt musikanlæg, og specielt et, der skal styres ved brug af semaforiske gestikker, kan der være en tilvænningsperiode, hvor brugeren skal lære at kende produktet og forstå, hvor og hvornår diverse gestikker kan udføres. Til denne tilvænningsperiode kan det være farvorabelt at have augmenteret feedback, der viser brugeren, hvornår en bevægelse er registreret rigtigt eller slet ikke registreret. Når brugeren har lært anlægget at kende og samtidig stoler på, at der bliver reageret på gestikkerne, så er det farvorabelt at kunne slå feedbacken fra, så der ikke bliver tiltrukket unødvendigt opmærksomhed fra musikanlægget.\blankline
%
Hvorvidt augmenteret feedback skal implementeres i Bang $\&$ Olufsens fremtidig musikanlæg kommer dels an på, hvordan designet fremtræder og dels, hvor hurtig systemet kan reagere på gestikkerne. I den forbindelse understreger \textcite[s. 8]{PDF:AChairAsUbiquitousInputDevice} vigtigheden i et godt genkendelsessytem, der er med til af undgå forvirring ved interaktion. Da både pause, start og skift musiknummer frem og tilbage er bevægelser, der skal registreres efter om de er der eller ej, giver det nødvendigvis ikke mening at give tydelig augmenteret feedback på disse. Derimod kan der være behov for at systemet giver augmenteret feedback, så brugeren ved om de har fat i lyden og kan justere lydstyrken. Ud fra testen i den sociale kontekst er der ingen af testpersonerne, der nævner noget om feedback fra systemet. Dette kan skyldes dels, at testleder 2 har reageret tilfredsstillende hurtigt på deres gestikker og dels, at den funktionelle feedback fra systemet har været alt den feedback de behøvede. 

Igen er det svært at fastslå, hvordan en virkelig situation vil være, da en virkende prototype ikke haves, og der derfor kun kan laves laboratorietests. Der kan nemlig være forskel på behovet af feedback, om der testes i et laboratorium eller ved et felteksperiment, hvor anlægget hænger hjemme på væggen, jævnfør \fullref{Feedbackformer}. Ydermere er det svært at fastslå, hvilken augmenteret feedback, der vil egne sig bedst, da dette, som nævnt, afhænger meget af udseendet på musikanlægget. Da interaktionen stadig ønskes at foregå i den perifere del af opmærksomheden er det ikke farvorabelt at have forskellige kraftige lamper til at lyse op eller danne et mønster, der fortæller brugeren at de gør det rigtigt. Derimod vil en diskret feedback i form af et let oplyst område eller en lille animation, der indikere at musikanlægget har fanget og forstået gestikken være med til at understøtte den afslappede og perifere interaktion.
%