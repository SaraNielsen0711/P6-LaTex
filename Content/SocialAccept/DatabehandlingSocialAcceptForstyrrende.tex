\section{Forstyrrende elementer ved interaktionsformen}
\label{TestresultaterSocialAcceptForstyrrende}
%
Følgende analyse og diskusion foretages på baggrund af værternes respons til spørgsmålene: \textit{Oplevede du, at det var forstyrrende at lave gestikkerne imens I planlagde ferie?} og \textit{Følte du, at du var i stand til både at planlægge ferien og styre musikken?} samt gæsternes respons til spørgsmålene: \textit{Oplevede du, at det var forstyrrende at din ven(inde)/kæreste lavede gestikker?} og \textit{Følte du, at din ven(inde)/kæreste var i stand til både at planlægge ferien sammen med dig og styre musikken?}. \blankline
%
Ingen af de 12 testpersoner oplevede, at gestikkerne var forstyrrende. Derimod giver seks testpersoner, herunder tre værter og tre gæster, udtryk for, at de forskellige hints forstyrrede, fordi de afbrød samtalen. Det er primært opringningen, der er forstyrrende for samtalen og derudover påpeger gæst 3, at det var forstyrrende, at musikken i nogle situationer blev meget høj. Tre af værterne giver udtryk for, at de koncentrerede sig om at lytte efter de forskellige hints, mens de planlagde ferien. Heriblandt giver vært 4 udtryk for, at være opmærksom på om musikken begyndte at skratte, men nævner samtidig, at det i en virkelig situation ikke vil være et problem, da der ikke kommer hints på samme måde. Ifølge vært 6, sad værten nogle gange og lyttede til musikken for at vurdere, om det var kvaliteten af musiknummeret, der var dårlig, eller om der var en decideret forvrængning af musikken, som værten skulle reagere på. To af testpersonerne, vært 3 og gæst 3, giver udtryk for, at det efter kort tid blev mere naturligt, at interaktionen foregik med gestikker. Gæst 3 giver udtryk for at være fascineret af interaktionsformen i starten, hvorefter det blev mere naturligt. Vært 3 nævner derimod, at værten i første omgang havde i baghovedet at reagere på hints, men at det med tiden blev naturligt at reagere på de forskellige hints. 

Når der spørges ind til, om gestikkerne var forstyrrende at udføre, nævner vært 5, at det ikke forstyrrer mere, at interaktionen foregår via semaforiske gestikker sammenlignet med, at interaktionen foregår på sin mobiltelefon. Gæst 5 og vært 6 giver i denne sammenhæng udtryk for, at det er mindre forstyrrende, at interaktionen foregår via semaforiske gestikker sammenliget med at benytte sin mobiltelefon eller en fjernbetjening. Dette er favorabelt, hvis formålet, på sigt, er at opnå en perifer interaktion ved gentagne brug af de udvalgte semaforiske gestikker, samtidig med at de, med alt forventning, ikke har en negativ effekt på det sociale samværd. \blankline
%
Ydermere giver samtlige testpersoner, uanset om de er værter eller gæster, udtryk for, at de følte, at værten var i stand til både at deltage i ferieplanlægning og styre musikanlægget samtidig. Selvom gæst 1 giver udtryk for, at vært 1 var i stand til at deltage i begge opgaver, gav gæsten udtryk for at værten nogle gange mistede fokus. Dog nævner gæst 1, at der selvfølgelig interageres mere med musikanlægget i en testsituation end hvad tilfældet vil være i virkeligheden, hvilket kan være årsagen til værtens, til tider, manglende fokus på ferieplanlægningen. Gæst 3 nævner, at begge parter blev distraheret, når musikken blev skruet op, hvilket resulterede i at de kortvarigt fjernede fokus fra ferieplanlægningen for at skrue ned. Dog giver gæst 3 udtryk for, at vært 3 derefter bare skruede ned, hvorefter samtalen genoptages. \blankline
%
Selvom det ikke er muligt, entydigt, at påvise, at interaktionen kan foregå i den perifere opmærksomhed forefindes der evidens for, at interaktionen kan foregå helt eller delvist sideløbende med en primær opgave, hvilket indikerer, at de udvalgte semaforiske gestikker ved gentagende brug kan danne grundlag for en perifer interaktion.

