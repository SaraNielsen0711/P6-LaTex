\newcommand{\Intorduktionpartname}{Introduktion}
\newcommand{\Introduktionparttext}{I en verden, hvor alle elektroniske produkter er smarte og befinder sig i den centrale del af opmærksomheden, findes der interessant at undersøge muligheden for at designe produkter til den perifere del af opmærksomheden. I et samarbejde med Bang $\&$ Olufsen undersøges det, hvordan gestikker kan bruges i den perifere del af opmærksomheden til at styre primære funktioner i et musikbaseret produkt. Her er det interessant at undersøge hvordan den perifere opmærksomhed virker, hvilke gestikker der egner sig til perifer interaktion og hvordan det sociale samliv blive påvirket, når brugeren af et produkt pludselig interagere helt anderledes end normalt. 
	
	
	
	%Flere og flere elekroniske apparater kommer frem på markedet og kræver vores opmærksomhed. Når vi skal interagere med flere af disse apparater på en gang bliver de kognitive ressourcer belastet, i nogle tilfælde overbelastet, hvilket kan ende ud i en dårlig oplevelse af et eller flere produkter. For at benytte de kognitive ressourcer bedre, kan det derfor være en idé at flytte interaktionen af nogle produkter til den perifere opmærksomhed. På den måde kan der interageres med produkter, uden at den fulde opmærksomhed behøver at blive rettet mod dem. Udover de kognitive ressourcer elektroniske apparater belaster, så kan det også være socialt hæmmende hele tiden at skulle rette blikket og opmærksomheden mod en smartphone, der vibrerer, eller et musikanlæg, der spiller for højt.
	
%I vores nymoderne verden er alt smart. Interaktionen med disse smart-produkter befinder sig i den centrale del af vores opmærksomhed - og det er næsten udelukkende det visuelle system produkterne retter sig imod \parencite[s. 249]{PDF:PeripheralInteraction}. Det kan derfor være interessant ikke blot at flytte interaktionen med produkter fra den centrale til den perifere opmærksomhed, men også flytte interaktionen med produkterne til den perifere interaktion, hvor det visuelle system ikke nødvendigvis skal bruges. 

%Perifer interaktion er ikke noget nyt. Selvom der skal bruges fuld opmærksomhed de første gange et objekt skal interageres med, kan interaktionen gennem gentagelser blive overført til det perifere. Når vi nu alle sammen bruger den perifere interaktion i løbet af dagen til eksempelvis at flytte en kaffekop op til munden eller tage tøj på, hvorfor så ikke også interagere med vores elektronik perifert?\\



%Flere og flere elektroniske apparater kommer ud på markedet og kræver vores opmærksomhed. Når vi skal interagere med flere af disse apparater på en gang bliver de kognitive ressourcer belastet, i nogle tilfælde overbelastet, hvilket kan ende ud i en dårlig oplevelse af et eller flere produkter. (mangler kilde!). For bedre at udnytte de kognitive ressourcer, kan det derfor være favorabelt at interaktion med elektroniske produkter i højere grad foregår i den perifere opmærksomhed. På den måde kan interaktionen med produkterne foregå uden at den fulde(centrale?) opmærksomhed behøver at blive rettet mod dem. Udover de kognitive ressourcer, som de elektroniske apparater belaster, så kan det også være socialt hæmmende hele tiden at skulle rette blikket og opmærksomheden mod en smartphone, der vibrerer, eller et musikanlæg, der spiller for højt. I vores nymoderne verden er alt smart. Interaktionen med disse smart-produkter befinder sig i den centrale del af vores opmærksomhed - og det er næsten udelukkende det visuelle system produkterne retter sig imod \parencite[s. 249]{PDF:PeripheralInteraction}. Det kan derfor være interessant ikke blot at flytte interaktionen med produkter fra den centrale til den perifere opmærksomhed, men også flytte interaktionen med produkterne til den perifere interaktion, hvor det visuelle system ikke nødvendigvis skal bruges. Perifer interaktion er ikke noget nyt. Selvom der skal bruges fuld opmærksomhed de første gange et objekt skal interageres med, kan interaktionen gennem gentagelser blive overført til det perifere. Når vi nu alle sammen bruger den perifere interaktion i løbet af dagen til eksempelvis at flytte en kaffekop op til munden eller tage tøj på, hvorfor så ikke også interagere med vores elektronik perifert?\\
	
%	..Vi får flere og flere elektroniske enheder, som kræver vores opmærksomhed, hvilket belaster de kognitive ressourcer. Det kan derfor være en idé at flytte noget af denne byrde til det perifere, så vi kan interagere med produkter uden at det overbelaster vores kognitive ressourcer. Elektronik kræver både vores opmærksomhed men det kan også hæmme os socialt, da vi hele tiden skal rette opmærksomheden mod vores elektroniske apparater. Alt er smart (smart TV, smartphone..) og det befinder sig i vores centrale system og kræver vores visuelle opmærksomhed, (henvisning til kapitel 11 i bogen). Flyt produkter (eller interaktionen med dem) fra det centrale til det perifere. Perifer interaktion er ikke noget nyt, vi gør det alle sammen i løbet af dagen, vi går, vi spiser.... så hvorfor ikke også interagere med vores elektronik i det perifere? 
%
\newpage 
}
%
\part[\Intorduktionpartname]{\Intorduktionpartname
\label{\Intorduktionpartname}
\vspace{8mm}
	\begin{center}
		\begin{minipage}[l]{14cm}
			\textnormal{\normalsize\noindent\Introduktionparttext}
		\end{minipage}
	\end{center}
}