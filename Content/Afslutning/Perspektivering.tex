\chapter{Perspektivering}
\label{Perspektivering}
%
I følgende afsnit fokuseres der på perspektiverende overvejelser omkring projektet, herunder hvordan videre arbejde kan forløbe.

Da der på nuværende tidspunkt ikke er udarbejdet en fuld funktionel prototype af Bang $\&$ Olufsens fremtidige musikanlæg, har det ikke været muligt at afvikle brugerundersøgelser, der kan påvise en perifer interaktion. Til videre undersøgelse af perifer interaktion, er det fordelagtigt at undersøge interaktionen med musikanlægget i en feltundersøgelse, hvor testpersonerne over en længere periode får mulighed for at lære gestikkerne til de respektive funktioner og igennem repetition vænner sig til både gestikkerne og interaktionsformen. Ved en fuld funktionel prototype er det, udover muligheden for at foretage undersøgelsen over en længere periode, muligt at undgå menneskelige fejl, som kan opstå ved et \textit{Wizard of Oz}-eksperiment. I det henseende er det muligt at undersøge aspekter, såsom musikanlæggets reaktionstid og kvaliteten af genkendelsessystemet for at afgøre, hvorvidt brugeren har behov for augmenteret feedback og om der opstår et \textit{Midas touch problem}.\blankline
%
For at undgå potentielle problemer ved den valgte semaforiske gestik til justering af lydstyrken, i forhold til hvor meget lydstyrken justeres, er det fordelagtigt at foretage en nærmere undersøgelser deraf. En bruger skruer højst sandsynligt aldrig op fra det laveste til det højeste lydtryksniveau, som anlægget kan spille, hvorfor det bør være muligt at koble en rotation på cirka 180$^{\circ}$ til en forudbestemt procentvis del af spektret, hvor lydstyrken kan justeres indefor. Ønsker brugeren at skrue højere op for lydstyrken, skal brugeren tage fat i den fiktive drejeknap igen. 

Et andet forslag er, at brugerens rotation er dobbelt så stor, som hvis brugeren roterede på anlæggets drejeknap. Roterer brugeren gestikken 90$^{\circ}$, svarer det til at drejeknappen på anlægget roteres 45$^{\circ}$.\blankline 
%
Forhåbentlig kan \textit{Midas touch problem} undgås ved at rette gestikkerne hen imod musikanlægget. Er dette ikke tilfældet, kan der foretages yderligere brugerundersøgelser, hvor der fokuseres på Bang $\&$ Olufsens egne funktioner; \textit{initiate interaction} eller \textit{confirm} og hvilken gestik, der kan tilknyttes funktionen. Denne gestik kan betragtes som værende en aktiveringsgestik. Alternativt kan der implementeres en form for \textit{eyetracker}, hvor produktet først reagerer på brugerens gestikker, efter produktet har registreret lysets reflektion fra brugerens pupil. Dog strider begge forslag imod perifer interaktion, da det ved en \textit{eyetracker} kræver, at brugeren retter sin visuelle opmærksomhed mod produktet og at det i \fullref{DiskussionUdvalgteGestikker} frarådes at anvende aktiveringsgestikker.\blankline   
%
I de foregående undersøgelser er der hovedsaligt testet på studerende, da det dels har været et ønske fra Bang $\&$ Olufsen, at teste på denne målgruppe og dels har det været de testpersoner, der har været til rådighed. 

Ved videre arbejde er det interessant at undersøge, hvordan en ældre målgruppe dels vurderer interaktionen via semaforiske gestikker og dels vil udføre de valgte semaforiske gestikker. Derudover er det interessant at undersøge, om en ældre generation forbinder de udvalgte semaforiske gestikker med noget andet end testpersonerne og hvordan de semaforiske gestikker påvirker den sociale accept. 
