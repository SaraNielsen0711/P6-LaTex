\section{Gestikker i sociale sammenhænge}
\label{TestresultaterSocialAcceptSocial}
%
Følgende analyse og diskusion foretages på baggrund af værternes respons til spørgsmålene: \textit{Hvordan tror du interaktion med gestikker vil fungere, hvis man er alene og i sociale sammenhænge?}, \textit{Hvordan vil du have det med at bruge gestikker, både når du er alene og i sociale sammenhænge?} og \textit{Vil det gøre en forskel, hvor du er henne?} samt gæsternes respons til spørgsmålene: \textit{Hvordan vil du føle, hvis du så din ven(inde)/kæreste lave gestikkerne i et virkeligt scenarie?}, \textit{Vil det gøre en forskel, hvem I er sammen med?} og \textit{Hvordan tror du interaktion med gestikker vil fungere i sociale sammenhænge?}. \blankline
%
Ud af de 12 testpersoner har alle et overordnet positivt indtryk af interaktion med semaforiske gestikker, både alene og i sociale sammenhænge. Vært 1 giver udtryk for, at det vil fungere godt at interagere med gestikker, men at kristisk i forhold til overvågningen, som vil gøre at det for værten ikke vil være relevant at benytte denne interaktionsform. Alle seks værter giver udtryk for at interaktionsformen vil fungere godt alene og at de vil have det fint med at lave gestikkerne, når de er alene. Når testpersonerne bliver spurgt ind til den sociale sammenhæng, giver ni testpersoner udtryk for, at det ikke vil være et problem at gestikulere i en social sammenhæng. Fire af værterne vil ikke have et problem med at lave gestikker hjemme hos en ven, såfremt de har fået lov til at styre musikanlægget. Fem af gæsterne, med undtagelse af gæst 4, giver udtryk for, at det ikke vil gøre en forskel, hvem de er sammen med. Gæst 4 nævner i den forbindelse at gæstens venner formentlig vil lege med systemet hele aftenen. 

Der er altså ingen gæster, der giver udtryk for at et bestemt selskab vil gøre det mindre passende at gestikulere til sit anlæg, men at interaktionen måske vil forekomme hyppigere med flere mennesker end alene. Det kommer også til udtryk, når testpersonerne bliver spurgt ind til interaktionsformen i en social sammenhæng generelt, hvortil syv testpersoner giver udtryk for, at der kan opstå problemer, hvis der holdes fest eller har børn på besøg. Gæst 2 nævner, at det kunne blive en form for konkurrence om hvem, der først kan skrue op og ned, når interaktionen foregår via semaforiske gestikker i en større forsamling, men at der ved en lille forsamling højst sandsynligt ikke vil opstå et lignende problem. Vært 5 nævner, at interaktionsformen netop kan være et problem til fester, hvor folk render frem og tilbage med store armbevægelser, eller med børn, der hurtigt lærer gestikkerne og laver bevægelserne hele tiden. For at løse denne problematik giver gæst 2, gæst 4, vært 5 og vært 6 udtryk for nogenlunde den samme holdning: At det vil være fint, hvis gestikkerne kan aktiveres og deaktiveres efter behov eller at det kun er bestemte personer, der kan styre musikanlægget med gestikker. Selvom der kan opstå nogle problemer, når der er flere mennesker samlet, så er der ingen indikation af at testpersonerne har et problem med den sociale accept af de udvalgte semaforiske gestikker, hvilket kun er positivt. 

To testpersoner giver ydermere udtryk for, at det er en god interaktionsform, fordi musikanlægget kan styres uden at ødelægge den pågældende samtale. Derudover nævner fem testpersoner, at det er fedt ikke at skulle rejse sig for at gå hen til anlægget og interagere eller lede efter fjernbetjeningen og én enkelt vært nævner, at det er dejligt at have hænderne frie. Gæst 3 er, som nævnt, meget fascineret og giver udtryk for, at have lyst til at vise det frem til sine venner, men at der nok er behov for tilvænning, så folk kan forstå hvad der foregår, når en vært lige pludselig gestikulerer til sit anlæg. \blankline
%
Ud fra resultaterne konkluderes det at interaktionsformen og gestikkerne er socialt acceptable på tomandshånd hvor parterne kender hinanden i forvejen. Derudover indikerer testpersonernes respons på de forskellige spørgsmål, at gestikkerne også er socialt acceptable i større forsamlinger, selvom der kan opstå nogle problemer med interaktionsformen, hvis flere vil interagere med musikanlægget på sammetid. Når den sociale accept er fastslået, er det interessant at undersøge, hvorvidt testpersonerne overhovedet har lyst til at interagere med musikanlægget på den pågældende måde. 

