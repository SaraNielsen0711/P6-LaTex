\chapter{Diskussion}
\label{Diskussion}
%
Det er uvist hvorvidt filtrering i henhold til \textit{Equal-Loudness Level Contours}, som fremlagt i \fullref{ISO226}, bør foretages på musik, da der ikke forefindes videnskabeligt belæg for at det faktisk er repræsentativt på andet end rene sinustoner. Det er derfor oplagt at teste om filtreringen også fungerer på komplekse toner, som dem der findes i musik. Ydermere er der i projektet afgrænset til kun at undersøge og kompensere for frekvenser til og med 1000Hz, hvor det måske kunne have været en fordel at udnytte hele det hørbare spektrum. På den måde ville det også være muligt at udvikle filtre, som tager højde for \textit{Loudness} ved høje frekvenser. Grunden til at der i projektet afgrænses til kun at arbejde med frekvenser til og med 1000Hz, skyldes blandt andet at det er begrænset hvor meget musik der rent faktisk indeholder frekvenser i det højfrekvente område. \\[5mm]
%
Efter computerens fremtog er det blevet muligt at lagre og behandle lydsignaler digitalt og uden behov for specialiceret analog elektronik. På baggrund af det, er det muligt at en digital løsning, i nogle sammenhænge, er et bedre alternativ til den analoge løsning udarbejdet i dette projekt. Et eksempel på hvordan dette kan lade sig gøre digitalt fremgår af \textcite{PDF:Sofus}. Artiklen understreger behovet for modificering af de lave frekvenser, når lydtryksniveauet der lyttes ved ikke længere matcher lydtryksniveauet musikstykket er masterede ved. Artiklen fremlægger et digitalt løsningsforslag til filtrering af musik i det lavfrekvente område.

Udover at det kunne være en fordel at udvikle produktet digitalt, så er det uvist om produktet overhovedet kan finde indpas i dagligdagen, særligt for brugere der ikke er akustisk interesserede. I den forbindelse kan der forekomme en kommunikationsbarriere i forhold til at formidle hvad \textit{Loudness} egentlig er, hvordan og hvorfor brugere skal forholde sig til det. Ydermere er det også uvist om der overhovedet er efterspørgsel for et sådan produkt, eller om den almindelige lytter, lever i lykkelig uvidenhed i forhold til de muligheder, der er for at forbedre de musikalske oplevelser.   

Et af problemerne, der belyses i \fullref{Loudness}, vedrører den store befolkningsgruppe der er i risiko for at udvikle høreskader, grundet uhensigtmæssigt brug af auditive enheder, hvor dette produkt muligvis kan forebygge denne stigende tendens. Det er endnu uvist hvorvidt en bruger, i besiddelse af det udviklede produkt, vil sænke lydtryksniveauet for hvor musik afspilles ved og i det henseende hvad incitamentet for at sænke lydtryksniveauet er. \\[5mm]
%
I forbindelse med udviklingen af produktet var det ikke muligt at finde et definitivt svar på hvilket lydtryksniveau musik er produceret og masteret ved. Efter at have været i kontakt med forskellige pladeselskaber i Danmark, var det bedste svar, at de lytter til musikken under produktionen, ved forskellige lydtryksniveauer og i øvrigt ikke har noget fast system. Der forefindes heller ikke nogen standard som dikterer dette. Den bedste måde, for brugeren, at vælge referencespændingen på, er derfor ved at lytte til musikken og vælge indstillingen på den baggrund. At der i projektet fokuseres på en referencespænding på 80dB skyldes dels at det antages at musik er masteret omkring dette lydtryksniveau. Det ville være hensigtsmæssigt hvis pladeselskaber opgav hvilket lydtryksniveau musikken er tiltænkt at lyde bedst, eller hvis de fulgte en fælles standard. \blankline
%
Selvom der i projektet er afgrænset til at produktet skal udvikles til at kunne understøtte hjemmeanlæg, så bør der foretages en grundigere undersøgelse af hvor produktet kan gavne mest i forhold til forskellige brugssituationer. Udover at undersøge brugssituation bør der foretages undersøgelser om produktet rent faktisk kan forbedre musikoplevelsen, idet produktet er bygget op omkring \textcite{STD:ISO226} som er udarbejdet i forhold til renetoner.  
%