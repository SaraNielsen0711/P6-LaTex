\chapter{Fravælgelse af gestik-par}
\label{app:TestresultaterFravaelgelse}
%
Hat
%
\section{Fravælgelse af gestik-par til pause og start}
\label{app:TestresultaterPauseDaarlig} 
%
I følgende afsnit analyseres hvilke af de syv semaforiske gestik-par testpersonerne fravælger samt hvorfor testpersonerne netop fravælger disse gestik-par. På baggrund af analysen bør det være muligt at udpege hvilke semaforiske gestikker, der hvertfald ikke skal knyttes til hverken pause eller start. Analysen bygger på testpersonerne respons til spørgsmålet: \textit{Hvilken gestik kan du mindst lide? og hvorfor?}, hvor testpersonernes samlede data er vedlagt i ELEKTRONISK BILAG.
%
\begin{figure}[H]
	\centering
	\includegraphics[resolution=300,width=0.5\textwidth]{Test1/DatabehandlingGrafer/DaarligstGestikPause.pdf}
	\caption{Barplot over hvilke gestik-par testpersonerne fravælger i forbindelse med pause og start. Søjlerne bygger på testpersonernes respons, hvorfor det kun er de fravalgte gestik-par, der udgør plottet.}
	\label{fig:DaarligstGestikPause}
\end{figure}
\noindent
% 
På \autoref{fig:DaarligstGestikPause} fremgår det hvilke gestik-par de 18 testpersoner fravælger i forbindelse med at skulle pause og starte musikken. Det fremgår tydeligt at gestik-par 3 er det par, som flest testpersoner fravælger i forhold til de tre andre gestik-par, der ellers er repræsenteret på \autoref{fig:DaarligstGestikPause}. Gestik-par nummer 1 fravælges af én testperson, testperson 13, fordi testpersonen forbinder det med at række en hånd i vejret. Testperson 1 og testperson 8 fravælger gestik-par 2 på baggrund af bevægelsesmængden, hvor begge testpersoner giver udtryk for, at der er for meget unødvendig bevægelse. Årsagen til at testperson 5 fravælger gestik-par 4 skyldes at testpersonen generelt ikke bryder sig om at pege. I og med at testperson 5 både kommenterer at den er lidt aggressiv og at det vil føles lidt mærkeligt at stå derhjemme og pege, så tyder det på at testpersonen oplever gestik-par 4 som værende social uacceptabel. 

De resterende 14 testpersoner fravælger alle gestik-par 3. De to største årsager til at parret fravælges er på grund af kompleksiteten i bevægelsen og at det simpelthen er for besværeligt at udføre. I forhold til bevægelsen giver flere testpersoner udtryk for at den, foruden at være kompleks, enten er for underlig, testperson 3, for mærkelig, testperson 4, for lang og akavet, testperson 7 og for indviklet, testperson 12. I tillæg kommenterer testperson 17, at det bliver en større øvelse at skulle pause og starte musikken, og så vil der gå langtid før musikken rent faktisk bliver sat på pause. Derudover giver testperson 14 og testperson 15 udtryk for at det ikke er sikkert, at de kan huske hvordan bevægelsen er. I tillæg kommenterer testperson 14 at hun simpelthen ikke ville vide hvordan hun skulle tænde eller slukke for lyden, svarende til at sætte musikken på pause og starte den igen. Testperson 9 giver udtryk for lignede bekymringer for hvordan musikken sættes på pause og startes igen.

Af de 18 testpersoner, der har deltaget i testen, er der kun to, som har inkluderet gestik-par 3 i deres rangering, det er testperson 8 og testperson 13. Begge testpersoner vælger gestik-par 3 som deres første prioritet, lige ind til at testperson 8 til sidst skal gengive de fortrukne gestikker og ind til testperson 13 skal lave en forbedring, hvorefter parret, i begge tilfælde, rangeres, som den næste bedste. Årsagerne til at testperson 8 har inkluderet gestik-par 3 blandt de tre bedste er dels at den er unik, speciel og sjov, men den gav også en fornemmelse af at det ikke bare var en knap der blev trykket på. Testperson 13 inkluderer derimod gestik-par 3 fordi det ikke er typisk ting at gøre og dermed ikke vil komme til at pause musikken ved et uheld.\blankline
%
For at afgøre hvilke af de syv gestik-par, der skal fravælges er det nødvendigt at sammenholde hvilke gestikker testpersonerne fravælger med de gestikker, som indgår i testpersonernes top tre rangering. Der opstilles derfor en tabel over de fire fravalgte gestik-par og hvordan de indgår i testpersonernes rangering.
%
\begin{table}[H]
	\centering
	\begin{tabular}{ | p{2.4cm} | p{2.4cm} | p{2.4cm} | p{2.4cm} | p{2.4cm} |}
	\hline
		 & Gestik-par 1 & Gestik-par 2 & Gestik-par 3 & Gestik-par 4 \\ \hline
		1. Plads & 8 & 1 & 0 & 0\\ \hline
		2. Plads & 3 & 3 & 2 & 1\\ \hline
		3. Plads & 2 & 0 & 0 & 2\\ \hline
	\end{tabular}
	\caption{Oversigt over hvor ofte og hvor de fire fravalgte gestikker indgår i testpersonernes top tre rangering.}
	\label{tab:FravalgteTopTrePause}
\end{table}
\noindent
%
På baggrund af \autoref{tab:FravalgteTopTrePause} tyder det på, at det eneste gestik-par, der flerstemmigt kan fravælges, er gestik-par 3, dels fordi parret kun indgår i to testpersoners top tre og dels fordi 14 testpersoner har fravalgt det. Derudover er gestik-par 4 kun inkluderet tre gange i testpersonernes samlede top tre rangering, jævnfør \autoref{tab:FravalgteTopTrePause}, og da parret ydermere er fravalgt, blandt andet på baggrund af det er socialt uacceptabelt, besluttes det at dette par ligeledes ekskluderes fra fremtidige undersøgelser. I forhold til gestik-par 2 så tyder det på, at parret kun indgår i testperson 3's rangering fordi testpersonen forbinder de andre foreslag med mute og testpersonen giver derefter udtryk for at det ikke giver meningen at lave et dynamisk stop-tegn, svarende til gestik-par 2. Der skal dog tages forbehold for, at ud fra testperson 3's respons da det på at testpersonen var forvirret og modsagde sig selv. Ifølge testperson 7 bliver gestik-par 2 inkluderet i top tre, fordi den virkede lige til og at det er det testpersonen forstår ved et stop-tegn. Årsagen til at testperson 10 inkluderer gestik-par 2 kan ikke direkte udledes af det indsamlede data, men generelt har testpersonen valgt gestikker, som var nemme at huske og med en lille risiko for at gestikken enten gengives forkert ved en forkert bevægelse eller udføres ved et uheld. Den eneste testperson, som har gestik-par 2 på en første plads i sin top tre er testperson 12 og årsagen til det er fordi testpersonen som udgangspunkt ville have valgt gestik-par 1, som den bedste og forbedret parret ved at tilføje bevægelsen, der allerede er inkluderet i gestik-par 2. Men derudover nævner testperson 12 også at være fascineret af gestik-par 5, hvorfor det vurderes at det måske også vil tilfredsstille testpersonen, hvis gestik-par 5 blev valgt. 

Argumenterne for at fravælge gestik-par 2 kan, blandt andet, sammenholdes med responsen fra både testperson 1 og testperson 8, som begge vurderer at der er unødvendigt meget bevægelse i. Derudover blev i \fullref{Socialaccept}, blandt andet, inddraget en undersøgelse hvori det blev konkluderet at testpersonerne følte sig mest komfortable når afstanden til det elektroniske apparat, som de interagerede med, var lille, hvilket ligeledes understøtter argumenterne for at fravælge gestik-par 2. Selvom fokus for denne del af undersøgelsen ikke vedrører social accept, så tyder det på, at desto tættere på kroppen gestikkerne udføres, desto større er sandsynligheden for at de indgår på testpersonernes top tre. Så baseret på de foregående argumenter og fordi gestik-par 2 kræver en bevægelse langt fra kroppen, i forhold til de andre foreslag og fordi parret kun fremgår fire gange i testpersonernes top tre, jævnfør \autoref{tab:FravalgteTopTrePause}, så vælges det at ekskludere gestik-par 2 fra fremtidige undersøgelser. 
%
\section{Fravælgelse af gestik-par til at skifte musiknummer}
\label{app:TestresultaterSkiftDaarlig}
%
I følgende afsnit analyseres hvilke af de syv semaforiske gestik-par testpersonerne fravælger samt hvorfor testpersonerne netop fravælger disse gestik-par. På baggrund af analysen bør det være muligt at udpege hvilke semaforiske gestikker, der hvertfald ikke skal knyttes til at skifte musiknummer. Analysen bygger på testpersonerne respons til spørgsmålet: \textit{Hvilken gestik kan du mindst lide? og hvorfor?}, hvor testpersonernes samlede data er vedlagt i ELEKTRONISK BILAG.
%
\begin{figure}[H]
	\centering
	\includegraphics[resolution=300,width=0.5\textwidth]{Test1/DatabehandlingGrafer/DaarligstGestikSkift.pdf}
	\caption{Barplot over hvilke gestik-par testpersonerne fravælger i forbindelse med at skifte musiknummer frem og tilbage. Søjlerne bygger på testpersonernes respons, hvorfor det kun er de fravalgte gestik-par, der udgør plottet.}
	\label{fig:DaarligstGestikSkift}
\end{figure}
\noindent
%
På \autoref{fig:DaarligstGestikSkift} fremgår det hvilke gestik-par de 18 testpersoner fravælger i forbindelse med at skifte musiknummer. Det fremgår tydeligt at gestik-par 4 er det par, som flest testpersoner fravælger i forhold til de resterende par på \autoref{fig:DaarligstGestikSkift}. Årsagen til at testperson 14 har valgt gestik-par 1, som værende den testpersonen mindst kan lide, begrundes med at testpersonen forestiller sig, at det er en bevægelse testpersonen vil komme til at gøre gentagende gange foran sit anlæg eller under en samtale. Testperson 1, testperson 5 og testperson 16 har alle valgt gestik-par 2, som værende den de mindst kan lide, fordi gestikken er modsat af, hvad de forventer er frem og tilbage. At gestik-par 3 fravælges begrunder testperson 3, testperson 11 og testperson 13 med at hvis der skulle peges i en retning så skulle det være med pegefingeren og ikke tommelfingeren, det er en akavet bevægelse og fordi gestikken er statisk. I den forbindelse kommenterer testperson 5 at vedkommende heller ikke bryder sig om hverken gestik-par 3 eller gestik-par 7, netop fordi de er statiske.

Der er forskellige årsager til at syv testpersoner fravælger gestik-par 4. Testperson 4 og testperson 6 fravælger gestikken fordi den er mærkelig, hvor testperson 4 forbinder det med at skulle tage telefonen. Testperson 8 forstår ikke hvorfor det er lige præcis er det håndtegn der skal bruges. Testperson 10 oplever ikke at gestikken bringer noget nyt men snarre er en besværlig version af gestik-par 5. I forhold til håndtegnet i gestik-par 4, så kommenterer testperson 12 at det er unaturligt at gøre noget med lillefingere, testperson 7 kommenterer dels at det er en stor armbevægelse og dels at det er et sjovt håndtegn, som vil blive glemt, hvis det ikke bliver brugt. I tillæg kommenterer testperson 18 ligeledes at det er en unaturlig gestik, som vil blive glemt dog pointerer testpersonen at det er en effektiv gestik i og med at den ikke vil blive lavet ved en fejl. 

Årsagen til at gestik-par 7 fravælges skyldes ifølge testperson 2, testperson 9 og testperson 17, dels at den virkede mærkelig, dels at den ikke blev opfattet og dels at den minder lidt om en pistol. Ligesom gestik-par 3 blandt andet blev fravalgt fordi den er statisk, så fravælger testperson 15 af samme årsag gestik-par 7 og fordi det er et enkelt tegn.\blankline
%
For at afgøre hvilke af de syv gestik-par, der skal fravælges er det nødvendigt at sammenholde hvilke gestikker testpersonerne fravælger med de gestikker, som indgår i testpersonernes top tre rangering. Der opstilles derfor en tabel over de fem fravalgte gestik-par og hvordan de indgår i testpersonernes rangering.    
%
\begin{table}[H]
	\centering
	\begin{tabular}{ | p{1.5cm} | p{2.1cm} | p{2.1cm} | p{2.1cm} | p{2.1cm} | p{2.1cm} |}
	\hline
		 & Gestik-par 1 & Gestik-par 2 & Gestik-par 3 & Gestik-par 4 & Gestik-par 7 \\ \hline
		1. Plads & 10 & 3 & 1 & 0 & 0\\ \hline
		2. Plads & 2 & 3 & 3 & 0 & 1\\ \hline
		3. Plads & 0 & 0 & 7 & 5 & 2\\ \hline
	\end{tabular}
	\caption{Oversigt over hvor ofte og hvor de fem fravalgte gestikker indgår i testpersonernes top tre rangering.}
	\label{tab:FravalgteTopTreSkift}
\end{table}
\noindent
%
På baggrund af \autoref{tab:FravalgteTopTreSkift} sammenholdt med \autoref{fig:DaarligstGestikSkift}, tyder det på at gestik-par 4 og gestik-par 7 kan ekskluderes fra fremtidige undersøgelser. Det skyldes at selvom gestik-par 4 indgår fem gange i testpersonernes top tre, så fravælges parret af syv testpersoner. Derudover tyder det på, at de fem testpersoner, som har inkluderet gestik-par 4 i deres top tre, har gjort det på baggrund af bevægelsen snarre end kombinationen af både håndtegnet og bevægelsen. Gestik-par 7 ekskluderes da parret kun indgår tre gange i top tre og da den derudover fravælges af fire testpersoner. 

De tre testpersoner, som fravælger gestik-par 2, gør det formentligt fordi de har en mental model af, at hvis der swipes fra højre mod venstre så afspilles det næste musiknummer i playlisten, modsat hvis der swipes fra venstre mod højre så afspilles det forrige musiknummer. Hvorimod de tre testpersoner, der har inkluderet gestik-par 2, som deres første valg, formentligt har den modsatte mentale model af hvilken swipe bevægelse der skal udføres for at skifte til det næste musiknummer. For at det kan afgøres hvorvidt gestik-par 2 skal ekskluderes eller ej, så er det nødvendigt at undersøge nærmere hvilke gestik-par de seks testpersoner, som har inkluderet gestik-par 2 i deres top tre, ellers har inkluderet. 
%
\begin{table}[H]
	\centering
	\begin{tabular}{ | p{3cm} | p{3cm} | p{3cm} | p{3cm} |}
	\hline
		 & 1. Plads & 2. Plads & 3. Plads \\ \hline
		Testperson 4 & Gestik-par 2 & Gestik-par 5 & Gestik-par 3 \\ \hline
		Testperson 17 & Gestik-par 2 & Gestik-par 3 & Gestik-par 4 \\ \hline
		Testperson 18 & Gestik-par 2 & Gestik-par 1 & Gestik-par 3 \\ \hline
		Testperson 8 & Gestik-par 3 & Gestik-par 2 & Gestik-par 7 \\ \hline
		Testperson 12 & Gestik-par 1 & Gestik-par 2 & Gestik-par 5\\ \hline
		Testperson 15 & Gestik-par 1 & Gestik-par 2 & Gestik-par 5 \\ \hline
	\end{tabular}
	\caption{Oversigt over de seks testpersoner, som enten har tildelt gestik-par 2 en første eller en anden plads i top tre, samt hvilke gestik-par de ellers har inkluderet.}
	\label{tab:GestikPar2ITopTre}
\end{table}
\noindent
%
Sammenholdes testperson 4's top tre rangering med testpersonens udsagn og bevægelser i videooptagelserne, så tyder det på at denne testperson har en mental model af at hvis der swipes fra højre mod venstre så afspilles det forrige musiknummer kontra et swipe fra venstre mod højre, som vil afspille det næste musiknummer. Testpersonen kommenterer ydermere at gestik-par 5 også vil fungere såfremt retning var omvendt, svarende til hvad der sker i gestik-par 2. Testperson 17, som også har rangeret gestik-par 2 som det bedste, udfører også de korrekte bevægelser i forhold til testpersonens egne kommenterer, dog opstår der en smule usikkerhed, når testpersonen afslutningsvist skal gengive sine fortrukne gestikker. Efter en diskusion frem og tilbage med testlederen konkluderer testperson 17 dog at det stadig er gestik-par 2, der er bedst. Selvom testperson 18 virker sikker i sit valg om, at det er gestik-par 2, der er det bedste gestik-par, så gengiver testpersonen rent faktisk bevægelserne fra gestik-par 1, dog med venstre hånd. Når testpersonen i tillæg forklarer hvad swipe bevægelserne gør i forhold til at skifte til det forrige eller det næste musiknummer, relaterer det sig ligeledes til gestik-par 1. Det er derfor ikke til at vide hvorfor testpersonen har rangeret gestik-par 2 højere end gestik-par 1. Det tyder derfor på at den eneste testperson, der med sikkerhed vil vælge swipe bevægelserne i gestik-par 2, er testperson 4. 

Rettes fokus mod de tre testpersoner, som har rangeret gestik-par 2 på en anden plads, så tyder det på at de to testpersoner, som har gestik-par 1 på en første plads, hovedsageligt har inkluderet gestik-par 2 på grund af bevægelsen. Dog kommenterer testperson 15, at gestik-par 2 er modsat af hvad testpersonen finder logisk, i forhold til at skifte musiknummer.  Derudover pointere testperson 12 at det var svært at adskille gestik-par 1 fra gestik-par 2. Testperson 8 har derimod rangeret gestik-par to mellem de to statiske gestik-par og når testpersonen gengiver bevægelserne for gestik-par 2 så stemmer det både overens med testpersons udsagn samt hvordan gestik-parret er designet. Det tyder derfor på at testpersonen rent faktisk har valgt gestik-par 2 fordi det stemmer overens med testpersonens mentale model.\blankline 
%
Ud af de seks testpersoner er det kun testperson 4 og testperson 8, der rent faktisk giver entydigt udtryk for at deres mentale model af gestik-par 2 stemmeroverens både med deres bevægelser og deres udsagn. Derudover er der i Bang $\&$ Olufsen's produkter desuden truffet en design beslutning om, at en swipe bevægelse fra højre mod venstre resulterer i at det er det næste musiknummer, der afspilles. På baggrund af den foregående analyse og da det ikke ønskes at gå i mod Bang $\&$ Olufsen's designvalg, vurderes det derfor at der er tilstrækkeligt belæg for at ekskludere gestik-par 2. 

På baggrund af foregående analyse forefindes der ikke et tilstrækkeligt belæg for hverken at ekskludere gestik-par 1 eller gestik-par 3. 
%
\section{Fravælgelse af gestik-par til at skrue op og ned for musikken}
\label{app:TestresultaterVolumenDaarlig}
%
I følgende afsnit analyseres hvilke af de ni semaforiske gestik-par testpersonerne fravælger samt hvorfor testpersonerne netop fravælger disse gestik-par. På baggrund af analysen bør det være muligt at udpege hvilke semaforiske gestikker, der hvertfald ikke skal knyttes til at skrue op og ned for musikken. Analysen bygger på testpersonerne respons til spørgsmålet: \textit{Hvilken gestik kan du mindst lide? og hvorfor?}, hvor testpersonernes samlede data er vedlagt i ELEKTRONISK BILAG.
%
\begin{figure}[H]
	\centering
	\includegraphics[resolution=300,width=0.5\textwidth]{Test1/DatabehandlingGrafer/DaarligstGestikVolumen.pdf}
	\caption{Barplot over hvilke gestik-par testpersonerne fravælger i forbindelse med at skrue op og ned for musikken. Søjlerne bygger på testpersonernes respons, hvorfor det kun er de fravalgte gestik-par, der udgør plottet.}
	\label{fig:DaarligstGestikVolumen}
\end{figure}
\noindent
%
På \autoref{fig:DaarligstGestikVolumen} fremgår det hvilke gestik-par de 18 testpersoner fravælger i forbindelse med at skrue op og ned for musikken. Det fremgår tydeligt at testpersonerne hyppigst fravælger gestik-par 7, parret er i alt fravalgt otte gange, og indgår ikke på en eneste top tre rangering (henvisning til en tabel jeg laver om lidt). Årsagen til at gestik-par 7 fravælges varierer mellem de otte testpersonerne, som har valgt parret. Testperson 2 har svært ved at vurdere hvilken vej der er op og hvilken vej der er ned. Ifølge testperson 3 er gestik-par 7 underligt og derudover så giver den ikke mening. At gestik-parret ikke giver mening pointere testperson 15 også og tilføjer, at det er en mellem ting mellem gestik-par 5 og gestik-par 6. Testperson 4 fravælger gestikken fordi den er mærkelig, hvilket også er en af årsagerne til at testperson 11 fravælger parret. Derudover kommenterer testperson 11 at det kræver koncentration at lave en bue, hvilket er en del af bevægelsen. Da formålet med at anvende semaforiske gestikker til at interagere med Bang $\&$ Olufsen's fremtidige musikanlæg, blandt andet er for at interaktionen på sigt kan foregå i den perifere opmærksomhed, så er det ikke hensigtsmæssigt at gestikkerne kræver mere koncentration end andre løsninger. Derudover påpeger testperson 10 at gestik-par 7 er væsentligt mindre intuitiv end de andre forslag og dertil er det svært at vurdere den nødvendige bevægelsesmængde for at skrue op og ned. Ifølge testperson 6 så er gestik-par 7 det par, som skiller sig mest ud i forhold til de andre forslag, hvilket er årsagen til at parret fravælges. Testperson 14's respons afvigere fra de andre testpersoners, i det at testperson 14 fravælger gestik-par 7 fordi testpersonen anser det som værende en meget naturlig bevægelse, som testpersonen giver udtryk for at ville komme til at lave ubevidst.

Selvom gestik-par 7 blev inkluderet som et forsøg på at overfører gestikken, der anvendes til at skrue op og ned på en A9, \parencite{WEB:BeoplayA9}, til en semaforisk gestik, så er gestik-par 7 det par, som oftes fravælges og som ikke indgår i nogen af testpersonernes top tre, hvorfor parret ekskluderes.\blankline
%
To ud af de fire testpersoner, som fravælger gestik-par 8, giver udtryk for at det er besværligt at pege nedad. Den ene af de to, testperson 18, giver stærkt udtryk for at det er både besværligt og ubehagligt at pege nedad samt at have sin arm i den position. Hvor den anden af de to testpersoner, testperson 9, giver udtryk for at det besværligt at brug og det er underligt at pege nedad. Derudover pointere testperson 9 at det er svært at kontrollere hvor meget der enten skal skrues og eller ned, hvilket også er årsagen til at testperson 5 fravælger gestik-par 8; der er ingen mulighed for at kontrollere hvor meget der skrues op. Den sidste af de fire testpersoner, testperson 13, fravælger gestik-par 8 fordi der mangler bevægelse og fordi det er noget testpersonen godt kunne forestille sig komme til at gøre ved et uheld.  

Der er to forskellige årsager til hvorfor gestik-par 2 fravælges af testperson 12 og testperson 16. Testperson 12 fravælger gestik-par 2 fordi testpersonen ikke bryder sig om cirkelbevægelsen, selvom testpersonen pointere at det burde virke naturligt og det egentlig er sådan der normaltvist skrues op på et anlæg. Det skal dog pointeres at testperson 12 ikke endegyldigt fastslår at det er gestik-par 2, da testpersonen egentlig heller ikke bryder sig om gestik-par 1. Grunden til at det fremgår, som om at testperson 12 har valgt gestik-par 2 er på baggrund af de bevægelser, der opstår når testpersonen skal forklare hvorfor der vælges som der gør og i det tilfælde stemte bevægelsen overens med bevægelsen i gestik-par 2. Den årsager til at gestik-par 2 fravælges er fordi bevægelsen, ifølge testperson 16, er modsat af hvad testpersonen ville forvente. Det skal dog pointeres at der ved gestik-par 2 skrues op for musikken ved at dreje hånden med uret og mod uret for at skrue ned, hvilket er det testpersonen egentlig forventer. Det tyder derfor på at testperson 16 har misforstået videooptagelsen af gestik-par 2. Testlederen spørger derfor ind til hvordan gestik-par 2 kunne gøres bedre, hvor testperson 16 først og fremmest foreslår at bevægelsen foregår i den rigtige retning og derudover foreslår testpersonen at gestikken skulle være mere ligesom gestik-par 1, hvor testpersonen referer til armens position.  

Gestik-par 6 fravælges af testperson 9 fordi det vil være svært at gengive bevægelsen med et barn på armen, hvilket også gør sig gældende for gestik-par 5. Testperson 17 fravælger gestik-par 6 fordi det er ulogisk at lave den bevægelse i forbindelse med musik og derudover virker det mærkelig at begge hænder skal være involveret. 

Ifølge testperson 7 så fravælges gestik-par 9 fordi den ikke tillader kontrol over hvor meget der skrues op og ned, hvorimod testperson 8 fravælger gestik-par 9 fordi testpersonen vil have det akavet med at lave den bevægelse.\blankline
%
For at afgøre hvilke af de ni gestik-par, foruden gestik-par 7, som allerede er ekskluderet, der skal fravælges er det nødvendigt at sammenholde hvilke gestikker testpersonerne fravælger med de gestikker, som indgår i testpersonernes top tre rangering. Der opstilles derfor en tabel over de fem fravalgte gestik-par og hvordan de indgår i testpersonernes rangering.    
%
\begin{table}[H]
	\centering
	\begin{tabular}{ | p{1.5cm} | p{2.1cm} | p{2.1cm} | p{2.1cm} | p{2.1cm} | p{2.1cm} |}
	\hline
		 & Gestik-par 2 & Gestik-par 6 & Gestik-par 7 & Gestik-par 8 & Gestik-par 9 \\ \hline
		1. Plads & 4 & 1 & 0 & 0 & 2\\ \hline
		2. Plads & 2 & 2 & 0 & 0 & 1\\ \hline
		3. Plads & 3 & 1 & 0 & 2 & 3\\ \hline
	\end{tabular}
	\caption{Oversigt over hvor ofte og hvor de fem fravalgte gestikker indgår i testpersonernes top tre rangering.}
	\label{tab:FravalgteTopTreVolumen}
\end{table}
\noindent
%
På baggrund af \autoref{tab:FravalgteTopTreVolumen} hvor gestik-par 9 fremgår seks gange i testpersonernes top tre rangering sammenholdt med \autoref{fig:DaarligstGestikVolumen} samt testpersonernes begrundelser for hvorfor de fravælger gestik-par 9, vurderes det at der er ikke er belæg for at ekskludere gestik-par 9. Det vurderes ydermere at der ikke forefindes tilstrækkeligt belæg for at ekskludere gestik-par 2, for selvom det ikke kan antages med sikkerhed at testperson 16 ville have udpeget et andet gestik-par, i tilfælde af at testpersonen ikke havde misforstået bevægelsesretningen i gestik-par 2, så tyder det på, at det godt kunne være tilfældet. I så fald, så er det kun testperson 12, som har givet udtryk for at en cirkulærbevægelse ikke foretrækkes. Da det heller ikke entydigt kan konkluderes hvorvidt testperson 12 mindst kan lide gestik-par 2 i forhold til gestik-par 1 og da gestik-par 2 sammenlagt indgår ni gange i testpersonernes top tre, jævnfør \autoref{tab:FravalgteTopTreVolumen}.\blankline 
%
Baseret på \autoref{tab:FravalgteTopTreVolumen} hvor gestik-par 8 kun indgår to gange i testpersonernes samlede top tre rangering sammenholdt med \autoref{fig:DaarligstGestikVolumen} samt testpersonernes begrundelser for hvorfor de fravælger gestik-par 8, vurderes det at der er belæg for at ekskludere gestik-par 8.  

For at have belæg for at ekskludere gestik-par 6 er det nødvendigt at inkludere hvad testpersonerne, som har rangeret gestik-par 6 i deres top tre, har kommenteret. Ifølge testperson 1 så indgår gestik-par 6 på en anden plads dels fordi den følger et princip om noget, der er større eller mindre og dels fordi bevægelsen er naturlig. Det tyder på at testperson 14 rangere gestikker alt efter hvad der føles akavede og unaturligt i frygt for at komme til at lave gestikkerne ved et uheld, hvilket er årsagen til at testperson 14 har tildelt gestik-par 6 en anden plads. Testperson 2 forklarer at årsagen til at gestik-par 6 rangeres på en tredje plads, er fordi den minder om gestik-par 4 og gestik-par 5 bare sidelæns, men at det ikke giver lige så meget mening, som de to andre. Baseret på testperson 5's udsagn tyder det på at årsagen til at gestik-par 6 tildeles en første plads er fordi testpersonen ønsker at have fuld kontrol over hvor meget der skrues op og ned, hvilket testpersonen oplever ved at bruge begge hænder. Grunden til at testpersonen vælger gestik-par 6 fremfor gestik-par 5 skyldes, at testpersonen derved føler sig mindre i rummet. Sammenholdes de fire testpersoners udsagn med hvordan de rent faktisk gengiver gestik-par 6, så tyder det på, at ingen af testpersonerne formår, at gengive bevægelsen korrekt. Det fremgår af optagelserne at ingen af de fire testpersoner formår at fastholde deres ikke-dominante hånd, som reference og derfra kontrollere hvor meget der skal skrues op eller ned med den dominante hånd. Til gengæld tyder det på	 at de fire testpersoner fortolker og gengiver gestik-par 6 ens; begge hænder bevæger sig horisontalt mod eller væk fra hinanden med håndfladerne vendt ind mod hinanden. 

Så med udgangspunkt i testpersonerne begrundelser for hvorfor de enten fravælger gestik-par 6 eller inkluderer gestik-par 6 i deres top tre, samt hvad de rent faktisk gør, når de gengiver gestik-parret, så vurderes det at der er belæg for at ekskludere gestik-par 6.  





