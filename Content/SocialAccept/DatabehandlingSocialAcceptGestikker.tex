\section{Gestikker}
\label{TestresultaterSocialAcceptGestikker}
%
Følgende analyse og diskusion foretages på baggrund af værternes respons til spørgsmålene: \textit{Hvordan syntes du det var at lære gestikkerne?}, \textit{Havde du problemer med at huske gestikkerne?}, \textit{Hvor rigtigt syntes du, at du fik gengivet gestikkerne til de enkelte funktioner?} og \textit{Var der nogle gestikker, som du følte dig mere ukomfortabel ved at udføre end andre?} samt gæsternes respons til spørgsmålene: \textit{Bemærkede du at din ven(inde)/kæreste lavede gestikker?}, \textit{Kan du gengive de gestikker som din ven(inde)/kæreste lavede?} og \textit{Hvad synes du om dem?}. Derudover inddrages observationer af hvordan værten reagerer på diverse hints. Førstedel vil omhandle dels, hvordan værterne vurderer det at lære gestikkerne, om de havde problemer med at huske gestikkerne samt hvor rigtigt de syntes de gengav gestikkerne og dels om gæsterne bemærkede det og var i stand til at gengive dem, samt deres vurdering af gestikkerne. Derefter vil fokus rettes mod om værterne fandt nogle af gestikkerne mere ukomfortable at udføre end andre.\blankline
%
I forhold til hvordan værterne responderede på spørgsmålet: \textit{Hvordan syntes du det var at lære gestikkerne?}, så er den gennemgående tendens at det var nemt. Vært 2, vært 3 og vært 6 giver alle udtryk for, at gestikkerne kan associeres med hvordan der normaltvist interageres på en telefon, i forhold til swipe-bevægelsen, samt hvordan der normaltvist interageres med et musikanlæg. I det henseende pointerer vært 3 og vært 4, at det er de rigtige gestikker, der er blevet udvalgt til hver funktion og ifølge vært 3 er det nærmest ikke muligt at vælge nogle gestikker, som vil passe bedre. Vært 5 påpeger, at det var en fordel at gestikkerne blev præsenteret via video og ikke på papirform, da det gjorde gestikkerne mere levende. Generelt er der ingen af de seks værter, som giver udtryk for at have problemer med at huske de udvalgte gestikker. Dog kommenterer vært 1, at værten normalvist ikke pauser musikken når telefonen ringer og derfor var værten, ifølge værten selv, nødsaget til at minde sig selv om det når telefonen ringede. Vært 3 gav udtryk for, at være mere opmærksom på forvrængningen fordi det er blevet pointeret, af andre, at værten har problemer med hørelsen. Første gang vært 5 skiftede til det næste musiknummer blev det gjort med hele hånden og eftersom det gik op for værten at det var forkert, så blev det rettet. Det skal dog påpeges at fejlen opstod under familiariseringen, hvorefter værten fejlfrit gengav den korrekte gestik for at skifte til det næste musiknummer. Vært 2 påpeger at fordi gestikkerne ligger i håndleddet, så er der ingen problemer med at huske dem. 

Til spørgsmålet: \textit{Hvor rigtigt syntes du, at du fik gengivet gestikkerne til de enkelte funktioner?} giver de seks værter udtryk for, at de syntes at gestikkerne blev gengivet korrekt. Vært 3 påpeger, at værten kom til at bruge venstre hånd istedet for højre, hvilket ud fra optagelserne skyldes, at værten spiser cookies med højre hånd. Derudover kommenterer vært 3, at værten glemte at starte musikken igen efter den sidste opringning i familiariseringsrunden fordi værten troede at familiariseringen var færdig, grundet en længere samtale med testleder 2. Vært 4 pointerer at værten ikke tænkte over det, da der blev reageret på gestikkerne hver gang så der opstod ikke situationer hvor værten var nødsaget til at repetere gestikkerne. \blankline
%
De seks gæster har alle bemærket at værten lavede gestikker, men det er ikke alle, der er formår at gengive gestikkerne korrekt. Gæst 1 gengiver gestikken til at skrue op og ned korrekt, men er i tvivl om hvilken vej hånden skal dreje for enten at skrue op eller ned. I tillæg bemærker gæst 1, at vært 1 laver en swipe-bevægelse for at skifte musiknummer, men formår ikke at gengive gestikken korrekt. Til gengæld bemærker gæsten ikke hvilken gestik værten gengiver for at pause eller starte musikken. Gæst 2 kan gengive de forskellige gestikker, dog lukker gæsten hånden i en knytnæve for at pause musikken og swiper med hele hånden istedet for med to fingre. Gæst 3 kan ligeledes gengive gestikkerne korrekt med forbehold for at gæsten alternerer mellem to fingre og hele hånden i swipe-bevægelsen, hvorimod gæst 4 formår at gengive gestikkerne korrekt og uden nogen forbehold. Gæst 5 kan gengive hvordan musikken pauses men ikke hvordan den startes igen, hvilket formentlig skyldes, at gæsten slet ikke kigger på værten når musikken startes igen. Derudover formår gæst 5 at gengive gestikken til at skrue op og ned for musikken korrekt og ligesom tilfældet ved de tidligere gæster, gengiver gæst 5 swipe-bevægelsen med hele hånden. Gæst 6 gengiver gestikkerne korrekt, men formår kun at gengive gestikkerne til at pause musikken, skrue ned for musikken og til at skifte musiknummer. 

Ifølge gæst 2 og gæst 5 gav gestikkerne god meningen i forhold til hvilken funktion de tilhører. Gæst 1 tænkte ikke over dem men giver udtryk for at de var diskrete og fungerede. Derudover kommenterer gæst 1 at værten ikke skulle stå og fægte for at styre musikken. Ifølge gæst 3 er gestikkerne intuitive og derudover kiggede gæsten kun enkelt gang på gestikkerne og vurderede at de gav god mening og at de var lige til. Gæst 5 pointerer at gestikkerne var nemme og at gæsten selv var i stand til at bruge dem. Den eneste gæst, der udviser en negativ holdning er gæst 6, som giver udtryk for gestikkerne til at starte med er blæret, hvorefter gæsten vil blive træt af dem. Gæsten giver udtryk for at det er et fedt festtrick men forestiller sig, at det er træls når der er flere mennesker.
%
\subsection{Var der nogle gestikker, som du følte dig mere ukomfortabel ved at udføre end andre?}
\label{TestresultaterSocialAcceptGestikkerUkomfortabelt}
%
Ud af de seks værter er det kun vært 1 og vært 6, som direkte giver udtryk for at en af gestikkerne føltes mere ukomfortabel end de andre; den til at skrue op og ned for musikken. Tre af de fire værter, som ikke oplevede, at nogen gestikker var mere ukomfortable at udføre end andre, giver udtryk for bekymringer relateret til den specifikke gestik. Vært 5 kommenterer at der kan være noget kulturelt ved gestikken til at pause musikken, som kan misforståes, i og med at det forbindes med en "klap i"-bevægelse. Dog pointere vært 5 at ingen i værtens omgangskreds vil misforstå gestikker, hvorfor det ikke vil opstå et problem. Vært 3 og vært 4 giver begge udtryk for bekymringer relateret til gestikken, som er tilknyttet skru op og ned. Fælles for de to værter er, at de giver udtryk for, at der kan opstå problemer i forhold til hvor meget der skal drejes for at skrue en hvis mængde op eller ned. Hvor vært 3 ydermere kommenterer at gestikker kræver mere finmotorik sammenlignet med de andre udvalgte gestikker. Derudover påpeger værten, at det kan være et problem, hvis bevægelsen kræver en helt rolig og præcis hånd for at kontrollere volumen. I tillæg kommenterer værten at ved fysisk kontakt, eksempelvis med en drejeknap, så er det muligt at lave en mindre bevægelse sammenlignet med semaforiske gestikker, fordi den fysiske kontakt tillader bedre føling med den pågældende knap. 

Der er to årsager til at vært 1 finder gestikken til at skrue op og ned for musikken mere ukomfortabel end de andre gestikker; 1) der er risiko for at der pludseligt skrues for højt op for musikken og 2) hvis rotationen begyndes et dårligt sted, så håndleddet fysisk ikke tillader mere rotation. Værten påpeger dog at når grænsen for, hvor meget ens håndled kan roteres nåes, så kan grebet slippes, hvorefter det er muligt at gribe fat igen. Grebet referer til, at der tages fat i en imaginær drejeknap. Vært 6 giver udtryk for, at der var noget underligt med responsen når værten skruede op og ned. Dette kan formentlig skyldes flere ting; dels at det er testleder 2, som styrer volumen afhængigt af værtens gestikulering, så i tilfælde af at det ikke blev gjort i overensstemmelse med værtens forventning, så er det forståeligt, at responsen kan virke underlig. Derudover kan det skyldes, at testleder 2 havde svært ved præcis at tolke bevægelsesmængden i vært 6's gestikulering, da fingrenes position ændrede sig i takt med rotationen. Derudover påpeger vært 6, at det kræver tilvænning i forhold til hvor meget der skal drejes og hvornår rotationen skal stoppe. Dog pointerer værten, at volumen normaltvist ikke ændres fra 0 til 100. 
%
\section{Observationer af vært og gæst}
\label{TestresultaterSocialAcceptGestikkerObservationer}
%
I følgende afsnit analyseres og diskuteres observationer fra videooptagelserne af de seks par. Observationerne dækker blandt andet over om gæsterne spørger ind til gestikkerne, hvordan værternes interaktion med musikanlægget forløber og hvordan samtalen mellem vært og gæst påvirkes af værtens interaktion med musikanlægget.
%
\subsection{Spørger gæsterne ind til gestikkerne}
\label{TestresultaterSocialAcceptGaestSPGGestikker}
%
Baseret på videooptagelserne er der ingen af gæsterne, der decideret spørger ind til værtens gestikker. Gæst 3 kommenterer når værten bliver bedt om at indstille volumen til et passende samtale niveau ved det første musiknummer. Kommentaren retter sig ikke direkte mod selve gestikken men mod det at værten bruger gestikker, hvor gæsten giver udtryk for, at det er fedt. I musiknummer 2 reagerer gæst 3 på, at musikken bliver skruet højt op og forsøger selv at skrue ned, dette gøres dog i takt med at værten selv skruer ned, hvorfor det er værtens gestik testleder 2 reagerer på. Gæst 3 giver udtryk for gerne, at ville prøve gestikkerne men det realiseres ikke, da samtalen drejes tilbage på ferieplanlægningen. Gæst 4 kommenterer ligeledes på, at værten i starten skal indstille volumen til et passende samtale niveau. Igen retter kommentaren sig ikke direkte mod selve gestikke med mod det at der bruges gestikker, hvorefter gæsten selv får lov til at prøve at indstille volumen og spørger efterfølgende ikke ind til gestikkerne igen. De fire resterende gæster hverken prøver eller kommenterer på værtens gestikker.  
%
\subsection{Hvordan værtens interaktion med musikanlægget forløber}
\label{TestresultaterSocialAcceptVaertsGestikker}
%
I følgende afsnit fokuseres der på; hvordan værterne generelt gestikulerer når de ikke reagerer på hints, om værterne begår fejl, og hvis de gør hvornår og hvilke fejl begåes, om værterne interagerer med musikken uden der har været et hint, om nogen af værterne bruger venstre hånd frem for højre og i hvilke situationer det sker, hvor mange gange værterne retter fokus mod enten anlægget, kommunikationskameraet eller selve gestikken, hvor mange gange gæsterne retter fokus mod værternes gestikker eller hints og om værterne ændre adfærd efter en opringning.\blankline
%
I forhold til hvordan værterne generelt gestikulerer, når de ikke reagerer på hints, så er værterne generelt meget stillesiddende og ingen af værterne laver gestikker, som misforståes af testleder 2. Dog laver vært 4 en "pyt-med-det"-bevægelse, som tilnærmelsesvist minder om et swipe med hele hånden, bortset fra at værtens bevægelsen er mere vertikal end horisontal, som egner sig bedre til swipe-bevægelsen. Værtens "pyt-med-det"-bevægelse kommer kun en gang og kommer i forbindelse med at værten og gæsten taler om budgetterne til de tre ferier. Ud fra observationerne af vært 2, så tyder det på, at værten koncentrerer sig om at høre de forskellige hints, eksempelvis retter værten ofte fokus mod telefonen, selvom den ikke ringer.\blankline
%
  


\subsection{Hvordan påvirkes samtalen mellem vært og gæst}
\label{TestresultaterSocialAcceptSamtale}
%
I følgende afsnit fokuseres der på; hvordan samtalen mellem vært og gæst påvirkes af værtens interaktion med musikanlægget og om værten selv benytter sig af computeren eller papirerne, som gæsten har medbragt. \blankline
%
Baseret på observationerne af de seks par, tyder det på at det primært er ved par 2 og par 6 samtalen påvirkes mellem vært og gæst. Ved par 2 tyder det på at samtalen afbrydes fordi værten koncentrerer sig om at registrerer og reagerer på hints, hvilket får gæsten til at stoppe op og vente til værten er færdig. Dette ændres dog efter værten får af vide dels at værten gerne må benytte sig af computeren, hvilket sker i opringningen i musiknummer 1 og dels når værten bliver opfordret til at gøre gestikkerne mere diskrete, hvilket sker efter opringningen i musiknummer 2. Efterfølgende tyder det på dels at samtalen mellem vært og gæst ikke påvirkes i lige så høj grad som før og dels at værten ikke koncentrerer sig lige så meget om at registrere diverse hints. Ud fra både observationer og værtens egen respons i exit-interviewet, så tyder det på, at værten havde problemer med at registrer forvrængningen. I forhold til par 6, så tyder det på, ud fra observationerne, at hver gang værten reagerede på et hint, så tiltrækker det gæstens opmærksomhed, hvilket afbryder samtalen, dog er de begge i stand til at vende tilbage til samtale emnet. Årsagen til at værtens interaktion med musikanlægget tiltrækker gæstens opmærksomhed, er formentlig fordi at værten både retter sit fokus og sin krop mod anlægget, som er placeret modsat gæsten, hvilket derfor resulterer i at gæsten ligeledes retter fokus fra samtalen til værtens interaktion. I opringningen i musiknummer 4 opfordres vært 6 til at lave gestikkerne mere diskrete, hvorefter det tyder på, at samtalen ikke længere påvirkes i lige så høj grad. Det tyder på, at både opringningen og det at værten for alvor begynder at bruge gæstens medbragte computer er med til dels at samtalen ikke afbrydes lige så brat, som før og dels at værten interagerer med musikanlægget mere afslappet og uden at rette sit fokus og sin kroppen mod anlægget.

Ud fra observationerne af de resternede fire par; par 1, par 3, par 4 og par 5, så tyder det ikke på at værtens interaktion med musikanlægget har påvirket samtalen med gæsten negativt. Vært 1 bliver ved den først opringning dog opfordret til kun at reagerer på hints, da værten skifter musiknummer i slutningen af det første musiknummer, der ikke indholder hints. Når vært 1 reagerer på en opringning arbejder gæsten som regel videre med ferieplanlægningen, hvorefter værten hurtigt falder ind i samtalen igen. Baseret på observationerne af par 3 tyder det på, at den eneste gang et hint påvirker samtalen er ved musiknummer 4 hvor der bliver skruet højt op for musikken, hvilket får gæsten til at reagerer. Generelt tyder det ikke på, at samtalen mellem vært 3 og gæst 3 påvirkes negativt, da de holder sig til samtale emnet og samtidig inkluderer opkaldende som en naturlig del af deres samtale. Lignende er gældende for par 5, hvis samtale ikke påvirkes af diverse hints og hvor værten ligeledes inkluderer opkaldende som en naturlig del af deres samtale. Ud fra observationerne af par 4 tyder det heller ikke på, at værtens interaktion med musikanlægget påvirker samtalen med gæsten negativt, da både vært og gæst formår at opretholde samtalen og gæsten arbejder videre når værten reagerer på et opkald. \blankline
%
I forhold til om værterne benytter sig af gæstens medbragte computer eller de medbragte papirer, så benytter halvdelen af værterne sig af computeren og ingen benytter papirerne. Vært 2 benytter sig af computeren allerede efter den første opringning i musiknummer 1 og benytter den stort set resten af tiden. Vært 5 benytter computeren i starten af musiknummer 3, hvor gæsten hurtigt overtager inden værten reagerer på hintet om at skifte musiknummer og kortvarigt i musiknummer 6. Vært 6 benytter computeren kortvarigt i musiknummer 1 og igen efter opkaldet i musiknummer 4, hvorefter værten benytter sig af computeren stort set resten af tiden. 

På baggrund af observationerne vurderes det, med forbehold for at vært 2 og vært 6 først skulle opfordres til at gestikulere mere diskret, at værternes interaktion med musikanlægget ikke påvirkede samtalen med gæsten negativt. Ydermere opstod der ikke akavede pauser mellem vært og gæst hvor de ikke vidste, hvad de skulle sige til hinanden eller hvor de kom fra. 

