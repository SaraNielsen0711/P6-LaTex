\section{Udvælgelse af gestik-par til pause og start}
\label{TestresultaterPauseStart}
%
Udvælgelsen af hvilket gestik-par, der skal knyttes til pause og start foretages på baggrund af testpersonernes udsagn. I \fullref{app:TestresultaterPauseDaarlig} analyseres testpersonernes respons i forhold til hvilke gestik-par de mindst kan lide og på baggrund af den analyse ekskluderes gestik-par 2, gestik-par 3 og gestik-par 4 fra yderligere undersøgelser. Dette medfører at udvælgelsen af hvilket gestik-par, der skal knyttes til pause og start kun foretages på gestik-par 1, gestik-par 5, gestik-par 6 og gestik-par 7. \blankline
%  
I nedenstående \autoref{tab:GestikParITopTrePause} fremgår samtlige testpersoners top tre rangering, hvor der ikke er taget forbehold for hvorvidt testpersonerne har inkluderet et gestik-par, som der på baggrund af \fullref{app:TestresultaterPauseDaarlig} er blevet ekskluderet.
%
\begin{table}[H]
	\centering
	\begin{tabular}{ | p{3cm} | p{3cm} | p{3cm} | p{3cm} |}
	\hline
		 & 1. Plads & 2. Plads & 3. Plads \\ \hline
		Testperson 1 & Gestik-par 1 & Gestik-par 7 & Gestik-par 5 \\ \hline
		Testperson 2 & Gestik-par 7 & Gestik-par 6 & Gestik-par 1 \\ \hline
		Testperson 3 & Gestik-par 1 & Gestik-par 2 & Gestik-par 4 \\ \hline
		Testperson 4 & Gestik-par 1 & Gestik-par 5 & Gestik-par 7 \\ \hline
		Testperson 5 & Gestik-par 5 & Gestik-par 1 & Gestik-par 6 \\ \hline
		Testperson 6 & Gestik-par 1 & Gestik-par 5 & Gestik-par 7 \\ \hline 
		Testperson 7 & Gestik-par 5 & Gestik-par 2 & Gestik-par 6 \\ \hline
		Testperson 8 & Gestik-par 7 & Gestik-par 3 & Gestik-par 6 \\ \hline
		Testperson 9 & Gestik-par 7 & Gestik-par 1 & Gestik-par 6 \\ \hline
		Testperson 10 & Gestik-par 1 & Gestik-par 2 & Gestik-par 4 \\ \hline
		Testperson 11 & Gestik-par 5 & Gestik-par 7 & Gestik-par 6 \\ \hline
		Testperson 12 & Gestik-par 2 & Gestik-par 1 & Gestik-par 5 \\ \hline
		Testperson 13 & Gestik-par 6 & Gestik-par 3 & Gestik-par 5 \\ \hline
		Testperson 14 & Gestik-par 6 & Gestik-par 5 & Gestik-par 7 \\ \hline
		Testperson 15 & Gestik-par 1 & Gestik-par 4 & Gestik-par 5 \\ \hline
		Testperson 16 & Gestik-par 1 & Gestik-par 5 & Gestik-par 7 \\ \hline
		Testperson 17 & Gestik-par 1 & Gestik-par 5 & Gestik-par 7 \\ \hline
		Testperson 18 & Gestik-par 5 & Gestik-par 7 & Gestik-par 1 \\ \hline
	\end{tabular}
	\caption{Oversigt over samtlige testpersoners top tre i forbindelse med pause og start.}
	\label{tab:GestikParITopTrePause}
\end{table}
\noindent
%
Da der er tre gestik-par, som er blevet ekskluderet, kan ovenstående  \autoref{tab:GestikParITopTrePause} med fordel opsummeres både i forhold til at fjerne de tre gestik-par men også i forhold til at opsummere hvor mange gange de fire tilbageværende gestik-par indgår i top tre rangeringen. 
%
\begin{table}[H]
	\centering
	\begin{tabular}{ | p{2.4cm} | p{2.4cm} | p{2.4cm} | p{2.4cm} |p{2.4cm}|}
	\hline
		 & 1. Plads & 2. Plads & 3. Plads & I alt \\ \hline
		Gestik-par 1 & 8 & 3 & 2 & 13\\ \hline
		Gestik-par 5 & 4 & 5 & 4 & 13\\ \hline
		Gestik-par 6 & 2 & 1 & 5 & 8\\ \hline 
		Gestik-par 7 & 3 & 3 & 5 & 11\\ \hline
	\end{tabular}
	\caption{Oversigt over dels hvor mange gange hvert gestik-par indgår i samtlige testpersoners top tre i forbindelse med pause og start og dels over hvor mange gange et gestik-par sammenlagt indgår i en top tre.}
	\label{tab:GestikParITopTrePauseOversigt}
\end{table}
\noindent
%
Det tyder på at de otte testpersoner, som alle rangerer gestik-par 1 på en første plads, gør det fordi de vurderer den til at være en kombination af at være; simpel, enkel, logisk, naturlig, oplagt, nem at huske, nem at udføre, hurtig at udføre og fordi den giver mening. Ud fra \autoref{tab:GestikParITopTrePause} fremgår det at testperson 3 og testperson 10 har fuldstændig den samme top tre rangering og sammenholdes det med testpersonernes respons og videooptagelser, så er det også de eneste to testpersoner, som har gestik-par 1 på en første plads, som ikke er i tvivl om at gestik-par 1 skal være på en første plads. Ydermere giver testperson 10 udtryk for, at der ved de gestik-par testpersonen har valgt ikke er så stor risiko for at testpersonen kommer til at lave dem anderledes og derudover er det ikke en bevægelse som testpersonen hyppigt laver. Ifølge testperson 5, som har rangeret gestik-par 1 på en anden plads, så er bevægelsen; et statisik stop-tegn, normal i ens kropssproget.   

De seks andre testpersoner, som har rangeret gestik-par 1 på en første plads, har alle inkluderet gestik-par 5 enten på en anden plads eller en tredje plads. Testperson 1 svinger mellem at have gestik-par 1 og gestik-par 5 på en første plads, hvor testpersonen til at starte med har top tre rangeringen, der fremgår af \autoref{tab:GestikParITopTrePause}, hvorefter den ændres til; gestik-par 5, gestik-par 1 og gestik-par 7 på henholdvis en første, anden og tredje plads. Når testpersonen afslutningsvist skal gengive gestikkerne foretrækker testpersonen gestik-par 1 igen. På baggrund af det er det ikke muligt at afgøre hvorvidt testperson 1's top tre rangeringen bør være gestik-par 1, gestik-par 5 og gestik-par 7 på henholdvis en første, anden og tredje plads, dog tyder det på at det er tilfældet. Ifølge testperson 16 er de tre gestik-par som testpersonen har rangeret i sin top-tre næsten lige gode, da de alle tre er simple, det tyder derfor på at der ikke er stor forskel mellem en første, anden og tredje plads. Dog vurderer testpersonen at gestik-par 1 er oplagt og den er ikke er til at glemme. Selvom testperson 4 ikke giver udtryk for at være i tvivl om sin top tre rangering, så tyder det på at testpersonen har rangeret gestik-par 1 højere end gestik-par 5 fordi parret er mere simpel. Dog giver testperson 4 udtryk for at gestik-par 5 er meget intuitiv. Ligende tendens forefindes ved testperson 17, som argumentere for hvorfor testpersonen har rangeret gestik-par 1 som værende den bedste; det er logisk. Når testperson 17 afslutningsvist bliver bedt om at gengive sine fortrukne gestikker, så er det gestik-par 5, som testpersonen knytter til pause og start. I og med at testpersonen tidligere har givet udtryk for at gestikkerne skal falde testpersonen naturligt ind og at testpersonen ikke skal tænke over hvilken bevægelse der laves, så tyder det på at gestik-par 5 måske bør være rangeret over gestik-par 1.   

Selvom testperson 6 og testperson 15 har rangeret gestik-par 1 på en første plads, så tyder det på at de foretrækker at der er bevægelse i gestikken. Testperson 6 gengiver en bevægelse, som minder om en kombination af gestik-par 1 og gestik-par 7, hvor håndens position fra gestik-par 1 bibeholdes mens fingrenes bevægelse i gestik-par 7 bibeholdes. Derudover giver testperson 6 også udtryk for at mekanikken i gestik-par 5, hvor fingrene lukkes sammen for at pause og åbnes igen for at starte musikken, er god. Selvom det ikke nøjagtigt gengiver bevægelsen i gestik-par 5, så tyder det på at det testperson 6 efterlyser formentlig kan opfyldes ved gestik-par 5. Testperson 15 giver udtryk for godt at kunne lide at der er forskel på pause og start i gestik-par 5, hvilket ikke er tilfældet for de to gestik-par testpersonen har rangeret over gestik-par 5. \blankline
%
Der er flere årsager til at fire testpersoner har rangeret gestik-par på en første plads, hvor den primære årsag er at testpersonerne forbinder bevægelsen med en ti stille bevægelse, som de forbinder med at lukke munden på anlægget eller at de lukker musikken. Testperson 18 pointerer at det er en familiær bevægelse og at alle ved hvad det betyder når en anden person, laver luk-delen i et krokodillenæb til en; ti stille. Derudover er det ord som logisk, intuitiv og naturlig testpersonerne beskriver gestik-par 5 med. I relation til bevægelsen i gestik-par 5 så tyder det på testperson 5 favoriserer parret dels fordi bevægelsen kan udføres tæt på kroppen, der skal ikke huskes et bestemt mønster og bevægelsen foregår ikke ud fra kroppen. Derudover kommenterer testperson 11 at det er en lille bevægelse, faktisk den mindste bevægelse der kan laves, hvor det stadig giver mening.   





Skrue op og ned: testperson 1 (gestik 5), mute testperson 3 (5, 6 og 7)









BEDSTE: 5
TP5: Her vil jeg tage den her (gestik 5) som den bedste, der lukker man bare munden på anlægget.  Så kan jeg også godt lide gestik 6, den er også fint. Men igen, den er lidt mere aktiv i forhold til hvad jeg måske vil gøre derhjemme. Så vil jeg hellere bare gøre sådan her (gestik 5). I stedet for at skulle sådan (bevæger armen frem), men det kommer også an på hvor stor en rolle det her med genkendelse og normal kropssprog. Så jeg ville tage 5, 1 og 6. 
TP5: Fordi nummer 5 det er også, igen, typisk kropssprog. Især hvis man snakker med folk om at nu skal du bare lukke, ti stille eller ro. Det synes jeg sådan, det er meget naturligt. Og den er heller ikke så langt fra kroppen og man skal ikke huske et eller andet mønster, man skal ikke bevæge sig sådan udad, man skal bare op og så luk. Det synes jeg er meget godt. Og så rangering 2 ved  


TP7: Den bedste det må være 5’eren og den næste må være 2’eren, og så 6’eren. 
TP7: Altså 5’eren synes jeg var lidt sjov fordi det er sådan lidt “zip it”. Jeg kunne forestille mig hvis man sådan skulle vise det frem, så kunne folk synes det var sjovt, det kan de godt huske og sådan noget. Og så 2’eren det var bare sådan, det virkede sådan rimelig lige til også. Det var sådan stop. Det er sådan noget man ligesom forstår ved et stop tegn og sådan. 6’eren der kunne jeg godt lide der hvor hun skulle starte den igen, hvor hun skulle sådan, hvor hun sådan gav slip på lyden så den fik lov til at spille, det var også ret godt. 
TP7: Jeg tror bare det var sådan generelt var det sådan jeg bedst kunne lide det, hvis jeg selv skulle bruge det. 

TP11: Den bedste det er nok 5’eren synes jeg. Jeg er egentlig ikke så stor fan af de andre, fordi jeg synes de kræver rimelig store bevægelser. Men så skulle det nok være 7’eren som nummer to og så 6’eren som nummer 3. 
TP11: Jeg kan godt lide den her idé med at du lukker musikken ned og så er det meget intuitivt også. Især 5’eren kan jeg godt lide, fordi det er bare en lille bevægelse. Og så 6 og 7 minder meget om 5’eren. 5’eren er nok den mindste bevægelse du kan lave, hvor det giver mening. 

TP18: 5,7,1 
TP18: Fordi den her (laver gestik 5) virker logisk, altså det giver mening for mig at du lukker musikken og du åbner igen (laver gestik 5) og det samme med at du trækker noget ned (laver gestik 7) for at slukke den og kunne åbne den igen (laver gestik 7). Og det er heller ikke gestures du gør ofte, så hvis du har brug for at gøre det, jamen så giver det mening, den bliver ikke mistaken for at være et eller andet andet. 1’eren igen fordi den er den letteste, altså det er flad hånd opad (TP løfter hånden og laver gestik 1) og flad hånd opad igen for at starte og stoppe den (TP løfter hånden og laver gestik 1). Men jeg kan godt lide det med at der faktisk, at der er forskel på, altså at det er modsatte bevægelser den ene der åbne nu og den anden lukker alt efter om du tænder eller slukker. 
TP18: Fordi det er logisk for mig at det her betyder ti stille (laver gestik 5 luk) og det her betyder åben munden (laver gestik 5 åben) altså det er en gestus der er familiær med, altså man ved alle sammen at hvis der er en der gør sådan her til en (laver gestik 5 luk) så er det fordi man skal tie stille og hvis man gør sådan her (laver gestik 5 åben) så er det okay. Men det er ikke så tit at man ser nogen der gør sådan her (laver gestik 5 åben) men man ser tit den her (gestik 5 luk) og så er det let at vende sig til at det omvendte er start. (snakker ikke om de to andre)

BEDST: 6 

TP13: Skal de også være i orden eller er det bare samlet (du må gerne sætte dem i orden) okay, hvad var det 6’eren gjorde? (demonstrer) jeg tror hvertfald 3’eren er en af de bedste, fordi det er sådan (laver gestik 3) og så åben (laver gestik 3) fordi det er ikke sådan noget man typisk vil gøre. Og så nok også, ja så 6’eren og 7’eren der ved jeg slet ikke hvad for en der er bedst af dem , fordi ja det er jo det samme bare hvad for en rækkefølge man gør det. Men så vil jeg nok sige at det var 5’eren, 6’eren og så 3’eren, hvor 3’eren så er bedst (så 3, 6, 5?) ja.
Ved forbedringer: (ændre til at hun bedst kan lide 6’eren) - ny rækkefølge 6, 3 og 5. (Får musik på til at lave forbedringer)
TP13: Fordi at det er sådan en ting, som jeg tænker at man ikke sådan typisk vil gøre, for eksempel det der med at løfte hånden (gestik 1) det er sådan noget man godt bare sådan kunne komme til at gøre når man sidder der hjemme, altså den der prik (gestik 4) den var også meget fin men det kan man jo godt komme til, sådan lige at pege på et eller andet og også den der 2eren med at stikke hånden frem, så er det med at sætte hånden frem og man skal gøre noget, det gør en forskel. 
TP13: Det er sådan jeg tænker at, hvad man sådan typisk vil, hvad for noget der er sådan mest, hvad det hedder, altså hvad man ikke sådan umiddelbart vil komme til at gøre ved et uheld. Altså den der med (laver gestik 5) med at stoppe den, den syntes jeg er meget smart fordi det giver mening at man sådan (laver gestik 5) stop og så den anden (gestik 2) det giver også mening. (laver gestik 3) det der med kryds.


TP14: Jeg kunne bedste lide 6, 5 og 7 tror jeg
TP14: Igen det der med at jeg syntes at det er tilpas akavet (laver gestik 6) men alligevel en meget naturlig bevægelse i forhold til det der med at slukke åbne lyd og lukke lyd (laver gestik 6). Så syntes jeg egentlig at det virkede meget fedt. 
TP14: 6’eren var den i mod skærmen (laver gestik 6) og det tænker jeg at det giver meget god mening, det tror jeg er meget naturligt i forhold til at hvis jeg skulle tænde eller slukke på et eller andet bestemt så jeg ville sigte i mod det jeg skulle tænde eller slukke for (laver gestik 6). (Hvad med de to andre?) Jamen det var egentlig mest på grund af den der lukke mekanisme (laver gestik 7) at jeg syntes at de var fede os (laver gestik 5), så det var egentlig bare om det var op eller ned, det ved jeg ikke, det har jeg egentlig ikke så mange holdninger til, derudover.  





BEDST: 7
TP2: 7, 6, 1, umiddelbart.
TP2: Fordi jeg synes de var simple. Jeg synes det gav meget mening det der - ja eller simpelt, og det gav mening at man ligesom lukkede og tændte i hvert fald med 7’eren der. Ja, og 6’eren der var også sådan luk og tænd. Og 1’eren den var meget simpel synes jeg. Jeg synes også det gav mening at det bare var sådan.
TP2: 1’eren den manglede lidt det der med at der var forskel på pause og play, hvor jeg synes at 7’eren og 6’eren gav mere mening, at den ene den var pause når man lukkede og play når man åbnede. 



TP8: Jeg kunne godt lide gestik 3 som den bedste. Og så tænker jeg 7’eren og 6’eren i den rækkefølge. (efter at prøve det til sidst, hvor der bruges gestik 7) Det (gestik 7) var det der gav mest mening, når man ligesom sad her. 
TP8: Gestik 3 det var ligesom, den var sådan unik i det, hvor man ligesom havde en fornemmelse af at det var ikke bare en knap man trykkede på, men der var ligesom sådan en motion der giver mening. Så igen gestik 6 og 7, det var bare den der med at du lukker sammen og du åbner op igen. Det er sådan en bevægelse der giver mening i mit hoved, at det er sådan man stopper noget musik og sådan man får noget musik til at komme frem igen. 
TP8: Umiddelbart så gestik 3, den var sådan speciel i det, sådan lidt sjov eller hvad man kan sige. Og de to andre det var sådan lidt der åbner man op og kører ned agtigt. 



TP9: Jeg har faktisk lidt en konkurrence mellem 4 af dem. 1, 5, 6 og 7. Jeg tror faktisk jeg vil sige 1, 5 og 7. I rækkefølgen 7, 1, 6. 
TP9: 1’eren - dejlig simpel. 
TP9: Det virkede logisk, gribe musikken og slippe den fri igen. 1’eren er igen stop og ja, du fortsætter bare. 6’eren, den er lidt mere ud af. Og igen, jeg har ungen her på armen og jeg står hjemme i dagligstuen, så hvad kan jeg gøre uden at hun farer efter armen. 












