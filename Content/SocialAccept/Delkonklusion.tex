\section{Delkonklusion}
\label{SocialAcceptDelkonklusion}
%
På baggrund af foregående analyse og diskussion kan der konkluderes på forskellige aspekter vedrørende værternes evne til at interagere med et musikanlæg via semaforiske gestikker som en sideløbende opgave til den primære opgave; samtalen med gæsten.

Først og fremmest fandt testpersonerne de udvalgte gestikker nemme at lære og de gav udtryk for, at gestikkerne passede godt til de individuelle funktioner. Fælles for de seks gæster er, at de bemærkede værternes gestikker, men det var ikke alle, der var i stand til at gengive de udvalgte gestikker. Hvorvidt gæsterne formåede at gengive værtens gestikker fremgår af \fullref{TestresultaterSocialAcceptGestikker}. Ud fra dette konkluderes det, at gestikkerne kan udføres diskret, de er nemme at lære og de er nemme at relatere til den specifikke funktion. Baseret på \fullref{TestresultaterSocialAcceptGestikkerUkomfortabelt} er den eneste gestik, der potentielt kan opstå problemer ved gestik-par 2 til at justere lydstyrken. Værterne gav udtryk for, at de ikke altid følte, at de havde fuld kontrol over lydstyrken, hvilket formentlig skyldes at testen udføres som et \textit{Wizard of Oz}-eksperiment, hvor testpersonerne får en fornemmelse af, at de rent faktisk interagerer med et funktionelt system, men hvor det i virkeligheden er en testleder, som styrer hvad der foregår, afhængigt af testpersonernes input. Hvis testleder 2 ikke nødvendigvis justerer lydstyrken nøjagtigt efter værtens forventning, kan dette være årsagen til værternes, til tider, manglende følelse af kontrol.\blankline
%    
Med udgangspunkt i \fullref{TestresultaterSocialAcceptVaertsGestikker}, konkluderes det, at ingen af testpersonerne i forbindelse med deres samtale ubevidst lavede gestikker, som kunne misforståes som decideret kommandoer til musikanlægget. Da det er uvist, hvordan testpersonerne normalvis gestikulerer i en samtale, hvorvidt samtaleemnet har haft en indflydelse på hvad testpersonerne gestikulerer samt hvorvidt selve testscenariet har haft en indflydelse, er det ikke muligt endegyldigt at konkludere, hvorvidt der i en reel kontekst kan opstå misforståelser. Dog skal det påpeges, at det er forsøgt at udvælge gestikker, som normalvis ikke indgår i ens naturlige kropssprog, hvorfor det vurderes at risikoen for misforståelser er minimal.\blankline   
%
Baseret på observationerne, som belyses i \fullref{TestresultaterSocialAcceptVaertsGestikker}, begår værterne generelt få fejl og udover vært 1, som afslutningsvist glemmer at pause musikken ved en opringning, så er fejlene relateret til at skifte musiknummer. Ydermere tyder det på, at desto bedre kendskab og tiltro værterne har til gestikkerne, desto mere afslappet og dovne bliver gestikkerne. I tillæg er der tendens til, at når værtens fokus tydeligt vendes mod musikanlægget, har det indflydelse på, hvor meget gæsten forstyrres samt hvordan samtalen forløber. Dog forefindes der ingen indikationer på, at selve interaktionsformen har en negativ indflydelse på samtalen og der opstod ingen akavede pauser, når hints eller interaktion midlertidigt afbrød samtalen. For at undgå, at værterne interagerede med musikken uden hints samt minimere antallet af fejl, uden en decideret afbrydelse fra testlederne, konkluderes det, at det er fordelagtigt at anvende et telefonopkald som hint, da dette gav testleder 2 mulighed for at kommunikere til værten uden gæsten nødvendigvis hørte det og for værten at stille spørgsmål til testlederen.\blankline
%
Ingen af testpersonerne giver udtryk for, at det var forstyrrende at interagere med musikanlægget og samtlige testpersoner giver tilmed udtryk for, at værten var i stand til at udføre begge opgaver. To af testpersonerne påpeger endda, at det er mindre forstyrrende for en samtale at interagere med gestikker end ved at bruge sin mobiltelefon. Selvom det ikke er muligt at påvise en perifer interaktion ud fra denne test, er der evidens for at interaktionen kan foregå sideløbende med en primær opgave, hvilket indikerer, at de udvalgte semaforiske gestikker ved gentagende og vedvarende brug højst sandsynligt kan foregå som perifer interaktion. \blankline
%
Da ingen testpersonerne giver udtryk for at vil have et problem med enten selv at udføre gestikkerne eller se andre udføre dem i forskellige sammenhængen, så konkluderes det, at de udvalgte semaforiske gestikker er socialt acceptable, hvorfor de ikke påvirker det sociale samværd med andre, negativt. Dog påpeger flere testpersoner, at der kan opstå problemer i festsituationer, hvor flere vil være fristet til at interagere med produktet. I den forbindelse foreslår nogle af testpersonerne, at det kan være en fordel, at systemet kan aktiveres og deaktiveres efter behov. Endvidere fastslås det, at testpersonerne gerne vil bruge de udvalgte semaforiske gestikker til at styre et musikanlæg, så længe det er under de rette omstændigheder, jævnfør \fullref{TestresultaterSocialAcceptBrug}.  

Endeligt konkluderes det på baggrund af både observationer og testpersonernes egne udsagn, at interaktion med et musikanlæg sagtens kan foregå ved brug af semaforiske gestikker som en sideløbende opgave til den primære opgave; samtalen med gæsten.



