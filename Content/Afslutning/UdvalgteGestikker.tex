\section{De udvalgte gestikker}
\label{DiskussionUdvalgteGestikker}
%
Diskuter de udvalgte gestikker, der var andre muligheder til at skrue op og ned, i forhold til bevægelsesmængde og afstand til kroppen, midas touch problem, maskering af gestikker, retningsbestemt fremfor aktiveringsgestik (perifert, socialt accept)\blankline
%
Midas touch problem. Der er allerede under komplikationer ved gestik åbnet op for at gestikkerne skal rettes mod anlægget. Så skygger man heller ikke for gestikken.\blankline
%
Retning på swipe\blankline
%
Standardisering af gestikker
%


I henhold til hvilken semaforisk gestik, der anvendes så nævner flere testpersoner, at statiske gestikker er utilregnelige fordi de ikke tillader kontrol over, eksempelvis hvor meget lydstyrken justeres eller hvilket musiknummer der skiftes til. Dertil er den gennegående tendens at statiske gestikker, særligt dem der indeholder en form for peg er  


 Dette kommer blandt andet til udtryk ved følgende: 
%
\begin{quotation}
	\noindent
	\textit{Og så 5'eren fordi at det virker sådan at man har mere kontrol der i forhold til 9'eren, hvor du bare står sådan og vifter opad. Det kunne jo være hvilken som helst volumen du var kommet op på.} TP7, \autoref{app:NoterValgAfGestikker}.
\noindent
\end{quotation}
%