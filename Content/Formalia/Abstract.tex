\begin{center}
	\textbf{{\huge Abstract}}
\end{center}
%
As most of our electronic devices demands our central attention and a focused interaction the need for designing devices which can be operated in our peripheral attention increases. Peripheral interaction relates to the possibility of operating a device with minimum attention and cognitive resources, which allow the user to solve a secondary task meanwhile focusing on a primary task. The primary goals for this project is to; 1) investigate how an interaction with a Bang $\&$ Olufsen soundsystem can take in the peripheral attention. 2) select which semaphoric gestures that should be connected to the following chosen functions; pause, play, previous, next and volumen adjustment. 3) investigate the chosen semaphoric gestures in a social context where the interaction with a soundsystem is defined as a sidetask to a primary task; a conversation with a guest. 4) determine if and how the chosen semaphoric gestures affects social acceptance according to the host and the guest.            

The first goal is achieved by an investigation where it is determined that semaphoric gestures supports peripheral interaction. The second goal is achieved by a qualitative investigation performed on 18 test subjects, preferably students from Aalborg University. From the investigation it is determined that a gesture similar to crocodile beak connects to pause when the beak closes and play when the beak opens. A semaphoric gesture related to previous and next is a swipe motion with both index finger and middle finger stretched from right to left for next and left to right for previous. For volumen adjustment the semaphoric gesture is similar to how one adjusts the volume by a physical volumen-knop. The third goal is achieved by a qualitative investigation performed on 12 test subjects divided into pairs of two, preferably students from Aalborg University who knows each other. The investigation is sought to determine the hosts ability to perform the interaction with a soundsystem while conducting a conversation with a guest. Based on the investigation the conclusion is that the hosts manages to solve both tasks. The fourth goal is achieved by conducting seperat exit-interviews. Based on the exit-interviews it is concluded that semaphoric gestures is social acceptable as an interactionform with a soundsystem.