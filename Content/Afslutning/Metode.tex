\section{Metodevalg}
\label{DiskussionMetodevalg}
%
Ved første undersøgelse, hvor de semaforiske gestikker blev udvalgt, skulle testpersonerne først se de forskellige forslag til semforiske gestikker, hvilket blev vist på de tre videoer, jævnfør \autoref{app:VideooptagelserValgAfGestikker}. Herefter blev de udspurgt om hvor nemt de syntes det var at udpege, hvilke gestikker de bedst og mindst kunne lide samt hvordan de syntes filmene illustrerede de forskellige gestikker. Her var den gennemgående tendens, at det var nemt at udpege, hvilke gestikker de bedst og mindst kunne lide, men at tredjepladsen nogle gange var svær at placere. Alle testpersoner syntes, at filmene illustrerede gestikker godt og at det var fint der kun var fokus på gestikkerne. Gestikkerne er optaget og vist, for at testpersonerne dels kan få et indtryk af hvilke gestikker, der er mulige at bruge og dels kan få et bedre indtryk af, hvordan de udføres, frem for hvis de fik vist stationære billeder. Derudover er de optaget, for at alle testpersoner ser den samme gestik udført på samme måde. Det skal bemærkes, at testpersonerne kun får det indtryk, som de bliver givet. Hvis demonstatoren havde svært ved at udføre gestikken ordentligt, er det også dét indtryk testpersonen tager med sig, hvilket kommer til udtryk ved TP16, der synes at GP2 til justering af lydstyrke ser ukomfortabel ud. Derudover kan der både være fordele og ulemper ved at testpersonerne kun ser gestikker udført i en klinisk sammenhæng. Testpersonerne kan i den givne situation koncentrere sig om gestikken, uden at blive forstyrret af andet, men de kan ikke se gestikken udført i den kontekst, hvortil den skal bruges. På baggrund af foregående argumenter vælges det til testen i den sociale kontekst at optage en ny video, jævnfør \autoref{app:VideooptagelseSocialAccept}, hvor gestikkerne vises i en mere naturlig kontekst.\blankline 
%
Testen i den sociale kontekst blev udført som et \textit{Wizard of Oz}-eksperiment, da en virkende prototype på nuværende tidspunkt ikke eksisterer. For at få værterne til at interagere med musikanlægget, var det nødvendigt at give dem hints, da der ellers var risiko for at værterne glemte at interagere med musikanlægget og i stedet kun fordybede sig i ferieplanlægningen. Ud fra \autoref{fig:KiggerImodAnlaeg} er det ikke muligt at konkludere, at værterne kigger mere ved nogle funktioner end andre, hvilket i dette tilfælde er positivt, da det kan betyde at der ikke er én gestik, som kræver mere opmærksomhed and andre. 

Fokuseres der generelt på de forskellige hints, kan disse have været med til at gøre situationen mindre økologisk, da det hverken er naturligt at ens musik ændrer lydstyrke midt i et musiknummer, at musiknumrene på ens egen playliste er forvrængede eller at telefonen ringer så mange gange på så kort tid. Derimod virkede det som om, at testpersonerne blev mere forstyrret af de forskellige hints end af selve interaktionen. Specielt hintet med det forvrængede stykke musik voldte problemer hos værterne, da nogle havde svært ved at høre forvrængningen, mens andre skiftede musiknummer, selvom der ikke var forvrægning. Alternativt til at forvrænge musikken kunne værterne introduceres til, at hver gang et dansk musiknummer eller en reklame blev afspillet, skulle der skiftes videre. Problemet med det er, at testsessionerne enten bliver meget længere eller at et testleder 2 skal skifte musiknummer midt i et andet nummer, hvilket kan skabe forvirring. Endnu et alternativ vil så være at korte sangene ned, så det danske musiknummer eller reklamen kunne afspilles hyppigere. 

Testen vil højst sandsynligt blive mere økologisk, hvis værten får længere tid under en familiarisering til at lære interaktionsformen at kende. Dette kan eventuelt ske over flere dage. I tilfælde af at værten lærer gestikkerne over længere tid, kan det samtidig være nemmere at påvise en perifer interaktion.

 Ydermere kan det overvejes, om det har en effekt at gæsterne slet ikke ved hvad der skal foregå med musikanlægget op imod at de både kender musikanlægget og interaktionsformen på forhånd. Det forventes, at gæster der kender både musikanlæg og interaktionsform er mindre interesseret i interaktionen end personer, der aldrig har set det før. 