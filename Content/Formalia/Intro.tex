\chapter{Hvorfor Animationer?}
\label{HvorforAnimationer}
Note: Skriv en form for indledning så det ikke kommer så lige på og hårdt. \\[5mm]
%
For at kunne svare på hvorfor der skal bruges animationer, er det essentielt at præcisere hvilken type animationer, der involveres i hvilke situationer. Der er særligt fokus på to forskellige animationstyper; funktionelle animationer og instruktionelle animationer. De funktionelle animationer er de animationer, som eksempelvis anvendes i brugergrænsefladen på en computer. Hvor instruktionelle animationer typisk anvendes som et undervisningsværktøj. Der er flere fællestræk mellem de funktionelle og de instruktionelle animationer blandt andet at de begge kan gengive elementer fra virkeligheden (isomorfisk) i form af bevægelsesmønstre eller betydning. Ligeledes er animationer gode til at illustrere vægt, kræfter og hastigheder på fysiske objekter. 

Med udgangspunkt i funktionelle og instruktionelle animationer vil følgende afsnit belyse hvorfor der bruges animationer i brugergrænseflader, samt hvorfor animationer bruges som et undervisningsværktøj. Dernæst belyses mulige problemstillinger vedrørende kognitiv belastning forårsagede af animationer, efterfulgt af hvilken grad af interaktion brugeren kan have med animationen. \\[5mm]

HUSK: Skriv hatten færdig når de andres afsnit spiller. 

Animationer i brugerflader er pr. natur ofte resultatet af en handling, hvilket hjælper med at definere et åbenlyst konsistenskriterie; at en given handling altid skal medføre den samme animation\fxnote{kilde! eller op i 'hatten'}. 
%
\input{Chapter/Problemanalyse/Hvorfor_animationer}
\input{Chapter/Problemanalyse/Inkonsistens}
%\input{Chapter/Problemanalyse/Alder_og_praeferencer}
\input{Chapter/Problemanalyse/Varighed}
\input{Chapter/Problemanalyse/Problemafgraensning}
\section{Problemformulering}
\label{Problemformulering}

Ved samarbejde med Bang $\&$ Olufensen er det altså et ønske at undersøge, hvordan der kan interageres med et interaktivt kunstværk, uden at det skal forstyrre primære opgaver som eksempelvis samtale. Måden, hvorpå der skal interageres, er ved brug af semaforiske gestikker, der signalerer hvordan musikken skal ændre sig uden at dette skal forstyrre, fejltolkes eller fremstå socialt uaccetabelt. For at få interaktionen med produktet til at fremstå gnidningsløst og magisk opstilles følgende problemstillinger, som der i projektet vil blive forsøgt svaret på.\\

\begin{itemize}
	\item Hvilke af de primære funktioner kan udføres med semaforiske gestikker?
	\item Hvilke semaforiske gestikker kan bruges til de primære funktioner, som det giver mening at have tiknyttet gestikker?
	\item Kan de semaforiske gestikker bruges til perifer interaktion uden at gestikkerne bliver lavet ubevidst?
	\item Er de semaforiske gestikker socialt acceptable?\\
\end{itemize}
For at besvare de forskellige problemstillinger kan flere brugerundersøgelser udføres, herunder kliniske undersøgelser, hvor brugerens holdning til forskellig gestikker kommer til udtryk, og mere virkelighedsnære undersøgelser, hvor gestikkerne bliver testet i en brugssituation.