\section{Udvælgelse af gestikker}
\label{UdvaelgelseAfGestikker}
%
Hat - handler om at finde de gestikker fra litteraturen og dem vi selv vælger at lave test med + formål og problemformulering fra testprotokol
%

\begin{itemize}
  \item opsummering af hvilke gestikker vi ikke vil og gerne vil kigge på.
  \item Forklaring af de forskellige gestikker der kan bruges
  \begin{itemize}
  	\item Gestikker fra litteraturen
  	\item Gestikker overført fra Bang $\&$ Olufsens eksisterende produkter
  	\item Vores egne idéer til gestikker
  \end{itemize}
  \item Fordele og ulemper ved gestikkerne
\end{itemize}

For at vælge hvilke gestikker der skal bruges til at styre et musikanlæg undersøges det først og fremmest, hvilke gestikker der er blevet brugt i andre studier - enten til at styre et musikanlæg eller eksempelvis et TV med lignende funktioner. Derudover blev det undersøgt hvilke gestikker der i forvejen bruges i Bang $\&$ Olufsens produkter og hvorvidt disse kan overføres fra en 2D flade til 3D. Både gestikker fra andre studier og Bang $\&$ Olufsens egne gestikker blev ydermere brugt som inspiration til forslag på andre gestikker. Studerende i de omkringliggende grupperum blev spurgt ind til deres input og der blev designet nogle gestikker af projektgruppen selv. Alle de indsamlede forslag fremgår af \autoref{tab:IndsamledeGestikkerPause}, \autoref{tab:IndsamledeGestikkerSkift} og \autoref{tab:IndsamledeGestikkerVolumen}. Er forslaget kommet fra tidligere studier eller Bang $\&$ Olufsens produkter er der blevet henvist. Er forslaget derimod kommet fra andre studerende eller projektgruppen selv er der ikke skrevet en henvisning. 

Vi har fundet nogle af gestikkerne flere steder. TV-kilden bruger også sweipe ligesom Bang og Olufesen gør. BMW bruger den cirkulære bevægelse med pegefingeren, ligesom Bang og Olufsen gør. 


\begin{table}[H]
	\centering
	\begin{tabular}{| p{6cm} | p{6cm} | }
		\hline
		\textbf{Pause} & \textbf{Start} \\ \hline
		Stationær vertikal hånd, \parencite[s. 166]{PDF:ComparingInputModalities} & Stationær vertikal hånd, \parencite[s. 166]{PDF:ComparingInputModalities} \\ \hline
		Dynamisk vertikal hånd, der starter ved brystet og ender i udstrakt arm  & Dynamisk vertikal hånd, der starter ved brystet og ender i udstrakt arm  \\ \hline
		Kryds i luften, \parencite[s. 48]{PDF:UserDefinedGesturesTV} & Flueben i luften \\ \hline
		Peg og tryk i luften hen mod anlægget & Peg og tryk i luften hen mod anlægget, \parencite[s. 48]{PDF:UserDefinedGesturesTV} \\ \hline
		Krokodillenæb med hånden, der lukker sammen, \parencite[s. 48]{PDF:UserDefinedGesturesTV} & Krokodillenæb med hånden, der åbner op \\ \hline
		Hånden lukker sammen hen mod anlægget & Hånden åbner op hen mod anlægget \\ \hline
		Hånden lukker sammen vertikalt nedad & Hånden åbner op vertikalt opad  \\ \hline
	\end{tabular}
	\caption{Oversigt over forslag til gestikker, der pauser og starter musikken. Hvis forslaget stammer fra tidligere studier eller Bang $\&$ Olufsens produkter er der lavet en henvisning, ellers stammer forslagene fra andre studerende og projektgruppen selv.}
	\label{tab:IndsamledeGestikkerPause}
\end{table}
\noindent

\begin{table}[H]
	\centering
	\begin{tabular}{| p{6cm} | p{6cm} |}
		\hline
		\textbf{Skift musiknummer frem} & \textbf{Skift musiknummer tilbage} \\ \hline
		Swipe bevægelse fra højre mod venstre, \parencite[s. 48]{WEB:Beosound2, WEB:BeosoundMoment, PDF:UserDefinedGesturesTV} & Swipe bevægelse fra venstre mod højre, \parencite[s. 48]{PDF:UserDefinedGesturesTV} \\ \hline
		Swipe hånden mod højre, \parencite[s. 166] {PDF:ComparingInputModalities} & Swipe hånden mod venstre, \parencite[s. 166]{PDF:ComparingInputModalities}  \\ \hline
		Peg tommelfingeren mod højre, \parencite[s. 166]{PDF:ComparingInputModalities} & Peg tommelfingeren mod venstre, \parencite[s. 166]{PDF:ComparingInputModalities} \\ \hline
		Swipe bevægelse fra højre mod venstre med tommel- og lillefinger strakt, mens de andre fingre bøjes & Swipe bevægelse fra venstre mod højre med tommel- og lillefinger strakt, mens de andre fingre bøjes \\ \hline
		Swipe bevægelse fra højre mod venstre med tommel-, pege- og langefinger strakt, mens de andre fingre bøjes & Swipe bevægelse fra venstre mod højre med tommel-, pege- og langefinger strakt, mens de andre fingre bøjes \\ \hline
		Tommel-, pege- og langefinger holdes strakt, de andre fingre bøjes samtidig med hånden køres i en bue fra venstre mod højre og ender med at pege til højre & Tommel-, pege- og langefinger holdes strakt, de andre fingre bøjes samtidig med hånden køres i en bue fra højre mod venstre og ender med at pege til højre\\ \hline
		Peg mod højre & Peg mod venstre \\ \hline
		
		
	\end{tabular}
	\caption{Oversigt over forslag til gestikker, der skifter musiknummer. Hvis forslaget stammer fra tidligere studier eller Bang $\&$ Olufsens produkter er der lavet en henvisning, ellers stammer forslagene fra andre studerende og projektgruppen selv.}
	\label{tab:IndsamledeGestikkerSkift}
\end{table}
\noindent

\begin{table}[H]
	\centering
	\begin{tabular}{| p{6cm} | p{6cm} |}
		\hline
		\textbf{Skru op for musikken} & \textbf{Skrue ned for musikken} \\ \hline
		Pegefingeren køres i en cirkulær bevægelse med uret, (BogO PLAY-kilde) & Pegefingeren køres i en cirkulær bevægelse mod uret, (BogI PLAY-kilde) \\ \hline
		 
		
		
	\end{tabular}
	\caption{Oversigt over forslag til gestikker, der ændrer volumen på musikken. Hvis forslaget stammer fra tidligere studier eller Bang $\&$ Olufsens produkter er der lavet en henvisning, ellers stammer forslagene fra andre studerende og projektgruppen selv.}
	\label{tab:IndsamledeGestikkerVolumen}
\end{table}
\noindent
