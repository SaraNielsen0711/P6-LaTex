\section{Interaktion med gestikker}
\label{TestresultaterSocialAcceptBrug}
%
Følgende analyse og diskusion foretages på baggrund af samtlig testpersoners respons til spørgsmålet: \textit{Kunne du se dig selv bruge gestikker til at styre din musik?}. \blankline
%
Ud af de 12 testpersoner så er det kun gæst 2, som giver udtryk for ikke at ville bruge gestikker i forbindelse med at styre et musikanlæg, hvilket skyldes at gæsten ikke selv benytter sig af et musikanlæg. Dog påpeger gæst 2 at for folk, som normaltvist benytter sig af et musikanlæg, vil det formentlig være nemmere at interagere via gestikker fremfor at skulle have fysisk kontakt med eksempelvis en fjernbetjening. Selvom gæst 4 og gæst 5 begge giver udtryk for, at ville bruge gestikker i forbindelse med at styre et musikanlæg, så pointerer de begge, at det er afhænger af prisen på produktet. I tillæg kommenterer vært 1, at det skal være det værd, at bruge gestikker, i forhold til bare at bruge telefonen eller en fjernbetjening. På nuværende tidspunkt giver gæst 6 udtryk for ikke at ville erhverve sig sådan et produkt, da det ifølge gæsten selv kræver et stort rum, som gæsten ikke har. Dog giver gæst 6 udtryk for, at såfremt at rummet er stort nok, så vil det være lækkert, blandt andet fordi der så ikke er behov for en fjernbetjening. Gæst 3, vært 4 og vært 6 kommenterer, i forhold til interaktion med gestikker, henholdvis at det er fedt, sjovt og at det er fremtiden. Ifølge vært 3 så skal produktet virke fra start og ikke først igennem flere opdateringer, før produktet er i stand til at registrere gestikkerne. Gæst 1 giver udtryk for, at det er træls hele tiden at skulle have fysisk kontakt for at kunne styre musikken. Ydermere kommenterer gæst 1, at det er forstyrrende at skulle sidde med telefonen fremme i forhold til at kunne interagere via gestikker, som ikke forstyrre samtalen.

Vært 5 giver udtryk for godt at ville bruge gestikker, men er bekymret for, om værtens forlovede vil være skeptisk i forhold til brugen af gestikker. Baseret på gæst 5's udsagn tyder det ikke på, at gæsten er skeptisk, da gæsten giver udtryk for gerne at ville bruge gestikker, men at det afhænger af prisen på produktet. Dette kan indikerer, at hvis den ene part i et forhold godt selv kan forestille sig at benytte semaforiske gestikker til at interagere, eksempelvist, med et musikanlæg, men antager at partneren ikke kan, så er der en sandsynlighed for, at førstnævnte vil være tilbageholdende og dermed ikke vil overveje at anskaffe sig sådan et produkt.    

