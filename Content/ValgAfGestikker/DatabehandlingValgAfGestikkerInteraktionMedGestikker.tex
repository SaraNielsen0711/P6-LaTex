\section{Interaktion med gestikker}
\label{TestresultaterInteraktioner}
%
Testpersonerne udfyldte som beskrevet et spørgeskema i starten af testen, hvori der blev blandt andet blev stillet spørgsmål til erfaring med semaforiske gestikker. Hertil svarede 12 testpersoner, at de før havde brugt gestikker, hvor de ikke skulle røre ved en skærm. De testpersoner der svarede, at de havde brugt gestikker før blev efterfølgende spurgt ind til med hvilke produkter, de havde brugt gestikker, hvor de ikke skulle røre ved en skærm. Af de 12 testpersoner havde tre brugt Xbox Kinect, ni havde brugt Nintendo Wii, seks havde brugt smartphones, fem havde brugt tablets og to havde brugt Virtual Reality briller. Tre testpersoner afkrydsede andet-kategorien og skrev at de havde brugt henholdsvis et Myo armbånd eller EyeToy. Ud af de 12 testpersoner følte én sig meget erfaren med semaforiske gestikker, fem følte sig erfarne, fem følte sig uerfarne og én følte sig meget uerfaren. Fra spørgeskemaet blev der også indsamlet data om hvordan testpersonerne betragtede sig selv i forhold til hvor hurtigt de anskaffer sig ny teknologi. Hertil svarede ni, at de betragter som værende lige så hurtigt som alle andre til at anskaffe sig ny teknologi og ni svarede at de betragter sig som værende de sidste til at anskaffe sig ny teknologi. Information som denne analyseres efterfølgende sammen med testpersonernes udtalelse om hvad de synes om interaktion med et interface ved brug af semaforiske gestikker.  \blankline

Efter at have udvalgt semaforiske gestikker til interaktion med musikanlæggets forskellige funktioner fik testpersonerne mulighed for at komme med deres egne holdninger til en interaktionsform som denne. 15 testpersoner gav udtryk for at de godt kunne lide idéen med at interagere ved brug af semaforiske gestikker, hvoraf nogle kommenterede det selvfølgelig var med forbehold for at det virkede hver gang og at systemet reagerede, når vedkommende lavede bevægelsen. Ud af de 15 er der ni testpersoner, der siger at de godt kunne tænke sig at bruge denne slags interaktion. Testperson 3 og 17 siger, at de ikke ville bruge det, hvortil testperson 17 tilføjer at vedkommende i hvert fald først skulle vende sig til det, da det på nuværende tidspunkt er lige så nemt for vedkommende at trykke på en telefon. Testperson 3 føler sig lidt for gammeldags til at bruge denne slags interaktion, men kan godt se det smarte i ikke at skulle rejse sig eller lede efter fjernbetjeningen. Begge testpersoner betragter sig selv som værende de sidste til at anskaffe sig ny teknologi og testperson 17, der har prøvet at interagere med semaforiske gestikker før, føler sig meget uerfaren. Testperson 8, der føler sig erfaren i brug af semaforiske gestikker og betragter sig som den sidste til at anskaffe sig ny teknologi,  nævner også, at vedkommende i hvert fald lige skulle vende sig til at bruge gestikker til interaktion og siger: \textsl{"Jeg har ikke så store armbevægelser i mig selv, så at skulle stå og få et fjernsyn til at reagere, det er sådan lidt.. Jo mindre bevægelser man kan lave det gør det sådan mere tilpas."} Udtalelser fra testpersonerne indikerer, at det at vende sig til at bruge semforiske gestikker hænger sammen med størrelsen af bevælgelsen og hvor meget arbejde der ligger i det i forhold til allerede kendte interaktionsformer. 

Testpersonerne udtrykker, udover en positiv indstilling, lidt bekymringer, der skal tages med i overvejelserne, når semaforiske gestikulering skal udvikles som interaktionsform. Testperson 1 nævner, at der ikke skal være for mange funktioner og i samme stil nævner testperson 17 og 18 at det ville være en fordel yderligere at kunne interagere med musikanlæggets skærm ved at trykke på den. Da det fra start har været målet kun at udvælge semaforiske gestikker til de mest gængse funktioner på et musikanlæg, vil det give mening at have en tilknyttet applikation, hvori de mest gængse funktioner samt de resterende funktioner vil befinde sig. Testperson 14 er bekymret for om musikanlægget vil opfange vedkommendes bevægelser og forstå dem som kommandoer uden at dette er intentionen, da vedkommende opfatter sig selv som meget gestikulerende i dagligdagen. Dette udtrykker andre testpersoner også bekymring om, når de skal tage stilling til at systemet altid vil kunne opfange dem, hvilket ved blive uddybet senere. 

Selvom testen er udført på 18 testpersoner, hvoraf ingen af dem betragter sig som værende de hurtigste til at anskaffe sig ny teknologi, er testpersonerne positivt stillet overfor en interaktionsform som denne, der alligevel ikke er så udbredt. Det er størstedelen af testpersonerne, der sagtens kan forestille sig at interagere med et musikanlæg ved brug af semaforiske gestikker og dem der ikke kan se sig selv gøre det, kan alligevel godt fordelen i ikke at skulle lede efter sin fjernbetjening eller røre ved en skærm med fedtede fingre. I alt er der ni testpersoner, der selv nævner fordelen ved enten ikke at skulle rejse sig, lede efter fjernbetjeningen, røre ved en skærm med fedtede fingre eller sidde og kigge på sin telefon, hver gang man vil interagere med musikanlægget.

Med udgangspunkt i at interaktion ved brug af semaforiske gestikker til at styre et musikanlæg vil falde i god jord er det relevant at undersøge, hvordan testpersonerne egentlig har det med altid at blive opfanget af musikanlægget.
