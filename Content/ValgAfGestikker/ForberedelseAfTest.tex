\section{Testforberedelse}
\label{Testforberedelse}
%
Hat
%
\begin{itemize}
  \item Materialer
  \item Formål
  \item Testpersoner - hvem og hvorfor
  \item Instruktioner
  \item Hvordan videoer er filmet
  \item Præsentationsrækkefølge 
  \item Deres opgaver
  \item Fremgangsmåde for testen 
  \item Pilottest (playliste, nyt spørgsmål, observatørens position)
\end{itemize}
%

\subsection{Testpersoner}
\label{TestpersonerValgAfGestikker}
%
Testen udføres på studerende fra Aalborg Universitet


Vi har valgt studerende, der ikke læser EIT og PDP. Dette er gjort på baggrund af pilotforsøg, hvor forsøgspersonerne snakkede om teknologien bag – og hvordan gestikkerne skulle designes, så teknologien kunne forstå dem. Det er vigtigt for os, at brugeren ikke nødvendigvis tænker på teknologien bag, men lige så meget hvordan det bliver den bedste oplevelse at bevæge hænderne og styre musikken. Side 10 i OUE bogen beskriver hvordan man ikke skal medtage folk der ved for meget om designprocesser og brugertest. 
Det kunne være en idé kun at rekruttere folk der allerede havde eller sværger til BogO og andre fine musikanlæg – men der er ikke mange studerende der har råd til det endnu, selvom det kunne tænkes at være universitetsstuderende der lige netop ville ende med at have penge nok til at købe BogO-produkter. Samtidig er konceptet ikke udviklet endnu, og det tager højst sandsynligt nogle år før det er færdigudviklet – den samme tid det tager det kan tage de universitetsstuderende der testes på at færdiggøre deres uddannelse, få et job og anskaffe sig et lækkert musikanlæg. 
Det kan være et problem kun at teste på brugere, der allerede har et BogO anlæg, da der ikke nødvendigvis vil komme nye input ind på den måde. Ved at bruge en vifte af forskellige mennesker fås der nye input på, hvordan det kan være smart at bruge gestikker og hvordan bevægelserne skal være for netop at fange de mindre interesseredes interesse (OUE bogen, starten af kapitel 6). 
