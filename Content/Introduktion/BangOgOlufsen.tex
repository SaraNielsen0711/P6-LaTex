\chapter{Perifer interaktion til musikkontrol}
\label{PeriferInteraktionTilMusikKontrol}
%
Skriv hat til hele kapitlet

\section{Sammenspil med Bang $\&$ Olufsen}
\label{SammenspilMedBO}
%
I vores højteknologiske verden er det vigtigt konstant at være i udvikling for at kunne følge med teknologien. Det gælder både om at vide, hvilke nye produkter brugeren ønsker, men også hvilken form for interaktion, der skal være med produktet. Specielt interaktions-delen er interessant i samarbejde med Bang $\&$ Olufsen, hvor implementering af nye interaktionsformer, herunder perifer interaktion, kan være med til at forbedre brugeroplevelsen. Perifer interaktion er nemlig en interaktionsform, der i større grad vil blive brugt i fremtiden, for at give et mere gnidningsløst sammenspil mellem mennesker og produkter, \parencite[s. 1]{PDF:PIIntroduction}. I samarbejde med Bang $\&$ Olufsen er det interessant at undersøge, hvordan perifer interaktion og tilhørende gestikker kan bruges til at styre et musikanlæg ophængt på væggen, så det ikke er nødvendigt at rejse sig fra sofaen eller afbryde en samtale, når eksempelvis lydstyrken skal ændres.

Selvom perifer interaktion med elektronik er uundgåeligt, jævnfør \textcite[s. 1]{PDF:PIIntroduction}, er det ikke nødvendigvis alle funktioner, der kan styres perifert. Ved interview med Lyle Clarke hos Bang $\&$ Olufsen findes der frem til både primære og sekundære funktioner, som Bang $\&$ Olufsens produkter skal indeholde. De primære funktioner kan ses i \autoref{tab:BogOsPrimaereFunktioner} og de sekundære funktioner i \autoref{tab:BogOsSekundaereFunktioner}.

%
\begin{table}[H]
	\centering
	\begin{tabular}{ | l | p{8cm} |}
		\hline
		\multicolumn{1}{|l|}{\textbf{Primære funktioner}} & \multicolumn{1}{l|}{\textbf{Formål}} \\ \hline
		Start & Start musikken \\ \hline
		Stop & Stop Musikken \\ \hline
		Standby & Musik sat på pause \\ \hline
		Forward/backward in time & Skift sang frem eller tilbage \\ \hline
		Intensity up/down & Justering af intensiteten op og ned (lyd, lys osv.) \\ \hline
		Like & Sæt en sang til farvoritter \\ \hline
		Dismiss & Fjern denne og lignende sange fra playlister \\ \hline
		Shift source & Skift musikkilde (radio, CD, AUX osv.) \\ \hline
		Shift experiences & Skift mellem hvilken højtaler der skal spilles sammen med (skift mellem zoner) \\ \hline
		Initiate interaction & Start interaktion med produktet \\ \hline
		Confirm & Bekræft interaktion med produktet \\ \hline
	\end{tabular}
	\caption{Primære funktioner i Bang $\&$ Olufsens produkter.}
	\label{tab:BogOsPrimaereFunktioner}
\end{table}
\noindent
%

%
\begin{table}[H]
	\centering
	\begin{tabular}{ | l | p{8cm} |}
		\hline
		\multicolumn{1}{|l|}{\textbf{Sekundære funktioner}} & \multicolumn{1}{l|}{\textbf{Formål}} \\ \hline
		Fastforward/fastbackward & Spol frem eller tilbage i musikken \\ \hline
		Stay in mood & Bliv i pågældende genre/stemning \\ \hline
		Send/expand & Flyt den afspillede musik til en anden højtaler \\ \hline
		Store & Gem sang/kunstner/playliste \\ \hline
		Add to.. & Tilføj til.. (eksempelvis playliste) \\ \hline
		Play next & Sæt i kø som det næste nummer \\ \hline
		Play later & Sæt i kø til sidst \\ \hline
	\end{tabular}
	\caption{Sekundære funktioner i Bang $\&$ Olufsens produkter.}
	\label{tab:BogOsSekundaereFunktioner}
\end{table}
\noindent
%

Af disse funktioner er det ikke alle, der giver mening have en tilknyttet gestik, da interaktionen ikke nødvendigvis behøver at være perifer for alle funktioner. Eksempelvis giver det god mening at skulle bruge den centrale opmærksomhed, når der skal spoles i et musikstykke eller en bestemt sang skal tilføjes en bestemt playliste. Det er hovedsaligt de sekundære funktioner, men også flere af de primære funktioner, der tænkes at kunne styres på selve produktet eller på en tilhørende applikation. På den måde bliver det kun de vigtigste funktioner, der kan styres perifert med gestikker. Ud fra interviewet med Lyle Clarke ligges det fast, at de vigtigste funktioner, hvortil der ønskes en tilknyttet gestikker, er start, pause, intensity (i det her tilfælde volumen) up/down og forward/backward in time. De overskydende funktioner skal naturligvis være til stede, men egner sig højst sandsynligt bedst til direkte fysisk interaktion med musikanlægget eller ved brug af en tilhørende applikation.

For at få en idé om hvordan perifer interaktion kan bruges vil der i de følgende afsnit blive beskrevet teoretisk hvad perifer interaktion er og hvordan gestikker kan bruges til at styre apparater perifert. 

