\subsection{Problemformulering}
\label{Problemformulering}
%
Da virksomheder konstant skal forny sig og samtidig skal finde nye interaktionsformer for de nye produkter på og at det tilmed tyder på, at perifer interaktion i fremtiden vil finde et større indpas i den måde, hvorpå vi interagere med vores elektroniske apparater, så virker det yderst relevant, for virksomheder som Bang $\&$ Olufsen at udnytte det store potentiale, der er ved perifer interaktion, jævnfør \fullref{Potentiale}. På baggrund af de afgrænsinger, der er foretaget i \fullref{Afgraensning} er det nu muligt at specificere hvilken retning projektet vil tage.  

I projektet er der fokus på, at undersøge hvilke semaforiske gestikker, der skal afspejle de seks mest gængse funktioner et musikanlæg indeholder; start, pause, skru op og ned og skift musiknummer frem og tilbage. Derudover vil det blive undersøgt om interaktionen med Bang $\&$ Olufsens fremtidige musikanlæg kan foregå i den perifere opmærksomhed. Ydermere vil det undersøges hvorvidt de valgte semaforiske gestikker forårsager social accept eller ej. Dette leder op til følgende problemstillinger:\blankline
%
\begin{quotation}
\noindent
\textit{Hvilke specifikke semaforiske gestikker skal knyttes til hver af de seks mest gængse funktioner i Bang $\&$ Olufsens fremtidige musikanlæg, for at interaktionen bliver perifer?\blankline
%
Hvordan vil disse semaforiske gestikke påvirke den sociale accept?}\blankline
\end{quotation}
%
For at besvare den første problemstilling er det nødvendigt først, at foretage en undersøgelse af hvilke semaforiske gestikker eventuelle brugere tilknytter hver af de seks førnævnte funktioner, for derefter at undersøge den perifere interaktion. Når den perifere interaktion undersøges, vil feedback stadig blive taget in mente og såfremt det tyder på, at feedback vil være nødvendig for at fremme den perifere interaktion så vil dette blive undersøgt og diskuteret nærmere. Som nævnt tidligere bliver en interaktion først perifer, når den er blevet en rutine og derfor kun kræver få mentale ressourcer, jævnfør \fullref{PeriferInteratkion}. Der skal derfor tages højde for at de testpersoner, som må deltage i de nødvendige undersøgelser, ikke nødvendigvis oplever interaktionen med musikanlægget som værende i den perifere opmærksomhed.    

Da det tyder på, at brugere generelt bliver påvirket af den sociale kontekst både i forhold til lokation og tilskuere, jævnfør \fullref{Socialaccept}, så vil det være fordelagtigt at undersøge perifer interaktion i hvertfald to scenarier; når brugeren er alene og når brugeren er sammen med andre. På baggrund af den undersøgelse bør det være muligt, at besvare den sidst nævnte problemstilling og tilmed vurdere om de semaforiske gestikker opfattes, som værende socialt acceptable eller ej. 



