\chapter{Diskussion}
\label{SamletDiskussion}
%

Hat - opsummering af BogO og hvad de gerne vil have i forhold til perifer interaktion og hvad der kommer til at blive diskuteret i efterfølgende
%
\section{Valg af funktioner til perifer interaktion}
\label{DiskussionFunktionerCasualInteraction}
%
Valg af funktioner til perifer interaktion sammen med Casual interaction og figuren fra intro 
Der er kun nogle funktioner, der egner sig til perifer interaktion/interaktion med semaforiske gestikker, som TP1 i test 1 også siger. F.eks. det der med at man aldrig vil tænde hele sit anlæg perifert alligevel 
Casual interaction - det virker til at testpersonerne, når de ikke får andre muligheder, sagtens kan opgive den fulde kontrol over musikanlæget. Bliver de spurgt, er der der dog flere der kommer ind på (manglende?) kontrol ved justering af lydstyrke. (Diskuter sammen med funktioner)\blankline
%
Calm technology, sammen med delt opmærksomhed, kan værten holde gang i samtalen samtidig med at føre en samtale med gæsten - kan værterne finde tilbage ind i samtalen med gæsten - burde vi have spurgt værterne om hvor nemt de syntes det var at falde ind i samtalen efter et hint. (Kan formentlig diskuteres flere steder) \blankline
%
\section{De udvalgte gestikker}
\label{DiskussionUdvalgteGestikker}
%
Diskuter de udvalgte gestikker, der var andre muligheder til at skrue op og ned, i forhold til bevægelsesmængde og afstand til kroppen, midas touch problem, maskering af gestikker, retningsbestemt fremfor aktiveringsgestik (perifert, socialt accept)\blankline
%
Midas touch problem. Der er allerede under komplikationer ved gestik åbnet op for at gestikkerne skal rettes mod anlægget. Så skygger man heller ikke for gestikken.\blankline
%
Retning på swipe\blankline
%
Standardisering af gestikker
%
\section{Indflydelse på social accept}
\label{DiskussionSocialAccept}
%
Social accept, venstre vs. højre, fra både vært og gæst, hvilken gestik det er - hemmelighedsfyldte og magiske mm., generelt til venstrehåndede f.eks. i forhold til swipe, varighed i forhold til selve gestikken (den er kort), tid i forhold til teknologi
 Er der et socialt problem, hvis de skal rette interaktionen mod venstre? Teorien siger de har det bedst med at rette den mod højre
 Synlige, usylinge; Hemmelighedsfyldte, udtryksfyldte, 
	magiske og spændingsfyldte gestikker
	Det med om gestikkerne er magiske, hvis interaktionen er perifer og diskret. 
%
\section{Metodevalg}
\label{DiskussionMetodevalg}
%
Metode diskusion af begge tests, wizard of oz, kigger de efter anlægget fordi hintet kommer der fra - ændringen i lyden tiltrækker deres opmærksomhed. 
Diskuter metoder for begge forsøg. Test 1: Videoer og præsentation af gestikker (hvis det overhoved er noget ved). Test 2: Kigger testpersonerne mere imod anlægget, fordi hintet kommer herfra? Kan holdes op med noget teori omkring opmærksomhed, måske?
%
\section{Feedback}
\label{DiskussionFeedback}
%
Feedback
AChairAsUbiquitours... understreger vigtigtheden af et godt genkendelsessystem, da testpersonerne ikke får andet end funktional feedback fra musikken - manglen på feedback under tests - hvd kan man foreslå af feeback - generelt teori vi har om feedback
% 
\section{Perifer interaktion}
\label{DiskussionPeriferInteraktion}
%
Er det overhovedet perifer interaktion
%



\begin{itemize}
  \item Har i et bud på hvorfor der er et gab, mellem fokuseret og automatisk interaktion (perifer) Lars Bos kommentar!
  \item Gamle telefoner med knapper var perifert
\end{itemize}
%
%Taget fra afgrænsningen men passer bedre i diskussionen 
Selvom der i \fullref{RelateretUndersoegelser} blev præsenteret nogle forskellige undersøgelser, som alle vedrører perifer interaktion med en musikafspiller, kan disse kun bruges som inspiration. Det skyldes blandt andet at interaktionen var med en musikafspiller, som eksempelvist iTunes, og ikke et decideret musikanlæg. Derudover var brugeren positioneret ved et skrivebord med en computer, hvor musikafspilleren befinder sig, og hvor interaktionen foregik i området tæt omkring computeren. Grunden til at undersøgelserne kun kan anvendes som inspiration er, at interaktionen, som undersøges i dette projekt, er med et musikanlæg og at selve brugssituationen kan være anderledes i og med, at brugeren ikke nødvendigvis behøver, at sidde ved sit skrivebord foran sin computer, men frit kan bevæge sig rundt i rummet. Der kan drages inspiration fra undersøgelserne i forhold til valg af semaforiske gestikker og hvordan disse kan testes.\blankline
%



