\section{Udvælgelse af gestikker}
\label{UdvaelgelseAfGestikker}
%
Hat - handler om at finde de gestikker fra litteraturen og dem vi selv vælger at lave test med + formål og problemformulering fra testprotokol
%

\begin{itemize}
  \item opsummering af hvilke gestikker vi ikke vil og gerne vil kigge på.
  \item Forklaring af de forskellige gestikker der kan bruges
  \begin{itemize}
  	\item Gestikker fra litteraturen
  	\item Gestikker overført fra Bang $\&$ Olufsens eksisterende produkter
  	\item Vores egne idéer til gestikker
  \end{itemize}
  \item Fordele og ulemper ved gestikkerne
\end{itemize}

For at vælge hvilke gestikker der skal bruges til at styre et musikanlæg undersøges det først og fremmest, hvilke gestikker der er blevet brugt i andre studier - enten til at styre et musikanlæg eller eksempelvis et TV med lignende funktioner. Derudover blev det undersøgt hvilke gestikker der i forvejen bruges i Bang $\&$ Olufsens produkter og hvorvidt disse kan overføres fra en 2D flade til 3D. Ydermere blev studerende i de omkringliggende grupperum spurgt ind til deres input og der blev designet nogle gestikker af projektgruppen selv. Alle de indsamlede forslag fremgår af \autoref{tab:IndsamledeGestikkerPause}, \autoref{tab:IndsamledeGestikkerSkift} og \autoref{tab:IndsamledeGestikkerVolumen}, hvor forslag skrevet i fed skrift er dem der er valgt at arbejde videre med under testen.




\begin{table}[H]
	\centering
	\begin{tabular}{| p{4cm} | p{4cm} | p{4cm} |}
		\hline
		\textbf{Pause} & \textbf{Start} & \textbf{Kilde} \\ \hline
		Stationært stop tegn & Stationært stop tegn & \parencite{PDF:ComparingInputModalities} \\ \hline
		Dynamisk stop tegn, der starter ved brystet og ender i udstrakt arm  & Dynamisk stop tegn, der starter ved brystet og ender i udstrakt arm & (kilde) \\ \hline
		Kryds i luften & Flueben i luften & (kilde) \\ \hline
		Hånden lukker sammen hen mod anlægget & Hånden åbner op hen mod anlægget & (kilde)\\ \hline
		Hånden lukker sammen vertikalt nedad & Hånden åbner op vertikalt opad & (kilde) \\ \hline
		Peg og tryk i luften hen mod anlægget & Peg og tryk i luften hen mod anlægget & (kilde) \\ \hline
		Krokodillenæb med hånden, der lukker sammen & Krokodillenæb med hånden, der åbner op & (kilde) \\ \hline
	\end{tabular}
	\caption{Oversigt over præsentationsrækkefølgen for de tre videoer; start og pause, skru op og ned og skift musiknummer.}
	\label{tab:IndsamledeGestikkerPause}
\end{table}
\noindent




%Sara prøver

\begin{table}[H]
\begin{center}
    \begin{tabular}{ | p{4cm} | p{4cm} | p{4cm} |}
    \hline
    \textbf{Pause} & \textbf{Start} & \textbf{Kilde} \\ \hline
    11C & 22C & A clear day with lots of sunshine.  
    However, the strong breeze will bring down the temperatures. \\ \hline
    9C & 19C & Cloudy with rain, across many northern regions. Clear spells 
    across most of Scotland and Northern Ireland, 
    but rain reaching the far northwest. \\ \hline
    10C & 21C & Rain will still linger for the morning. 
    Conditions will improve by early afternoon and continue 
    throughout the evening. \\ \hline
    \end{tabular}
     \caption {Should be a caption}
     \label{tab:}
\end{center}
\end{table}


