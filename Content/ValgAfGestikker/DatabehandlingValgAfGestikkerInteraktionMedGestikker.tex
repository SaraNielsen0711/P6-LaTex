\section{Interaktion med gestikker}
\label{TestresultaterInteraktioner}
%
I starten af testen blev testpersonerne bedt om at udfylde et spørgeskema, hvori der, blandt andet, blev stillet spørgsmål til deres erfaring med semaforiske gestikker. Hertil responderede 12 testpersoner, at de før har brugt gestikker, hvor de ikke skulle røre ved en skærm. De testpersoner, der responderede at de har brugt gestikker før, blev efterfølgende spurgt ind til med hvilke produkter, de havde brugt semaforiske gestikker. Af de 12 testpersoner, som har tidligere erfaringer med semaforiske gestikker, har tre brugt Xbox Kinect, ni har brugt Nintendo Wii, seks har brugt smartphones, fem har brugt tablets og to har brugt Virtual Reality briller. Tre testpersoner afkrydsede andet-kategorien og angav at de har brugt henholdsvis et Myo armbånd eller EyeToy, som er et PlayStation 2 spil. Ud af de 12 testpersoner følte én sig meget erfaren med semaforiske gestikker, fem følte sig erfarne, fem følte sig uerfarne og én følte sig meget uerfaren. Fra spørgeskemaet blev der ydermere indsamlet data om, hvordan testpersonerne betragter sig selv i forhold til hvor hurtigt de anskaffer sig den nyeste teknologi. Hertil responderede ni, at de betragter sig som værende lige så hurtig som alle andre til at anskaffe sig den nyeste teknologi og ni responderede at de betragter sig som værende de sidste til at anskaffe sig den nyeste teknologi. Information som denne analyseres efterfølgende i forbindelse med testpersonernes udtalelser om hvad de synes om, at interagere med en brugergrænseflade ved brug af semaforiske gestikker. \blankline
%
Efter at have udvalgt hvilke semaforiske gestikker, der skal knyttes til henholdvis pause og start, skift musiknummer samt skru op og ned, blev testpersonerne, i exit-interviewet, spurgt indtil hvordan de forholder sig til denne interaktionsform. 14 testpersoner giver udtryk for, at de godt kan lide idéen om at interagere via semaforiske gestikker, hvoraf nogle kommenterede, at det selvfølgelig er med forbehold for, at det virker hver gang og at systemet reagerer på ens bevægelser. Ud af de 14 testpersoner giver ni testpersoner udtryk for, at de godt forestille sig at bruge denne slags interaktion. 

Testperson 3 og testperson 17 giver begge udtryk for, at de ikke vil bruge semaforiske gestikker, hvortil testperson 17 pointerer at det på nuværende tidspunkt ikke er aktuelt, da det er lige så nemt at trykke på en telefon. Dog afviser testpersonen ikke, at det på sigt kan blive aktuelt. Testperson 3 føler sig lidt for gammeldags til at bruge den slags interaktion, men kan godt se det smarte i ikke at skulle rejse sig eller lede efter fjernbetjeningen. Begge testpersoner betragter sig selv som værende de sidste til at anskaffe sig den nyeste teknologi og testperson 17, der har prøvet at interagere med semaforiske gestikker før, føler sig meget uerfaren. Testperson 15 giver udtryk for ikke at have et behov for at interaktionen skal foregå på det niveau, hvor niveau referer til brugen af semaforiske gestikker. I tillæg kommenterer testperson 15, at det ikke er et produkt testpersonen vil købe. Testperson 15's synspunkter stemmer overens med hvad testpersonen responderer i spørgeskemaet, hvor testepersonen betragter sig som værende den sidste til at anskaffe sig den nyeste teknologi og føler sig derudover uerfaren i brugen af semaforiske gestikker. Testperson 8, der føler sig erfaren i brugen af semaforiske gestikker samt betragter sig som værende den sidste til at anskaffe sig den nyeste teknologi, giver ligeledes udtryk for, at have brug for en tilvænningsperiode, hvis interaktionen skal foregå med semaforiske gestikker.

Sammenholdes resultaterne fra spørgeskemaet med testpersonernes udsagn i exit-interviewet, så fremgår det at de testpersoner, der er skeptiske overfor interaktionsformen også er de testpersoner, som har angivet at de er de sidste til at anskaffe sig den nyeste teknologi. Dette er ikke nødvendigvis tilfældet for alle testpersoner, som betragter sig selv som værende de sidste til at anskaffe sig den nyeste teknologi, da fem af disse testpersoner giver udtryk for, at være åbne for at interagere med et musikanlæg via semaforiske gestikker.

Ud af de ni testpersoner, der godt kan forestille sig at bruge interaktionsformen, anser tre af dem sig som værende de sidste til at anskaffe sig den nyeste teknologi. Derudover har otte ud af ni testpersoner tidligere erfaringer med semaforiske gestikker, hvoraf én føler sig meget erfaren, fire føler sig erfarne og tre føler sig uerfaren. Selvom ingen af de 18 testpersoner betragter sig som værende de første til at anskaffe sig den nyeste teknologi, så er størstedelen af testpersonerne åbne omkring denne interaktionsform. I tillæg giver ni testpersoner udtryk for, at de godt kan forestille sig at eje et produkt, hvor interaktionen foregår via semaforiske gestikker. Det tyder derfor på, at der er en spirrende interesse for at erhverve sig denne type produkter, selvom brugeren ikke nødvendigvis betragter sig selv som værende den første til at anskaffe sig den nyeste teknologi. Selv de testpersoner, som giver udtryk for ikke at bryde sig om, at skulle interagere via semaforiske gestikker, giver alligevel udtryk for, at det er en fordel ikke at skulle lede efter sin fjernbetjening eller røre ved en skærm med fedtede fingre. Halvdelen af testpersonerne giver udtryk for, at det er en fordel at de hverken skal rejse sig, lede efter fjernbetjeningen, røre ved en skærm med fedtede fingre eller rette fokus mod en telefon for at interagere med et musikanlæg.

Foruden den positive indstilling rettet mod de semaforiske gestikker, så giver testpersonerne udtryk for nogle bekymringer, der bør tages til eftertragtning for, at fremme brugen af semaforiske gestikker som interaktionsform. Ifølge testperson 1 skal der ikke være for mange funktioner, der er tilknyttet semaforiske gestikker. Testperson 17 og testperson 18 efterspørger begge, at der stadig er mulighed for at interagere med produktet, eksempelvist ved at trykke på produktet. Da det fra starten af blev besluttet kun at udvælge semaforiske gestikker til de seks mest gængse funktioner på et musikanlæg, vil det være en fordel at have en tilknyttet applikation, som både indeholder de seks valgte funktioner; pause, start, næste musiknummer, forrige musiknummer, skru op og skru ned samt de resterende funktioner Bang $\&$ Olufsen gør brug af, jævnfør \fullref{SamspilMedBO}.\blankline
% 
Testperson 2 kommer med en vigtig pointe i forhold til, at gøre brug af semaforiske gestikker, som interaktionsmulighed: 
%
\begin{quotation}
\textit{Hvis det fungerer lige så godt som at trykke på en knap, så ja, så kunne jeg godt finde på at bruge det.} Testperson 2, \autoref{app:NoterValgAfGestikker}. 
\end{quotation}
%
Det er en vigtig pointe fordi det er nødvendigt, at de semaforiske gestikker enten fungerer lige så godt eller bedre end hvordan interaktionen foregår nu. Hvis det ikke er tilfældet kan brugeren lige så godt bruge sin telefon eller sin fjernbetjeningen til at styre musikanlægget, hvilket potentielt kan resulterer i at brugeren ikke længere bruger produktet og måske føler at det har været et fejlkøb og i værste fald fraråder andre at købe det. Det kan ikke understreges nok, hvor vigtigt det er, at produktet fungerer hver gang og ikke skaber frustration eller tvivl hos brugeren. \blankline
%
Med udgangspunkt i og en forventning om at interaktion ved brug af semaforiske gestikker til at styre de seks mest gængse funktioner i et musikanlæg, falder i god jord er det relevant at undersøge, hvordan testpersonerne egentlig vil have det med, at produkter, som skal reagere på semaforiske gestikker også bliver nødt til altid at kunne registrere gestikkerne. 

  


