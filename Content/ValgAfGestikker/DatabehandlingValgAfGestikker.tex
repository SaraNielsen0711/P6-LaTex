\chapter{Testresultater}
\label{Testresultater}
%
Igennem de følgende afsnit vil testresultaterne blive analyseret og diskuteret. I de tre første sektioner vil fokus være på at udvælge hvilke semaforiske gestikker, der skal knyttes til henholdvis at pause og starte musikken, skifte musiknummer og til at skrue op og ned. Derefter belyses testpersonernes holdning til hvordan det er at interagere med gestikker samt deres holdning til hvordan gestikker registreres af et produkt.\blankline
%
Der er i alt indsamlet data fra 18 testpersoner; 7 kvinder og 11 mænd, hvoraf én af de mandlige testpersoner er venstrehåndet. Testpersonernes alder spænder fra 22 år til 32 år med en gennemsnitalder på 24 år og ud af de 18 testpersoner er der én, som ikke er studerende på Aalborg Universitet. Ingen af de 18 testpersoner angiver at de har et eller flere Bang $\&$ Olufsen produkter. Data fra spørgeskemaet er vedlagt i \autoref{app:RaaDataSpoergeskema}, testpersonernes top tre rangering fremgår af \autoref{app:TestpersonernesTopTre}, videoptagelserne er vedlagt i HENVISNING TIL ELEKTRONISK BILAG og observatørens notater er vedlagt i \autoref{app:NoterValgAfGestikker}.     