\section{Indflydelse på social accept}
\label{DiskussionSocialAccept}
%
For at besvare anden del af problemformuleringen; \textit{Hvordan vil disse semaforiske gestikker påvirke den sociale accept?}, blev testen i den sociale kontekst udført. Ud fra observationerne samt testpersonernes respons konkluderes det, at de valgte semaforiske gestikker ikke påvirker den sociale accept negativt, hverken fra værten eller gæstens side. 
Teorien beskriver, hvordan gestikker kan inddeles i fire kategorier, hvor gestikkerne, udført i den sociale kontekst, kan diskuteres at være enten udtryksfyldte eller magiske, \parencite[s. 276]{PDF:WouldYouDoThat}. Umiddelbart anses de udvalgte gestikker som værende udtryksfyldte, da både manipulationen og effekten er synlig; gestikkerne udføres, så de kan ses af begge parter og musikken ændres, hvilket begge parter kan høre. Det ses dog ud fra observationerne, at det ikke er alle gæster, som lægger mærke til at gestikkerne udføres, hvorfor gestikken kan anses som værende magisk, da effekten stadig vil være hørbar for begge parter. Gestikkerne er designet ud fra et ønske om, at interaktionen med tiden vil blive perifer. Bliver interaktionen faktisk perifer, er der belæg for at brugere og gæster vil lægge endnu mindre mærke til de udførte gestikker og derfor opleve interaktionen magisk, da musikken ændres efter ønske og det sker uden at koncentrationen skal rettes mod kommandoen. Da begge former for gestikker er socialt acceptable ifølge \textcite[s. 277]{PDF:WouldYouDoThat}, bør der ikke være et problem, om gæsterne lægger mærke til gestikken eller ej, men da Bang $\&$ Olufsen generelt har et ønske om at skabe magiske produkter, vil det være farvorabelt at interaktionsformen også findes magisk.\blankline
%
Udover hvilken type af gestikker der bruges, kan der være flere ting der spiller ind, før gestikker er socialt acceptable. Vært 3 gav i forbindelse med spørgsmål om social accept udtryk for, at venner og bekendte godt kan kigge lidt mærkeligt, hvis vedkommende var den eneste der havde sådan et produkt, og så begyndte at interagere med gestikker hjemme i stuen. Ifølge \textcite[s. 276]{PDF:WouldYouDoThat} kan den sociale accept øges, når et produkt eller en teknologi har været på markedet et stykke tid, hvilket henvender sig godt til, hvad vært 3 giver udtryk for. Ydermere kan problematikker inden for den sociale accept komme til udtryk, hvis brugeren ikke ønsker at bruge gestikker som interaktionsform, \parencite[s. 276]{PDF:WouldYouDoThat}, hvis gestikkerne tager mere end seks sekunder at udføre, hvis de er for store eller hvis der er for langt hen til det elektroniske apparat, \parencite[s. 199]{PDF:AreYouComfortableDoingThat}. Da testpersonerne generelt virker positivt stillede overfor interaktionen med gestikker, gestikkerne ikke tager mere end seks sekunder at udføre og ikke er for store er dette med til at støtte den sociale accept af interaktionsformen. Hvor stor afstanden mellem bruger og musikanlæg kommer til at være, er ikke muligt at vurdere, da det kan afhænge af brugernes forskellige hjem. Ud fra testen i den sociale kontekst, og med en understøttende kommentar fra gæst 6, der godt kunne tænke sig at bruge interaktionsformen i en større stue, konkluderes det, at det interaktionsformen stadig vil være social acceptabel ved større afstande.  

Under testen i den sociale kontekst blev interaktionen rettet enten direkte mod anlægget eller til højre for anlægget. Det skyldes dels, at alle værterne var højrehåndet og dels, at sofaen stod sådan, at værterne havde deres højre side mod anlægget. Ifølge \textcite[s. 197]{PDF:AreYouComfortableDoingThat} føler testpersoner sig mest komfortable ved at udføre gestikken lige over eller til højre for apparatet. Da ingen af testpersonerne har rettet gestikkerne til venstre for musikanlægget, er det ikke muligt at sige noget om den sociale accept, hvis interaktionen foregår fra en anden vinkel, men dette lægger op til at teste en virkende prototype i et felteksperiment, så interaktionen fra alle vinkler kan vurderes. \blankline 
