\subsection{Mikrogestikker}
\label{Mikrogestikker}

Indenfor gestikker findes der både mikro- og makrogestikker, \parencite[s. 6]{PDF:UsabilityofMicroVsMacroGestures}. Mikrogestikker er små bevægelser, der ikke nødvendigvis kræver at hele hånden bevæger sig og er specielt lovende indenfor interaktion i den perifere del af opmærksomheden, \parencite[s. 95]{PDF:PeripheralInteraction}. Mikrogestikker er derudover gestikker, der er hurtige at udføre, hvilket gør dem til en god kandidat for perifer interaktion, \parencite[s. 96]{PDF:PeripheralInteraction}. Makrogestikker er derimod store bevægelser, der kræver signifikant bevægelse som eksempelvis at løfte en arm eller rejse sig fra en siddende position, \parencite[s. 6]{PDF:UsabilityofMicroVsMacroGestures}. Ifølge \textcite[s. 9]{PDF:UsabilityofMicroVsMacroGestures} kan makrogestikker udføres meget mindre præcist end mikrogestikker, da en løftet arm tolkes som en løftet arm, om den er løftet 60$^{\circ}$ eller 90$^{\circ}$. En af ulemperne ved makrogestikker er dog, at der skal bruges et stort område til at udføre de pågældende gestikker og at en sensor der opfanger makrogestikker ofte vil dække et helt lokale. Det vil derfor være svært at træde ud af interaktionsområdet, hvilket betyder brugeren fysisk skal flytte sig langt væk for at undgå interaktion. Mikrogestikker behøver modsat makrogestikker ikke et stort interaktionsområde, men skal til gengæld udføres meget mere præcist \parencite[s. 10]{PDF:UsabilityofMicroVsMacroGestures}.Fordelen ved mirkogestikker kan være, at interaktionen med systemet ikke sker ved en fejl, da det er mindre sandsynligt at præcise bevægelser bliver udført ved en fejl. Ifølge \textcite[s. 10]{PDF:UsabilityofMicroVsMacroGestures} er mikrogestikker mest passende til professionel brug, blandt andet fordi det ikke altid er ønskværdigt at skulle udføre store bevægelser for at interagere med et produkt.

Ved brug af gestikker til perifer interaktion .... noget om at mikrogestikker er gode, men de ikke skal være lige så præcise som de beskrevne mikrogestikker - og at det måske er en blanding af de to, så vi får mindre, upræcise gestikker frem for enten eller.
