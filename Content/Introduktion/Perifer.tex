\chapter{Perifer interaktion}
\label{PeriferInteratkion}
%
\begin{itemize}
  \item Perifer interaktion - de to yderpunkter
  \item Opmærksomhed (perifer opmærksomhed)
  \item Inddrag figur fra projektforslag (side 118 i bogen eller side 6, der er den røde cirkel der ikke)
  \item Hvorfor er det interesant at designe produkter, som kan reagere på perifer interaktion? (fordele og ulemper) 
  \item Inddrag at det er relativt nyt og ukendt at have produkter, som man kan interagere med perifert 
  \item Microgestures 
  \item Socialt acceptabelt 
  \item Hvordan har perifer interaktion ændret sig? (nu er det blevet skærmbaseret)
  \item Adfærdsændringer
  \item Relateret produkter  
  \item Baggrund og teori
  \item Calm og Casual
\end{itemize}
%
Som nævnt tidligere er perifer interaktion ikke nyt. Vi gør det alle sammen, når vi i løbet af vores dagligdag foretager adskillige aktiviteter i vores perifere opmærksomhed, hvilket forudsætter at disse perifere interaktioner primært retter sig mod ikke-computer-relateret aktiviteter, \parencite[s. 1]{PDF:PeripheralInteraction}. Rettes fokus derimod til \textit{Human-Computer-Interaction}, HCI, så forlanger diverse elektroniske enheder vores centrale opmærksomhed ved at blinke, ringe og vibrere, \parencite[s. 1]{PDF:PeripheralInteraction}. Ifølge \textcite[s. 3]{PDF:PeripheralInteraction} så bevæger interaktionen med elektroniske enheder sig uforudsigeligt mellem den centrale og perifere opmærksomhed, i modsætning til ikke-computer-relateret aktiviteter, som i større grad kan kontrolleres.      
%
\begin{figure}[H]
	\centering
	\includegraphics[resolution=300,width=\textwidth]{LevelsOfInteraction}
	\caption{ny \textcite[s. 118]{PDF:PeripheralInteraction}.}
	\label{fig:LevelsOfInteraction}
\end{figure}
\noindent
%


%
\begin{table}[H]
\centering
\begin{tabular}{|l|l|}
\hline
\multicolumn{1}{|l|}{\textbf{Primære funktioner}} & \multicolumn{1}{l|}{\textbf{Formål}} \\ \hline
Start & .. \\ \hline
Stop & .. \\ \hline
Standby & .. \\ \hline
Forward/backward in time & .. \\ \hline
Intensity up/down & .. \\ \hline
Like/dislike & .. \\ \hline
Shift source & .. \\ \hline
Shift experiences & .. \\ \hline
Initiate interaction & .. \\ \hline
Confirm & .. \\ \hline
\end{tabular}
\caption{NY - Primære funktioner}
\label{tab:BogOsPrimaereFunktioner}
\end{table}
\noindent
%

%
\begin{table}[H]
\centering
\begin{tabular}{|l|l|}
\hline
\multicolumn{1}{|l|}{\textbf{Sekundære funktioner}} & \multicolumn{1}{l|}{\textbf{Formål}} \\ \hline
Fastforward/fastbackward & .. \\ \hline
Stay in mood & .. \\ \hline
Send/expand & .. \\ \hline
Store & .. \\ \hline
Add to.. & .. \\ \hline
Play next & .. \\ \hline
Play later & .. \\ \hline
\end{tabular}
\caption{NY - Senkundære funktioner}
\label{tab:BogOsSekundaereFunktioner}
\end{table}
\noindent
%

